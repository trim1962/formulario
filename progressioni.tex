
% !BIB TS-program = biber
% !TeX encoding = UTF-8
% !TeX spellcheck = it_IT
\chapter{Progressioni}
\section{Progressione aritmetica}
\begin{defn}[Progressione aritmetica]\label{defn:ProgAritm1}
	Successione di numeri $a_1,a_2,a_3,\dots,a_n$ in cui la differenza tra un termine e il suo precedente è costante. Tale termine $d=a_r-a_{r-1}$ è chiamato ragione della successione.
\end{defn}\index{Progressione!aritmetica}\index{Progressione!aritmetica!ragione}
\begin{prop}
	Data una successione aritmetica $a_1,a_2,a_3,\dots,a_n$ di ragione $d$ allora\[a_r=a_1+(r-1)d\]
\end{prop}
\begin{thm}[Somma progressione aritmetica]\label{thm:SommaProgAritm1}
Data una progressione aritmetica $a_1,a_2,a_3,\dots,a_n$ allora
\begin{align*}
	S_n=&a_1+a_2+a_3+\dots+a_n\\
	=&\dfrac{a_1+a_n}{2}n\\
\end{align*}
\end{thm}\index{Progressione!aritmetica!somma}
\section{Progressione geometrica}
\begin{defn}[Progressione geometrica]\label{defn:ProgGeom1}
		Successione di numeri $a_1,a_2,a_3,\dots,a_n$ in cui il rapporto tra un termine e il suo precedente è costante. Tale termine $d=\dfrac{a_r}{a{r-1}}$ è chiamato ragione della successione.
\end{defn}\index{Progressione!geometrica}\index{Progressione!geometrica!ragione}
\begin{prop}
	Data una successione geometrica $a_1,a_2,a_3,\dots,a_n$ di ragione $d$ allora\[a_r=a_1d^{r-1}\]
\end{prop}
\begin{thm}[Somma progressione geometrica]\label{thm:SommaProgGeo}
		Data una successione geometrica $a_1,a_2,a_3,\dots,a_n$ di ragione $d$ allora 
		\begin{align*}
			S_n=&a_1+a_2+a_3+\dots+a_n\\
			=&a_1\dfrac{d^n-1}{d-1}
		\end{align*}
\end{thm}\index{Progressione!geometrica!somma}