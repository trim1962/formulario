\chapter{Logaritmi}
\section{Definizione}
Il logaritmo $x$ di $b$ in base $a$ è
\begin{equation}
x=\log_{a}b \quad\Longleftrightarrow\quad a^x=b\quad a>0,\; a\neq 1,\; b>0
\end{equation}\index{Logaritmo!definizione}
\section{Logaritmo decimale}
se la base è $e$ il logaritmo si dice naturale e si scrive $\ln$
\begin{equation}
\log_{e}b=\ln b
\end{equation}\index{Logaritmo!naturale}
\section{Proprietà fondamentali}
\begin{align}
a^{\log_{a}b}={}&b&a>0,\; a\neq 1,\; b>0\\
\log_{a}a^c=&c&a>0,\; a\neq 1\\
\log_{a}1=&0&a>0,\; a\neq 1\\
\log_{a}a=&1&a>0,\; a\neq 1
\end{align}\index{Logaritmo!proprietà!fondamentali}
\section{Logaritmo prodotto}
\begin{equation}
\log_{a}b\cdot c=\log_{a}b+\log_{a}c\quad a>0,\;a\neq 1,\;b>0,\;c>0 
\end{equation}\index{Logaritmo!prodotto}
\section{Logaritmo somma}
\begin{equation}
\log_{a}b+\log_{a}c=\log_{a}b\cdot c\quad a>0,\;a\neq 1,\;b>0,\;c>0 
\end{equation}\index{Logaritmo!somma}
\section{Logaritmo quoziente}
\begin{equation}
\log_{a}\dfrac{b}{c}=\log_{a}b-\log_{a}c\quad a>0,\;a\neq 1,\;b>0,\;c>0 
\end{equation}\index{Logaritmo!quoziente}
\section{Logaritmo differenza}
\begin{equation}
\log_{a}b-\log_{a}c=\log_{a}\dfrac{b}{c}\quad a>0,\;a\neq 1,\;b>0,\;c>0 
\end{equation}\index{Logaritmo!differenza}
\section{Logaritmo potenza}
\begin{equation}
\log_{a}b^m=m\log_{a}b\quad a>0,\;a\neq 1,\;b>0,\;c>0,\;m\in\R
\end{equation}\index{Logaritmo!potenza}
\section{Logaritmo radicale}
\begin{equation}
\log_{a}\sqrt[n]{b}=\dfrac{1}{n}\log_{a}b\quad a>0,\;a\neq 1,\;b>0,\;n\in\Nz
\end{equation}\index{Logaritmo!radicale}
\section{Cambiamento di base}
\begin{equation}
\log_{a}b=\dfrac{\log_{c}b}{\log_{c}a}\quad a>0,\;a\neq 1,\;c>0,\;c\neq 1,\;b>0
\end{equation}\index{Logaritmo!cambio di base}
\section{Utili}
\begin{align}
\log_{a}b=&\dfrac{1}{\log_{b}a}&a>0,\;a\neq 1,\;b>0,\;b\neq 1\\
\log_{\frac{1}{a}}b=&-\log_{a}b&a>0,\;a\neq 1,\;b>0
\end{align}\index{Logaritmo!reciproco!base}\index{Logaritmo!scambio!base argomento}