% !TeX encoding = UTF-8
% !TeX spellcheck = it_IT
% !TeX root = formulario.tex
\chapter{Funzioni irrazionali}\index{Funzione!irrazionale}
\section{Irrazionale intera pari}\index{Funzione!irrazionale!intera}
\begin{equation*}
f(x)=\sqrt[n]{A(x)}\quad n=2,4,6,\dots
\end{equation*}
\subsection{Dominio e positività}
\begin{enumerate}
	\item Dominio: $A(x)\geq 0$
	\item Positività: Sempre positiva
\end{enumerate}
\section{Irrazionale intera dispari}
\begin{equation*}\index{Funzione!irrazionale!dispari}
f(x)=\sqrt[n]{A(x)}\quad n=3,5,7,\dots
\end{equation*}\index{Funzione!irrazionale!pari}
\subsection{Dominio e positività}
\begin{enumerate}
	\item Dominio: Sempre definita
	\item Positività: $A(x)\geq 0$
\end{enumerate}
\section{Irrazionali fratte pari}\index{Funzione!irrazionale!fratta}
\begin{align*}
f(x)=&\dfrac{\sqrt{A(x)}}{B(x)}
\intertext{Dominio}\index{Funzione!irrazionale!dominio}
&\begin{cases}
A(x)\geq 0\\
B(x)=0
\end{cases}
\intertext{Positività}
B(x)>0&\wedge Dominio\\
f(x)=&\dfrac{A(x)}{\sqrt{B(x)}}
\intertext{Dominio}
B(x)>&0
\intertext{Positività}
A(x)\geq0&\wedge Dominio\\
f(x)=\sqrt{\dfrac{A(x)}{B(x)}}\\
\intertext{Dominio}
A(x)\geq0 \wedge B(x)>0
\intertext{Positività}
\intertext{Sempre positiva}
\end{align*}
