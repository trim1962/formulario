% !TeX encoding = UTF-8
% !TeX spellcheck = it_IT
% !TeX root = formulario.tex
\chapter{Forma goniometrica numeri complessi}
{\centering
	\includestandalone{geometria/polarerettangolare}
	\captionof{figure}{Polare rettangolare}\par}
\section{Argomento}
\begin{align*}
z=&a+b\uimm& z\in\Co\quad a,b\in\R\\[8pt]
\theta=\arg(z)=&\\
=a&\arctan(\dfrac{b}{a})&\text{se}\ a>0 \\[8pt]
=&\arctan(\dfrac{b}{a})+\pi&\text{se}\ a<0 \ \text{e} \ b\geq0\\[8pt]
=&\arctan(\dfrac{b}{a})-\pi&\text{se}\ a<0\ \text{e} \ b<0\\[8pt]
=&+\dfrac{\pi}{2}&\text{se}\ a=0 \ \text{e} \ b>0\\[8pt]
=&-\dfrac{\pi}{2}&\text{se}\ a=0 \ \text{e} \ b<0\\[8pt]
=&\text{non definito}&\text{se}\ a=0 \ \text{e} \ b=0
\end{align*}\index{Numero!complesso!argomento}
Se $\theta\in]-\pi,\pi]$

\begin{align*}
z=&a+b\uimm& z\in\Co\quad a,b\in\R\\[8pt]
\theta=\arg(z)=&\\
=&\arctan(\dfrac{b}{a})&\text{se}\ a>0\ b\geq 0 \\[8pt]
=&\arctan(\dfrac{b}{a})+2\pi&\text{se}\ a>0 \ \text{e} \ b<0\\[8pt]
=&\arctan(\dfrac{b}{a})+\pi&\text{se}\ a<0\ \text{e b qualsiasi} \ \\[8pt]
=&\dfrac{\pi}{2}&\text{se}\ a=0 \ \text{e} \ b>0\\[8pt]
=&\dfrac{3\pi}{2}&\text{se}\ a=0 \ \text{e} \ b<0\\[8pt]
=&\text{non definito}&\text{se}\ a=0 \ \text{e} \ b=0
\end{align*}\index{Numero!complesso!argomento}
Se $\theta\in[0,2\pi)$
 \section{Modulo}
\begin{equation*}
z=a+b\uimm\quad r=\abs{z}=\sqrt{a^2+b^2}\quad a,b\in\R
\end{equation*}\index{Numero!complesso!modulo}
\section{Da forma algebrica a goniometrica di numero complesso}
\begin{align*}
z=&a+b\uimm\quad z\in\Co\quad a,b\in\R\\
\theta=&\arg(z)\\
r=&\abs{z}\\
z=&r[\cos\theta+\uimm\sin\theta]
\end{align*}\index{Numero!complesso!forma goniometrica}\index{Numero!complesso!modulo}\index{Numero!complesso!argomento}
\section{Da forma goniometrica ad algebrica di numero complesso}
\begin{align*}
z=&r[\cos\theta+\uimm\sin\theta]\\
z=&a+b\uimm	\\
a=&r\cos\theta\\
b=&r\sin\theta
\end{align*}\index{Numero!complesso!forma algebrica}\index{Numero!complesso!modulo}\index{Numero!complesso!argomento}
