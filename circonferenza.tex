\chapter{Circonferenza}
\section{Equazione}
Centro $C(\alpha,\beta)$ raggio $r$
\begin{equation}
(x-\alpha)^2+(y-\beta)^2=r^2
\end{equation}\index{Circonferenza!equazione}
\section{Forma canonica}
\begin{align}
x^2+y^2+ax+by+c=&0\\
a=&-2\alpha\\
b=&-2\beta\\
c=&\alpha^2+\beta^2-r^2\\
\alpha=&-\dfrac{a}{2}\\
\beta=&-\dfrac{b}{2}\\
r=&\sqrt{\alpha^2+\beta^2-c}
\end{align}\index{Circonferenza!forma canonica}\index{Circonferenza!raggio}\index{Circonferenza!coordinate centro}

\section{Circonferenza per tre punti}
\begin{align}
P(x_1;y_1)&\\
Q(x_2;y_2)&\\
M(x_3;y_3)&\\
\begin{cases}
x_1^2+y_1^2+ax_1+by_1+c=0\\
x_2^2+y_2^2+ax_2+by_2+c=0\\
x_3^2+y_3^2+ax_3+by_3+c=0\\
\end{cases}&
\end{align}\index{Circonferenza!per tre punti}
\chapter{Circonferenza e retta}
\begin{equation}
\begin{cases}
x^2+y^2+ax+by+c=0\\
y=mx+q
\end{cases}
\end{equation}\index{Circonferenza!retta}\index{Retta!circonferenza}
\begin{equation}
x^2+(mx+q)^2+ax+b(mx+q)+c=0\quad\begin{cases}
\text{Se $\Delta >0$ Secante}\\
\text{Se $\Delta =0$ Tangente}\\
\text{Se $\Delta <0$ Esterna}\\
\end{cases}
\end{equation}\index{Circonferenza!intersezione!retta}\index{Discriminante}\index{Retta!secante}\index{Retta!tangente}\index{Retta!esterna}\index{Delta}
