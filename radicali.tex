% !TeX encoding = UTF-8
% !TeX spellcheck = it_IT
% !TeX root = formulario.tex
\chapter{Radicali}
\section{Glossario}
\begin{equation*}
\sqrt[n]{a}=b\\
\end{equation*}
$n$ indice, $a$ radicando, $b$ radice e $\sqrt[n]{a}$ radicale
\section{Definizione}
\begin{align*}
\intertext{n pari}
\sqrt[n]{a}=&b\Longleftrightarrow b^n=a&a\geq0\quad b\geq0\quad n\in\Nz\\
\intertext{n dispari}
\sqrt[n]{a}=&b\Longleftrightarrow b^n=a&a,b\in\R\quad n\in\Nz
\end{align*}\index{Radicale!definizione}
\section{Segno}
\begin{align*}
\intertext{n pari}
\sqrt[n]{a}&&\text{sempre positivo}\\
\intertext{n dispari}
\sqrt[n]{a}&&\text{segno radicando}
\end{align*}\index{Radicale!segno}
\section{Proprietà invariantiva}
\begin{align*}
\sqrt[n\cdot s]{a^{m\cdot s}}=&\sqrt[n]{a^m}&\forall a\geq 0\quad n,m,s\in\Nz\\
\sqrt[n]{a^m}=&\sqrt[n\cdot s]{a^{m\cdot s}}&\forall a\geq 0\quad n,m,s\in\Nz
\end{align*}\index{Radicale!proprietà invariantiva}\index{Proprietà!invariantiva}
\section{Proprietà}
\begin{align*}
\intertext{n pari}
\sqrt[n]{a^n}=&\abs{a}\\
\intertext{n dispari}
\sqrt[n]{a^n}=&a
\end{align*}
\section{Prodotto con indici uguali}
\begin{align*}
\sqrt[n]{a}\sqrt[n]{b}=&\sqrt[n]{ab}\\
\sqrt[n]{ab}=&\sqrt[n]{a}\sqrt[n]{b}
\end{align*}\index{Radicale!prodotto indici uguali}
\section{Prodotto con indici diversi}
\begin{enumerate}
	\item Per eseguire $\sqrt[n]{a^p}\cdot\sqrt[m]{b^q}$
	\item calcolo il $\mcm(m,n)$ che diventerà l'indice  delle due radici
	\item divido il  $\mcm$ per $n$ e il risultato lo moltiplico per $p$
	\item divido il  $\mcm$ per $m$ e il risultato lo moltiplico per $q$
	\item eseguiamo  il prodotto tra le due radici che ora hanno lo stesso indice.
\end{enumerate}\index{Radicale!prodotto indici diversi}
\section{Divisione con indici uguali}
\begin{align*}
\dfrac{\sqrt[n]{a}}{\sqrt[n]{b}}=&\sqrt[n]{\dfrac{a}{b}}&b\neq 0\\
\sqrt[n]{\dfrac{a}{b}}=&\dfrac{\sqrt[n]{a}}{\sqrt[n]{b}}&b\neq 0
\end{align*}\index{Radicale!quoziente indici uguali}
\section{Divisione con indici diversi}
\begin{enumerate}
	\item Per eseguire
	$\dfrac{\sqrt[n]{a^p}}{\sqrt[m]{b^q}}$
	\item Calcolare il $\mcm(m,n)$ che diventerà l'indice  delle due radici
	\item Dividere il  $\mcm$ per $n$ e il risultato lo moltiplico per $p$
	\item Divider il  $\mcm$ per $m$ e il risultato lo moltiplico per $q$
	\item Eseguire la divisione fra le due radici che ora hanno lo stesso indice.
\end{enumerate}\index{Radicale!quoziente indici diversi}
\section{Potenza radicale}
\begin{align*}
\left(\sqrt[n]{a}\right)^m=&\sqrt[n]{a^m}\\
\sqrt[n]{a^m}=&\left(\sqrt[n]{a}\right)^m\\
\end{align*}\index{Radicale!potenza}
\section{Radice di radice}
\begin{equation*}
\sqrt[n]{\sqrt[m]{a}}=\sqrt[n\cdot m]{a}
\end{equation*}\index{Radicale!radice di radice}
\section{Trasporto di un termine dentro al segno di radice}
\begin{equation*}
b\sqrt[n]{a}=\sqrt[n]{b^na}\quad a\geq 0,b\geq 0 
\end{equation*}\index{Radicale!trasporto dentro}
\section{Trasporto di un termine fuori dal segno di radice}
\begin{equation*}
\sqrt[n]{a^nb}=\abs{a}\sqrt[n]{b}
\end{equation*}
\begin{align*}
\sqrt[n]{a^m}&\quad m\geq n\\
m=&n\cdot q+r\\
\sqrt[n]{a^m}=&\sqrt[n]{a^{n\cdot q+r}}\\
=&a^q\sqrt[n]{a^r}
\end{align*}\index{Radicale!trasporto fuori}
\section{Somma di radicali}
Due radicali sono simili se hanno stesso indice e radicando.\index{Radicale!radici!simili}
\begin{align*}
b\sqrt[n]{a^m}+c\sqrt[n]{a^m}=&(b+c)\sqrt[n]{a^m}\\
b\sqrt[n]{a}+c\sqrt[n]{a^m}=&b\sqrt[n]{a}+c\sqrt[n]{a^m}
\end{align*}
\chapter{Razionalizzazione di frazioni}
\section{Radicale quadrato al denominatore}
\begin{align*}
\dfrac{b}{\sqrt{a}}=&\dfrac{b}{\sqrt{a}}\cdot\dfrac{\sqrt{a}}{\sqrt{a}}\\
=&\dfrac{b\sqrt{a}}{a}
\end{align*}\index{Radicale!razionalizzazione}
\section{Radicale di indice qualunque al denominatore}
\begin{align*}\index{Radicale!razionalizzazione}
\intertext{radici qualunque $n>m$}
\dfrac{b}{\sqrt[n]{a^m}}=&\\
=&\dfrac{b}{\sqrt[n]{a^m}}\cdot\dfrac{\sqrt[n]{a^{n-m}}}{\sqrt[n]{a^{n-m}}}\\
=&\dfrac{b\sqrt[n]{a^{n-m}}}{a}
\intertext{radici qualunque $m>n$}
\intertext{Prima trasporto fuori il termine e poi procediamo come prima.}\nonumber
\end{align*}
\section{Differenza di quadrati}
\begin{align*}
\dfrac{b}{\sqrt{a}+\sqrt{c}}=&\\
=&\dfrac{b}{\sqrt{a}+\sqrt{c}}\cdot\dfrac{\sqrt{a}-\sqrt{c}}{\sqrt{a}-\sqrt{c}}\\
=&\dfrac{b(\sqrt{a}-\sqrt{c})}{a-c}
\end{align*}\index{Radicale!razionalizzazione!differenza quadrati}
\begin{align*}
\dfrac{b}{\sqrt{a}-\sqrt{c}}=&\\
=&\dfrac{b}{\sqrt{a}+\sqrt{c}}\cdot\dfrac{\sqrt{a}+\sqrt{c}}{\sqrt{a}+\sqrt{c}}\\
=&\dfrac{b(\sqrt{a}-\sqrt{c})}{a-c}
\end{align*}
\section{Somma e differenza di cubi}
\begin{align*}
\dfrac{b}{\sqrt[3]{a}+\sqrt[3]{c}}=&\\
=&\dfrac{b}{\sqrt[3]{a}+\sqrt[3]{c}}\cdot\dfrac{\sqrt[3]{a}-\sqrt[3]{ac}+\sqrt[3]{c}}{\sqrt[3]{a}-\sqrt[3]{ac}+\sqrt[3]{c}}\\
=&\dfrac{b(\sqrt[3]{a}-\sqrt[3]{ac}+\sqrt[3]{c})}{a+c}
\end{align*}\index{Radicale!razionalizzazione!differenza cubi}
\begin{align*}
\dfrac{b}{\sqrt[3]{a}-\sqrt[3]{c}}=&\\
=&\dfrac{b}{\sqrt[3]{a}-\sqrt[3]{c}}\cdot\dfrac{\sqrt[3]{a}+\sqrt[3]{ac}+\sqrt[3]{c}}{\sqrt[3]{a}+\sqrt[3]{ac}+\sqrt[3]{c}}\\
=&\dfrac{b(\sqrt[3]{a}+\sqrt[3]{ac}+\sqrt[3]{c})}{a-c}
\end{align*}\index{Radicale!razionalizzazione!somma cubi}
\chapter{Esponente frazionario}
\section{Definizione}
\begin{align*}
\sqrt[n]{a}=&a^{\frac{1}{n}}&a\geq 0\quad n\in\Ni
\quad n\neq 0\\
\sqrt[n]{a^m}=&a^{\frac{m}{n}}&a\geq 0\quad m,n\in\Ni
\quad n\neq 0
\end{align*}\index{Potenza!esponente frazionario}
\section{Prodotto tra radicali}
\begin{equation*}
a^{\frac{p}{n}}\cdot a^{\frac{q}{m}}=a^{\frac{p}{n}+\frac{q}{m}}
\end{equation*}\index{Radicale!prodotto}
\section{Quoziente tra radicali}
\begin{equation*}
a^{\frac{p}{n}}\div a^{\frac{q}{m}}=a^{\frac{p}{n}-\frac{q}{m}}
\end{equation*}\index{Radicale!quoziente}
\section{Radicale di radicale}
\begin{equation*}
\left(a^{\frac{p}{n}}\right)^{\frac{q}{m}}=a^{\frac{p}{n}\cdot\frac{q}{m}}
\end{equation*}\index{Radicale!radice di radice}
\section{Prodotto fra radicali che hanno stesso indice}
\begin{equation*}
a^{\frac{p}{n}}\cdot b^{\frac{p}{n}}=\left(a\cdot b\right)^{\frac{p}{n}}
\end{equation*}
\section{Quoziente fra radicali che hanno stesso indice}
\begin{equation*}
a^{\frac{p}{n}}\div b^{\frac{p}{n}}=\left(a\div b\right)^{\frac{p}{n}}
\end{equation*}
\section{Potenze con esponente negativo}
\begin{equation*}
a^{-\frac{p}{n}}=\sqrt[b]{\left(\dfrac{1}{a}\right)^p}\quad a>0\quad n\in\Ni
\quad n\neq 0
\end{equation*}\index{Potenza!esponente frazionario!negativo}
