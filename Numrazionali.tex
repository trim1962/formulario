\chapter{Numeri razionali}
\section{Simbolo}
\begin{equation}
\Q
\end{equation}
\section{Frazione definizione}
\begin{equation}
\text{frazione}=\dfrac{\text{numeratore}}{\text{denominatore}}
\end{equation}\index{Numeratore}\index{Denominatore}
\section{Proprietà invariantiva}
\begin{align}
\dfrac{a}{b}=&\dfrac{a\times c}{b\times c}&c\neq 0\\
\dfrac{a}{b}=&\dfrac{a\div c}{b\div c}&c\neq 0
\end{align}\index{Raziona li!proprietà!invariantiva}
\section{Frazioni equivalenti}
\begin{equation}
\dfrac{a}{b}\equiv\dfrac{c}{d}\quad\Longleftrightarrow\quad bc=ad
\end{equation}\index{Razionali!frazioni!equivalenti}
\section{Frazioni proprie}
\begin{equation}
\dfrac{a}{b}\quad\text{frazione propria}\quad\Longleftrightarrow\quad a<b
\end{equation}\index{Razionali!frazioni!proprie}
\section{Frazioni improprie}
\begin{equation}
\dfrac{a}{b}\quad\text{frazione impropria}\quad\Longleftrightarrow\quad a>b\quad\text{a non multiplo di b}
\end{equation}\index{Razionali!frazioni!improprie}
\section{Frazioni apparenti}
\begin{equation}
\dfrac{a}{b}\quad\text{frazione apparente}\quad\Longleftrightarrow\text{$a$ multiplo di $b$}
\end{equation}\index{Razionali!frazioni!apparenti}
\section{Minimo comune denominatore}
\begin{enumerate}
	\item Scomporre in fattori i singoli denominatori
	\item Prendere i fattori comuni e non comuni con il maggiore esponente
	\item Il  numero ottenuto è il massimo comune denominatore.
\end{enumerate}
%{\centering\captionof{table}{Proprietà numeri razionali}
%	\begin{tabular}{lLL}
%		\toprule
%		Proprietà	& Somma & Prodotto  \\ 
%		\midrule
%		associativa	& (a+b)+c=a+(b+c) & (a\times b)\times c=a\times(b\times c) \\ 
%		commutativa	&a+b=b+a  &a\times b=b\times a  \\ 
%		elemento neutro	&a+0=0+a=a  & a\times 1=1\times a=a \\ 
%		inverso&(-a)+a=a=(-a)=0&\dfrac{1}{a}\times a=a\times\dfrac{1}{a}=1\\
%		distributiva	&(a+b)\times c=a\times c+b\times c  &  \\ 
%		assorbimento	&  & a\times 0=0\times a=0 \\ 
%		\bottomrule
%	\end{tabular}
%	\par}\index{Proprietà!associativa}\index{Proprietà!commutativa}\index{Elemento!neutro}\index{Proprietà!distributiva}\index{Proprietà!assorbimento}\index{Proprietà!inverso}
\section{Ordinamento}
\begin{align}
\frac{a}{b}<\frac{c}{d} &\Longleftrightarrow ad<bc\\
\frac{a}{b}=\frac{c}{d} &\Longleftrightarrow ad=bc\\
\frac{a}{b}>\frac{c}{d} &\Longleftrightarrow ad>bc
\end{align}\index{Razionali!ordinamento}
\section{Operazioni}
\subsection{Somma}
\begin{align}
\frac{a}{b}\pm\frac{c}{b}=&\frac{a\pm c}{b}\\
\frac{a}{b}\pm\frac{c}{d}=&\frac{(\mcm(b,d):b)\cdot a\pm(\mcm(b,d):d)\cdot c}{\mcm(b,d)}
\end{align}\index{Razionali!somma}
\subsection{Prodotto}
\begin{align}
\frac{a}{b}\cdot\frac{c}{d}=&\frac{a\cdot c}{b\cdot d}
\end{align}\index{Razionali!prodotto}
\subsection{Divisione}
\begin{align}
\frac{a}{b}:\frac{c}{d}=&\frac{a}{b}\cdot\frac{d}{c}\\
\frac{\frac{a}{b}}{\frac{c}{d}}=&\frac{a}{b}\cdot\frac{d}{c}
\end{align}\index{Razionali!divisione}
