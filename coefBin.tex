\chapter{Calcolo combinatorio}\label{ch:calcolo-combinatorio}
\section{Fattoriale}\label{sec:fattoriale}
\begin{defn}[Fattoriale di un numero]\label{defn:Fattoriale-num}
Dato un numero $n\in\Ni$ diremo fattoriale di un numero il prodotto	
\begin{align*}
n!=&1\cdot 2\cdot 2\cdots n&&n\in\Ni\\
0!=&1\\
\end{align*}
\end{defn}\index{Fattoriale}
\begin{defn}[Disposizioni semplici]\label{defn:Diposizioni-semplici}
	Dati $n$ oggetti distinti, $n\in\Ni$ e un $k\in\Ni$ $k\leqslant n$ diremo disposizione semplice di $n$ oggetti a gruppi di $k$ tutti i gruppi di $k$ oggetti in modo che ogni gruppo differisca o per gli oggetti che lo compongono o per l'ordine degli oggetti nel gruppo.
\end{defn}\index{Disposizioni!semplici}
\begin{defn}[Disposizioni con ripetizione]\label{defn:Diposizioni-ripetizione}
	Raggruppamento di $n\in\Ni$ oggetti in gruppi di $k\in\Ni$. In ogni gruppo un oggetto può comparire più volte. Ogni gruppo differisce dall'altro o per gli elementi o per l'ordine con cui compaiono. 
\end{defn}