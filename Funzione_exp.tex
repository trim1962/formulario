% !TeX encoding = UTF-8
% !TeX spellcheck = it_IT
% !TeX root = formulario.tex
\chapter{Funzione esponenziale}
\section{Definizione}
Una funzione esponenziale è una funzione del tipo
\begin{equation}
\function{f}{\R}{\Rpos}{x}{a^x}\quad a>0\quad a\neq1
\end{equation}\index{Funzione!esponenziale!definizione}
\section{Caso base maggiore di uno}
Se la base a è maggiore di uno
\begin{itemize}
	\item La funzione è trascendente\index{Funzione!trascendente}
	\item Il dominio è $\R$
	\item Il codominio è $\Rpos$
	\item La funzione passa per $A(0,1)$
	\item La funzione è crescente in senso stretto $
	\forall\; x_1,x_2\in I\quad x_1< x_2\Longrightarrow f(x_1)<f(x_2)$\index{Funzione!crescente in senso stretto}
	\item L'asse delle $x$ è un asintoto orizzontale per la funzione $\lim_{x\to-\infty} f(x)=0$\index{Asintoto!orizzontale} 
\end{itemize}
\begin{center}
	\includestandalone{geometria/expMadiuno}
	\captionof{figure}{Funzione esponenziale a >1}
\end{center}\index{Funzione!esponeneziale}
\section{Caso base compresa tra zero e uno}
Se la base a è compresa tra zero e uno
\begin{itemize}
	\item La funzione è trascendente\index{Funzione!trascendente}
	\item Il dominio è $\R$
	\item Il codominio è $\Rpos$
	\item La funzione passa per $A(0,1)$
	\item La funzione è decrescente in senso stretto $
	\forall\; x_1,x_2\in I\quad x_1< x_2\Longrightarrow f(x_1)>f(x_2)$\index{Funzione!decrescente in senso stretto}
	\item L'asse delle $x$ è un asintoto orizzontale per la funzione $\lim_{x\to +\infty} f(x)=0$\index{Asintoto!orizzontale} 
\end{itemize}
\begin{center}
	\includestandalone{geometria/expMidiuno}
	\captionof{figure}{Funzione esponenziale 0<a<1}
\end{center}\index{Funzione!esponeneziale}
