% !TeX encoding = UTF-8
% !TeX spellcheck = it_IT
% !TeX root = formulario.tex
\chapter{Equazioni di primo grado}
\section{Definizione}
\begin{align*}
ax+b&={}0\quad a\neq 0\\
x_1&=-\dfrac{b}{a}
\end{align*}\index{Equazione!primo grado}
\section{Classificazione delle equazioni}
{\centering\captionof{table}{Classificazione equazioni primo grado}
	\begin{tabular}{cl}
		\toprule
		Coefficienti&Tipo\\
		\midrule
		$a=0$\quad $b\neq 0$	& Equazione impossibile  \\ 
		$a=0$\quad $b=0$	& Equazione indeterminata\\ 
		$a\neq0$	& Equazione determinata  \\ 
		\bottomrule
	\end{tabular}\par}\index{Equazione!primo grado!classificazione}
\section{Risoluzione}
Metodo separazione variabili\index{Equazione!primo grado!separazione}
\chapter{Equazioni di secondo grado}
\section{Equazioni pure}
\begin{align*}
ax^2+c=0&\quad a\neq 0
\intertext{se $a$ e $c$ discordi}
x_{1,2}=&\pm\sqrt{-\dfrac{c}{a}}
\intertext{se $a$ e $c$ concordi non si hanno soluzioni}\nonumber
\end{align*}\index{Equazione!secondo grado!pure}
\section{Equazioni spurie}
\begin{align*}
ax^2+bx=&0\quad a\neq 0\\
x(ax+b)=&0\\
x_1=&0\\
ax+b=&0\\
x_2=&-\dfrac{b}{a}
\end{align*}\index{Equazione!secondo grado!spurie}
\section{Equazioni monomia}
\begin{equation*}
ax^2=0\quad a\neq 0
\end{equation*}\index{Equazione!secondo grado!monomia}
\begin{equation*}
x_{1,2}=0
\end{equation*}
\section{Equazioni complete}
\begin{align*}
ax^2+bx+c=&0\quad a\neq 0\\
x_{1,2}=&\dfrac{-b\pm\sqrt{b^2-4ac}}{2a}
\end{align*}\index{Equazione!secondo grado!complete}
\section{Il delta e la classificazione delle soluzioni}
\begin{equation*}
\Delta=b^2-4ac
\end{equation*}\index{Equazione!secondo grado!discriminante}\index{Discriminante}\index{Delta}
\index{Equazione!secondo grado!delta}
\begin{equation*}
ax^2+bx+c=0\quad\begin{cases}
\text{Se $\Delta >0$ Soluzioni distinte}\\
\text{Se $\Delta =0$ Soluzioni coincidenti}\\
\text{Se $\Delta <0$ Nessuna soluzione reale}\\
\end{cases}
\end{equation*}\index{Equazione!secondo grado!classificazione soluzioni}
\section{Proprietà soluzioni}
\begin{equation*}
\begin{cases}
x_1+x_2=-\dfrac{b}{a}\\
x_1\cdot x_2=\dfrac{c}{a}
\end{cases}\index{Equazione!secondo grado!somma soluzioni}\index{Equazione!secondo grado!prodotto soluzioni}
\end{equation*}
\section{Scomposizione trinomio}
\begin{align*}
\intertext{$\Delta>0$}
ax^2+bx+c=&a(x-x_1)(x-x_2)\\
\intertext{$\Delta=0$}
ax^2+bx+c=&a(x-x_1)^2\\
\intertext{$\Delta<0$}
ax^2+bx+c=&a\left[\left(x+\dfrac{b}{2a}\right)^2+\dfrac{4ac-b^2}{4a^2}\right]
\end{align*}\index{Equazione!secondo grado!scomposizione trinomio}\index{Scomposizione!trinomio}\index{Delta}\index{Discriminante}
\chapter{Equazioni binomie}
\begin{align*}
ax^n+b=&0&a\neq 0\\
x^n=&-\frac{b}{a}
\intertext{Se $n$ pari}
x=&\pm\sqrt[n]{-\frac{b}{a}}&\begin{cases}
\text{$a$ e $b$ concordi}& \text{non ha soluzione}\\
\text{$a$ e $b$ discordi}& x=\pm\sqrt[n]{-\frac{b}{a}}\\
\end{cases}\\
\intertext{Se $n$ dispari}
x=&\sqrt[n]{-\frac{b}{a}}\\
\end{align*}\index{Equazione!binomie}
\chapter{Equazioni biquadratiche}
\begin{align*}
ax^4+bx^2+c=&0&a\neq 0\\
x^2=&y\\
ay^2+by+c=&0\\
x^2=&y_1\\
x^2=&y_2
\end{align*}\index{Equazione!biquadratica}
\chapter{Equazioni trinomie}
\begin{align*}
ax^{2n}+bx^n+c=&0&a\neq 0\\
x^n=&y\\
ay^2+by+c=&0\\
\intertext{$\Delta<0$}
\intertext{l'equazione trinomia non ha soluzione}
\intertext{$\Delta=0$}
x^n=-\frac{b}{2a}
\intertext{$\Delta>0$}
x^n=&\dfrac{-b+\sqrt{b^2-4ac}}{2a}\\
x^n=&\dfrac{-b-\sqrt{b^2-4ac}}{2a}\\
\end{align*}\index{Equazione!trinomie}