% !TeX encoding = UTF-8
% !TeX spellcheck = it_IT
% !TeX root = formulario.tex
\chapter{Trigonometria}
\section{Triangoli rettangolo}
\begin{center}
		\includestandalone{geometria/triangolopitagorico1}
	\captionof{figure}{Triangolo rettangolo}
\end{center}
\begin{equation*}
\dfrac{\pi}{2}+\beta+\gamma=\pi
\end{equation*}\index{Triangolo!somma!angoli interni}
\begin{align*}
b=&a\sin\beta=a\cos\gamma\\
c=&a\sin\gamma=a\cos\beta
\end{align*}\index{Triangolo!rettangolo!cateto}
\begin{align*}
\sin\beta=&\dfrac{b}{a}=\dfrac{\text{opposto}}{\text{ipotenusa}}\\
\cos\beta=&\dfrac{c}{a}=\dfrac{\text{adiacente}}{\text{ipotenusa}}\\
\sin\gamma=&\dfrac{c}{a}=\dfrac{\text{opposto}}{\text{ipotenusa}}\\
\cos\gamma=&\dfrac{b}{a}=\dfrac{\text{adiacente}}{\text{ipotenusa}}
\end{align*}\index{Triangolo!rettangolo!adiacente}
\index{Triangolo!rettangolo!opposto}
\begin{equation*}
a=\dfrac{b}{\sin\beta}=\dfrac{b}{\cos\gamma}=\dfrac{c}{\sin\gamma}=\dfrac{c}{\cos\beta}
\end{equation*}
\begin{align*}
b=&c\tan\beta\\
c=&b\tan\gamma
\end{align*}
\begin{align*}
\tan\beta=&\dfrac{b}{c}=\dfrac{\text{opposto}}{\text{adiacente}}\\
\tan\gamma=&\dfrac{c}{b}=\dfrac{\text{opposto}}{\text{adiacente}}
\end{align*}\index{Triangolo!rettangolo!adiacente}
\index{Triangolo!rettangolo!opposto}
\section{Triangolo qualunque}
\begin{center}
	\includestandalone{geometria/triangoloisc}
	\captionof{figure}{Triangolo qualunque}
\end{center}
\begin{equation*}
\alpha+\beta+\gamma=\pi
\end{equation*}\index{Triangolo!somma!angoli interni}
\section{Teorema della corda}
\begin{align*}
a=&2R\sin\alpha\\
b=&2R\sin\beta\\
c=&2R\sin\gamma
\end{align*}\index{Corda!teorema}\index{Teorema!corda}
\section{Teorema dei seni}
\begin{equation*}
\dfrac{a}{\sin\alpha}=\dfrac{b}{\sin\beta}=\dfrac{c}{\sin\gamma}=2R
\end{equation*}\index{Triangolo!teorema seni}\index{Seno!teorema}\index{Teorema!seno}
\section{Raggio cerchio circoscritto}
\begin{equation*}
R=\dfrac{a}{2\sin\alpha}=\dfrac{b}{2\sin\beta}=\dfrac{c}{2\sin\gamma}
\end{equation*}\index{Triangolo!raggio!circoscritto}\index{Circonferenza!triangolo!circoscritto}
\section{Teorema di Carnot}
\begin{align*}
a^2=&b^2+c^2-2bc\cos\alpha\\
b^2=&a^2+c^2-2ac\cos\beta\\
c^2=&a^2+b^2-2ab\cos\gamma
\end{align*}\index{Triangolo!teorema Carnot}\index{Carnot!teorema}\index{Teorema!Carnot}\index{Triangolo!lato}
\section{Area triangolo}
\begin{align*}
S=&\dfrac{1}{2}bc\sin\alpha\\
S=&\dfrac{1}{2}ba\sin\gamma\\
S=&\dfrac{1}{2}ac\sin\beta
\end{align*}\index{Triangolo!area}
\begin{align*}
S=&\dfrac{1}{2}a^2\dfrac{\sin\beta\sin\gamma}{\sin\alpha}\\
S=&\dfrac{1}{2}b^2\dfrac{\sin\alpha\sin\gamma}{\sin\beta}\\
S=&\dfrac{1}{2}c^2\dfrac{\sin\alpha\sin\beta}{\sin\gamma}
\end{align*}\index{Triangolo!area}