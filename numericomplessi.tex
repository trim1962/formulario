% !TeX encoding = UTF-8
% !TeX spellcheck = it_IT
% !TeX root = formulario.tex
\chapter{Numeri complessi}
\section{Unità immaginaria}
\begin{equation*}
\uimm^2=-1
\end{equation*}\index{Unità immaginaria}
\begin{equation*}
\uimm=\sqrt{-1}
\end{equation*}
\section{Numero complesso}
\begin{equation*}
z=a+b\uimm\quad z\in\Co\quad a,b\in\R
\end{equation*}\index{Numero!complesso}
\section{Uguaglianza numeri complessi}
\begin{equation*}
a+b\uimm=c+d\uimm\quad\Longleftrightarrow\quad a=c\quad b=d  \quad a,b,c,d\in\R
\end{equation*}\index{Numero!complesso!uguaglianza}
\section{Parte reale}
\begin{equation*}
z=a+b\uimm\quad\Re(z)=a\quad a,b\in\R
\end{equation*}\index{Numero!complesso!parte reale}
\section{Parte immaginaria}
\begin{equation*}
z=a+b\uimm\quad\Im(z)=b\quad a,b\in\R
\end{equation*}\index{Numero!complesso!parte immaginaria}
\section{Modulo}
\begin{equation*}
z=a+b\uimm\quad r=\abs{z}=\sqrt{a^2+b^2}\quad a,b\in\R
\end{equation*}\index{Numero!complesso!modulo}
\section{Complessi coniugati}
\begin{align*}
z=a+b\uimm\quad\conj{z}=a-b\uimm\quad a,b\in\R\\
\conj{\conj{z}}=&z\\
\end{align*}\index{Numero!complesso!coniugato}
\section{Complessi opposti}
\begin{equation*}
z=a+b\uimm\quad\-z=-a-b\uimm\quad a,b\in\R
\end{equation*}\index{Numero!complesso!opposto}
\section{Somma di numeri complessi}
\begin{align*}
z_1=&a+b\uimm\\
z_2=&c+d\uimm\\
z_1+z_2=&a+c+(b+d)\uimm
\end{align*}\index{Numero!complesso!somma}
\section{Somma di numeri complessi coniugati}
\begin{align*}
z=&a+b\uimm\\
\conj{z}=&a-b\uimm\\
z+\conj{z}=&a+a=2a\\
\conj{z\pm w}=&\conj{z}\pm\conj{w}\\
\end{align*}\index{Numero!complesso!coniugato}
\section{Differenza di numeri complessi}
\begin{align*}
z_1=&a+b\uimm\\
z_2=&c+d\uimm\\
z_1+z_2=&a-c+(b-d)\uimm
\end{align*}\index{Numero!complesso!differenza}
\section{Differenza di numeri complessi coniugati}
\begin{align*}
z=&a+b\uimm\\
\conj{z}=&a-b\uimm\\
z-\conj{z}=&+2b\uimm
\end{align*}\index{Numero!complesso!coniugato}
\section{Prodotto di numeri complessi}
\begin{align*}
z_1=&a+b\uimm\\
z_2=&c+d\uimm\\
z_1\cdot z_2=&(ac-bd)+(ad+bc)\uimm
\end{align*}\index{Numero!complesso!prodotto}
\section{Prodotto di numeri complessi coniugati}
\begin{align*}
z=&a+b\uimm\\
\conj{z}=&a-b\uimm\\
z\cdot\conj{z}=&a^2+b^2\\
\conj{zw}=&\conj{z}\conj{w}
\end{align*}
\section{Reciproco numero complesso }
\begin{align*}
z_1=&a+b\uimm\\
z_2=&c+d\uimm\\
z_1\cdot z_2=&1\\
z_2=&\dfrac{\conj{z_1}}{\abs{z_1}^2}=z^{-1}_1\\
\end{align*}\index{Numero!complesso!reciproco}
\section{Quoziente di due numeri complessi}
\begin{align*}
z_1=&a+b\uimm\\
z_2=&c+d\uimm\\
\dfrac{z_1}{z_2}=&\dfrac{z_1}{z_2}\cdot\dfrac{\conj{z}_2}{\conj{z}_2}\\
\dfrac{z_1}{z_2}=&\dfrac{a+b\uimm}{c+d\uimm}\cdot\dfrac{c-d\uimm}{c-d\uimm}\\
=&\dfrac{ac+bd+(bc-ad)\uimm}{c^2+d^2}
\end{align*}\index{Numero!complesso!quoziente}
\section{Prodotto per scalare}
\begin{align*}
z=&a+b\uimm\\
k\in&\R\\
k\cdot z=&k\cdot a+k\cdot b\uimm\\
\end{align*}\index{Numero!complesso!prodotto per scalare}
