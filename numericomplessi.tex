% !TeX encoding = UTF-8
% !TeX spellcheck = it_IT
% !TeX root = formulario.tex
\chapter{Numeri complessi}
\section{Unità immaginaria}
\begin{equation}
\uimm^2=-1
\end{equation}\index{Unità immaginaria}
\begin{equation}
\uimm=\sqrt{-1}
\end{equation}
\section{Numero complesso}
\begin{equation}
z=a+b\uimm\quad z\in\Co\quad a,b\in\R
\end{equation}\index{Numero!complesso}
\section{Uguaglianza numeri complessi}
\begin{equation}
a+b\uimm=c+d\uimm\quad\Longleftrightarrow\quad a=c\quad b=d  \quad a,b,c,d\in\R
\end{equation}\index{Numero!complesso!uguaglianza}
\section{Parte reale}
\begin{equation}
z=a+b\uimm\quad\Re(z)=a\quad a,b\in\R
\end{equation}\index{Numero!complesso!parte reale}
\section{Parte immaginaria}
\begin{equation}
z=a+b\uimm\quad\Im(z)=b\quad a,b\in\R
\end{equation}\index{Numero!complesso!parte immaginaria}
\section{Modulo}
\begin{equation}
z=a+b\uimm\quad r=\abs{z}=\sqrt{a^2+b^2}\quad a,b\in\R
\end{equation}\index{Numero!complesso!modulo}
\section{Complessi coniugati}
\begin{equation}
z=a+b\uimm\quad\conj{z}=a-b\uimm\quad a,b\in\R
\end{equation}\index{Numero!complesso!coniugato}
\section{Complessi opposti}
\begin{equation}
z=a+b\uimm\quad\-z=-a-b\uimm\quad a,b\in\R
\end{equation}\index{Numero!complesso!opposto}
\section{Somma di numeri complessi}
\begin{align}
z_1=&a+b\uimm\\
z_2=&c+d\uimm\\
z_1+z_2=&a+c+(b+d)\uimm
\end{align}\index{Numero!complesso!somma}
\section{Somma di numeri complessi coniugati}
\begin{align}
z=&a+b\uimm\\
\conj{z}=&a-b\uimm\\
z+\conj{z}=&a+a=2a
\end{align}\index{Numero!complesso!coniugato}
\section{Differenza di numeri complessi}
\begin{align}
z_1=&a+b\uimm\\
z_2=&c+d\uimm\\
z_1+z_2=&a-c+(b-d)\uimm
\end{align}\index{Numero!complesso!differenza}
\section{Differenza di numeri complessi coniugati}
\begin{align}
z=&a+b\uimm\\
\conj{z}=&a-b\uimm\\
z-\conj{z}=&+2b\uimm
\end{align}\index{Numero!complesso!coniugato}
\section{Prodotto di numeri complessi}
\begin{align}
z_1=&a+b\uimm\\
z_2=&c+d\uimm\\
z_1\cdot z_2=&(ac-bd)+(ad+bc)\uimm
\end{align}\index{Numero!complesso!prodotto}
\section{Prodotto di numeri complessi coniugati}
\begin{align}
z=&a+b\uimm\\
\conj{z}=&a-b\uimm\\
z\cdot\conj{z}=&a^2+b^2
\end{align}
\section{Quoziente di due numeri complessi}
\begin{align}
z_1=&a+b\uimm\\
z_2=&c+d\uimm\\
\dfrac{z_1}{z_2}=&\dfrac{a+b\uimm}{c+d\uimm}\cdot\dfrac{c-d\uimm}{c-d\uimm}\\
=&\dfrac{ac+bd+(bc-ad)\uimm}{c^2+d^2}
\end{align}\index{Numero!complesso!quoziente}
\chapter{Forma cartesiana numeri complessi}
\section{Definizione}
\begin{equation}
z=a+b\uimm\quad z\in\Co\quad a,b\in\R
\end{equation} 
\begin{center}
	\includestandalone{geometria/ncomplessi}
	\captionof{figure}{Numero complesso nel piano}
\end{center}\index{Numero!complesso!forma cartesiana}
\section{Complessi opposti}
\begin{equation}
z=a+b\uimm\quad\-z=-a-b\uimm\quad a,b\in\R
\end{equation}\index{Numero!complesso!opposto}
\begin{center}
	\includestandalone{geometria/ncomplessiopposti}
	\captionof{figure}{Numeri complessi opposti}
\end{center}\index{Numero!complesso!opposto}
\section{Complessi coniugati}
\begin{equation}
z=a+b\uimm\quad\conj{z}=a-b\uimm\quad a,b\in\R
\end{equation}\index{Numero!complesso!coniugato}
\begin{center}
	\includestandalone{geometria/ncomplessiconiugati}
	\captionof{figure}{Numeri complessi coniugati}
\end{center}\index{Numero!complesso!coniugato}
\section{Somma di numeri complessi}
\begin{equation}
w=u+v
\end{equation}\index{Numero!complesso!somma}
\begin{center}
	\includestandalone{geometria/ncomplessisomma}
	\captionof{figure}{Numeri complessi somma}
\end{center}\index{Numero!complesso!somma}