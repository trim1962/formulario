% !TeX encoding = UTF-8
% !TeX spellcheck = it_IT
\chapter{Insiemi}
\section{Inclusione}
\begin{defn}[Sottoinsieme]
$A\subseteq B\; \Leftrightarrow\; \forall x\in A\;\Rightarrow\; x\in B$\index{Insieme!inclusione} 
\end{defn}
{\centering
	\includestandalone{geometria/sottoinsieme}
	\captionof{figure}{Inclusione}\par}\index{Insieme!inclusione}
\begin{defn}[Sottoinsieme proprio]
$A\subset B\; \Leftrightarrow\; A\subseteq B\; \wedge \; A\neq B$\index{Insieme!inclusione!propria}
\end{defn}
\section{Insieme vuoto}
\begin{defn}[Insieme vuoto]
$\emptyset=\lbrace\text{insieme senza elementi}\rbrace$\index{Insieme!vuoto}
\end{defn}
\begin{thm}[Il vuoto sottoinsieme proprio di ogni insieme]
$\emptyset\subseteq A\;\forall\; A$
\end{thm}
\section{Uguaglianza}
\begin{defn}[Uguaglianza]
	$A=B\quad\Longleftrightarrow\quad\; A\subseteq B\;\wedge\; B\subseteq A$
\end{defn}\index{Insieme!uguaglianza}
\section{Unione}
\begin{defn}[Unione]
$A\cup B=\lbrace x:\;x\in A\;\vee\; x\in B\rbrace$\index{Insieme!unione}
\end{defn}
{\centering
	\includestandalone{geometria/unione}
	\captionof{figure}{Unione}
\par}\index{Insieme!unione}
\begin{thm}[Unione proprietà]
	Se $A$, $B$ e $C$ sono insiemi allora:
	\begin{enumerate}
		\item $A\cup A=A$\index{Insieme!unione!idempotenza} Idempotenza
		\item $A\cup B=B\cup A$\index{Insieme!unione!commutativa} Commutativa
		\item $A\cup\emptyset=A$
		\item $\left(A\cup B\right)\cup C=A\cup\left(B\cup C\right)$\index{Insieme!unione!associativa} Associativa
		\item $A\cup B=B\;\Leftrightarrow\; A\subset B$
		\item $A\subseteq A\cup B\;\wedge\;A\subseteq A\cup B$ 
	\end{enumerate}
\end{thm}
\section{Intersezione}
\begin{defn}[Intersezione]
$A\cap B=\lbrace x:\;x\in A\;\wedge\; x\in B\rbrace$\index{Insieme!intersezione}
\end{defn}
{\centering
	\includestandalone{geometria/intersezione}
	\captionof{figure}{Intersezione}\par
}\index{Insieme!intersezione}
\begin{thm}[Intersezione proprietà]
	Se $A$, $B$ e $C$ sono insiemi allora:
	\begin{enumerate}
		\item $A\cap A=A$\index{Insieme!intersezione!idempotenza} Idempotenza
		\item $A\cap B=B\cap A$\index{Insieme!intersezione!commutativa} Commutativa
		\item $A\cap\emptyset=\emptyset$
		\item $\left(A\cap B\right)\cap C=A\cap\left(B\cap C\right)$\index{Insieme!intersezione!associativa } Associativa
		\item $A\cap B=A\;\Leftrightarrow\; A\subset B$
		\item $A\cup B\subseteq A;\wedge\;A\cup B\subseteq B$ 
	\end{enumerate}
\end{thm}
\section{Distributiva unione intersezione}
\begin{thm}[Distributiva]
	Se $A$, $B$ e $C$ sono insiemi allora:
	\begin{align*}
A\cap\left(B\cup C\right)=&\left(A\cap B\right)\cup\left(A\cap C\right)\\
A\cup\left(B\cap C\right)=&\left(A\cup B\right)\cap\left(A\cup C\right)
	\end{align*}\index{Insieme!distributiva}
\end{thm}
	\section{Differenza}
	\begin{defn}[Differenza]
	$B-A=\lbrace x:\; x\in B\wedge\; x\notin A\rbrace$
	\end{defn}\index{Insieme!differenza}
{\centering
	\includestandalone{geometria/differenza}
	\captionof{figure}{Differenza}
\par}\index{Insieme!differenza}
\begin{thm}[Proprietà differenza]
Se $A$, $B$ e $C$ sono insiemi allora:
\begin{enumerate}
	\item $B-A=\emptyset$
	\item $B-\emptyset=B$
	\item $\emptyset-A=\emptyset$
	\item $\left(A-B\right)-C=A-\left(B\cup C\right)=\left(A-C\right)-B$
\end{enumerate}
\end{thm}
\section{Differenza simmetrica}
\begin{defn}[Differenza simmetrica]
	$A\difs B=\left(A-B\right)\cup\left(B-A\right)=\left(A\cup B\right)-\left(A\cap B\right)$\index{Insieme!differenza simmetrica}
\end{defn}
{\centering
	\includestandalone{geometria/differenzasimmetrica}
	\captionof{figure}{Differenza simmetrica}
\par}\index{Insieme!differenza simmetrica}
\begin{thm}[Proprietà differenza simmetrica]
	Se $A$, $B$ e $C$ sono insiemi allora:
	\begin{enumerate}
		\item $A\difs A=\emptyset$
		\item $A\difs B=B\difs A$\index{Insieme!differenza simmetrica!commutativa} Commutativa
		\item $A\difs\emptyset=A$
		\item $\left(A\difs B\right)\difs\left(B\difs C\right)=A\difs C$
		\item $A\difs\left(B\difs C\right)=\left(A\difs B\right)\difs C$\index{Insieme!differenza simmetrica!associativa} Associativa
		\item $A\cap\left(B\difs C\right)=\left(A\cap B\right)\difs\left(A\cap C\right)$\index{Insieme!differenza simmetrica!distributiva}
	\end{enumerate}
\end{thm}
\chapter{Complemetare}
\section{Insieme universo}
\begin{defn}[Insieme universo]
Insieme universo $X$ è formato da tutti gli elementi, da tutti gli insiemi e che contiene anche se stesso e l'insieme vuoto.
\end{defn}\index{Insieme!Universo}
\begin{thm}[Insime universo proprietà]
Dato un insieme $A$ allora:
\begin{align*}
A\cup X=& X\\
A\cap X=& A\\
A\subseteq X&\\
X\subseteq X&\\
\end{align*}
\end{thm}
\begin{defn}[Insieme complementare]
$\overline{A}=X-A=\lbrace x:\; x\in X\wedge\; x\notin A\rbrace$\index{Insieme!complementare}
\end{defn}
\section{Insieme complementare}
{\centering
	\includestandalone{geometria/complementare}
	\captionof{figure}{Complementare}
\par}\index{Insieme!complementare}
\begin{thm}[Proprietà insieme complementare]
Dato un insieme $A$ abbiamo:
\begin{enumerate}
	\item $\overline{\overline{A}}=A$
	\item $A\cap\overline{A}=\emptyset$
	\item $A\cup\overline{A}=X$
	\item $\overline{\emptyset}=X$
	\item $\overline{X}=\emptyset$
\end{enumerate}
\end{thm}
\section{Leggi di De Morgan}
\begin{thm}[Leggi di De Morgan]
Dati due insiemi $A$ e $B$ abbiamo:
\begin{align*}
\overline{\left(A\cup B\right)}=&\overline{A}\cap\overline{B}\\
\overline{\left(A\cap B\right)}=&\overline{A}\cup\overline{B}
\end{align*}
\end{thm}\index{Insieme!de Morgan}
