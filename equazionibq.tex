\chapter{Equazioni binomie}
\begin{align*}
ax^n+b=&0&a\neq 0\\
x^n=&-\frac{b}{a}
\intertext{Se $n$ pari}
x=&\pm\sqrt[n]{-\frac{b}{a}}&\begin{cases}
\text{$a$ e $b$ concordi}& \text{non ha soluzione}\\
\text{$a$ e $b$ discordi}& x=\pm\sqrt[n]{-\frac{b}{a}}\\
\end{cases}\\
\intertext{Se $n$ dispari}
x=&\sqrt[n]{-\frac{b}{a}}\\
\end{align*}\index{Equazione!binomie}
\chapter{Equazioni biquadratiche}
\begin{align*}
ax^4+bx^2+c=&0&a\neq 0\\
x^2=&y\\
ay^2+by+c=&0\\
x^2=&y_1\\
x^2=&y_2
\end{align*}\index{Equazione!biquadratica}
\chapter{Equazioni trinomie}
\begin{align*}
ax^{2n}+bx^n+c=&0&a\neq 0\\
x^n=&y\\
ay^2+by+c=&0\\
\intertext{$\Delta<0$}
\intertext{l'equazione trinomia non ha soluzione}
\intertext{$\Delta=0$}
x^n=-\frac{b}{2a}
\intertext{$\Delta>0$}
x^n=&\dfrac{-b+\sqrt{b^2-4ac}}{2a}\\
x^n=&\dfrac{-b-\sqrt{b^2-4ac}}{2a}\\
\end{align*}\index{Equazione!trinomie}