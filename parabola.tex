% !TeX encoding = UTF-8
% !TeX spellcheck = it_IT
% !TeX root = formulario.tex
\chapter{Parabola con asse parallelo asse ordinate}
\section{Equazione}
\begin{equation}
y=ax^2+bx+c\quad a\neq0
\end{equation}\index{Parabola!asse parallelo a asse y}\index{Funzione!parabola}\index{Funzione!quadratica}
\section{Asse di simmetria parabola}
\begin{equation}
x=-\dfrac{b}{2a}\quad a\neq0
\end{equation}\index{Parabola!asse simmetria}
\section{Fuoco}
\begin{equation}
F\coord{-\dfrac{b}{2a}}{\dfrac{1-\Delta}{4a}}\quad  a\neq0\; \Delta=b^2-4ac
\end{equation}\index{Parabola!fuoco}\index{Discriminante}\index{Delta}
\section{Vertice}
\begin{equation}
V\coord{-\dfrac{b}{2a}}{-\dfrac{\Delta}{4a}}\quad  a\neq0\; \Delta=b^2-4ac
\end{equation}\index{Parabola!vertice}\index{Discriminante}\index{Delta}
\section{Direttrice parabola}
\begin{equation}
y=-\dfrac{1+\Delta}{4a}\quad  a\neq0,\; \Delta=b^2-4ac
\end{equation}\index{Parabola!direttrice}\index{Discriminante}
\chapter{Parabola con asse parallelo asse ascisse}
\section{Equazione}
\begin{equation}
x=ay^2+by+c\quad a\neq0
\end{equation}\index{Parabola!asse parallelo a asse x}
\section{Asse di simmetria parabola}
\begin{equation}
y=-\dfrac{b}{2a}\quad a\neq0
\end{equation}\index{Parabola!asse simmetria}
\section{Fuoco}
\begin{equation}
F\coord{\dfrac{1-\Delta}{4a}}{-\dfrac{b}{2a}}\quad  a\neq0\; \Delta=b^2-4ac
\end{equation}\index{Parabola!fuoco}\index{Discriminante}\index{Delta}
\section{Vertice}
\begin{equation}
V\coord{-\dfrac{\Delta}{4a}}{-\dfrac{b}{2a}}\quad  a\neq0\; \Delta=b^2-4ac
\end{equation}\index{Parabola!vertice}\index{Discriminante}\index{Delta}
\section{Direttrice parabola}
\begin{equation}
x=-\dfrac{1+\Delta}{4a}\quad  a\neq0\;\Delta=b^2-4ac
\end{equation}\index{Parabola!direttrice}\index{Discriminante}
\chapter{Parabola e retta}
\section{Posizioni retta parabola}
\begin{equation}
\begin{cases}
y=ax^2+bx+c\\
y=mx+q
\end{cases}
\end{equation}\index{Parabola!retta}\index{Retta!parabola}
\begin{equation}
ax^2+(b-m)x+c-q=0\quad\begin{cases}
\text{Se $\Delta >0$ Secante}\\
\text{Se $\Delta =0$ Tangente}\\
\text{Se $\Delta <0$ Esterna}\\
\end{cases}
\end{equation}\index{Parabola!intersezione!retta}\index{Discriminante}\index{Retta!secante}\index{Retta!tangente}\index{Retta!esterna}\index{Delta}
