\chapter{Potenze}
\section{Definizione}
\begin{align*}
a^n=&\overbrace{a\times a\times\cdots\times a}^{n{}\mbox{volte}}\\
a^0=&1&a\neq0\\
a^1=&a\\
0^n=&0&n>0\\
1^n=&1%
% 1/10/2017 :: 13:56:37 :: a^{-n}=&\left(\dfrac{1}{a}\right)^n
\end{align*}\index{Potenza!definzione}\index{Potenza!proprietà}
\section{Prodotto di potenze che hanno la stessa base}
\begin{equation*}
a^n\cdot a^m=a^{n+m}
\end{equation*}\index{Potenza!prodotto!stessa base}
\section{Quoziente di potenze che hanno la stessa base}
\begin{equation*}
a^n\div a^m=a^{n-m}
\end{equation*}\index{Potenza!quoziente!stessa base}
\section{Potenza di potenze}
\begin{equation*}
(a^n)^m=a^{n\cdot m}
\end{equation*}\index{Potenza!di potenze}
\section{Prodotto di potenze che hanno lo stesso esponente}
\begin{equation*}
a^n\times b^n=(a\times b)^n
\end{equation*}\index{Potenza!prodotto!stesso esponente}
\section{Divisione di potenze che hanno lo stesso esponente}
\begin{equation*}
a^n\div b^n=(a\div b )^n
\end{equation*}\index{Potenza!quoziente!stesso esponente}
\section{Esponente negativo}
\begin{equation*}
a^{-n}=\left(\dfrac{1}{a}\right)^n
\end{equation*}\index{Potenza!esponente!negativo}
\section{Esponente frazionario}
\begin{align*}
\sqrt[n]{a}=&a^{\frac{1}{n}}\\
\sqrt[n]{a^m}=&a^{\frac{m}{n}}
\end{align*}\index{Potenza!esponente!frazionario}
