\chapter{Potenze}
\section{Definizione}
\begin{align}
a^n=&\overbrace{a\times a\times\cdots\times a}^{n{}\mbox{volte}}\\
a^0=&1&a\neq0\\
a^1=&a\\
0^n=&0&n>0\\
1^n=&1%
% 1/10/2017 :: 13:56:37 :: a^{-n}=&\left(\dfrac{1}{a}\right)^n
\end{align}\index{Potenza!definzione}\index{Potenza!proprietà}
\section{Prodotto di potenze che hanno la stessa base}
\begin{equation}
a^\rosso{n}\cdot a^\verde{m}=a^{\rosso{n}+\verde{m}}
\end{equation}\index{Potenza!prodotto!stessa base}
\section{Quoziente di potenze che hanno la stessa base}
\begin{equation}
a^\rosso{n}\div a^\verde{m}=a^{\rosso{n}-\verde{m}}
\end{equation}\index{Potenza!quoziente!stessa base}
\section{Potenza di potenze}
\begin{equation}
(a^\rosso{n})^\verde{m}=a^{\rosso{n}\cdot \verde{m}}
\end{equation}\index{Potenza!di potenze}
\section{Prodotto di potenze che hanno lo stesso esponente}
\begin{equation}
a^\rosso{n}\times b^\rosso{n}=(a\times b)^\rosso{n}
\end{equation}\index{Potenza!prodotto!stesso esponente}
\section{Divisione di potenze che hanno lo stesso esponente}
\begin{equation}
a^\rosso{n}\div b^\rosso{n}=(a\div b )^\rosso{n}
\end{equation}\index{Potenza!quoziente!stesso esponente}
\section{Esponente negativo}
\begin{equation}
a^{-n}=\left(\dfrac{1}{a}\right)^n
\end{equation}\index{Potenza!esponente!negativo}
\section{Esponente frazionario}
\begin{align}
\sqrt[n]{a}=&a^{\frac{1}{n}}\\
\sqrt[n]{a^m}=&a^{\frac{m}{n}}
\end{align}\index{Potenza!esponente frazionario}
