\chapter{Retta}
\section{Forma esplicita}
\begin{equation*}
y=mx+q
\end{equation*}\index{Retta!forma!esplicita}
Coefficiente angolare
\begin{equation*}
m
\end{equation*}\index{Retta!coefficiente!angolare}
Termine noto
\begin{equation*}
q
\end{equation*}\index{Retta!termine noto}
\section{Forma implicita}
\begin{equation*}
ax+by+c=0
\end{equation*}\index{Retta!forma!implicita}
Coefficiente angolare
\begin{equation*}
m=-\dfrac{a}{b}\quad b\neq 0
\end{equation*}\index{Retta!coefficiente!angolare}
Termine noto
\begin{equation*}
q=-\dfrac{c}{b}\quad b\neq 0
\end{equation*}\index{Retta!termine!noto}
\section{Parallelismo}
\subsection{Forma esplicita}
\begin{equation*}
r_1\parallel r_2\quad\Longleftrightarrow\quad	m_1=m_2
\end{equation*}\index{Retta!parallela}\index{Retta!coefficiente! angolare}
\subsection{Forma implicita}
\begin{equation*}
r_1\parallel r_2\quad\Longleftrightarrow\quad \dfrac{a_1}{b_1}=\dfrac{a_2}{b_2}
\end{equation*}\index{Retta!parallela}\index{Retta!coefficiente angolare}
\section{Perpendicolarità}
\subsection{Forma esplicita}
\begin{equation*}
r_1\perp r_2\quad\Longleftrightarrow\quad	m_1 \cdot m_2=-1\quad\Longleftrightarrow\quad	m_1=-\dfrac{1}{m_2}
\end{equation*}\index{Retta!perpendicolare}\index{Retta!coefficiente angolare}
\subsection{Forma implicita}
\begin{equation*}
r_1\perp r_2\quad\Longleftrightarrow\quad	\dfrac{a_1}{b_1}\cdot\dfrac{a_2}{b_2}=-1\quad\Longleftrightarrow\quad a_1\cdot a_2+b_1\cdot b_2=0
\end{equation*}\index{Retta!perpendicolare}\index{Retta!coefficiente angolare}
\section{Fascio proprio di rette}
\begin{equation*}
y-y_1=m(x-x_1)
\end{equation*}\index{Retta!fascio!proprio}
\section{Fascio improprio di rette}
\begin{equation*}
y=mx+k\quad k\in\R
\end{equation*}\index{Retta!fascio!improprio}
\section{Coefficiente angolare retta per due punti}
\begin{equation*}
m=\dfrac{y_2-y_1}{x_2-x_1}\quad x_2\neq x_1
\end{equation*}\index{Retta!coefficiente angolare!per due punti}
\section{Retta per un punto parallela a retta data}
\subsection{Forma esplicita}
\begin{align*}
P\coord{x_1}{y_1}&\nonumber\\
y=&mx+q\nonumber\\
y-y_1=&m(x-y_1)
\end{align*}\index{Retta!per punto!parallela}
\subsection{Forma implicita}
\begin{align*}
P\coord{x_1}{y_1}&\nonumber\\
ax+by+c=&0\nonumber\\
a(x-x_1)+b(y-y_1)=&0
\end{align*}\index{Retta!per punto!parallela}
\section{Retta per un punto perpendicolare a retta data}
\subsection{Forma esplicita}
\begin{align*}
P\coord{x_1}{y_1}&\nonumber\\
y=&mx+q\nonumber\\
y-y_1=&-\dfrac{1}{m}(x-y_1)
\end{align*}\index{Retta!per punto!perpendicolare}
\subsection{Forma implicita}
\begin{align*}
P\coord{x_1}{y_1}&\nonumber\\
ax+by+c=&0\nonumber\\
b(x-x_1)-a(y-y_1)=&0
\end{align*}\index{Retta!per punto!perpendicolare}
\section{Retta per due punti}
\begin{equation*}
\dfrac{x-x_1}{x_2-x_1}=\dfrac{y-y_1}{y_2-y_1}\quad x_2\neq x_1\quad y_2\neq y_1
\end{equation*}\index{Retta!per due punti}
\section{Condizione allineamento tre  punti}
\begin{equation*}
\dfrac{x_3-x_1}{x_2-x_1}=\dfrac{y_3-y_1}{y_2-y_1}\quad x_2\neq x_1\quad y_2\neq y_1
\end{equation*}\index{Retta!tre punti!allineati}
\section{Posizioni due rette}
\begin{equation*}
\begin{cases}
a_1x+b_1y+c_1=0\\
a_2x+b_2y+c_2=0
\end{cases}
\end{equation*}
\begin{equation*}
\begin{cases}
\text{Se il sistema ha soluzione}& \text{Incidenti}\\
\text{Se il sistema è indeterminato}& \text{Coincidenti}\\
\text{Se il sistema non ha soluzione}& \text{Parallele}\\
\end{cases}
\end{equation*}\index{Retta!incidente}\index{Retta!coincidente}\index{Retta!parallela}
\section{Distanza punto retta}
\begin{align*}
P(x_1;y_1)&\quad ax+by+c=0\\
d=&\dfrac{\abs{ax_1+by_1+c}}{\sqrt{a^2+b^2}}
\end{align*}
\section{Traslazioni}
\begin{gather*}
 P_{xoy}\coord{x}{y}\ P_{XOY}\coord{X}{Y}\ O_{xoy}\coord{a}{b}\notag\\  x=a+X\  y=b+Y
\end{gather*}\index{Traslazioni}
\chapter{Luoghi geometrici}
\section{Asse segmento}
\begin{gather*}
A\coord{x_1}{y_1}\ B\coord{x_2}{y_2}\ P\coord{x}{y}\notag\\
\sqrt{(x_1-x)^2+(y_1-y)^2}=\sqrt{(x_2-x)^2+(y_2-y)^2}\\
2x(x_2-x_1)+2y(y_2-y_1)+x_1^2+y_1^2-x_2^2-y_2^2=0\\		
\end{gather*}\index{Asse!segmento}
