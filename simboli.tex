% !TeX encoding = UTF-8
% !TeX spellcheck = it_IT
% !TeX root = formulario.tex
\chapter{Simboli}
\section{Simboli matematici}
\label{sec:simbolimatematici}
%\begin{table}[H]
%\centering
\begin{center}
	\begin{tabular}{WlWl}
\toprule
\multicolumn{1}{c}{Simbolo}&\multicolumn{1}{l}{Significato}&\multicolumn{1}{c}{Simbolo}&\multicolumn{1}{l}{Significato}\\
\midrule
=&uguale&\abs{a}&valore assoluto di a\\[.25cm]
\neq&diverso& \Ni &Numeri naturali\\[.25cm]
\approx&circa&\Nz&Numeri naturali meno lo zero\\[.25cm]
<&minore&\Z&Numeri interi\\[.25cm]
>&maggiore&\Zn&Numeri interi negativi\\[.25cm]
\leq&minore o uguale&\Zp&Numeri interi positivi\\[.25cm]
\geq&maggiore o uguale&\Q&Numeri razionali\\[.25cm]
\exists&esiste&\Qn&Numeri razionali negativi\\[.25cm]
\nexists&non esiste&\Qp&Numeri razionali positivi\\[.25cm]
\forall &comunque&\R&Numeri reali\\[.25cm]
\in&appartiene&\Rneg&Numeri reali negativi\\[.25cm]
\notin&non appartiene&\Rpos&Numeri reali positivi\\[.25cm]
\propto&proporzionale&\Co&Numeri complessi\\[.25cm]
\pm&più meno&\perp&perpendicolare\\[.25cm]
\mp&meno più&\equiv&equivalente\\[.25cm]
\Longrightarrow&allora&\parallel&parallele\\[.25cm]
\Longleftrightarrow&se e solo se&\infty&infinito\\[.25cm]
\tikz{\draw[gray] (0,0)--(1,0);\fill[gray] (.5,0) circle (3pt);}&valore compreso&\tikz{\draw[gray] (0,0)--(1,0);\draw[gray] (.5,0) circle (3pt);}&valore non compreso\\
2P&perimetro&P&semiperimetro\\
\bottomrule\index{Simboli!matematici}
\end{tabular}
\captionof{table}{Simboli matematici}
\end{center}
\label{tab:simolimatimatici}
%\end{table}

