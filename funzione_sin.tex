% !TeX encoding = UTF-8
% !TeX spellcheck = it_IT
% !TeX root = formulario.tex
\chapter{Funzione seno}
\section{Definizione}
\begin{equation*}
y=\sin(x)
\end{equation*}
\section{Proprietà}
\begin{itemize}
	\item La funzione è trascendente\index{Funzione!trascendente}
\item Il dominio è $\R$
\item La funzione è limitata\index{Funzione!limitata!seno}, il codominio è $[-1,1]$
\item La funzione è periodica di periodo $k\ang{360;;}$ o $2k\pi$\index{Funzione!periodica!seno}
\end{itemize}
{\centering
	\includestandalone{geometria/senografico}
	\captionof{figure}{Funzione seno grafico}\par}\index{Funzione!seno}
\chapter{Funzione sinusoide}
\begin{equation*}
y=A\sin(\omega t+\phi)
\end{equation*}
\section{Proprietà}
\begin{itemize}
	\item La funzione è trascendente\index{Funzione!trascendente}
	\item Il dominio è $\R$
	\item $t$ si misura  in \SI[parse-numbers=false]{}{\second}
	\item $\tau=\dfrac{2\pi}{\omega}$ periodo si misura  in \SI[parse-numbers=false]{}{\second}
	\item $f$ frequenza si misura in  \SI[parse-numbers=false]{}{1\per\second}= \SI[parse-numbers=false]{}{\hertz}
	\item $\phi$ sfasamento si misura in \SI[parse-numbers=false]{}{\radian}
\item $A$ ampiezza\index{Funzione!sinusoidale!ampiezza}
\item $\omega=\dfrac{2\pi}{\tau}$ si misura in  \SI[parse-numbers=false]{}{\radian\per\second} pulsazione\index{Funzione!sinusoide!pulsazione} o velocità angolare 
	\item La funzione è limitata\index{Funzione!limitata!seno}, il codominio è $[-A,A]$
\end{itemize}