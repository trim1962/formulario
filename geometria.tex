\chapter{Geometria}
\section{Triangolo}
\begin{tcolorbox}[sidebyside,righthand width=7cm,colback=white,colframe=white,fonttitle=\bfseries	]
\includestandalone[width=5.5cm]{geometria/triangoloSca}
\tcblower
	\begin{align}
		A=&\dfrac{b\cdot h}{2}&h=&\dfrac{2A}{b}\\
		b=&\dfrac{2A}{h}&2P=&a+b+c\\
		A=&\sqrt{p(p-a)(p-b)(p-c)}&
	\end{align}
\end{tcolorbox}\index{Triangolo!area}\index{Triangolo!perimetro}\index{Triangolo!altezza}\index{Erone!formula}
\section{Triangolo rettangolo}
\begin{tcolorbox}[sidebyside,righthand width=9cm,colback=white,colframe=white,fonttitle=\bfseries	]
	\includestandalone[width=5.5cm]{geometria/triangoloRet}
	\tcblower
	\begin{align}
	A=&\dfrac{b\cdot c}{2}&h=&\dfrac{b\cdot c}{a}\\
	a^2=&b^2+c^2&a=&\sqrt{b^2+c^2}\\
	b=&\sqrt{a^2-c^2}&c=&\sqrt{a^2-b^2}\\
	b^2=&p_1a&c^2=&p_2a\\
	h^2=&p_1p_2&h=&\sqrt{p_1p_2}\\
	a=&p_1+p_2
	\end{align}
\end{tcolorbox}\index{Triangolo!rettangolo!area}\index{Triangolo!rettangolo!altezza}\index{Triangolo!rettangolo!cateto}\index{Triangolo!rettangolo!ipotenusa}\index{Pitagora!teorema}\index{Euclide!primo teorema}\index{Euclide!secondo teorema}
\section{Triangolo equilatero}
\begin{tcolorbox}[sidebyside,righthand width=9cm,colback=white,colframe=white,fonttitle=\bfseries	]
	\includestandalone[width=4.5cm]{geometria/triangoloiso}
	\tcblower
	\begin{align}
	A=&\dfrac{h\cdot l}{2}&h=&\dfrac{\sqrt{3}}{2}l\\
	A=&\dfrac{\sqrt{3}}{4}h&l=&\dfrac{2\sqrt{3}}{3}h\\
	2P=&3l
	\end{align}\index{Triangolo!equilatero!area}\index{Triangolo!equilatero!altezza}\index{Triangolo!equilatero!lato}\index{Triangolo!equilatero!perimetro}
\end{tcolorbox}
\section{Triangolo isoscele}
\begin{tcolorbox}[sidebyside,righthand width=9cm,colback=white,colframe=white,fonttitle=\bfseries	]
	\includestandalone[width=4.5cm]{geometria/triangoloequi}
	\tcblower
	\begin{align}
	A=&\dfrac{h\cdot b}{2}&h=&\dfrac{2A}{b}\\
	b=&\dfrac{2A}{h}&2P=&b+2l\\
	l=&\sqrt{h^2+\frac{b^2}{4}}	
	\end{align}\index{Triangolo!isoscele!area}\index{Triangolo!isoscele!altezza}\index{Triangolo!isoscele!lato}\index{Triangolo!isoscele!perimetro}
\end{tcolorbox}
\section{Trapezio}
\begin{tcolorbox}[sidebyside,righthand width=9cm,colback=white,colframe=white,fonttitle=\bfseries	]
	\includestandalone[width=5.5cm]{geometria/trapezio}
	\tcblower
	\begin{align}
	A=&\dfrac{B+b}{2}h\\
	2P=&B+b+l_1+l_2
	\end{align}
\end{tcolorbox}\index{Trapezio}\index{Trapezio!area}\index{Trapezio!perimetro}

\section{Trapezio isoscele}
\begin{tcolorbox}[sidebyside,righthand width=9cm,colback=white,colframe=white,fonttitle=\bfseries	]
	\includestandalone[width=5.5cm]{geometria/trapezioiso}
	\tcblower
	\begin{align}
	2P=&B+b+2l\\
	l=&\sqrt{h^2+\left(\frac{B-b}{2}\right)^2}\\
	h=&\sqrt{l^2-\left(\frac{B-b}{2}\right)^2}\\
	\frac{B-b}{2}=&\sqrt{l^2-h^2}
	\end{align}
\end{tcolorbox}\index{Trapezio!isoscele}\index{Trapezio!isoscele!perimetro}\index{Trapezio!isoscele!lato obliquo}\index{Trapezio!isoscele!altezza}
\section{Trapezio rettangolo}
\begin{tcolorbox}[sidebyside,righthand width=9cm,colback=white,colframe=white,fonttitle=\bfseries	]
	\includestandalone[width=5.5cm]{geometria/trapezioret}
	\tcblower
	\begin{align}
	l=&\sqrt{(B-b)^2+h^2}&\\
	h=&\sqrt{l^2-(B-b)^2}&\\
	B-b=&\sqrt{l^2-h^2}&\\
	D=&\sqrt{h^2+B^2}&d=&\sqrt{h^2+b^2}\\
	B=&\sqrt{D^2-h^2}&b=&\sqrt{d^2-h^2}\\
	h=&\sqrt{D^2-B^2}&h=&\sqrt{d^2-b^2}\\	
	\end{align}
\end{tcolorbox}\index{Trapezio!rettangolo}\index{Trapezio!rettangolo!lato obliquo}\index{Trapezio!rettangolo!altezza}\index{Trapezio!rettangolo!base}\index{Trapezio!rettangolo!diagonali}
\section{Parallelogramma}
\begin{tcolorbox}[sidebyside,righthand width=9cm,colback=white,colframe=white,fonttitle=\bfseries	]
	\includestandalone[width=5cm]{geometria/parallelogramma}
	\tcblower
	\begin{align}
	A=&b\cdot h&h=&\dfrac{A}{c}&b=&\dfrac{A}{h}\\
	2P=&2a+2b
	\end{align}
\end{tcolorbox}\index{Parallelogramma}
\section{Quadrato}\index{Quadrato!area}\index{Quadrato!lato}\index{Quadrato!diagonale}\index{Quadrato!periodo}
\begin{tcolorbox}[sidebyside,righthand width=9cm,colback=white,colframe=white,fonttitle=\bfseries	]
	\includestandalone[width=4.5cm]{geometria/quadrato}
	\tcblower
	\begin{align}
	A=&l^2&d=&l\sqrt{2}\\
	l=&\sqrt{A}&l=&\dfrac{\sqrt{2}}{2}d\\
	2P=&4l&l=\dfrac{2P}{4}
	\end{align}
\end{tcolorbox}
\section{Rettangolo}
\begin{tcolorbox}[sidebyside,righthand width=9cm,colback=white,colframe=white,fonttitle=\bfseries	]
	\includestandalone[width=4.5cm]{geometria/rettangolo}
	\tcblower
	\begin{align}
	A=&b\cdot h&d=&\sqrt{b^2+h^2}\\
	b=&\sqrt{d^2-h^2}&h=&\sqrt{d^2-b^2}\\
	2P=&2(b+h)
	\end{align}
\end{tcolorbox}\index{Rettangolo!area}\index{Rettangolo!perimetro}\index{Rettangolo!diagonale}
\section{Rombo}
\begin{tcolorbox}[sidebyside,righthand width=9cm,colback=white,colframe=white,fonttitle=\bfseries	]
	\includestandalone[width=4.5cm]{geometria/rombo}
	\tcblower
	\begin{align}
	A=&\dfrac{d_1\cdot d_2}{2} & 2P=&4a\\
	d_1=&\dfrac{2A}{d_2}
		\end{align}
\end{tcolorbox}\index{Rombo!area}\index{Rombo!perimetro}
\section{Circonferenza}\index{Circonferenza}\index{Cerchio}
\begin{tcolorbox}[sidebyside,righthand width=9cm,colback=white,colframe=white,fonttitle=\bfseries	]
	\includestandalone[width=4.5cm]{geometria/circonferenza}
	\tcblower
	\begin{align}
	A=&\pi r^2 & 2P=&2\pi r	\\
	r=&\sqrt{\dfrac{A}{\pi}}
	\end{align}
\end{tcolorbox}\index{Circonferenza!perimetro}\index{Cerchio!area}\index{Circonferenza!raggio}
\section{Poligoni regolari}
\begin{tcolorbox}[sidebyside,righthand width=9cm,colback=white,colframe=white,fonttitle=\bfseries	]
	\includestandalone[width=4.5cm]{geometria/poligoniregolari}
	\tcblower
	\begin{align}
	A=&\dfrac{2P\cdot a}{2} & 2P=&nl	\\
	a=&l\cdot f & l=&\dfrac{a}{f}	\\
	A=&l^2\cdot\phi&l=&\sqrt{\dfrac{A}{\phi}}
	\end{align}
\end{tcolorbox}
\begin{center}
	\begin{tabular}{lcll}
		\toprule
Poligono	&  Lati&  Fisso f&Fisso $\phi$ \\ 
Triangolo equilatero	& 3 & 0.289 &0.433\\ 
Quadrato	& 4 & 0.5&1 \\ 
Pentagono	& 5 &0.688 &1.720 \\ 
Esagono	& 6 &0.866 &2.598 \\ 
Ettagono	& 7 &1.038&3.634 \\ 
Ottagono	& 8 &1.207&4.828 \\ 
Ennagono	& 9&  1.374&6.182\\ 
Decagono	& 10 & 1.539&7.694 \\ 
Dodecagono	&  12&  1.866&11.196\\ 
\bottomrule
\end{tabular}\captionof{table}{Poligoni regolari }\index{Poligono!regolare}
\end{center}
