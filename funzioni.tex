\chapter{Funzioni}
\section{Definizione}
Una funzione è una relazione che ad ogni elemento di $x\in A$ dominio\index{Funzione!dominio}, fa corrispondere uno e uno solo elemento appartenente  a $B$ codominio\index{Funzione!codominio}
\begin{equation*}
\function{f}{A}{B}{x}{f(x)}
\end{equation*}\index{Funzione!definizione}
\section{Funzione pari}
Una funzione $\funzione{f}{A}{B}$ è una funzione pari se 
\begin{equation*}
f(-x)=f(x)\quad\forall x\in A
\end{equation*}\index{Funzione!pari}
\begin{center}
	\includestandalone{geometria/funzionepari}
	\captionof{figure}{Funzione pari}
\end{center}\index{Funzione!pari}
\section{Funzione dispari}
Una funzione $\funzione{f}{A}{B}$ è una funzione dispari se 
\begin{equation*}
f(-x)=-f(x)\quad\forall x\in A
\end{equation*}\index{Funzione!dispari}
\begin{center}
	\includestandalone{geometria/funzionedispari}
	\captionof{figure}{Funzione dispari}
\end{center}\index{Funzione!dispari}
\section{Funzione limitata}
Una funzione $\funzione{f}{A}{B}$ è una funzione limitata se 
\begin{equation*}
\exists\; M\quad \abs{f(x)}<M \quad\forall x\in A
\end{equation*}\index{Funzione!limitata}
\begin{center}
	\includestandalone{geometria/funzionelimitata}
	\captionof{figure}{Funzione limitata}
\end{center}\index{Funzione!limitata}
\section{Funzione iniettiva}
Una funzione $\funzione{f}{A}{B}$ è iniettiva se
\begin{equation*}
 x_1\neq x_2\quad\Longrightarrow\quad f(x_1)\neq f(x_2)\quad \forall x_1,x_2\in A
\end{equation*}
oppure
\begin{equation*}
f(x_1)= f(x_2)\quad\Longrightarrow\quad  x_1= x_2\quad \forall x_1,x_2\in A
\end{equation*}\index{Funzione!iniettiva}
\begin{center}
	\includestandalone{geometria/funzioneiniettiva}
	\captionof{figure}{Funzione iniettiva}
\end{center}\index{Funzione!iniettiva}
\section{Funzione suriettiva}
Una funzione $\funzione{f}{A}{B}$ è suriettiva se
\begin{equation*}
	f(A)=B
\end{equation*}\index{Funzione!suriettiva} 
\begin{center}
	\includestandalone{geometria/funzionesuriettiva}
	\captionof{figure}{Funzione suriettiva}
\end{center}\index{Funzione!suriettiva}
\section{Funzione biettiva}
Una funzione $\funzione{f}{A}{B}$ è biettiva se è contemporaneamente suriettiva e iniettiva\index{Funzione!biettiva}
\begin{center}
	\includestandalone{geometria/funzionebiettiva}
	\captionof{figure}{Funzione biettiva}
\end{center}\index{Funzione!biettiva}
\section{Funzione periodica}
Una funzione $\funzione{f}{A}{B}$ è periodica di periodo $T>0$ se
\begin{equation*}
f(x+kT)=f(x)\quad k\in Z
\end{equation*}\index{Funzione!periodica}
\section{Funzione crescente in senso stretto}
Una funzione $\funzione{f}{A}{B}$ si dice crescente in senso stretto nell'intervallo  $I\subset A$ se
\begin{equation*}
\forall\; x_1,x_2\in I\quad x_1< x_2\Longrightarrow f(x_1)<f(x_2)
\end{equation*}\index{Funzione!crescente!in senso stretto}
\section{Funzione non decrescente}
Una funzione $\funzione{f}{A}{B}$ si dice non decrescente nell'intervallo  $I\subset A$ se
\begin{equation*}
\forall\; x_1,x_2\in I\quad x_1< x_2\Longrightarrow f(x_1)\leq f(x_2)
\end{equation*}\index{Funzione!non decrescente}
\section{Funzione decrescente in senso stretto}
Una funzione $\funzione{f}{A}{B}$ si dice decrescente in senso stretto nell'intervallo  $I\subset A$ se
\begin{equation*}
\forall\; x_1,x_2\in I\quad x_1< x_2\Longrightarrow f(x_1)>f(x_2)
\end{equation*}\index{Funzione!decrescente!in senso stretto}
\section{Funzione non crescente}
Una funzione $\funzione{f}{A}{B}$ si dice non crescente nell'intervallo  $I\subset A$ se
\begin{equation*}
\forall\; x_1,x_2\in I\quad x_1< x_2\Longrightarrow f(x_1)\geq f(x_2)
\end{equation*}\index{Funzione!non crescente}
..\section{Zeri funzione}
Data una funzione $\funzione{f}{A}{B}$ $a\in A$ è uno zero per la funzione se $f(a)=0$\index{Funzione!zero}
\section{Funzione algebrica}
Una funzione $\funzione{f}{A}{B}$ si dice algebrica se costruita utilizzando un numero finito di applicazione delle quattro operazioni dell'aritmetica, dell'elevazione a potenza e delle radici.\index{Funzione!algebrica} 
\section{Funzione algebrica razionale}
Una funzione $\funzione{f}{A}{B}$ algebrica è razionale quando la variabile indipendente non si trova sotto il segno di radice\index{Funzione!algebrica!razionale}
\section{Funzione algebrica irrazionale}
Una funzione $\funzione{f}{A}{B}$ algebrica è irrazionale quando la variabile indipendente  si trova sotto il segno di radice\index{Funzione!algebrica!irrazionale} 
\section{Funzione algebrica intera}
Una funzione $\funzione{f}{A}{B}$ algebrica è intera quando la variabile indipendente non si trova al denominatore di una frazione\index{Funzione!algebrica!intera}
\section{Funzione algebrica fratta}
Una funzione $\funzione{f}{A}{B}$ algebrica è fratta quando la variabile indipendente si trova al denominatore di una frazione\index{Funzione!algebrica!fratta}	
\section{Funzione trascendente}
	Una funzione $\funzione{f}{A}{B}$  è trascendente quando compaiono operazioni non algebriche\index{Funzione!trascendente} come logaritmo, esponenziale, goniometriche.
\section{Funzione insieme di definizione}
Data una funzione $\funzione{f}{A}{B}$, diremo insieme di definizione della funzione un sotto insieme del dominio in cui la funzione è effettivamente definita.\index{Funzione!insieme!definizione} L'insieme di definizione è noto come campo di esistenza.\index{Funzione!campo!esistenza}

\begin{center}
	\includestandalone[width=0.95\textwidth]{geometria/DominioFunzRazzioIrrazio}
	\captionof{figure}{Insieme di definizione funzioni razionali e irrazionali}
\end{center}\index{Funzione!razionale!insieme definizione}\index{Funzione!irrazionale!insieme definizione}
\section{Classificazione}
\begin{center}
	\includestandalone[width=0.95\textwidth]{geometria/FunzioniClassificazione}
	\captionof{figure}{Funzioni classificazione}
\end{center}
