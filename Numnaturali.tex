% !TeX encoding = UTF-8
% !TeX spellcheck = it_IT
% !TeX root = formulario.tex

\chapter{Numeri naturali}

\section{Simbolo}
\begin{equation}
\Ni
\end{equation}\index{Numero!naturale}
\section{Addizione}
L'addizione è una funzione definita:  $\funzione{+}{\Ni\times\Ni}{\Ni}$
\begin{equation}\index{Numero!naturale!addizione}
\overbrace{a}^{addendo}+\overbrace{b}^{addendo}=\overbrace{c}^{somma}
\end{equation}\index{Numero!naturale!addendo}\index{Numero!naturale!somma}
\subsection{Associativa}
\begin{equation}
(a+b)+c=a+(b+c)
\end{equation}\index{Numero!naturale!somma associativa}
\subsection{Commutativa}
\begin{equation}
a+b=b+a
\end{equation}\index{Numero!naturale!somma commutativa}
\subsection{Elemento neutro}\
\begin{equation}
a+0=0+a=a
\end{equation}\index{Numero!naturale!somma elemento neutro}\index{Elemento!neutro}
\section{Prodotto}
Il prodotto è una funzione che\index{Numero!naturale!prodotto} $\funzione{\times}{\Ni\times\Ni}{\Ni}$
\begin{equation}
\overbrace{a}^{fattore}\times\overbrace{b}^{fattore}=\overbrace{c}^{prodotto}
\end{equation}\index{Fattore}\index{Numero!naturale!prodotto}
\subsection{Associativa}
\begin{equation}
(a\times b)\times c=a\times(b\times c) 
\end{equation}\index{Numero!naturale!prodotto associativa}
\subsection{Commutativa}
\begin{equation}
a\times b=b\times a
\end{equation}\index{Numero!naturale!prodotto commutativa}
\subsection{Elemento neutro}
\begin{equation}
a\times 1=1\times a=a
\end{equation}\index{Numero!naturale!prodotto elemento neutro}
\subsection{Assorbente}
\begin{equation}
a\times 0=0\times a=0
\end{equation}\index{Numero!naturale!prodotto assorbente}
\section{Sottrazione}
\begin{equation}
\overbrace{a}^{minuendo}-\overbrace{b}^{sottraendo}=\overbrace{c}^{Sottrazione}\quad a\geq b
\end{equation}\index{Minuendo}\index{Sottraendo}\index{Numero!naturale!sottrazione}
\subsection{Invariantiva sottrazione}
\begin{equation}
a-b=(a+c)-(b+c)\quad a>b
\end{equation}\index{Numero!naturale!sottrazione invariantiva}
\begin{equation}
a-b=(a-c)-(b-c)\quad a>b\quad a-c,b-c\in\Ni
\end{equation}\index{Numero!naturale!sottrazione invariantiva}
 \section{Divisione}
 Non sempre possibile.
\begin{equation}
\overbrace{a}^{dividendo}\div\overbrace{b}^{divisore}=\overbrace{c}^{quoziente}\quad b\neq 0
\end{equation}\index{Dividendo}\index{Divisore}\index{Numero!naturale!quoziente}
\subsection{Invariantiva quoziente}
\begin{equation}
a\div b=(a\times c)\div (b\times c)\quad b\neq 0\quad c \neq 0
\end{equation}\index{Numero!naturale!quoziente invariantiva}
\begin{equation}
a\div b=(a\div c)\div (b\div c)\quad b\neq 0\quad c \neq 0
\end{equation}\index{Numero!naturale!quoziente invariantiva}
\section{Proprietà distributive}
\subsection{Moltiplicazione addizione}
\begin{equation}
(a+b)\times c=a\times c+b\times c
\end{equation}\index{Numero!naturale!distributiva moltiplicazione addizione}
\subsection{Moltiplicazione sottrazione}
\begin{equation}
(a-b)\times c=a\times c-b\times c\quad a>b\quad c\neq 0
\end{equation}\index{Numero!naturale!distributiva moltiplicazione sottrazione}
\subsection{Divisione addizione}
\begin{equation}
(a+b)\div c=a\div c+b\div c\quad c\neq 0
\end{equation}\index{Numero!naturale!distributiva divisione addizione}
\subsection{Divisione sottrazione}
\begin{equation}
(a-b)\div c=a\div c+b\div c\quad a>b\quad c\neq 0
\end{equation}\index{Numero!naturale!distributiva divisione sottrazione}
\section{Riepilogo}

{\centering\captionof{table}{Proprietà numeri naturali}
	\begin{tabular}{lLL}
\toprule
Proprietà	& Somma & Prodotto  \\ 
\midrule
associativa	& (a+b)+c=a+(b+c) & (a\times b)\times c=a\times(b\times c) \\ 
commutativa	&a+b=b+a  &a\times b=b\times a  \\ 
elemento neutro	&a+0=0+a=a  & a\times 1=1\times a=a \\ 
distributiva	&(a+b)\times c=a\times c+b\times c &  \\ 
assorbimento	&  & a\times 0=0\times a=0 \\ 
\bottomrule
\end{tabular}
\par}