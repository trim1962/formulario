\chapter{Polinomi}
\section{Definizione}
Un polinomio è la somma di monomi non simili.\index{Polinomio!definizione}
\section{Grado polinomio}
Il grado di un polinomio è il grado maggiore fra i 
monomi che lo compongono.\index{Polinomio!grado}
\section{Monomio per binomio}
\begin{equation*}
a(b+c)=ab+ac
\end{equation*}\index{Prodotto!monomio!binomio}
\begin{equation*}
1(1+2)=11+12
\end{equation*}
\section{Binomio per binomio}
\begin{equation*}
(a+b)(c+d)=ac+ad+bc+bd
\end{equation*}\index{Prodotto!binomio!binomio}
\begin{equation*}
(1+2)(1+2)=11+12+21+22
\end{equation*}
\section{Quadrato del binomio}
\begin{align*}
(a+b)^2=&a^2+b^2+2ab\\
(a-b)^2=&a^2+b^2-2ab
\end{align*}\index{Prodotto!quadrato!binomio}
\begin{equation*}
(1+2)^2=1^2+2^2+2(1)(2)
\end{equation*}
\section{Quadrato del trinomio}
\begin{equation*}
(a+b+c)^2=a^2+b^2+c^2+2ab+2ac+2bc
\end{equation*}\index{Prodotto!quadrato!trinomio}
\begin{equation*}
(1+2+3)^2=1^2+2^2+3^2+2(1)(2)+2(1)(3)+2(2)(3)
\end{equation*}
\section{Cubo binomio}
\begin{equation*}
(a+b)^3=a^3+b^3+3a^2b+3ab^2
\end{equation*}\index{Prodotto!cubo!binomio}
\begin{equation*}
(1+2)^3=1^3+2^3+3(1)^2 2+3(1)2^2
\end{equation*}
\section{Differenza di quadrati}
\begin{equation*}
(a-b)(a+b)=a^2-b^2
\end{equation*}\index{Prodotto!differenza!quadrati}
\begin{equation*}
(1-2)(1+2)=1^2-2^2
\end{equation*}
