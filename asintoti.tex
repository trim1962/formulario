\chapter{Asintoti}
\section{Asintoto verticale}
La funzione $\funzione{f}{A}{B}$ ha un asintoto verticale per $a\in A$ se
\begin{equation}
\lim_{x\to a^+} f(x)=\pm\infty
\end{equation}\index{Funzione!asintoto!verticale}\index{Asintoto!verticale}
oppure
\begin{equation}
\lim_{x\to a^-} f(x)=\pm\infty
\end{equation}\index{Funzione!asintoto!verticale}
\section{Asintoto orizzontale}
La retta $y=c$ è un asintoto orizzontale per la funzione  $\funzione{f}{A}{B}$  se \begin{equation}
\lim_{x\to\pm\infty} f(x)=c
\end{equation}\index{Asintoto!orizzontale}\index{Funzione!asintoto!orizzontale}
\section{Asintoto obliquo}
La retta $y=mx+q$ è un asintoto obliquo per la funzione  $\funzione{f}{A}{B}$ se
\begin{equation}
\lim_{x\to +\infty} [f(x)-(mx+q)]=0
\end{equation}
o analogamente
\begin{equation}
\lim_{x\to -\infty} [f(x)-(mx+q)]=0
\end{equation}\index{Asintoto!obliquo}\index{Funzione!asintoto!obliquo}