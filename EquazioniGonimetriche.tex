% !TeX encoding = UTF-8
% !TeX spellcheck = it_IT
% !TeX root = formulario.tex
\chapter{Equazioni goniometriche}
\section{Equazioni goniometriche elementari}
\begin{align*}
\intertext{$x\in[0,2\pi]$}
	\sin(x)=&m &0<m<1\\
	x_1=&\arcsin(m)+2k\pi &k\in\Z\\
	x_2=&\pi-\arcsin(m)+2k\pi\\
\sin(x)=&m &-1<m<0\\
x_1=&\pi+\arcsin(\abs{m})+2k\pi &k\in\Z\\
x_2=&2\pi-\arcsin(\abs{m})+2k\pi\\
\sin(x)=&1\\\index{Equazione!goniometrica!seno}
x_1=&\dfrac{\pi}{2}+2k\pi &k\in\Z\\
\sin(x)=&-1\\
x_1=&\dfrac{3}{2}\pi+2k\pi &k\in\Z\\
\sin(x)=&0\\
x_1=&k\pi &k\in\Z
\intertext{$x\in[0,2\pi]$}
\cos(x)=&m &0<m<1\\\index{Equazione!goniometrica!coseno}
x_1=&\arccos(m)+2k\pi &k\in\Z\\
x_2=&2\pi-\arccos(m)+2k\pi\\
\cos(x)=&m &-1<m<0\\
x_1=&\pi-\arccos(\abs{m})+2k\pi &k\in\Z\\
x_2=&\pi+\arccos(\abs{m}))+2k\pi\\
\cos(x)=&1\\
x_1=&2k\pi &k\in\Z\\
\cos(x)=&-1\\
x_1=&\pi+2k\pi &k\in\Z\\
\cos(x)=&0\\
x_1=&k\dfrac{\pi}{2} &k\in\Z
\intertext{$x\in[0,2\pi]$}
\tan(x)=&m&m\geq0\\\index{Equazione!goniometrica!tangente}
x=&\arctan(m)+k\pi&k\in\Z\\
\tan(x)=&m&m<0\\
x=&\pi-\arctan(\abs{m})+k\pi&k\in\Z
\intertext{$x\in[-\pi,\pi]$}
\sin(x)=&m &0<m<1\\
x_1=&\arcsin(m)+2k\pi &k\in\Z\\
x_2=&\pi-\arcsin(m)+2k\pi\\
\sin(x)=&m &-1<m<0\\
x_1=&-\arcsin(\abs{m})+2k\pi &k\in\Z\\
x_2=&-\pi+\arcsin(\abs{m})+2k\pi\\
\sin(x)=&1\\
x_1=&\dfrac{\pi}{2}+2k\pi &k\in\Z\\
\sin(x)=&-1\\
x_1=&-\dfrac{\pi}{2}+2k\pi &k\in\Z\\
\sin(x)=&0\\
x_1=&k\pi &k\in\Z
\intertext{$x\in[-\pi,\pi]$}
\cos(x)=&m &0<m<1\\
x_1=&\arccos(m)+2k\pi &k\in\Z\\
x_2=&-\arccos(m)+2k\pi\\
\cos(x)=&m &-1<m<0\\
x_1=&\pi-\arccos(\abs{m})+2k\pi &k\in\Z\\
x_2=&-\pi+\arccos(\abs{m})+2k\pi\\
\cos(x)=&1\\
x_1=&2k\pi &k\in\Z\\
\cos(x)=&-1\\
x_1=&-\pi+2k\pi &k\in\Z\\
\cos(x)=&0\\
x_1=&k\dfrac{\pi}{2} &k\in\Z
\intertext{$x\in[-\pi,\pi]$}
\tan(x)=&m&m\geq0\\
x=&\arctan(m)+k\pi&k\in\Z\\
\end{align*}
\section{Equazioni riconducibili ad equazioni elementari}
\begin{align*}
\intertext{Tipo uno}
\sin(\alpha)=&\sin(\beta)\\
\alpha=&\beta+2k\pi\\
\alpha=&\pi-\beta+2k\pi\\\index{Equazione!goniometrica!seno}
\intertext{Tipo due}
\cos(\alpha)=&\cos(\beta)\\
\alpha=&\pm\beta+2k\pi\\\index{Equazione!goniometrica!coseno}
\intertext{Tipo tre}
\tan(\alpha)=\tan(\beta)\\
\alpha=&\beta+k\pi\index{Equazione!goniometrica!coseno}
\end{align*}
\begin{align*}
\intertext{Riconducibili al tipo uno}
\sin(\alpha)=&-\sin(\beta)&&\sin(\alpha)=\sin(-\beta)\\
\sin(\alpha)=&\cos(\beta)&&\sin(\alpha)=\sin(\frac{\pi}{2}-\beta)\\
\sin(\alpha)=&-\cos(\beta)&&\sin(-\alpha)=\sin(\frac{\pi}{2}-\beta)\\
\sin^2(\alpha)=&\sin^2(\beta)
&&\begin{cases}
\sin(\alpha)=\sin(\beta)\\
\sin(\alpha)=-\sin(\beta)
\end{cases}\\
\intertext{Riconducibili al tipo due}
\cos(\alpha)=&-\cos(\beta)&&\cos(\alpha)=\cos(\pi-\beta)\\
\cos^2(\alpha)=&\cos^2(\beta)
&&\begin{cases}
\cos(\alpha)=\cos(\beta)\\
\cos(\alpha)=-\cos(\beta)
\end{cases}\\
\intertext{Riconducibili al tipo tre}
\tan(\alpha)=&-\tan(\beta)&&\tan(\alpha)=\tan(-\beta)\\
\tan^2(\alpha)=&\tan^2(\beta)
&&\begin{cases}
\tan(\alpha)=\tan(\beta)\\
\tan(\alpha)=-\tan(\beta)
\end{cases}
\end{align*}
\section{Equazioni risolvibili tramite  equazioni di secondo grado}
\begin{align*}
a\cos^2\alpha+b\cos\alpha+c=&0\quad a\neq 0\\
\cos\alpha=&t\\
at^2+bt+c=&0
\intertext{$\Delta>0$}
t_1=&\frac{-b+\sqrt{b^2-4ac}}{2a}\\
t_2=&\frac{-b-\sqrt{b^2-4ac}}{2a}
\intertext{se $t_1\in[-1,+1]$}
\cos\alpha=&t_1\\
\intertext{se $t_2\in[-1,+1]$}
\cos\alpha=&t_2
\intertext{$\Delta=0$}
t_1=&-\frac{b}{2a}\\
\intertext{se $t_1\in[-1,+1]$}
\cos\alpha=&t_1
\intertext{$\Delta<0$}
\intertext{Nessuna soluzione}
\end{align*}\index{Equazione!goniometrica!secondo grado}\index{Discriminante}\index{Delta}
\begin{align*}
a\sin^2\alpha+b\sin\alpha+c=&0\quad a\neq 0\\
\sin\alpha=&t\\
at^2+bt+c=&0
\intertext{$\Delta>0$}
t_1=&\frac{-b+\sqrt{b^2-4ac}}{2a}\\
t_2=&\frac{-b-\sqrt{b^2-4ac}}{2a}
\intertext{se $t_1\in[-1,+1]$}
\sin\alpha=&t_1\\
\intertext{se $t_2\in[-1,+1]$}
\sin\alpha=&t_2
\intertext{$\Delta=0$}
t_1=&-\frac{b}{2a}\\
\intertext{se $t_1\in[-1,+1]$}
\sin\alpha=&t_1\\
\intertext{$\Delta<0$}
\intertext{Nessuna soluzione}
\end{align*}\index{Equazione!goniometrica!secondo grado}\index{Discriminante}\index{Delta}
\begin{align*}
a\tan^2\alpha+b\tan\alpha+c=&0\quad a\neq 0\\
\tan\alpha=&t\\
at^2+bt+c=&0
\intertext{$\Delta>0$}
t_1=&\frac{-b+\sqrt{b^2-4ac}}{2a}\\
t_2=&\frac{-b-\sqrt{b^2-4ac}}{2a}
\tan\alpha=&t_1\\
\tan\alpha=&t_2
\intertext{$\Delta=0$}
t_1=&-\frac{b}{2a}\\
\tan\alpha=&t_1\\
\intertext{$\Delta<0$}
\intertext{Nessuna soluzione}
\end{align*}\index{Equazione!goniometrica!secondo grado}\index{Discriminante}\index{Delta}