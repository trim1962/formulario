% !TeX encoding = UTF-8
% !TeX spellcheck = it_IT
% !TeX root = formulario.tex
\chapter{Prodotto di disequazioni e equazioni}
\section{Forma normale}
Una disequazione prodotto è in forma normale se
\begin{equation*}
f(x)\cdot g(x)\begin{cases}
>0\\
\geq 0\\
<0\\
\leq 0
\end{cases}
\end{equation*}\index{Disequazione!prodotto}
\section{Risoluzione disequazione prodotto}
\begin{enumerate}
	\item Trovare il segno dei fattori
	\item Sovrapporre i grafici dei segni
	\item Leggere il grafico e trovare la soluzione
\end{enumerate}\index{Disequazione!prodotto!risoluzione}
\section{Forma normale}
Un'equazione prodotto è in forma normale se
\begin{equation*}
f(x)\cdot g(x)=0
\end{equation*}
\section{Risoluzione equazione prodotto}
\begin{enumerate}
	\item Risolvere, ponendoli uguali a zero, i fattori che compongono il prodotto.
\end{enumerate}\index{Equazione!prodotto!risoluzione}
{\centering\captionof{table}{Equazioni e disequazioni prodotto}\index{Disequazione!prodotto}\index{Equazione!prodotto}
	\begin{tabular}{Cp{0.4\textwidth}}
		\toprule
		Tipo& Soluzione \\ 
		\midrule
		f(x)\cdot g(x)=0	& Ha soluzione per quei valori di $x$ per cui $f(x)=0$ e $g(x)= 0$  \\ 
		f(x)\cdot g(x)>0	& Ha soluzione per quei valori di $x$ per cui $f(x)$ e $g(x)$ sono concordi\\ 
	f(x)\cdot g(x)<0& Ha soluzione per quei valori di $x$ per cui $f(x)$ e $g(x)$ sono discordi\\ 
		\bottomrule
	\end{tabular}\par}
\chapter{Disequazioni biquadratiche}
\section{Definizioni}
\begin{defn}[Disequazione biquadratica]\index{Disequazione!biquadratica}
Una disequazione razionale intera è un s disequazione biquadratica se è nella forma
\begin{align*}
	ax^4+bx^2+c>&0\\
	ax^4+bx^2+c<&0\\
	ax^4+bx^2+c\leq&0\\
	ax^4+bx^2+c\geq&0
\end{align*}
\end{defn}
\section{Proprietà}
\begin{thm}[Disequazioni biquadratiche]
	Data una disequazione biquadratica, posto $x^2=t$, indicato con $\Delta=b^2-4ac$  associato all'equazione $at^2+bt+c=0$ avremo
	\begin{description}
		\item[$\Delta>0$] allora $at^2+bt+c=0$ ha due soluzioni distinte $t_1=\alpha$ $t_2=\beta$ allora la disequazione biquadratica diviene \begin{align*}
			a(x^2-\alpha)(x^2-\beta)>0\\a(x^2-\alpha)(x^2-\beta)<0\\
		\end{align*}
			\item[$\Delta=0$] allora $at^2+bt+c=0$ ha due soluzioni coincidenti $t_1=\alpha$ allora la disequazione biquadratica diviene \begin{align*}
		a(x^2-\alpha)(x^2-\alpha)>0\\a(x^2-\alpha)(x^2-\alpha)<0\\
		\end{align*}
			\item[$\Delta<0$] allora $at^2+bt+c=0$ non ha  soluzioni  allora la disequazione biquadratica avrà sempre lo stesso segno di $a$.
	\end{description}
\end{thm}