% !TeX encoding = UTF-8
% !TeX spellcheck = it_IT
% !TeX root = formulario.tex
\chapter{Prodotto di disequazioni e equazioni}
\section{Forma normale}
Una disequazione prodotto è in forma normale se
\begin{equation}
f(x)\cdot g(x)\begin{cases}
>0\\
\geq 0\\
<0\\
\leq 0
\end{cases}
\end{equation}\index{Disequazione!prodotto}
\section{Risoluzione disequazione prodotto}
\begin{enumerate}
	\item Trovare il segno dei fattori
	\item Sovrapporre i grafici dei segni
	\item Leggere il grafico e trovare la soluzione
\end{enumerate}\index{Disequazione!prodotto!risoluzione}
\section{Forma normale}
Un'equazione prodotto è in forma normale se
\begin{equation}
f(x)\cdot g(x)=0
\end{equation}
\section{Risoluzione equazione prodotto}
\begin{enumerate}
	\item Risolvere, ponendoli uguali a zero, i fattori che compongono il prodotto.
\end{enumerate}\index{Equazione!prodotto!risoluzione}
\begin{center}
	\begin{tabular}{Cp{0.4\textwidth}}
		\toprule
		& Soluzione \\ 
		\midrule
		f(x)\cdot g(x)=0	& Ha soluzione per quei valori di $x$ per cui $f(x)=0$ e $g(x)= 0$  \\ 
		f(x)\cdot g(x)>0	& Ha soluzione per quei valori di $x$ per cui $f(x)$ e $g(x)$ sono concordi\\ 
	f(x)\cdot g(x)<0& Ha soluzione per quei valori di $x$ per cui $f(x)$ e $g(x)$ sono discordi\\ 
		\bottomrule
	\end{tabular}\captionof{table}{Equazioni e disequazioni prodotto}\index{Disequazione!prodotto}\index{Equazione!prodotto}
\end{center}