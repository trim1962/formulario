\chapter{Simmetrie}
\section{Simmetria assiale}
Una trasformazione assiale di asse $r$ è una trasformazione che per ogni ponto $P$ fa corrispondere il punto $P'$ se:
\begin{enumerate}
	\item il segmento $PP'$ è perpendicolare alla retta $r$
	\item Il punto medio del segmento $PP'$ appartiene alla retta $r$
\end{enumerate}
\subsection{Asse parallelo asse y}
\begin{tcolorbox}[sidebyside,righthand width=7cm,colback=white,colframe=white,fonttitle=\bfseries	]
\[\begin{cases}
x'=2a-x\\
y'=y
\end{cases}
\]
\tcblower
\includestandalone[width=6.5cm]{geometria/simmetriaassialeV}	
\end{tcolorbox}\index{Simmetria!assiale}
\subsection{Asse parallelo asse x}
\begin{tcolorbox}[sidebyside,righthand width=7cm,colback=white,colframe=white,fonttitle=\bfseries	]
		\[\begin{cases}
	x'=x\\
	y'=2b-y
	\end{cases}
	\]
	\tcblower
	\includestandalone[width=6.5cm]{geometria/simmetriaassialeO}
\end{tcolorbox}
\subsection{Asse $y=x$}
\begin{tcolorbox}[sidebyside,righthand width=7cm,colback=white,colframe=white,fonttitle=\bfseries	]
	\[\begin{cases}
	y'=x\\
	y'=y
	\end{cases}
	\]
\tcblower
		\includestandalone[width=6.5cm]{geometria/simmetriaassialeyux}
\end{tcolorbox}
\subsection{Asse $y=-x$}
\begin{tcolorbox}[sidebyside,righthand width=7cm,colback=white,colframe=white,fonttitle=\bfseries	]
	\[\begin{cases}
	y'=-x\\
	y'=-y
	\end{cases}
	\]
	\tcblower
	\includestandalone[width=6.5cm]{geometria/simmetriaassialeyumx}
\end{tcolorbox}
\subsection{Asse asse y}
\begin{tcolorbox}[sidebyside,righthand width=7cm,colback=white,colframe=white,fonttitle=\bfseries	]
	\[\begin{cases}
	x'=-x\\
	y'=y
	\end{cases}
	\]
	\tcblower
	\includestandalone[width=6.5cm]{geometria/simmetriaassialeVy}	
\end{tcolorbox}
\subsection{Asse asse x}
\begin{tcolorbox}[sidebyside,righthand width=7cm,colback=white,colframe=white,fonttitle=\bfseries	]
	\[\begin{cases}
	x'=x\\
	y'=-y
	\end{cases}
	\]
	\tcblower
	\includestandalone[width=6.5cm]{geometria/simmetriaassialeOx}
\end{tcolorbox}
\subsection{Asse asse $y=mx+q$}
\begin{tcolorbox}[sidebyside,righthand width=7cm,colback=white,colframe=white,fonttitle=\bfseries	]
	\[\begin{cases}
	x'=\frac{-2mq-m^2x+2my}{1+m^2}\\
	y'=\frac{2q+2mx-y+m^2y}{1+m^2}
	\end{cases}
	\]
	\tcblower
	\includestandalone[width=6.5cm]{geometria/simmetriaassialegenerale}
\end{tcolorbox}
\section{Simmetria centrale}
La simmetria rispetto ad un centro $C$ è una trasformazione che ad ogni punto $P$ a corrispondere un punto $P'$ in modo che il centro $C$ è punto medio del segmento $PP'$.
\begin{tcolorbox}[sidebyside,righthand width=7cm,colback=white,colframe=white,fonttitle=\bfseries	]
	\[\begin{cases}
	x'=-x+2x_0\\
	y'=-y+2y_0
	\end{cases}
	\]
	\tcblower
	\includestandalone[width=6.5cm]{geometria/simmetriacentrale}
\end{tcolorbox}\index{Simmetria!centrale}