\chapter{Cinematica}
\section{Medie}
\begin{equation*}
v_m=\dfrac{s_2-s_1}{\Delta t}=\dfrac{\Delta s}{\Delta t}
\end{equation*}\index{Velocità!media}
\begin{equation*}
a_m=\dfrac{v_2-v_1}{\Delta t}=\dfrac{\Delta v}{\Delta t}
\end{equation*}
\section{Moto rettilineo uniformemente accelerato}
\begin{align*}
v=&v_0+at\\
t=\dfrac{v-v_0}{a}\\
s=&s_0+\frac{1}{2}(v_x+v_0)t\\
s=&s_0+v_0t+\frac{1}{2}at^2\\
v^2=&v_{0}^2+2a(s-s_0)\\
a=&\frac{v^2-v_{0}^2}{2a(s-s_0)}\\
t=&\frac{2(x-x_0)}{v_0+v}\\
t=&\frac{v_x-v_0}{a}
\end{align*}\index{Moto uniforme!velocità}\index{Moto uniforme!spazio}\index{Moto uniforme!tempo}\index{Moto uniforme!accelerazione}
\section{Moto rettilineo costante}
\begin{align*}
a=&0\\
v=&v_0\\
s=&s_0+v_0t\\
\end{align*}\index{Moto uniforme!velocità}\index{Moto uniforme!spazio}\index{Moto uniforme!tempo}\index{Moto uniforme!accelerazione}
\section{Moto in caduta libera}
Sistema di riferimento verticale 
\begin{align*}
v=&v_0-gt\\
s=&\frac{1}{2}(v_0+v)t\\
s=&v_0t-\frac{1}{2}gt^2\\
v^2=&v_{0}^2-2gs\\
t=&\frac{v_0-v}{g}\\
s=&\frac{v_{0}^2-v^2}{2g}
\end{align*}
\chapter{Moto circolare uniforme}
{\centering
	\includestandalone[width=.6\linewidth]{geometria/velocitangolare}
	\captionof{figure}{Velocità e accelerazione centripeta}\par}
Modulo velocità, velocità angolare, modulo accelerazione centripeta, frequenza, periodo costanti. \index{Velocità!angolare}\index{Accelerazione!centripeta}\index{Frequenza}\index{Periodo}
\section{Frequenza Periodo}
\begin{align*}
f=&\dfrac{1}{T}\\
T=&\dfrac{1}{f}\\
\end{align*}
\section{Velocità tangenziale}
\begin{align*}
v=&\dfrac{s}{t}\\
v=&\dfrac{2\pi r}{T}\\
r=&\dfrac{vT}{2\pi}\\
T=&\dfrac{2\pi r}{v}\\
\end{align*}
\section{Velocità angolare}
\begin{align*}
\omega=&\dfrac{2\pi}{T}\\
\omega=&2\pi f\\
T=&\dfrac{2\pi}{\omega}\\
v=&\omega r\\
\omega=&\dfrac{v}{r}\\
r=&\dfrac{v}{\omega}
\end{align*}
\section{Accelerazione centripeta}
\begin{align*}
a_c=&\dfrac{v^2}{r}\\
a_c=&\omega^2 r\\
v=&\sqrt{a_c r}\\
r=&\dfrac{v^2}{a_c}\\
\omega=&\sqrt{\dfrac{a_c}{r}}\\
r=&\dfrac{a_c}{\omega^2}
\end{align*}
\section{Legge oraria}
\begin{equation*}
\phi_t=\phi_0+\omega t
\end{equation*}
\section{Equazioni parametriche}
\begin{align*}
x(t)=&r\cos(\omega t+\phi_0)\\
y(t)=&r\sin(\omega t+\phi_0)\\
v_x(t)=&-r\omega\sin(\omega t+\phi_0)\\
v_y(t)=&r\omega\cos(\omega t+\phi_0)\\
a_x(t)=&-r\omega^2\cos(\omega t+\phi_0)=-\omega^2 x(t)\\
a_y(t)=&-r\omega^2\sin(\omega t+\phi_0)=-\omega^2 y(t)\\
\end{align*}