\chapter{Misura di angoli}
\section{Gradi sessagesimali}
\begin{align}
\ang{1}=&\dfrac{angolo giro}{360}\\
\ang{;1;}=&\dfrac{\ang{1}}{60}\\
\ang{;;1}=&\dfrac{\ang{;1;}}{60}=\dfrac{\ang{1}}{3600}
\end{align}\index{Angoli!gradi!sessagesimali}
\begin{equation}
\alpha=x^\circ y'z''
\end{equation}
\section{Gradi sessa-decimali}
\begin{equation}
\beta=x.y^\circ\quad\text{x parte intera, y parte decimale}
\end{equation}\index{Angoli!gradi!sessa-decimali}
\section{Conversione da gradi sessagesimali a gradi sessa-decimali}
\begin{align}
\alpha=&x^\circ y'z''\\
\beta=&x^\circ+\left(\dfrac{y}{60}\right)^\circ+\left(\dfrac{z}{3600}\right)^\circ
\end{align}\index{Angoli!gradi!conversione}
\section{Conversione da gradi sessa-decimali  a gradi sessagesimali}
\begin{enumerate}
	\item Inizio
	\item $\beta=x.y^\circ$
	\item La parte intera di $\beta$ $x^\circ$ sono i gradi
	\item Per ottenerei i minuti $m'=(x.y^\circ-y^\circ)*60$
	\item $m'=z.k'$
	\item $z$ la parte intera di $m$ sono i minuti 
	\item Per ottenere i secondi
	$s''=(z.k''-z'')*60$ 
	\item $s''=p.q''$
	\item $p$ la parte intera di $s$ sono i secondi
	\item Stop
\end{enumerate}
\section{Radianti}
\begin{center}
	\includestandalone{geometria/radianti}
	\captionof{figure}{Radianti}
\end{center}\index{Angoli!radianti}
\begin{align}
\rho=\dfrac{arco}{raggio}=\dfrac{l}{r}
\end{align}\index{Angoli!radianti}
\section{Da radianti a gradi sessa-decimali}
\begin{equation}\index{Angoli!conversione!radianti gradi}
\alpha=\dfrac{\ang{180}}{\pi}\rho
\end{equation}
\section{Da gradi sessa-decimali a radianti}
\begin{equation}
\rho=\dfrac{\pi}{\ang{180}}\alpha
\end{equation}\index{Angoli!conversione!gradi radianti}