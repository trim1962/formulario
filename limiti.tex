\chapter{Limiti}
\section{Limite finito per x che tende a valore finito}
Data la funzione$\funzione{f}{D}{\R}$,definita nel suo dominio $D$ a valore in $\R$  diremo che la funzione $f$ tende al limite $l$ per $x$ che tende a $x_0$, quando, comunque preso un intorno $I(l)$, esiste un intorno $I(x_0)$ tale che: comunque preso $x$ diverso da $x_0$, appartenente  all'intorno $I(x_0)$ risulti che $f(x)$ appartenga all'intorno $I(l)$. In simboli
\begin{equation*}
\lim_{x\to x_0}f(x)=l
\end{equation*}
\begin{equation*}
\forall\; I(l)\; \exists\; I(x_0) : \forall x\in I(x_0)-\lbrace x_0\rbrace \longrightarrow f(x)\in I(l)
\end{equation*}\index{Limite!finito!finito}
\section{Limite infinito per x che tende a valore finito}
Data la funzione$\funzione{f}{D}{\R}$,definita nel suo dominio $D$ a valore in $\R$  diremo che la funzione $f$ tende al limite $\infty$ per $x$ che tende a $x_0$, quando, comunque preso un intorno $I(\infty)$, esiste un intorno $I(x_0)$ tale che: comunque preso $x$ diverso da $x_0$, appartenente  all'intorno $I(x_0)$ risulti che $f(x)$ appartenga all'intorno $I(\infty)$. In simboli
\begin{equation*}
\lim_{x\to x_0}f(x)=\infty
\end{equation*}
\begin{equation*}
\forall\; I(\infty)\; \exists\; I(x_0) : \forall x\in I(x_0)-\lbrace x_0\rbrace \longrightarrow f(x)\in I(\infty)
\end{equation*}\index{Limite!infinito!finito}
\section{Limite finito per x che tende a valore infinito}
Data la funzione$\funzione{f}{D}{\R}$,definita nel suo dominio $D$ a valore in $\R$  diremo che la funzione $f$ tende al limite $l$ per $x$ che tende a $\infty$, quando, comunque preso un intorno $I(l)$, esiste un intorno $I(\infty)$ tale che: comunque preso $x$ appartenente  all'intorno $I(\infty)$ risulti che $f(x)$ appartenga all'intorno $I(l)$. In simboli
\begin{equation*}
\lim_{x\to \infty}f(x)=l
\end{equation*}
\begin{equation*}
\forall\; I(l)\; \exists\; I(\infty) : \forall x\in I(\infty) \longrightarrow f(x)\in I(l)
\end{equation*}\index{Limite!finito!infinito}
\section{Limite infinito per x che tende a valore finito}
Data la funzione$\funzione{f}{D}{\R}$,definita nel suo dominio $D$ a valore in $\R$  diremo che la funzione $f$ tende al limite $\infty$ per $x$ che tende a $x_0$, quando, comunque preso un intorno $I(\infty)$, esiste un intorno $I(x_0)$ tale che: comunque preso $x$ diverso da $x_0$, appartenente  all'intorno $I(x_0)$ risulti che $f(x)$ appartenga all'intorno $I(\infty)$. In simboli
\begin{equation*}
\lim_{x\to x_0}f(x)=\infty
\end{equation*}
\begin{equation*}
\forall\; I(\infty)\; \exists\; I(x_0) : \forall x\in I(x_0)-\lbrace x_0\rbrace \longrightarrow f(x)\in I(\infty)
\end{equation*}\index{Limite!infinito!finito}
\section{Limite infinito per x che tende a valore infinito}
Data la funzione$\funzione{f}{D}{\R}$,definita nel suo dominio $D$ a valore in $\R$  diremo che la funzione $f$ tende al limite $\infty$ per $x$ che tende a $\infty$, quando, comunque preso un intorno $I(\infty)$, esiste un intorno $I(\infty)$ tale che: comunque preso $x$ appartenente  all'intorno $I(\infty)$ risulti che $f(x)$ appartenga all'intorno $I(\infty)$. In simboli
\begin{equation*}
\lim_{x\to \infty}f(x)=\infty
\end{equation*}
\begin{equation*}
\forall\; I(\infty)\; \exists\; I(\infty) : \forall x\in I(\infty) \longrightarrow f(x)\in I(\infty)
\end{equation*}\index{Limite!infinito!infinito}
\section{Operazioni}
%\begin{center}
% \begin{tabular}{FFFFFF}
%\toprule
%\lim_{x\to a}f(x) & \lim_{x\to a}g(x) & \lim_{x\to a}f(x)+g(x) &\lim_{x\to a}f(x)\cdot g(x) &\lim_{x\to a}\dfrac{1}{g(x)} &\lim_{x\to a}\dfrac{f(x)}{g(x)} \\[0.8cm] 
%
%m & \pm\infty& \pm\infty & \pm\infty &0 & 0 \\[0.8cm] 
%
%\pm\infty & m & \pm\infty & \pm\infty & \dfrac{1}{m} &\pm\infty \\[0.8cm]
% 
%0 &\pm\infty &\pm\infty & \text{Ind} & 0 & 0 \\[0.8cm] 
%
%\pm\infty & 0 & \pm\infty & \text{Ind} & \pm\infty & \pm\infty \\[0.8cm]
% 
%+\infty & +\infty & +\infty&+\infty &0 & \text{Ind} \\[0.8cm]
% 
%-\infty & -\infty & -\infty & +\infty & 0 & \text{Ind} \\[0.8cm] 
%
%+\infty & -\infty & \text{Ind} & -\infty & 0 & \text{Ind} \\[0.8cm]
%\bottomrule
%\end{tabular}\index{Limite!operazioni}
%\captionof{table}{Simboli matematici}
%\end{center}
\section{Limite della somma}
Date due funzioni $\funzione{f}{A}{\R}$ $\funzione{g}{A}{\R}$ se $\lim_{h \to x_0}f(x)=l$ e  $\lim_{h \to x_0}g(x)=m$ allora\[\lim_{h \to x_0}(f(x)+g(x)=\lim_{h \to x_0}f(x)+\lim_{h \to x_0}g(x)=l+m \]

\begin{center}
	\begin{tabular}{c*{4}{C}}
\diagbox{$g(x)$}{$f(x)$}&0&l&+\infty&-\infty \\ 
\cmidrule{2-5}
0&0&l&+\infty&-\infty\\
\cmidrule{2-5}
m&m&l+m&+\infty&-\infty\\
\cmidrule{2-5}
$+\infty$&+\infty&+\infty&+\infty&?\\
\cmidrule{2-5}
$-\infty$&-\infty&-\infty&?&-\infty\\
\cmidrule{2-5}
\end{tabular}\captionof{table}{Limite della somma}\index{Limite!somma}
\end{center}
\section{Limite della differenza}
Date due funzioni $\funzione{f}{A}{\R}$ $\funzione{g}{A}{\R}$ se $\lim_{h \to x_0}f(x)=l$ e  $\lim_{h \to x_0}g(x)=m$ allora\[\lim_{h \to x_0}(f(x)-g(x)=\lim_{h \to x_0}f(x)-\lim_{h \to x_0}g(x)=l-m \]

\begin{center}
	\begin{tabular}{c*{4}{C}}
		\diagbox{$g(x)$}{$f(x)$}&0&l&+\infty&-\infty \\ 
		\cmidrule{2-5}
		0&0&l&+\infty&-\infty\\	\cmidrule{2-5}
		m&-m&l-m&+\infty&-\infty\\\cmidrule{2-5}
		$+\infty$&-\infty&-\infty&?&-\infty\\cmidrule{2-5}
		$-\infty$&+\infty&+\infty&+\infty&?\\
		\cmidrule{2-5}
	\end{tabular}\captionof{table}{Limite della differenza}\index{Limite!differenza}
\end{center}
\section{Limite funzione razionale intera}
Data la funzione $\funzione{f}{\R}{\R}$ definita
$f(x)=ax^n+bx^{n-1}+cx^{n-2}+\dots+d$ allora il limite
\begin{align*}
\lim_{x\to \infty}f(x)=&\\
=&x^n\left(a+\dfrac{b}{x}+\dfrac{c}{x^2}+\cdots+\dfrac{d}{x^n}\right)
\intertext{se $n$ è pari}
=&+\infty\\
\intertext{se $n$ è dispari}
=&\begin{cases}
+\infty& \text{se} x\to +\infty\\[2ex]
-\infty& \text{se} x\to -\infty\\
\end{cases}
\end{align*}\index{Limite!funzione razionale!intera}
\section{Limite funzione razionale fratta}
Data la funzione $\funzione{f}{\R}{\R}$ definita
$f(x)=\dfrac{a_n x^n+a_{n-1}x^{n-1}+\cdots a_0}{b_m x^m+a_{m-1}x^{m-1}+\cdots b_0}$
 allora il limite
\begin{align*}
\lim_{x\to \infty}f(x)=&\\
=&\lim_{x\to \infty}\dfrac{x^n\left(a_n+\frac{a_{n-1}}{x}+\cdots+\dfrac{a_0}{x^n}\right)}{x^m\left(b_m+\frac{b_{m-1}}{x}+\cdots+\dfrac{b_0}{x^m}\right)}\\
=&\lim_{x\to \infty}x^{n-m}\dfrac{\left(a_n+\frac{a_{n-1}}{x}+\cdots+\dfrac{a_0}{x^n}\right)}{\left(b_m+\frac{b_{m-1}}{x}+\cdots+\dfrac{b_0}{x^m}\right)}\\
\intertext{Se $n>m$}
\lim_{x\to +\infty}f(x)=&+\infty\\
\lim_{x\to -\infty}f(x)=&\begin{cases}
+\infty& \text{se}\; n-m\quad \text{pari}\\[2ex]
-\infty& \text{se}\; n-m\quad \text{dispari}\\
\end{cases}
\intertext{Se $n<m$}
\lim_{x\to \infty}f(x)=&0\\
\intertext{Se $n=m$}
\lim_{x\to \infty}f(x)=&\dfrac{a_n}{b_m}
\end{align*}\index{Limite!funzione razionale!fratta}
\chapter{Funzioni continue}
\section{Definizione}
Una funzione $\funzione{f}{D}{\R}$ è continua se
\begin{equation*}
\lim_{x\to a}f(x)=f(a)
\end{equation*}\index{Funzione!continua}
