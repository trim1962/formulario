\section{Punti Notevoli triangolo}
\subsection{Incentro}
\begin{defn}[Bisettrice]
Retta che divide un angolo in due parti uguali
\end{defn}\index{Angolo!bisettrice}
\begin{defn}[Incentro]
Punto di incontro delle bisettrici del triangolo.
\end{defn}\index{Triangolo!bisettrice}\index{Incentro}
\begin{defn}[Incerchio]
Circonferenza che ha centro nell'incentro e tangente ai lati del triangolo.
\end{defn}\index{Triangolo!circonferenza!inscritta}\index{Incerchio}
{\centering
	\includestandalone{geometria/IncentroDef}
	\captionof{figure}{Incentro Triangolo}\par}\index{Incentro}
\begin{defn}[Inraggio]
	Il raggio incerchio è chiamato inraggio
\end{defn}\index{Triangolo!bisettrice}\index{Inraggio}
\begin{thm}[Inraggio]
	In un triangolo di superficie $S$ e semiperimetro $p$ e lati di lunghezza $a$, $b$ e $c$ avremo che il raggio della circonferenza inscritta è:
	\begin{align*}
r=&\dfrac{S}{p}\\
r=&\sqrt{\dfrac{(p-a)(p-b)(p-c)}{p}}
	\end{align*}
\end{thm}\index{Triangolo!bisettrice}\index{Inraggio}
{\centering
	\includestandalone{geometria/Inraggio}
	\captionof{figure}{Inraggio}\par}\index{Inraggio}
\begin{thm}[Area incerchio]
	In un triangolo di semiperimetro $p$ e lati di lunghezza $a$, $b$ e $c$ avremo che l'area dell'incerchio è
	\[ A=\pi\dfrac{(p-a)(p-b)(p-c)}{p}\]
\end{thm}\index{Triangolo!bisettrice}\index{Incerchio!area}
\section{Circocentro}
\begin{defn}[Circocentro]
	Centro della circonferenza circoscritta
\end{defn}\index{Triangolo!circonferenza!circoscritta}\index{Circocentro}
{\centering
	\includestandalone{geometria/circumcerchio}
	\captionof{figure}{Circumcerchio}\par}\index{Inraggio}