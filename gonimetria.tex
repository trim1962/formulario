% !TeX encoding = UTF-8
% !TeX spellcheck = it_IT
% !TeX root = formulario.tex
%\onecolumn
\chapter{Goniometria}\label{ch:goniometria}
%{\centering\captionof{table}{Valori particolari di funzioni goniometriche}
%	%	\renewcommand{\arraystretch}{2}
%	\begin{tabular}{cccccc}
%		\toprule
%		Gradi & Radianti & Seno & Coseno & Tangente & Cotangente \\ [.25cm]
%		%\midrule
%		$\ang{0}$ & 0 & 0 & 1 & 0 & n.e. \\ [.25cm] 
%%	\midrule%
%	$\ang{15}$ &$\dfrac{1}{12}\pi$ &$\dfrac{1}{4}\left(\sqrt{6}-\sqrt{2}\right)$&$\dfrac{1}{4}\left(\sqrt{6}+\sqrt{2}\right)$&$2-\sqrt{3}$& $2+\sqrt{3}$ \\ [.25cm]
%%		\hline%
%	%
%		$\ang{18}$&$\dfrac{1}{10}\pi$& $\dfrac{1}{4}\left(\sqrt{5}-1\right)$ & $\dfrac{1}{4}\sqrt{10+2\sqrt{5}}$ & $\dfrac{1}{5}\sqrt{25-10\sqrt{5}}$ & $\sqrt{5+2\sqrt{5}}$ \\ [.25cm]
%	%	\hline%
%		$\ang{22;30;}$&$\dfrac{1}{8}\pi$&$\dfrac{1}{2}\sqrt{2-\sqrt{2}}$&$\dfrac{1}{2}\sqrt{2+\sqrt{2}}$&$\sqrt{2}-1$&$\sqrt{2}+1$ \\ [.25cm]
%%		\hline%
%		$\ang{30}$&$\dfrac{1}{6}\pi$&$\dfrac{1}{2}$&$\dfrac{\sqrt{3}}{2}$&$\dfrac{\sqrt{3}}{3}$&$\sqrt{3}$\\ [.25cm]
%%		\hline%
%		$\ang{36}$&$\dfrac{1}{5}\pi$&$\dfrac{1}{4}\sqrt{10-2\sqrt{5}}$&$\dfrac{1}{4}\left(\sqrt{5}+1\right)$&$\sqrt{5-2\sqrt{5}}$&$\dfrac{1}{5}\sqrt{25+10\sqrt{5}}$\\ [.4cm]
%%		\hline%
%		$\ang{45}$&$\dfrac{1}{4}\pi$&$\dfrac{\sqrt{2}}{2}$& $\dfrac{\sqrt{2}}{2}$ & 1 & 1 \\ [.4cm]
%%		\hline%
%		$\ang{54}$&$\dfrac{3}{10}\pi$& $\dfrac{1}{4}\left(\sqrt{5}+1\right)$ & $\dfrac{1}{4}\sqrt{10-2\sqrt{5}}$ & $\dfrac{1}{5}\sqrt{25+10\sqrt{5}}$ & $\sqrt{5-2\sqrt{5}}$ \\ [.25cm]
%%		\hline%
%		$\ang{60}$&$\dfrac{1}{3}\pi$&$\dfrac{\sqrt{3}}{2}$&$\dfrac{1}{2}$&$\sqrt{3}$&$\dfrac{\sqrt{3}}{3}$\\ [.25cm]
%%		\hline%
%$\ang{65;30;}$&$\dfrac{3}{8}\pi$&$\dfrac{1}{2}\sqrt{2+\sqrt{2}}$&$\dfrac{1}{2}\sqrt{2-\sqrt{2}}$&$\sqrt{2}+1$&$\sqrt{2}-1$ \\ [.25cm]
%%		\hline%
%		$\ang{72}$&$\dfrac{2}{5}\pi$&$\dfrac{1}{4}\sqrt{10+2\sqrt{5}}$&$\dfrac{1}{4}\left(\sqrt{5}-1\right)$&$\sqrt{5+2\sqrt{5}}$&$\dfrac{1}{5}\sqrt{25-10\sqrt{5}}$\\ [.4cm]
%%		\hline%
%		$\ang{75}$ &$\dfrac{5}{12}\pi$ &$\dfrac{1}{4}\left(\sqrt{6}+\sqrt{2}\right)$&$\dfrac{1}{4}\left(\sqrt{6}-\sqrt{2}\right)$&$2+\sqrt{3}$& $2-\sqrt{3}$ \\ [.25cm]
%%		\hline%
%		$\ang{90}$&$\dfrac{\pi}{2}$&1&0&n.e.&0\\[.25cm]
%	%	\hline%
%		$\ang{180}$&$\pi$&0&-1& 0 &n.e.\\ [.25cm]
%	%	\hline%
%		$\ang{270}$&$\dfrac{3}{2}\pi$&-1&0&n.e.&0\\ [.25cm]
%%		\hline%
%		$\ang{360}$&$2\pi$&0&1&0&n.e.\\ [.25cm]
%		\bottomrule%
%	\end{tabular}
%\par}
%\twocolumn
\begin{table}
		\begin{tabular}{cccccc}
		\toprule
		Gradi & Radianti & Seno & Coseno & Tangente & Cotangente \\ [.25cm]
		%\midrule
		$\ang{0}$ & 0 & 0 & 1 & 0 & n.e. \\ [.25cm] 
		%	\midrule%
		$\ang{15}$ &$\dfrac{1}{12}\pi$ &$\dfrac{1}{4}\left(\sqrt{6}-\sqrt{2}\right)$&$\dfrac{1}{4}\left(\sqrt{6}+\sqrt{2}\right)$&$2-\sqrt{3}$& $2+\sqrt{3}$ \\ [.25cm]
		%		\hline%
		%
		$\ang{18}$&$\dfrac{1}{10}\pi$& $\dfrac{1}{4}\left(\sqrt{5}-1\right)$ & $\dfrac{1}{4}\sqrt{10+2\sqrt{5}}$ & $\dfrac{1}{5}\sqrt{25-10\sqrt{5}}$ & $\sqrt{5+2\sqrt{5}}$ \\ [.25cm]
		%	\hline%
		$\ang{22;30;}$&$\dfrac{1}{8}\pi$&$\dfrac{1}{2}\sqrt{2-\sqrt{2}}$&$\dfrac{1}{2}\sqrt{2+\sqrt{2}}$&$\sqrt{2}-1$&$\sqrt{2}+1$ \\ [.25cm]
		%		\hline%
		$\ang{30}$&$\dfrac{1}{6}\pi$&$\dfrac{1}{2}$&$\dfrac{\sqrt{3}}{2}$&$\dfrac{\sqrt{3}}{3}$&$\sqrt{3}$\\ [.25cm]
		%		\hline%
		$\ang{36}$&$\dfrac{1}{5}\pi$&$\dfrac{1}{4}\sqrt{10-2\sqrt{5}}$&$\dfrac{1}{4}\left(\sqrt{5}+1\right)$&$\sqrt{5-2\sqrt{5}}$&$\dfrac{1}{5}\sqrt{25+10\sqrt{5}}$\\ [.4cm]
		%		\hline%
		$\ang{45}$&$\dfrac{1}{4}\pi$&$\dfrac{\sqrt{2}}{2}$& $\dfrac{\sqrt{2}}{2}$ & 1 & 1 \\ [.4cm]
		%		\hline%
		$\ang{54}$&$\dfrac{3}{10}\pi$& $\dfrac{1}{4}\left(\sqrt{5}+1\right)$ & $\dfrac{1}{4}\sqrt{10-2\sqrt{5}}$ & $\dfrac{1}{5}\sqrt{25+10\sqrt{5}}$ & $\sqrt{5-2\sqrt{5}}$ \\ [.25cm]
		%		\hline%
		$\ang{60}$&$\dfrac{1}{3}\pi$&$\dfrac{\sqrt{3}}{2}$&$\dfrac{1}{2}$&$\sqrt{3}$&$\dfrac{\sqrt{3}}{3}$\\ [.25cm]
		%		\hline%
		$\ang{65;30;}$&$\dfrac{3}{8}\pi$&$\dfrac{1}{2}\sqrt{2+\sqrt{2}}$&$\dfrac{1}{2}\sqrt{2-\sqrt{2}}$&$\sqrt{2}+1$&$\sqrt{2}-1$ \\ [.25cm]
		%		\hline%
		$\ang{72}$&$\dfrac{2}{5}\pi$&$\dfrac{1}{4}\sqrt{10+2\sqrt{5}}$&$\dfrac{1}{4}\left(\sqrt{5}-1\right)$&$\sqrt{5+2\sqrt{5}}$&$\dfrac{1}{5}\sqrt{25-10\sqrt{5}}$\\ [.4cm]
		%		\hline%
		$\ang{75}$ &$\dfrac{5}{12}\pi$ &$\dfrac{1}{4}\left(\sqrt{6}+\sqrt{2}\right)$&$\dfrac{1}{4}\left(\sqrt{6}-\sqrt{2}\right)$&$2+\sqrt{3}$& $2-\sqrt{3}$ \\ [.25cm]
		%		\hline%
		$\ang{90}$&$\dfrac{\pi}{2}$&1&0&n.e.&0\\[.25cm]
		%	\hline%
		$\ang{180}$&$\pi$&0&-1& 0 &n.e.\\ [.25cm]
		%	\hline%
		$\ang{270}$&$\dfrac{3}{2}\pi$&-1&0&n.e.&0\\ [.25cm]
		%		\hline%
		$\ang{360}$&$2\pi$&0&1&0&n.e.\\ [.25cm]
		\bottomrule%
	\end{tabular}
	\caption{Valori particolari di funzioni goniometriche}\label{tab:Valori-funzioni-goniometriche}
\end{table}
\section{Circonferenza goniometrica}\label{sec:circonferenza-goniometrica}
\begin{defn}[Circonferenza goniometrica]\label{defn:Circoferenza-Gonimetrica}
Dato un sistema di riferimento cartesiana, una circonferenza goniometria è una circonferenza con centro nell'origine e raggio unitario
\begin{equation*}
x^2+y^2=1
\end{equation*}\index{Circonferenza!goniometrica}
\end{defn}
\section{Definizione coseno}\label{sec:definizione-coseno}
\begin{defn}[Coseno]\label{defn:coseno1}
Data una circonferenza goniometrica ed un angolo $\alpha$ diremo coseno dell'angolo $\alpha$, l'ascissa del punto $P$.
\end{defn}\index{Coseno!definizione}
\begin{figure}
	\centering
	\includestandalone{geometria/cosenodefinizione}
	\caption{Definizione coseno}
	\label{fig:cosenodefinizione}
\end{figure}\index{Funzione!coseno!definizione}
%{\centering
%	\includestandalone{geometria/cosenodefinizione}
%	\captionof{figure}{Definizione coseno}\par}\index{Funzione!coseno!definizione}
\begin{prop}[Il coseno è limitato]\label{prop:Cosenolimitato}
Il coseno è limitato\index{Funzione!limitata!coseno}\index{Coseno!limitato}
e assume i valori compresi
\begin{equation*}
-1\leq \cos\alpha \leq 1
\end{equation*}
\end{prop}
\begin{prop}[Periodo coseno]\label{Prop:periodocoseno}
Il coseno è periodico. 
Se l'angolo è misurato in gradi il coseno è periodico di periodo $k\ang{360}$\index{Funzione!periodica!coseno}
\begin{equation*}
\cos(\alpha+k\ang{360;;})=\cos\alpha
\end{equation*}
Se l'angolo è misurato in radianti il coseno è periodico di periodo $2k\pi$\index{Funzione!periodica!coseno}
\begin{equation*}
\cos(\alpha+2k\pi)=\cos\alpha
\end{equation*}
\end{prop}
\section{Definizione seno}\label{sec:definizione-seno}
\begin{defn}[Definizione seno]
Data una circonferenza goniometrica ed un angolo $\alpha$ chiamo seno l'ordinata del punto $P$.\index{Seno!definizione}
\end{defn}
%{\centering
%	\includestandalone{geometria/senodefinizione}
%	\captionof{figure}{Definizione seno}\par}\index{Funzione!seno!definizione}
\begin{figure}
	\centering
	\includestandalone{geometria/senodefinizione}
	\caption{Definzione seno}
	\label{fig:senodefinizione}
\end{figure}\index{Funzione!seno!definizione}
\begin{prop}[Il seno è limitato]\label{prop:senolimitato}
	Il seno è limitato\index{Funzione!limitata!seno}\index{Seno!limitato}
e	assume i valori compresi
	\begin{equation*}
	-1\leq \sin\alpha \leq 1
	\end{equation*}
\end{prop}
\begin{prop}[Periodo seno]\label{Prop:periodoseno}
	Il seno è periodico. 
	Se l'angolo è misurato in gradi il seno è periodico di periodo $k\ang{360}$\index{Funzione!periodica!seno}
	\begin{equation*}
	\sin(\alpha+k\ang{360;;})=\sin\alpha
	\end{equation*}
	Se l'angolo è misurato in radianti il seno è periodico di periodo $2k\pi$\index{Funzione!periodica!seno}
	\begin{equation*}
	\sin(\alpha+2k\pi)=\sin\alpha
	\end{equation*}
\end{prop}
\section{Definizione tangente}
\begin{defn}[Definizione tangente]
	Data una circonferenza goniometrica ed un angolo $\alpha$ diremo tangente dell'angolo $\alpha$ l'ordinata del punto $T$ intersezione tra il prolungamento del raggio e la tangente alla circonferenza passante per il punto $(1,0)$  .\index{Funzione!tangente!definizione}
\end{defn}
\begin{prop}[La tangente non è limitata]\label{prop:tangentenonlimitata}
La tangente è non limitata e assume i seguenti valori:
\begin{equation*}
-\infty<\tan\alpha<\infty
\end{equation*}\index{Funzione!illimitata!tangente}
\end{prop}
\begin{figure}
	\centering
	\includestandalone{geometria/tangentedefinizione}
	\caption{Definizione tangente}
	\label{fig:tangentedefinizione}
\end{figure}\index{Funzione!tangente!definizione}
%{\centering
%	\includestandalone{geometria/tangentedefinizione}
%	\captionof{figure}{Definizione tangente}\par}\index{Funzione!tangente!definizione}
\begin{prop}[Periodo tangente]\label{prop:PeriodoTangente}
	La tangente è periodica.
	Se l'angolo è espresso in gradi la tangente è periodica di periodo $k\ang{180}$\index{Funzione!periodica!tangente}
	\begin{equation*}
	\tan(\alpha+k\ang{180;;})=\tan\alpha
	\end{equation*}
	Se l'angolo è espresso in radianti la tangente è periodica di periodo $k\pi$\index{Funzione!periodica!tangente}
	\begin{equation*}
	\tan(\alpha+k\pi)=\tan\alpha
	\end{equation*}
\end{prop}
\section{Definizione cotangente}
\begin{defn}[Definizione cotangente]
	Data una circonferenza goniometrica ed un angolo $\alpha$ diremo cotangente dell'angolo $\alpha$ l'ascissa del punto $C$ intersezione tra il prolungamento del raggio e la tangente alla circonferenza passante per il punto $(0,1)$  .\index{Funzione!cotangente!definizione}
\end{defn}
\begin{prop}[La cotangente non è limitata]\label{prop:cotangentenonlimitata}
	La tangente è non limitata e assume i seguenti valori:
	\begin{equation*}
	-\infty<\cot\alpha< \infty
	\end{equation*}\index{Funzione!illimitata!cotangente}
\end{prop}
\begin{figure}
	\centering
	\documentclass[10pt]{standalone}
\usepackage{amsmath}
\usepackage{pgf,tikz}
\usetikzlibrary{calc}
\usepackage{mathrsfs}
\usetikzlibrary{arrows}
\pagestyle{empty}
\begin{document}
	

		\begin{tikzpicture}[>=triangle  45]
		\pgfmathsetmacro{\raggio}{3};
		\pgfmathsetmacro{\angolob}{50};
		\pgfmathsetmacro{\angolo}{\angolob};
		\pgfmathsetmacro{\y}{3};
		\coordinate [label=below right:$O$] (oo)  at (0,0);
		\coordinate[label=above:$P$] (P) at ({\raggio*cos(\angolo)},{\raggio*sin(\angolo)});
		\coordinate(Q) at (\raggio,0);
		\coordinate[label=above:$P$] (P) at ({\raggio*cos(\angolo)},{\raggio*sin(\angolo)});
		\coordinate(Q) at (0,\raggio);
		\coordinate[label=above right:$C$] (T) at ({\raggio*(cot(\angolo ))},\raggio);
		\coordinate[label=above:$\cot\alpha$] (M)at ($(T)!0.5!(Q)$);
		\draw[->] (-\raggio-1,0) -- (\raggio+1,0) node[above] {$x$} ;
		\draw[->] (0,-\raggio-1) -- (0,\raggio+1) node[left] {$y$} ;
		\draw (-\raggio-1,\raggio) -- (\raggio+1,\raggio) ;
		\draw (oo) circle (\raggio) ;
		\draw[->] (\raggio/\y,0 ) arc (0:\angolo:\raggio/\y) ;
		\draw (\angolo/2:\raggio/\y) node[ above right]  {$\alpha$};
		\draw (oo)-- (P) ;
		\draw (T)-- (oo) ;
		\filldraw[black] (oo) circle(1pt);
		\filldraw[black] (P) circle(1pt);
		\filldraw[black] (T) circle(1pt);
		\filldraw[black] (Q) circle(1pt);
		\end{tikzpicture}
\end{document}
	\caption{Definizione cotangente}
	\label{fig:cotangentedefinizione}
\end{figure}\index{Funzione!cotangente!definizione}
 %{\centering
%	\includestandalone{geometria/cotangentedefinizione}
%	\captionof{figure}{Definizione cotangente}\par}\index{Funzione!cotangente!definizione}
\begin{prop}[Periodo cotangente]\label{prop:PeriodoCotangente}
	La cotangente è periodica.
	Se l'angolo è espresso in gradi la cotangente è periodica di periodo $k\ang{180}$\index{Funzione!periodica!cotangente}
	\begin{equation*}
	\cot(\alpha+k\ang{180;;})=\cot\alpha
	\end{equation*}
	Se l'angolo è espresso in radianti la cotangente è periodica di periodo $k\pi$\index{Funzione!periodica!cotangente}
	\begin{equation*}
	\cot(\alpha+k\pi)=\cot\alpha
	\end{equation*}
\end{prop}

\section{Relazioni fondamentali goniometria}\label{sec:relazioni-fondamentali-goniometria}
\begin{align*}	
	&\cos^{2}\alpha+\sin^{2}\alpha=1\\[.25cm]\index{Goniometria!relazione!fondamentale}
	&\tan\alpha={}\dfrac{\sin\alpha}{\cos\alpha}&\alpha\neq\dfrac{\pi}{2}+k\pi\\[.25cm]\index{Tangente!definizione}
	&\cot\alpha={}\dfrac{\cos\alpha}{\sin\alpha}&\alpha\neq k\pi\\ \index{Cotangente!definizione}
	&\tan\alpha\cot\alpha={}1&\alpha\neq\dfrac{\pi}{2}+k\pi\quad\alpha\neq k\pi
\end{align*}\index{Funzione!seno}\index{Funzione!coseno}\index{Funzione!tangente}\index{Funzione!cotangente}
\section{Relazioni derivate}\label{sec:relazioni-derivate}
\begin{align*}
\cos^{2}\alpha=&{}1-\sin^{2}\alpha\\\index{Coseno!dato!seno}
\cos\alpha=&{}\pm\sqrt{1-\sin^{2}\alpha}\\
\sin^{2}\alpha=&{}1-\cos^{2}\alpha \\\index{Seno!dato!coseno}
\sin^{2}\alpha=&{}\pm\sqrt{1-\cos^{2}\alpha}\\
\cos^{2}\alpha=&{}\dfrac{1}{1+{\tan}^{2}\alpha} &&\alpha\neq\dfrac{\pi}{2}+k\pi\\
\cos\alpha=&{}\pm\dfrac{1}{\sqrt{1+{\tan}^{2}\alpha}} &&\alpha\neq\dfrac{\pi}{2}+k\pi\\\index{Coseno!dato!tangente}
\sin^{2}\alpha=&{}\dfrac{\tan^{2}\alpha}{1+\tan^{2}\alpha}&&\alpha\neq\dfrac{\pi}{2}+k\pi\\
\sin\alpha=&{}\pm\dfrac{\tan\alpha}{\sqrt{1+\tan^{2}\alpha}}&&\alpha\neq\dfrac{\pi}{2}k\pi\\\index{Seno!dato!tangente}
 \sin\alpha=&{}\tan\alpha\cos\alpha&&\alpha\neq\dfrac{\pi}{2}+k\pi\\\index{Seno!dato!tangente}
\sin^{2}\alpha=&{}\dfrac{1}{1+{\cot}^{2}\alpha} &&\alpha\neq k\pi\\
\cos\alpha=&{}\pm\dfrac{1}{\sqrt{1+{\cot}^{2}\alpha}} &&\alpha\neq k\pi\\\index{Coseno!dato!cotangente}
\cos^{2}\alpha=&{}\dfrac{\cot^{2}\alpha}{1+\cot^{2}\alpha}&&\alpha\neq k\pi\\
\cos\alpha=&{}\pm\dfrac{\cot\alpha}{\sqrt{1+\cot^{2}\alpha}}&&\alpha\neq+k\pi\\\index{Coseno!dato!cotangente}
\cos\alpha=&{}\cot\alpha\sin\alpha&&\alpha\neq k\pi\\\index{Coseno!dato!cotangente} 
\tan\alpha=&{}\pm\dfrac{\sin(x)}{\sqrt{1-\sin^2(x)}}&&\alpha\neq\dfrac{\pi}{2}+k\pi\\\index{Tangente!dato!seno}
\tan\alpha=&{}\pm\dfrac{\sqrt{1-\cos^2(x)}}{\cos(x)}
&&\alpha\neq\dfrac{\pi}{2}+k\pi\\ \index{Tangente!dato!coseno}
\cot\alpha=&{}\pm\dfrac{\cos(x)}{\sqrt{1-\cos^2(x)}}&&\alpha\neq k\pi\\\index{Cotangente!dato!coseno}
\cot\alpha=&{}\pm\dfrac{\sqrt{1-\sin^2(x)}}{\sin(x)}
&&\alpha\neq k\pi\index{Cotangente!dato!seno}
\end{align*}
%{\centering\index{Seno!quadrato}\index{Coseno!quadrato}\index{Tangente!quadrato}%\index{Cotangente!quadrato}
%	\begin{tabular}{LCCC}
%		\toprule 
%		& \sin(x) & \cos(x) &\tan(x)  \\
%		\midrule
%		\sin(x)	&\sin(x)  & \pm\sqrt{1-\cos^2(x)} & \pm\dfrac{\tan(x)}{\sqrt{1+\tan^2(x)}} \\[.5cm]
%		\cos(x)	&\pm\sqrt{1-\sin^2(x)}  & \cos(x) & \pm\dfrac{1}{\sqrt{1+\tan^2(x)}} \\[.5cm]
%		\tan(x)	& \pm\dfrac{\sin(x)}{\sqrt{1-\sin^2(x)}} &\pm\dfrac{\sqrt{1-\cos^2(x)}}{\cos(x)}   & \tan(x) \\[.5cm]
%		\bottomrule
%	\end{tabular}\captionof{table}{Relazioni derivate}
%\par}
\section{Formule di addizione e sottrazione}\label{sec:formule-di-addizione}
\begin{align*}
\cos\left(\alpha-\beta\right)=&{}\cos\alpha\cos\beta+\sin\alpha\sin\beta\\\index{Coseno!differenza!angoli}
\cos\left(\alpha+\beta\right)=&{}\cos\alpha\cos\beta-\sin\alpha\sin\beta\\\index{Coseno!somma!angoli}
\sin\left(\alpha-\beta\right)=&{}\sin\alpha\cos\beta-\cos\alpha\sin\beta\\\index{Seno!differenza angoli}
\sin\left(\alpha+\beta\right)=&{}\sin\alpha\cos\beta+\cos\alpha\sin\beta\\\index{Seno!somma!angoli}
\tan\left(\alpha-\beta\right)=&{}\dfrac{\tan\alpha-\tan\beta}{1+\tan\alpha\tan\beta} &\alpha,\beta,\left(\alpha-\beta\right)\neq\left(2k+1\right)\dfrac{\pi}{2}\\\index{Tangente!differenza!angoli}
\cot\left(\alpha+\beta\right)=&{}\dfrac{\cot\alpha\cot\beta-1}{\cot\beta+\cot\alpha}
&\alpha,\beta,\left(\alpha+\beta\right)\neq k\pi\index{Cotangente!somma!angoli}
\end{align*}
\section{Angoli supplementari}\label{sec:angoli-supplementari}
\begin{align*}
\cos\alpha=&-\cos(\pi-\alpha)\\
\sin\alpha=&\sin(\pi-\alpha)\\
\tan\alpha=&-\tan(\pi-\alpha)\\
\cot\alpha=&-\cot(\pi-\alpha)
\end{align*}
\section{Angoli la cui differenza è \texorpdfstring{$\pi$}{\textpi}}
\begin{align*}
\cos\alpha=&-\cos(\pi+\alpha)\\
\sin\alpha=&-\sin(\pi+\alpha)\\
\tan\alpha=&\tan(\pi+\alpha)\\
\cot\alpha=&\cot(\pi+\alpha)
\end{align*}
\section{Angoli esplementari}\label{sec:angoli-esplementari}
\begin{align*}
\cos\alpha=&\cos(2\pi-\alpha)\\
\sin\alpha=&-\sin(2\pi-\alpha)\\
\tan\alpha=&-\tan(2\pi-\alpha)\\
\cot\alpha=&-\cot(2\pi-\alpha)
\end{align*}
\section{Angoli opposti}\label{sec:angoli-opposti}
\begin{align*}
\cos\alpha=&\cos(-\alpha)\\
\sin\alpha=&-\sin(-\alpha)\\
\tan\alpha=&-\tan(-\alpha)\\
\cot\alpha=&-\cot(-\alpha)
\end{align*}
\section{Angoli complementari}\label{sec:angoli-complementari}
\begin{align*}
\cos\alpha=&\sin(\dfrac{\pi}{2}-\alpha)\\
\sin\alpha=&\cos(\dfrac{\pi}{2}-\alpha)\\
\tan\alpha=&\cot(\dfrac{\pi}{2}-\alpha)\\
\cot\alpha=&\tan(\dfrac{\pi}{2}-\alpha)\\
\end{align*}
\section{Angoli la cui differenza è \texorpdfstring{$\dfrac{\pi}{2}$}{\textpi/2} }
\begin{align*}
\cos\alpha=&\sin(\dfrac{\pi}{2}+\alpha)\\
\sin\alpha=&-\cos(\dfrac{\pi}{2}+\alpha)\\
\tan\alpha=&-\cot(\dfrac{\pi}{2}+\alpha)\\
\cot\alpha=&-\tan(\dfrac{\pi}{2}+\alpha)\\
\end{align*}
\section{Angoli la cui somma è \texorpdfstring{$\pi$}{\textpi}}
\begin{align*}
\cos\alpha=&-\sin(\dfrac{3}{2}\pi-\alpha)\\
\sin\alpha=&-\cos(\dfrac{3}{2}\pi-\alpha)\\
\tan\alpha=&\cot(\dfrac{3}{2}\pi-\alpha)\\
\cot\alpha=&\tan(\dfrac{3}{2}\pi-\alpha)\\
\end{align*}
\section{Angoli la cui differenza è \texorpdfstring{$\dfrac{3}{2}\pi$}{3/2 \textpi}}
\begin{align*} 
\cos\alpha=&-\sin(\dfrac{3}{2}\pi-\alpha)\\
\sin\alpha=&\cos(\dfrac{3}{2}\pi-\alpha)\\
\tan\alpha=&-\cot(\dfrac{3}{2}\pi-\alpha)\\
\cot\alpha=&-\tan(\dfrac{3}{2}\pi-\alpha)\\
\end{align*}
\section{Funzioni inverse}\label{sec:funzioni-inverse}
\begin{align*}
\alpha=&\arcsin x\quad\Longleftrightarrow\quad\sin\alpha=x&& x\in[-1,1]&\alpha\in[-\dfrac{\pi}{2},\dfrac{\pi}{2}]\\
\alpha=&\arccos x\quad\Longleftrightarrow\quad\cos\alpha=x&&x\in[-1,1]&\alpha\in[0,\pi]\\
\alpha=&\arctan x\quad\Longleftrightarrow\quad\tan\alpha=x&& x\in\R
&\alpha\in(-\dfrac{\pi}{2},\dfrac{\pi}{2})
\end{align*}\index{Funzione!inversa!tangente}\index{Funzione!inversa!seno}\index{Funzione!inversa!coseno}		