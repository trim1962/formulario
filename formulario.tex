% !TeX spellcheck = it_IT
% !TeX TS-program = pdflatex
% !BIB TS-program = biber\textbackslash\{\}usepackage[big]\{layaureo\}
% !TeX encoding = UTF-8
% !TeX root = formulario.tex
\listfiles
\documentclass[a4paper,oneside]{book}%
% 21/01/2018 :: 22:12:46 :: \documentclass{book}%
\usepackage{cmap}
\frenchspacing%
\usepackage{textgreek}
%\input{../Mod_base/base}
\usepackage{amsmath}
\providecommand*{\uimm}%
%{\ensuremath{\mathrm{j}}}
{j\mkern1mu}
%\usepackage{amsthm}
\usepackage[thmmarks,hyperref]{ntheorem}
% 10/09/2017 :: 10:07:45 :: \usepackage{amsfonts}
\usepackage{amssymb}
\usepackage[italian]{babel}
\usepackage{newtxtext,newtxmath}
%\usepackage{lmodern} % load vector font
\usepackage[T1]{fontenc} % font encoding
\usepackage[utf8]{inputenc} % input encoding
%\usepackage{noto}
\usepackage[babel=true]{microtype}
%\usepackage{geometry}
\usepackage{textcomp}
%\geometry{top=1.5cm,bottom=1.5cm}
\usepackage[big]{layaureo}
%Teorema
\theoremstyle{marginbreak}
\theoremheaderfont{\normalfont\bfseries}\theorembodyfont{\slshape}
\theoremsymbol{\ensuremath{\diamondsuit}}
\theoremseparator{:}
\newtheorem{thm}{Teorema}[section]
%Proprietà
%\theoremstyle{marginbreak}
\theoremheaderfont{\normalfont\bfseries}\theorembodyfont{\slshape}
\theoremsymbol{\ensuremath{\diamondsuit}}
\theoremseparator{:}
\newtheorem{prop}{Proprietà}[section]
%lemma
%\theoremstyle{changebreak}
\theoremsymbol{\ensuremath{\heartsuit}}
\theoremindent0.5cm
\theoremnumbering{greek}
\newtheorem{lem}[thm]{Lemma}
%corollario
\theoremindent0cm
\theoremsymbol{\ensuremath{\spadesuit}}
\theoremnumbering{arabic}
\newtheorem{cor}[thm]{Corollario}
%esempio
%\theoremstyle{change}
\theorembodyfont{\upshape}
\theoremsymbol{\ensuremath{\ast}}
\theoremseparator{}
\newtheorem{exmp}{Esempio}[section]
%definizione
%\theoremstyle{plain}
\theoremsymbol{\ensuremath{\clubsuit}}
\theoremseparator{.}
%\theoremprework{\hrule\bigskip}
%\theorempostwork{\hrule\bigskip}
\newtheorem{defn}{Definizione}[section]
%commento
\theoremstyle{plain}
\theorembodyfont{\upshape}
\theoremsymbol{\ensuremath{\blacklozenge}}
\theoremseparator{:}
\newtheorem{comm}{Commento}
%dimostrazione
\theoremheaderfont{\sc}\theorembodyfont{\upshape}
\theoremstyle{nonumberplain}
\theoremseparator{:}
\theoremsymbol{\rule{1ex}{1ex}}
\newtheorem{proof}{Dimostrazione}
\usepackage{grafica}
\usepackage{matematica}
\usepackage{tabelle}
\usepackage{adjustbox}
\usepackage{copyright}
\usepackage{CDloghi}
\usepackage{stand_class}

\usepackage{pagina}
\setlength{\headheight}{13pt}
\usepackage{unita_misura}
\usepackage{indice}
\usepackage{utili}
\newcommand{\dif}[1]{\left[#1\right]}
\DeclareSIUnit\Lunghezza{L}
\DeclareSIUnit\Massa{M}
\DeclareSIUnit\Tempo{T}
\DeclareSIUnit\Corrente{I}
%%%%%%%%%%%%%%%%%%%%%%%%%%%%%%%%
%%%lunghezza arrotondamenti%%%%%
\newcommand{\lungarrotandamento}{4}
\newcommand{\extralungarrotandamento}{6}
%%%%%%%%%%%%%%%%%%%%%%%%%%%%%%%%%%
% 28/12/2017 :: 18:57:00 :: \input{../Mod_base/glossario}


 
\usepackage{imakeidx}
\makeindex[options=-s ../Mod_base/oldclaudio.sti]

\newcolumntype{H}{>{\sffamily\Large $}l<{$}}
\newcolumntype{C}{>{\sffamily $}c<{$}}
\newcolumntype{M}[1]{>{\centering}p{#1}}
\newcolumntype{F}{>{$\displaystyle }c<{$}}
\newcolumntype{L}{>{\sffamily $}l<{$}}
\newcolumntype{T}{>{\centering\arraybackslash}p{1em}} 
\newcolumntype{O}{>{\centering\arraybackslash}p{1.5em}} 
\usepackage{diagbox}
%\include{simboli_operatori}


\newcommand{\HRule}{\rule{\linewidth}{0.5mm}}
\newlength{\gnat}
\newlength{\gnam}
\newcommand{\pilH}{\rule{0pt}{2.5ex}}
\newcommand{\pilD}{\rule[-1ex]{0pt}{0pt}}
\usepackage{altrebasitabelle}
% 10/09/2017 :: 10:18:49 :: \usepackage{draftwatermark}
% 10/09/2017 :: 10:19:00 :: \SetWatermarkText{BOZZA}

\usepackage{gn-logic14}
\newenvironment{truthtable}[2][3]
{\begin{tabular}{*{#1}{c}}
		\multicolumn{1}{l}{#2}\\}
	{\end{tabular}}
\newcommand{\cport}[1]{%
	\begin{circuitikz}
		\draw (0,0) node [#1 port] {};
\end{circuitikz}} 
 \makeatletter
 \renewcommand\frontmatter{%
 	\cleardoublepage
 	\@mainmatterfalse
 	%\pagenumbering{roman}
 }
 \renewcommand\mainmatter{%
 	\cleardoublepage
 	\@mainmattertrue
 	%\pagenumbering{arabic}
 }
 \makeatother

\usepackage[grumpy,mark,markifdirty,raisemark=0.95\paperheight]{gitinfo2}
% 10/02/2018 :: 20:05:58 :: \usepackage{parskip}
\usepackage[toc,page]{appendix}

\renewcommand{\appendixtocname}{Appendici}

\renewcommand{\appendixpagename}{Appendici}

\usepackage[style=italian]{csquotes}
\usepackage[%
style=philosophy-modern,
annotation=true,
hyperref,
backend=biber,
backref]{biblatex}
\addbibresource{formulario.bib}
\usepackage[italian]{varioref}
\usepackage{hyperxmp}
\usepackage[pdfpagelabels]{hyperref}
\usepackage[italian]{cleveref}
\crefname{defn}{definizione}{definizioni}
\Crefname{defn}{Definizione}{Definizioni}
\crefname{thm}{teorema}{teoremi}
\Crefname{thm}{Teorema}{Teoremi}
\crefname{cor}{corollario}{corollari}
\Crefname{cor}{Corollario}{Corollari}
\crefname{equation}{equazione}{equazioni}
\Crefname{equation}{Equazione}{Equazioni}
\crefname{sistema}{sistema}{sistemi}
\Crefname{sistema}{Sistema}{Sistemi}
\crefname{lem}{lemma}{lemmi}
\Crefname{lem}{Lemma}{Lemmi}
\crefname{prop}{proprietà}{proprietà}
\Crefname{prop}{Proprietà}{Proprietà}
\creflabelformat{equation}{#2\textup{#1}#3}
\usepackage{tcolorboxgest}

\title{Formulario}
\author{Claudio Duchi}
\date{\datetime}
\hypersetup{%
pdfencoding=auto,
urlcolor={blue},
pdftitle={Formulario},
pdfsubject={Formulario matematico},
pdfstartview={FitH},
pdfpagemode={UseOutlines},
pdflicenseurl={http://creativecommons.org/licenses/by-nc-nd/3.0/},
pdflang={it},
pdfmetalang={it},
pdfkeywords={Algebra, geometria, analisi},
pdfcopyright={Copyright (C) 2019, Claudio Duchi},
pdfcontacturl={http://breviariomatematico.altervista.org},
pdfcontactpostcode={06128},
pdfcontactphone={},
pdfcontactemail={claduc},
pdfcontactcountry={Italy},
pdfcontactcity={Perugia},
pdfcontactaddress={},
pdfcaptionwriter={Claudio Duchi},
pdfauthortitle={},%
pdfauthor={Claudio Duchi},
linkcolor={blue},
colorlinks=true,
citecolor={red},
breaklinks,
bookmarksopen,
verbose,
baseurl={http://breviariomatematico.altervista.org}
}
%\usepackage[symbols,acronyms,translate=false,nonumberlist,toc,numberedsection,counter=chapter,automake]{glossaries-extra}
\usepackage[symbols,acronyms,nonumberlist]{glossaries-extra}
\usepackage{glossary-mcols}
\usepackage{glossary-longragged}
\usepackage{glossaries-babel}

\makeglossaries
\renewcommand*{\glsxtrpostdescgeneral}{%
	\ifglshasfield{see}{\glscurrententrylabel}
	{, \glsxtrusesee{\glscurrententrylabel}}%
	{}%
}
\setglossarystyle{altlistgroup}
\loadglsentries{../glossario/glossari1}
\includeonly{%
simboli,
alfabetogreco,
cascii,
CriteriDivisibilita,
tabprimi,
tabella,
elencofattori,
tabPitagoriche,
altrebasi,
proporzioni,
catenarapporti,
percentuale,
sconto,
geometria,
geometriasolida,
insiemi,
funzionilogiche,
Numnaturali,
Numinteri,
NumRazionali,
%aproxerrori,
algebra,	
monomi,
prodottiAlgebrici,	
triantartaglia,
scomposizionipoli,
radicali,
%equazionipg,
%equazionisg,
%equazionibq,
equazionigen,
sistemilineari,
numericomplessi,
disequazioniPgrado,
disequazioni,
disequazionifraz,
prodottoDisEqua,
angoli,	
gonimetria,
trigonometria,
RisoluzioneTriangoloRettangolo,
RisoluzioneTriangoloQualunque,
EquazioniGonimetriche,
goniocomplessi,
 genAnalitica,
 genAnaliticaretta,
 simmetrie,
parabolaAPAO,
parabolaAPAA,
parabolaRetta,
complessicartesiano,
Circonferenza,
inttersezionecirc,
intervalli,
logaritmi,
funzioni,
Funzione_exp,
funzione_log,
funzione_irrazionale,
funzione_sin,
funzione_cos,
funzione_tan, 
derivate,
limiti,
asintoti,
Tabprefissi,
cinematica,
statistica,
}

\usepackage{enumitem}
%patch allieamento lista teoremi
\usepackage{regexpatch}
\makeatletter
%\xpatchcmd*{\thm@@thmline}{2.3em}{5em}{}{} % not really needed
\xpatchcmd*{\thm@@thmline@name}{2.3em}{3em}{}{} 
\xpatchcmd*{\thm@@thmline@noname}{2.3em}{3em}{}{}
\makeatother
%fine patch allieamento lista teoremi


\begin{document}
%\setcounter{page}{2}
		\frontmatter
		%\hypersetup{pageanchor=false}
		\begin{titlepage}\parindent=0pt
			\centering
			\parbox{0.8\textwidth}{\centering
%			\begin{center}
				\Lgrandedue\\[1cm]
			\textsc{\LARGE Claudio Duchi}\\[1.2cm]
				\HRule \\[0.4cm]
				{ \huge \bfseries Formulario}\\[0.4cm]
				\HRule \\[1.2cm]
				\vfill
				\polylogo[5.5]{18}		
			{\large $-$\DTMnow$-$}	
		\end{center}
	{\centering
	Release:\gitReln\ (\gitAbbrevHash)\ Autore:\gitAuthorName\ 
	\gitCommitterDate \\
}
		\end{titlepage}
	\hypersetup{pageanchor=true}
	\CDcopyright
		\tableofcontents
	\addcontentsline{toc}{chapter}{\listfigurename}%
		\listoffigures
	\addcontentsline{toc}{chapter}{\listtablename}%
			\listoftables
		\chapter*{Elenco teoremi}
	\theoremlisttype{allname}
	Elenco teoremi:
	\listtheorems{thm}
	Elenco definizioni:
	\listtheorems{defn}
	Elenco Corollari:
	\listtheorems{cor}
	Elenco commenti:
	\listtheorems{comm}
	Elenco Lemmi:
	\listtheorems{lem}
	Elenco Proprietà
	\listtheorems{prop}
			\mainmatter
% !TeX root = formulario.tex
% 4/10/2017 :: 22:38:04 :: \part{Matematica}			
% !TeX encoding = UTF-8
% !TeX spellcheck = it_IT
% !TeX root = formulario.tex
\chapter{Simboli}
\section{Simboli matematici}
\label{sec:simbolimatematici}
%\begin{table}[H]
%\centering
\begin{center}
	\begin{tabular}{WlWl}
\toprule
\multicolumn{1}{c}{Simbolo}&\multicolumn{1}{l}{Significato}&\multicolumn{1}{c}{Simbolo}&\multicolumn{1}{l}{Significato}\\
\midrule
=&uguale&\abs{a}&valore assoluto di a\\[.25cm]
\neq&diverso& \Ni &Numeri naturali\\[.25cm]
\approx&circa&\Nz&Numeri naturali meno lo zero\\[.25cm]
<&minore&\Z&Numeri interi\\[.25cm]
>&maggiore&\Zn&Numeri interi negativi\\[.25cm]
\leq&minore o uguale&\Zp&Numeri interi positivi\\[.25cm]
\geq&maggiore o uguale&\Q&Numeri razionali\\[.25cm]
\exists&esiste&\Qn&Numeri razionali negativi\\[.25cm]
\nexists&non esiste&\Qp&Numeri razionali positivi\\[.25cm]
\forall &comunque&\R&Numeri reali\\[.25cm]
\in&appartiene&\Rneg&Numeri reali negativi\\[.25cm]
\notin&non appartiene&\Rpos&Numeri reali positivi\\[.25cm]
\propto&proporzionale&\Co&Numeri complessi\\[.25cm]
\pm&più meno&\perp&perpendicolare\\[.25cm]
\mp&meno più&\equiv&equivalente\\[.25cm]
\Longrightarrow&allora&\parallel&parallele\\[.25cm]
\Longleftrightarrow&se e solo se&\infty&infinito\\[.25cm]
\tikz{\draw[gray] (0,0)--(1,0);\fill[gray] (.5,0) circle (3pt);}&valore compreso&\tikz{\draw[gray] (0,0)--(1,0);\draw[gray] (.5,0) circle (3pt);}&valore non compreso\\
2P&perimetro&P&semiperimetro\\
I(x_0)&intorno del punto $x_0$&I(x_0,\delta)&intorno del punto $x_0$ e raggio $\delta$\\
I^{-}(x_0)&intorno sinistro di $x_0$&I^{+}(x_0)&Intorno destro di $x_0$\\
\end{tabular}
\captionof{table}{Simboli matematici}
\end{center}
\label{tab:simolimatimatici}
%\end{table}


% !TeX encoding = UTF-8
% !TeX spellcheck = it_IT

\section{Alfabeto greco}
\label{sec:AlfabetoGreco}
%\begin{table}[h!]
%\centering

{\centering\captionof{table}{Alfabeto greco}
	\begin{tabular}{lcc}
\toprule
Lettera&Minuscola&Maiuscola\\
\midrule
Alfa&$\alpha$&A\\
Beta&$\beta$&B\\
Gamma&$\gamma$&$\Gamma$\\
Delta&$\delta$&$\Delta$\\
Epsilon&$\epsilon$&E\\
Zeta&$\zeta$&Z\\
Eta&$\eta$&H\\
Teta&$\theta$&$\Theta$\\
Iota&$\iota$&I\\
Kappa&$\kappa$&K\\
Lambda&$\lambda$&$\Lambda$\\
Mi,mu&$\mu$&M\\
Ni,nu&$\nu$&N\\
Xi&$\xi$&$\Xi$\\
Omicron&o&O\\
Pi&$\pi$&$\Pi$\\
Ro&$\rho$&P\\
Sigma&$\sigma$&$\Sigma$\\
Tau&$\tau$&T\\
Upsilon&$\upsilon$&$\Upsilon$\\
Fi&$\phi$&$\Phi$\\
Chi&$\chi$&X\\
Psi&$\psi$&$\Psi$\\
Omega&$\omega$&$\Omega$\\
\bottomrule	
\end{tabular}\index{Alfabeto!greco}
\par}
%\label{Tab:alfabetogreco}
%\end{table}

\include{cascii}
\chapter{Tavole Numeriche}
\section{Criteri di divisibilità}
\label{sec:CriteridiDivisibilita}
\begin{center}
	\begin{tabular}{ccp{0.5\textwidth}}
\toprule  N&  &\multicolumn{1}{c}{Regola}   \\ 
\midrule 2 & Se & l'ultima cifra è pari, cioè è  \numlist{0;2;4;6;8} \\ 
3 & Se & la somma delle cifre è divisibile per tre.Esempio \num{375} $3+7+5=15\div3=5$ infatti $375\div 3=125$ \\ 
 4 & Se & le ultime due cifre sono divisibili per quattro o sono due zeri $\mathbf{00}$. Esempio $4\mathbf{60}$ $60\div 4=15$ $469\div 4=115$ \\
 5 & Se & l'ultima cifra è divisibile per cinque \\  
 6 & Se & è divisibile contemporaneamente per tre e per due  \\  
 8 & Se & ultime tre cifre sono divisibili per 8 o sono tre zeri $\mathbf{000}$. Esempio $9\mathbf{872}$ le ultime tre cifre sono divisibili per otto $872\div 8= 109$ $9872\div 8=1234$ \\  
 9 & Se & la somma delle cifre è divisibile per 9. Esempio $405$ $4+0+5=9$ $405\div9=45$  \\
 10 & Se & l'ultima sua cifra è zero \\
 11 & Se& la differenza della somma delle cifre di posto pari e le cifre di posto dispari è zero o si divide per undici. Esempio $25652$ $(5+5)-(2+6+2)=0$ $25652\div 11=2332$. Esempio \num{4145889} $(4+4+8+9)-(1+5+8=11)$ $4145889\div 11=376899$  \\    
 12 & Se & è divisibile contemporaneamente per tre e per quattro  \\  
 25 & Se & il numero  formato dalle ultime due cifre è divisibile per venticinque\\
\bottomrule
\end{tabular}
\end{center}





% !TeX encoding = UTF-8
% !TeX spellcheck = it_IT

\section{Numeri primi}
\label{sec:TabellaNumeriPrrimi}
%\begin{table}[H]
%\centering
{\centering\captionof{table}{Numeri primi}\index{Numero!primo}
	\begin{tabular}{llllllllll}
\toprule
1 &2 &3 &5 &7 &11 &13 &17 &19 &23 \\
29 &31 &37 &41 &43 &47 &53 &59 &61 &67 \\
71 &73 &79 &83 &89 &97 &101 &103 &107 &109 \\
113 &127 &131 &137 &139 &149 &151 &157 &163 &167 \\
173 &179 &181 &191 &193 &197 &199 &211 &223 &227 \\
229 &233 &239 &241 &251 &257 &263 &269 &271 &277 \\
281 &283 &293 &307 &311 &313 &317 &331 &337 &347 \\
349 &353 &359 &367 &373 &379 &383 &389 &397 &401 \\
409 &419 &421 &431 &433 &439 &443 &449 &457 &461 \\
463 &467 &479 &487 &491 &499 &503 &509 &521 &523 \\
541 &547 &557 &563 &569 &571 &577 &587 &593 &599 \\
601 &607 &613 &617 &619 &631 &641 &643 &647 &653 \\
659 &661 &673 &677 &683 &691 &701 &709 &719 &727 \\
733 &739 &743 &751 &757 &761 &769 &773 &787 &797 \\
809 &811 &821 &823 &827 &829 &839 &853 &857 &859 \\
863 &877 &881 &883 &887 &907 &911 &919 &929 &937 \\
941 &947 &953 &967 &971 &977 &983 &991 &997 &1009 \\
1013 &1019 &1021 &1031 &1033 &1039 &1049 &1051 &1061 &1063 \\
1069 &1087 &1091 &1093 &1097 &1103 &1109 &1117 &1123 &1129 \\
1151 &1153 &1163 &1171 &1181 &1187 &1193 &1201 &1213 &1217 \\
1223 &1229 &1231 &1237 &1249 &1259 &1277 &1279 &1283 &1289 \\
1291 &1297 &1301 &1303 &1307 &1319 &1321 &1327 &1361 &1367 \\
1373 &1381 &1399 &1409 &1423 &1427 &1429 &1433 &1439 &1447 \\
1451 &1453 &1459 &1471 &1481 &1483 &1487 &1489 &1493 &1499 \\
1511 &1523 &1531 &1543 &1549 &1553 &1559 &1567 &1571 &1579 \\
1583 &1597 &1601 &1607 &1609 &1613 &1619 &1621 &1627 &1637 \\
1657 &1663 &1667 &1669 &1693 &1697 &1699 &1709 &1721 &1723 \\
1733 &1741 &1747 &1753 &1759 &1777 &1783 &1787 &1789 &1801 \\
1811 &1823 &1831 &1847 &1861 &1867 &1871 &1873 &1877 &1879 \\
1889 &1901 &1907 &1913 &1931 &1933 &1949 &1951 &1973 &1979 \\
1987 &1993 &1997 &1999 &2003 &2011 &2017 &2027 &2029 &2039 \\
2053 &2063 &2069 &2081 &2083 &2087 &2089 &2099 &2111 &2113 \\
2129 &2131 &2137 &2141 &2143 &2153 &2161 &2179 &2203 &2207 \\
2213 &2221 &2237 &2239 &2243 &2251 &2267 &2269 &2273 &2281 \\
2287 &2293 &2297 &2309 &2311 &2333 &2339 &2341 &2347 &2351 \\
2357 &2371 &2377 &2381 &2383 &2389 &2393 &2399 &2411 &2417 \\
2423 &2437 &2441 &2447 &2459 &2467 &2473 &2477 &2503 &2521 \\
2531 &2539 &2543 &2549 &2551 &2557 &2579 &2591 &2593 &2609 \\
2617 &2621 &2633 &2647 &2657 &2659 &2663 &2671 &2677 &2683 \\
2687 &2689 &2693 &2699 &2707 &2711 &2713 &2719 &2729 &2731 \\
2741 &2749 &2753 &2767 &2777 &2789 &2791 &2797 &2801 &2803 \\
\bottomrule
%2819 &2833 &2837 &2843 &2851 &2857 &2861 &2879 &2887 &2897 \\
\end{tabular}\index{Tabella!numeri primi}
\par}
%\end{table}
	
% !TeX encoding = UTF-8
% !TeX spellcheck = it_IT

\section{Tabella quadrati cubi radici}
\label{sec:Tabellaquadraticubiradici}
%\begin{longtable}{rrrrrrrrrrr} 
\begin{longtable}{SSSSSSSSSSS} 
	\toprule
	\bfseries {$n$} &  {$n^2$} & {$n^3$}&{$\sqrt{n}$}&{$\sqrt[3]{n}$}&{ }&{$n$} &  {$n^2$} & {$n^3$}&{$\sqrt{n}$}&{$\sqrt[3]{n}$}  \\
	\midrule \endhead
	\bottomrule \endfoot\index{Tabella!quadrati}\index{Tabella!cubi}\index{Tabella!radici}
1&1&1&1,0000&1,0000&&51&2601&132651&7,1414&3,7084\\
2&4&8&1,4142&1,2599&&52&2704&140608&7,2111&3,7325\\
3&9&27&1,7321&1,4422&&53&2809&148877&7,2801&3,7563\\
4&16&64&2,0000&1,5874&&54&2916&157464&7,3485&3,7798\\
5&25&125&2,2361&1,7100&&55&3025&166375&7,4162&3,8030\\
6&36&216&2,4495&1,8171&&56&3136&175616&7,4833&3,8259\\
7&49&343&2,6458&1,9129&&57&3249&185193&7,5498&3,8485\\
8&64&512&2,8284&2,0000&&58&3364&195112&7,6158&3,8709\\
9&81&729&3,0000&2,0801&&59&3481&205379&7,6811&3,8930\\
10&100&1000&3,1623&2,1544&&60&3600&216000&7,7460&3,9149\\
11&121&1331&3,3166&2,2240&&61&3721&226981&7,8102&3,9365\\
12&144&1728&3,4641&2,2894&&62&3844&238328&7,8740&3,9579\\
13&169&2197&3,6056&2,3513&&63&3969&250047&7,9373&3,9791\\
14&196&2744&3,7417&2,4101&&64&4096&262144&8,0000&4,0000\\
15&225&3375&3,8730&2,4662&&65&4225&274625&8,0623&4,0207\\
16&256&4096&4,0000&2,5198&&66&4356&287496&8,1240&4,0412\\
17&289&4913&4,1231&2,5713&&67&4489&300763&8,1854&4,0615\\
18&324&5832&4,2426&2,6207&&68&4624&314432&8,2462&4,0817\\
19&361&6859&4,3589&2,6684&&69&4761&328509&8,3066&4,1016\\
20&400&8000&4,4721&2,7144&&70&4900&343000&8,3666&4,1213\\
21&441&9261&4,5826&2,7589&&71&5041&357911&8,4261&4,1408\\
22&484&10648&4,6904&2,8020&&72&5184&373248&8,4853&4,1602\\
23&529&12167&4,7958&2,8439&&73&5329&389017&8,5440&4,1793\\
24&576&13824&4,8990&2,8845&&74&5476&405224&8,6023&4,1983\\
25&625&15625&5,0000&2,9240&&75&5625&421875&8,6603&4,2172\\
26&676&17576&5,0990&2,9625&&76&5776&438976&8,7178&4,2358\\
27&729&19683&5,1962&3,0000&&77&5929&456533&8,7750&4,2543\\
28&784&21952&5,2915&3,0366&&78&6084&474552&8,8318&4,2727\\
29&841&24389&5,3852&3,0723&&79&6241&493039&8,8882&4,2908\\
30&900&27000&5,4772&3,1072&&80&6400&512000&8,9443&4,3089\\
31&961&29791&5,5678&3,1414&&81&6561&531441&9,0000&4,3267\\
32&1024&32768&5,6569&3,1748&&82&6724&551368&9,0554&4,3445\\
33&1089&35937&5,7446&3,2075&&83&6889&571787&9,1104&4,3621\\
34&1156&39304&5,8310&3,2396&&84&7056&592704&9,1652&4,3795\\
35&1225&42875&5,9161&3,2711&&85&7225&614125&9,2195&4,3968\\
36&1296&46656&6,0000&3,3019&&86&7396&636056&9,2736&4,4140\\
37&1369&50653&6,0828&3,3322&&87&7569&658503&9,3274&4,4310\\
38&1444&54872&6,1644&3,3620&&88&7744&681472&9,3808&4,4480\\
39&1521&59319&6,2450&3,3912&&89&7921&704969&9,4340&4,4647\\
40&1600&64000&6,3246&3,4200&&90&8100&729000&9,4868&4,4814\\
41&1681&68921&6,4031&3,4482&&91&8281&753571&9,5394&4,4979\\
42&1764&74088&6,4807&3,4760&&92&8464&778688&9,5917&4,5144\\
43&1849&79507&6,5574&3,5034&&93&8649&804357&9,6437&4,5307\\
44&1936&85184&6,6332&3,5303&&94&8836&830584&9,6954&4,5468\\
45&2025&91125&6,7082&3,5569&&95&9025&857375&9,7468&4,5629\\
46&2116&97336&6,7823&3,5830&&96&9216&884736&9,7980&4,5789\\
47&2209&103823&6,8557&3,6088&&97&9409&912673&9,8489&4,5947\\
48&2304&110592&6,9282&3,6342&&98&9604&941192&9,8995&4,6104\\
49&2401&117649&7,0000&3,6593&&99&9801&970299&9,9499&4,6261\\
50&2500&125000&7,0711&3,6840&&100&10000&1000000&10,0000&4,6416\\
\newpage
101&10201&1030301&10,0499&4,6570&&151&22801&3442951&12,2882&5,3251\\
102&10404&1061208&10,0995&4,6723&&152&23104&3511808&12,3288&5,3368\\
103&10609&1092727&10,1489&4,6875&&153&23409&3581577&12,3693&5,3485\\
104&10816&1124864&10,1980&4,7027&&154&23716&3652264&12,4097&5,3601\\
105&11025&1157625&10,2470&4,7177&&155&24025&3723875&12,4499&5,3717\\
106&11236&1191016&10,2956&4,7326&&156&24336&3796416&12,4900&5,3832\\
107&11449&1225043&10,3441&4,7475&&157&24649&3869893&12,5300&5,3947\\
108&11664&1259712&10,3923&4,7622&&158&24964&3944312&12,5698&5,4061\\
109&11881&1295029&10,4403&4,7769&&159&25281&4019679&12,6095&5,4175\\
110&12100&1331000&10,4881&4,7914&&160&25600&4096000&12,6491&5,4288\\
111&12321&1367631&10,5357&4,8059&&161&25921&4173281&12,6886&5,4401\\
112&12544&1404928&10,5830&4,8203&&162&26244&4251528&12,7279&5,4514\\
113&12769&1442897&10,6301&4,8346&&163&26569&4330747&12,7671&5,4626\\
114&12996&1481544&10,6771&4,8488&&164&26896&4410944&12,8062&5,4737\\
115&13225&1520875&10,7238&4,8629&&165&27225&4492125&12,8452&5,4848\\
116&13456&1560896&10,7703&4,8770&&166&27556&4574296&12,8841&5,4959\\
117&13689&1601613&10,8167&4,8910&&167&27889&4657463&12,9228&5,5069\\
118&13924&1643032&10,8628&4,9049&&168&28224&4741632&12,9615&5,5178\\
119&14161&1685159&10,9087&4,9187&&169&28561&4826809&13,0000&5,5288\\
120&14400&1728000&10,9545&4,9324&&170&28900&4913000&13,0384&5,5397\\
121&14641&1771561&11,0000&4,9461&&171&29241&5000211&13,0767&5,5505\\
122&14884&1815848&11,0454&4,9597&&172&29584&5088448&13,1149&5,5613\\
123&15129&1860867&11,0905&4,9732&&173&29929&5177717&13,1529&5,5721\\
124&15376&1906624&11,1355&4,9866&&174&30276&5268024&13,1909&5,5828\\
125&15625&1953125&11,1803&5,0000&&175&30625&5359375&13,2288&5,5934\\
126&15876&2000376&11,2250&5,0133&&176&30976&5451776&13,2665&5,6041\\
127&16129&2048383&11,2694&5,0265&&177&31329&5545233&13,3041&5,6147\\
128&16384&2097152&11,3137&5,0397&&178&31684&5639752&13,3417&5,6252\\
129&16641&2146689&11,3578&5,0528&&179&32041&5735339&13,3791&5,6357\\
130&16900&2197000&11,4018&5,0658&&180&32400&5832000&13,4164&5,6462\\
131&17161&2248091&11,4455&5,0788&&181&32761&5929741&13,4536&5,6567\\
132&17424&2299968&11,4891&5,0916&&182&33124&6028568&13,4907&5,6671\\
133&17689&2352637&11,5326&5,1045&&183&33489&6128487&13,5277&5,6774\\
134&17956&2406104&11,5758&5,1172&&184&33856&6229504&13,5647&5,6877\\
135&18225&2460375&11,6190&5,1299&&185&34225&6331625&13,6015&5,6980\\
136&18496&2515456&11,6619&5,1426&&186&34596&6434856&13,6382&5,7083\\
137&18769&2571353&11,7047&5,1551&&187&34969&6539203&13,6748&5,7185\\
138&19044&2628072&11,7473&5,1676&&188&35344&6644672&13,7113&5,7287\\
139&19321&2685619&11,7898&5,1801&&189&35721&6751269&13,7477&5,7388\\
140&19600&2744000&11,8322&5,1925&&190&36100&6859000&13,7840&5,7489\\
141&19881&2803221&11,8743&5,2048&&191&36481&6967871&13,8203&5,7590\\
142&20164&2863288&11,9164&5,2171&&192&36864&7077888&13,8564&5,7690\\
143&20449&2924207&11,9583&5,2293&&193&37249&7189057&13,8924&5,7790\\
144&20736&2985984&12,0000&5,2415&&194&37636&7301384&13,9284&5,7890\\
145&21025&3048625&12,0416&5,2536&&195&38025&7414875&13,9642&5,7989\\
146&21316&3112136&12,0830&5,2656&&196&38416&7529536&14,0000&5,8088\\
147&21609&3176523&12,1244&5,2776&&197&38809&7645373&14,0357&5,8186\\
148&21904&3241792&12,1655&5,2896&&198&39204&7762392&14,0712&5,8285\\
149&22201&3307949&12,2066&5,3015&&199&39601&7880599&14,1067&5,8383\\
150&22500&3375000&12,2474&5,3133&&200&40000&8000000&14,1421&5,8480\\
\newpage
201&40401&8120601&14,1774&5,8578&&251&63001&15813251&15,8430&6,3080\\
202&40804&8242408&14,2127&5,8675&&252&63504&16003008&15,8745&6,3164\\
203&41209&8365427&14,2478&5,8771&&253&64009&16194277&15,9060&6,3247\\
204&41616&8489664&14,2829&5,8868&&254&64516&16387064&15,9374&6,3330\\
205&42025&8615125&14,3178&5,8964&&255&65025&16581375&15,9687&6,3413\\
206&42436&8741816&14,3527&5,9059&&256&65536&16777216&16,0000&6,3496\\
207&42849&8869743&14,3875&5,9155&&257&66049&16974593&16,0312&6,3579\\
208&43264&8998912&14,4222&5,9250&&258&66564&17173512&16,0624&6,3661\\
209&43681&9129329&14,4568&5,9345&&259&67081&17373979&16,0935&6,3743\\
210&44100&9261000&14,4914&5,9439&&260&67600&17576000&16,1245&6,3825\\
211&44521&9393931&14,5258&5,9533&&261&68121&17779581&16,1555&6,3907\\
212&44944&9528128&14,5602&5,9627&&262&68644&17984728&16,1864&6,3988\\
213&45369&9663597&14,5945&5,9721&&263&69169&18191447&16,2173&6,4070\\
214&45796&9800344&14,6287&5,9814&&264&69696&18399744&16,2481&6,4151\\
215&46225&9938375&14,6629&5,9907&&265&70225&18609625&16,2788&6,4232\\
216&46656&10077696&14,6969&6,0000&&266&70756&18821096&16,3095&6,4312\\
217&47089&10218313&14,7309&6,0092&&267&71289&19034163&16,3401&6,4393\\
218&47524&10360232&14,7648&6,0185&&268&71824&19248832&16,3707&6,4473\\
219&47961&10503459&14,7986&6,0277&&269&72361&19465109&16,4012&6,4553\\
220&48400&10648000&14,8324&6,0368&&270&72900&19683000&16,4317&6,4633\\
221&48841&10793861&14,8661&6,0459&&271&73441&19902511&16,4621&6,4713\\
222&49284&10941048&14,8997&6,0550&&272&73984&20123648&16,4924&6,4792\\
223&49729&11089567&14,9332&6,0641&&273&74529&20346417&16,5227&6,4872\\
224&50176&11239424&14,9666&6,0732&&274&75076&20570824&16,5529&6,4951\\
225&50625&11390625&15,0000&6,0822&&275&75625&20796875&16,5831&6,5030\\
226&51076&11543176&15,0333&6,0912&&276&76176&21024576&16,6132&6,5108\\
227&51529&11697083&15,0665&6,1002&&277&76729&21253933&16,6433&6,5187\\
228&51984&11852352&15,0997&6,1091&&278&77284&21484952&16,6733&6,5265\\
229&52441&12008989&15,1327&6,1180&&279&77841&21717639&16,7033&6,5343\\
230&52900&12167000&15,1658&6,1269&&280&78400&21952000&16,7332&6,5421\\
231&53361&12326391&15,1987&6,1358&&281&78961&22188041&16,7631&6,5499\\
232&53824&12487168&15,2315&6,1446&&282&79524&22425768&16,7929&6,5577\\
233&54289&12649337&15,2643&6,1534&&283&80089&22665187&16,8226&6,5654\\
234&54756&12812904&15,2971&6,1622&&284&80656&22906304&16,8523&6,5731\\
235&55225&12977875&15,3297&6,1710&&285&81225&23149125&16,8819&6,5808\\
236&55696&13144256&15,3623&6,1797&&286&81796&23393656&16,9115&6,5885\\
237&56169&13312053&15,3948&6,1885&&287&82369&23639903&16,9411&6,5962\\
238&56644&13481272&15,4272&6,1972&&288&82944&23887872&16,9706&6,6039\\
239&57121&13651919&15,4596&6,2058&&289&83521&24137569&17,0000&6,6115\\
240&57600&13824000&15,4919&6,2145&&290&84100&24389000&17,0294&6,6191\\
241&58081&13997521&15,5242&6,2231&&291&84681&24642171&17,0587&6,6267\\
242&58564&14172488&15,5563&6,2317&&292&85264&24897088&17,0880&6,6343\\
243&59049&14348907&15,5885&6,2403&&293&85849&25153757&17,1172&6,6419\\
244&59536&14526784&15,6205&6,2488&&294&86436&25412184&17,1464&6,6494\\
245&60025&14706125&15,6525&6,2573&&295&87025&25672375&17,1756&6,6569\\
246&60516&14886936&15,6844&6,2658&&296&87616&25934336&17,2047&6,6644\\
247&61009&15069223&15,7162&6,2743&&297&88209&26198073&17,2337&6,6719\\
248&61504&15252992&15,7480&6,2828&&298&88804&26463592&17,2627&6,6794\\
249&62001&15438249&15,7797&6,2912&&299&89401&26730899&17,2916&6,6869\\
250&62500&15625000&15,8114&6,2996&&300&90000&27000000&17,3205&6,6943\\
\newpage
301&90601&27270901&17,3494&6,7018&&351&123201&43243551&18,7350&7,0540\\
302&91204&27543608&17,3781&6,7092&&352&123904&43614208&18,7617&7,0607\\
303&91809&27818127&17,4069&6,7166&&353&124609&43986977&18,7883&7,0674\\
304&92416&28094464&17,4356&6,7240&&354&125316&44361864&18,8149&7,0740\\
305&93025&28372625&17,4642&6,7313&&355&126025&44738875&18,8414&7,0807\\
306&93636&28652616&17,4929&6,7387&&356&126736&45118016&18,8680&7,0873\\
307&94249&28934443&17,5214&6,7460&&357&127449&45499293&18,8944&7,0940\\
308&94864&29218112&17,5499&6,7533&&358&128164&45882712&18,9209&7,1006\\
309&95481&29503629&17,5784&6,7606&&359&128881&46268279&18,9473&7,1072\\
310&96100&29791000&17,6068&6,7679&&360&129600&46656000&18,9737&7,1138\\
311&96721&30080231&17,6352&6,7752&&361&130321&47045881&19,0000&7,1204\\
312&97344&30371328&17,6635&6,7824&&362&131044&47437928&19,0263&7,1269\\
313&97969&30664297&17,6918&6,7897&&363&131769&47832147&19,0526&7,1335\\
314&98596&30959144&17,7200&6,7969&&364&132496&48228544&19,0788&7,1400\\
315&99225&31255875&17,7482&6,8041&&365&133225&48627125&19,1050&7,1466\\
316&99856&31554496&17,7764&6,8113&&366&133956&49027896&19,1311&7,1531\\
317&100489&31855013&17,8045&6,8185&&367&134689&49430863&19,1572&7,1596\\
318&101124&32157432&17,8326&6,8256&&368&135424&49836032&19,1833&7,1661\\
319&101761&32461759&17,8606&6,8328&&369&136161&50243409&19,2094&7,1726\\
320&102400&32768000&17,8885&6,8399&&370&136900&50653000&19,2354&7,1791\\
321&103041&33076161&17,9165&6,8470&&371&137641&51064811&19,2614&7,1855\\
322&103684&33386248&17,9444&6,8541&&372&138384&51478848&19,2873&7,1920\\
323&104329&33698267&17,9722&6,8612&&373&139129&51895117&19,3132&7,1984\\
324&104976&34012224&18,0000&6,8683&&374&139876&52313624&19,3391&7,2048\\
325&105625&34328125&18,0278&6,8753&&375&140625&52734375&19,3649&7,2112\\
326&106276&34645976&18,0555&6,8824&&376&141376&53157376&19,3907&7,2177\\
327&106929&34965783&18,0831&6,8894&&377&142129&53582633&19,4165&7,2240\\
328&107584&35287552&18,1108&6,8964&&378&142884&54010152&19,4422&7,2304\\
329&108241&35611289&18,1384&6,9034&&379&143641&54439939&19,4679&7,2368\\
330&108900&35937000&18,1659&6,9104&&380&144400&54872000&19,4936&7,2432\\
331&109561&36264691&18,1934&6,9174&&381&145161&55306341&19,5192&7,2495\\
332&110224&36594368&18,2209&6,9244&&382&145924&55742968&19,5448&7,2558\\
333&110889&36926037&18,2483&6,9313&&383&146689&56181887&19,5704&7,2622\\
334&111556&37259704&18,2757&6,9382&&384&147456&56623104&19,5959&7,2685\\
335&112225&37595375&18,3030&6,9451&&385&148225&57066625&19,6214&7,2748\\
336&112896&37933056&18,3303&6,9521&&386&148996&57512456&19,6469&7,2811\\
337&113569&38272753&18,3576&6,9589&&387&149769&57960603&19,6723&7,2874\\
338&114244&38614472&18,3848&6,9658&&388&150544&58411072&19,6977&7,2936\\
339&114921&38958219&18,4120&6,9727&&389&151321&58863869&19,7231&7,2999\\
340&115600&39304000&18,4391&6,9795&&390&152100&59319000&19,7484&7,3061\\
341&116281&39651821&18,4662&6,9864&&391&152881&59776471&19,7737&7,3124\\
342&116964&40001688&18,4932&6,9932&&392&153664&60236288&19,7990&7,3186\\
343&117649&40353607&18,5203&7,0000&&393&154449&60698457&19,8242&7,3248\\
344&118336&40707584&18,5472&7,0068&&394&155236&61162984&19,8494&7,3310\\
345&119025&41063625&18,5742&7,0136&&395&156025&61629875&19,8746&7,3372\\
346&119716&41421736&18,6011&7,0203&&396&156816&62099136&19,8997&7,3434\\
347&120409&41781923&18,6279&7,0271&&397&157609&62570773&19,9249&7,3496\\
348&121104&42144192&18,6548&7,0338&&398&158404&63044792&19,9499&7,3558\\
349&121801&42508549&18,6815&7,0406&&399&159201&63521199&19,9750&7,3619\\
350&122500&42875000&18,7083&7,0473&&400&160000&64000000&20,0000&7,3681\\
\newpage
401&160801&64481201&20,0250&7,3742&&451&203401&91733851&21,2368&7,6688\\
402&161604&64964808&20,0499&7,3803&&452&204304&92345408&21,2603&7,6744\\
403&162409&65450827&20,0749&7,3864&&453&205209&92959677&21,2838&7,6801\\
404&163216&65939264&20,0998&7,3925&&454&206116&93576664&21,3073&7,6857\\
405&164025&66430125&20,1246&7,3986&&455&207025&94196375&21,3307&7,6914\\
406&164836&66923416&20,1494&7,4047&&456&207936&94818816&21,3542&7,6970\\
407&165649&67419143&20,1742&7,4108&&457&208849&95443993&21,3776&7,7026\\
408&166464&67917312&20,1990&7,4169&&458&209764&96071912&21,4009&7,7082\\
409&167281&68417929&20,2237&7,4229&&459&210681&96702579&21,4243&7,7138\\
410&168100&68921000&20,2485&7,4290&&460&211600&97336000&21,4476&7,7194\\
411&168921&69426531&20,2731&7,4350&&461&212521&97972181&21,4709&7,7250\\
412&169744&69934528&20,2978&7,4410&&462&213444&98611128&21,4942&7,7306\\
413&170569&70444997&20,3224&7,4470&&463&214369&99252847&21,5174&7,7362\\
414&171396&70957944&20,3470&7,4530&&464&215296&99897344&21,5407&7,7418\\
415&172225&71473375&20,3715&7,4590&&465&216225&100544625&21,5639&7,7473\\
416&173056&71991296&20,3961&7,4650&&466&217156&101194696&21,5870&7,7529\\
417&173889&72511713&20,4206&7,4710&&467&218089&101847563&21,6102&7,7584\\
418&174724&73034632&20,4450&7,4770&&468&219024&102503232&21,6333&7,7639\\
419&175561&73560059&20,4695&7,4829&&469&219961&103161709&21,6564&7,7695\\
420&176400&74088000&20,4939&7,4889&&470&220900&103823000&21,6795&7,7750\\
421&177241&74618461&20,5183&7,4948&&471&221841&104487111&21,7025&7,7805\\
422&178084&75151448&20,5426&7,5007&&472&222784&105154048&21,7256&7,7860\\
423&178929&75686967&20,5670&7,5067&&473&223729&105823817&21,7486&7,7915\\
424&179776&76225024&20,5913&7,5126&&474&224676&106496424&21,7715&7,7970\\
425&180625&76765625&20,6155&7,5185&&475&225625&107171875&21,7945&7,8025\\
426&181476&77308776&20,6398&7,5244&&476&226576&107850176&21,8174&7,8079\\
427&182329&77854483&20,6640&7,5302&&477&227529&108531333&21,8403&7,8134\\
428&183184&78402752&20,6882&7,5361&&478&228484&109215352&21,8632&7,8188\\
429&184041&78953589&20,7123&7,5420&&479&229441&109902239&21,8861&7,8243\\
430&184900&79507000&20,7364&7,5478&&480&230400&110592000&21,9089&7,8297\\
431&185761&80062991&20,7605&7,5537&&481&231361&111284641&21,9317&7,8352\\
432&186624&80621568&20,7846&7,5595&&482&232324&111980168&21,9545&7,8406\\
433&187489&81182737&20,8087&7,5654&&483&233289&112678587&21,9773&7,8460\\
434&188356&81746504&20,8327&7,5712&&484&234256&113379904&22,0000&7,8514\\
435&189225&82312875&20,8567&7,5770&&485&235225&114084125&22,0227&7,8568\\
436&190096&82881856&20,8806&7,5828&&486&236196&114791256&22,0454&7,8622\\
437&190969&83453453&20,9045&7,5886&&487&237169&115501303&22,0681&7,8676\\
438&191844&84027672&20,9284&7,5944&&488&238144&116214272&22,0907&7,8730\\
439&192721&84604519&20,9523&7,6001&&489&239121&116930169&22,1133&7,8784\\
440&193600&85184000&20,9762&7,6059&&490&240100&117649000&22,1359&7,8837\\
441&194481&85766121&21,0000&7,6117&&491&241081&118370771&22,1585&7,8891\\
442&195364&86350888&21,0238&7,6174&&492&242064&119095488&22,1811&7,8944\\
443&196249&86938307&21,0476&7,6232&&493&243049&119823157&22,2036&7,8998\\
444&197136&87528384&21,0713&7,6289&&494&244036&120553784&22,2261&7,9051\\
445&198025&88121125&21,0950&7,6346&&495&245025&121287375&22,2486&7,9105\\
446&198916&88716536&21,1187&7,6403&&496&246016&122023936&22,2711&7,9158\\
447&199809&89314623&21,1424&7,6460&&497&247009&122763473&22,2935&7,9211\\
448&200704&89915392&21,1660&7,6517&&498&248004&123505992&22,3159&7,9264\\
449&201601&90518849&21,1896&7,6574&&499&249001&124251499&22,3383&7,9317\\
450&202500&91125000&21,2132&7,6631&&500&250000&125000000&22,3607&7,9370\\
\newpage
501&251001&125751501&22,3830&7,9423&&551&303601&167284151&23,4734&8,1982\\
502&252004&126506008&22,4054&7,9476&&552&304704&168196608&23,4947&8,2031\\
503&253009&127263527&22,4277&7,9528&&553&305809&169112377&23,5160&8,2081\\
504&254016&128024064&22,4499&7,9581&&554&306916&170031464&23,5372&8,2130\\
505&255025&128787625&22,4722&7,9634&&555&308025&170953875&23,5584&8,2180\\
506&256036&129554216&22,4944&7,9686&&556&309136&171879616&23,5797&8,2229\\
507&257049&130323843&22,5167&7,9739&&557&310249&172808693&23,6008&8,2278\\
508&258064&131096512&22,5389&7,9791&&558&311364&173741112&23,6220&8,2327\\
509&259081&131872229&22,5610&7,9843&&559&312481&174676879&23,6432&8,2377\\
510&260100&132651000&22,5832&7,9896&&560&313600&175616000&23,6643&8,2426\\
511&261121&133432831&22,6053&7,9948&&561&314721&176558481&23,6854&8,2475\\
512&262144&134217728&22,6274&8,0000&&562&315844&177504328&23,7065&8,2524\\
513&263169&135005697&22,6495&8,0052&&563&316969&178453547&23,7276&8,2573\\
514&264196&135796744&22,6716&8,0104&&564&318096&179406144&23,7487&8,2621\\
515&265225&136590875&22,6936&8,0156&&565&319225&180362125&23,7697&8,2670\\
516&266256&137388096&22,7156&8,0208&&566&320356&181321496&23,7908&8,2719\\
517&267289&138188413&22,7376&8,0260&&567&321489&182284263&23,8118&8,2768\\
518&268324&138991832&22,7596&8,0311&&568&322624&183250432&23,8328&8,2816\\
519&269361&139798359&22,7816&8,0363&&569&323761&184220009&23,8537&8,2865\\
520&270400&140608000&22,8035&8,0415&&570&324900&185193000&23,8747&8,2913\\
521&271441&141420761&22,8254&8,0466&&571&326041&186169411&23,8956&8,2962\\
522&272484&142236648&22,8473&8,0517&&572&327184&187149248&23,9165&8,3010\\
523&273529&143055667&22,8692&8,0569&&573&328329&188132517&23,9374&8,3059\\
524&274576&143877824&22,8910&8,0620&&574&329476&189119224&23,9583&8,3107\\
525&275625&144703125&22,9129&8,0671&&575&330625&190109375&23,9792&8,3155\\
526&276676&145531576&22,9347&8,0723&&576&331776&191102976&24,0000&8,3203\\
527&277729&146363183&22,9565&8,0774&&577&332929&192100033&24,0208&8,3251\\
528&278784&147197952&22,9783&8,0825&&578&334084&193100552&24,0416&8,3300\\
529&279841&148035889&23,0000&8,0876&&579&335241&194104539&24,0624&8,3348\\
530&280900&148877000&23,0217&8,0927&&580&336400&195112000&24,0832&8,3396\\
531&281961&149721291&23,0434&8,0978&&581&337561&196122941&24,1039&8,3443\\
532&283024&150568768&23,0651&8,1028&&582&338724&197137368&24,1247&8,3491\\
533&284089&151419437&23,0868&8,1079&&583&339889&198155287&24,1454&8,3539\\
534&285156&152273304&23,1084&8,1130&&584&341056&199176704&24,1661&8,3587\\
535&286225&153130375&23,1301&8,1180&&585&342225&200201625&24,1868&8,3634\\
536&287296&153990656&23,1517&8,1231&&586&343396&201230056&24,2074&8,3682\\
537&288369&154854153&23,1733&8,1281&&587&344569&202262003&24,2281&8,3730\\
538&289444&155720872&23,1948&8,1332&&588&345744&203297472&24,2487&8,3777\\
539&290521&156590819&23,2164&8,1382&&589&346921&204336469&24,2693&8,3825\\
540&291600&157464000&23,2379&8,1433&&590&348100&205379000&24,2899&8,3872\\
541&292681&158340421&23,2594&8,1483&&591&349281&206425071&24,3105&8,3919\\
542&293764&159220088&23,2809&8,1533&&592&350464&207474688&24,3311&8,3967\\
543&294849&160103007&23,3024&8,1583&&593&351649&208527857&24,3516&8,4014\\
544&295936&160989184&23,3238&8,1633&&594&352836&209584584&24,3721&8,4061\\
545&297025&161878625&23,3452&8,1683&&595&354025&210644875&24,3926&8,4108\\
546&298116&162771336&23,3666&8,1733&&596&355216&211708736&24,4131&8,4155\\
547&299209&163667323&23,3880&8,1783&&597&356409&212776173&24,4336&8,4202\\
548&300304&164566592&23,4094&8,1833&&598&357604&213847192&24,4540&8,4249\\
549&301401&165469149&23,4307&8,1882&&599&358801&214921799&24,4745&8,4296\\
550&302500&166375000&23,4521&8,1932&&600&360000&216000000&24,4949&8,4343\\
\newpage
601&361201&217081801&24,5153&8,4390&&651&423801&275894451&25,5147&8,6668\\
602&362404&218167208&24,5357&8,4437&&652&425104&277167808&25,5343&8,6713\\
603&363609&219256227&24,5561&8,4484&&653&426409&278445077&25,5539&8,6757\\
604&364816&220348864&24,5764&8,4530&&654&427716&279726264&25,5734&8,6801\\
605&366025&221445125&24,5967&8,4577&&655&429025&281011375&25,5930&8,6845\\
606&367236&222545016&24,6171&8,4623&&656&430336&282300416&25,6125&8,6890\\
607&368449&223648543&24,6374&8,4670&&657&431649&283593393&25,6320&8,6934\\
608&369664&224755712&24,6577&8,4716&&658&432964&284890312&25,6515&8,6978\\
609&370881&225866529&24,6779&8,4763&&659&434281&286191179&25,6710&8,7022\\
610&372100&226981000&24,6982&8,4809&&660&435600&287496000&25,6905&8,7066\\
611&373321&228099131&24,7184&8,4856&&661&436921&288804781&25,7099&8,7110\\
612&374544&229220928&24,7386&8,4902&&662&438244&290117528&25,7294&8,7154\\
613&375769&230346397&24,7588&8,4948&&663&439569&291434247&25,7488&8,7198\\
614&376996&231475544&24,7790&8,4994&&664&440896&292754944&25,7682&8,7241\\
615&378225&232608375&24,7992&8,5040&&665&442225&294079625&25,7876&8,7285\\
616&379456&233744896&24,8193&8,5086&&666&443556&295408296&25,8070&8,7329\\
617&380689&234885113&24,8395&8,5132&&667&444889&296740963&25,8263&8,7373\\
618&381924&236029032&24,8596&8,5178&&668&446224&298077632&25,8457&8,7416\\
619&383161&237176659&24,8797&8,5224&&669&447561&299418309&25,8650&8,7460\\
620&384400&238328000&24,8998&8,5270&&670&448900&300763000&25,8844&8,7503\\
621&385641&239483061&24,9199&8,5316&&671&450241&302111711&25,9037&8,7547\\
622&386884&240641848&24,9399&8,5362&&672&451584&303464448&25,9230&8,7590\\
623&388129&241804367&24,9600&8,5408&&673&452929&304821217&25,9422&8,7634\\
624&389376&242970624&24,9800&8,5453&&674&454276&306182024&25,9615&8,7677\\
625&390625&244140625&25,0000&8,5499&&675&455625&307546875&25,9808&8,7721\\
626&391876&245314376&25,0200&8,5544&&676&456976&308915776&26,0000&8,7764\\
627&393129&246491883&25,0400&8,5590&&677&458329&310288733&26,0192&8,7807\\
628&394384&247673152&25,0599&8,5635&&678&459684&311665752&26,0384&8,7850\\
629&395641&248858189&25,0799&8,5681&&679&461041&313046839&26,0576&8,7893\\
630&396900&250047000&25,0998&8,5726&&680&462400&314432000&26,0768&8,7937\\
631&398161&251239591&25,1197&8,5772&&681&463761&315821241&26,0960&8,7980\\
632&399424&252435968&25,1396&8,5817&&682&465124&317214568&26,1151&8,8023\\
633&400689&253636137&25,1595&8,5862&&683&466489&318611987&26,1343&8,8066\\
634&401956&254840104&25,1794&8,5907&&684&467856&320013504&26,1534&8,8109\\
635&403225&256047875&25,1992&8,5952&&685&469225&321419125&26,1725&8,8152\\
636&404496&257259456&25,2190&8,5997&&686&470596&322828856&26,1916&8,8194\\
637&405769&258474853&25,2389&8,6043&&687&471969&324242703&26,2107&8,8237\\
638&407044&259694072&25,2587&8,6088&&688&473344&325660672&26,2298&8,8280\\
639&408321&260917119&25,2784&8,6132&&689&474721&327082769&26,2488&8,8323\\
640&409600&262144000&25,2982&8,6177&&690&476100&328509000&26,2679&8,8366\\
641&410881&263374721&25,3180&8,6222&&691&477481&329939371&26,2869&8,8408\\
642&412164&264609288&25,3377&8,6267&&692&478864&331373888&26,3059&8,8451\\
643&413449&265847707&25,3574&8,6312&&693&480249&332812557&26,3249&8,8493\\
644&414736&267089984&25,3772&8,6357&&694&481636&334255384&26,3439&8,8536\\
645&416025&268336125&25,3969&8,6401&&695&483025&335702375&26,3629&8,8578\\
646&417316&269586136&25,4165&8,6446&&696&484416&337153536&26,3818&8,8621\\
647&418609&270840023&25,4362&8,6490&&697&485809&338608873&26,4008&8,8663\\
648&419904&272097792&25,4558&8,6535&&698&487204&340068392&26,4197&8,8706\\
649&421201&273359449&25,4755&8,6579&&699&488601&341532099&26,4386&8,8748\\
650&422500&274625000&25,4951&8,6624&&700&490000&343000000&26,4575&8,8790\\
\newpage
701&491401&344472101&26,4764&8,8833&&751&564001&423564751&27,4044&9,0896\\
702&492804&345948408&26,4953&8,8875&&752&565504&425259008&27,4226&9,0937\\
703&494209&347428927&26,5141&8,8917&&753&567009&426957777&27,4408&9,0977\\
704&495616&348913664&26,5330&8,8959&&754&568516&428661064&27,4591&9,1017\\
705&497025&350402625&26,5518&8,9001&&755&570025&430368875&27,4773&9,1057\\
706&498436&351895816&26,5707&8,9043&&756&571536&432081216&27,4955&9,1098\\
707&499849&353393243&26,5895&8,9085&&757&573049&433798093&27,5136&9,1138\\
708&501264&354894912&26,6083&8,9127&&758&574564&435519512&27,5318&9,1178\\
709&502681&356400829&26,6271&8,9169&&759&576081&437245479&27,5500&9,1218\\
710&504100&357911000&26,6458&8,9211&&760&577600&438976000&27,5681&9,1258\\
711&505521&359425431&26,6646&8,9253&&761&579121&440711081&27,5862&9,1298\\
712&506944&360944128&26,6833&8,9295&&762&580644&442450728&27,6043&9,1338\\
713&508369&362467097&26,7021&8,9337&&763&582169&444194947&27,6225&9,1378\\
714&509796&363994344&26,7208&8,9378&&764&583696&445943744&27,6405&9,1418\\
715&511225&365525875&26,7395&8,9420&&765&585225&447697125&27,6586&9,1458\\
716&512656&367061696&26,7582&8,9462&&766&586756&449455096&27,6767&9,1498\\
717&514089&368601813&26,7769&8,9503&&767&588289&451217663&27,6948&9,1537\\
718&515524&370146232&26,7955&8,9545&&768&589824&452984832&27,7128&9,1577\\
719&516961&371694959&26,8142&8,9587&&769&591361&454756609&27,7308&9,1617\\
720&518400&373248000&26,8328&8,9628&&770&592900&456533000&27,7489&9,1657\\
721&519841&374805361&26,8514&8,9670&&771&594441&458314011&27,7669&9,1696\\
722&521284&376367048&26,8701&8,9711&&772&595984&460099648&27,7849&9,1736\\
723&522729&377933067&26,8887&8,9752&&773&597529&461889917&27,8029&9,1775\\
724&524176&379503424&26,9072&8,9794&&774&599076&463684824&27,8209&9,1815\\
725&525625&381078125&26,9258&8,9835&&775&600625&465484375&27,8388&9,1855\\
726&527076&382657176&26,9444&8,9876&&776&602176&467288576&27,8568&9,1894\\
727&528529&384240583&26,9629&8,9918&&777&603729&469097433&27,8747&9,1933\\
728&529984&385828352&26,9815&8,9959&&778&605284&470910952&27,8927&9,1973\\
729&531441&387420489&27,0000&9,0000&&779&606841&472729139&27,9106&9,2012\\
730&532900&389017000&27,0185&9,0041&&780&608400&474552000&27,9285&9,2052\\
731&534361&390617891&27,0370&9,0082&&781&609961&476379541&27,9464&9,2091\\
732&535824&392223168&27,0555&9,0123&&782&611524&478211768&27,9643&9,2130\\
733&537289&393832837&27,0740&9,0164&&783&613089&480048687&27,9821&9,2170\\
734&538756&395446904&27,0924&9,0205&&784&614656&481890304&28,0000&9,2209\\
735&540225&397065375&27,1109&9,0246&&785&616225&483736625&28,0179&9,2248\\
736&541696&398688256&27,1293&9,0287&&786&617796&485587656&28,0357&9,2287\\
737&543169&400315553&27,1477&9,0328&&787&619369&487443403&28,0535&9,2326\\
738&544644&401947272&27,1662&9,0369&&788&620944&489303872&28,0713&9,2365\\
739&546121&403583419&27,1846&9,0410&&789&622521&491169069&28,0891&9,2404\\
740&547600&405224000&27,2029&9,0450&&790&624100&493039000&28,1069&9,2443\\
741&549081&406869021&27,2213&9,0491&&791&625681&494913671&28,1247&9,2482\\
742&550564&408518488&27,2397&9,0532&&792&627264&496793088&28,1425&9,2521\\
743&552049&410172407&27,2580&9,0572&&793&628849&498677257&28,1603&9,2560\\
744&553536&411830784&27,2764&9,0613&&794&630436&500566184&28,1780&9,2599\\
745&555025&413493625&27,2947&9,0654&&795&632025&502459875&28,1957&9,2638\\
746&556516&415160936&27,3130&9,0694&&796&633616&504358336&28,2135&9,2677\\
747&558009&416832723&27,3313&9,0735&&797&635209&506261573&28,2312&9,2716\\
748&559504&418508992&27,3496&9,0775&&798&636804&508169592&28,2489&9,2754\\
749&561001&420189749&27,3679&9,0816&&799&638401&510082399&28,2666&9,2793\\
750&562500&421875000&27,3861&9,0856&&800&640000&512000000&28,2843&9,2832\\
\newpage
801&641601&513922401&28,3019&9,2870&&851&724201&616295051&29,1719&9,4764\\
802&643204&515849608&28,3196&9,2909&&852&725904&618470208&29,1890&9,4801\\
803&644809&517781627&28,3373&9,2948&&853&727609&620650477&29,2062&9,4838\\
804&646416&519718464&28,3549&9,2986&&854&729316&622835864&29,2233&9,4875\\
805&648025&521660125&28,3725&9,3025&&855&731025&625026375&29,2404&9,4912\\
806&649636&523606616&28,3901&9,3063&&856&732736&627222016&29,2575&9,4949\\
807&651249&525557943&28,4077&9,3102&&857&734449&629422793&29,2746&9,4986\\
808&652864&527514112&28,4253&9,3140&&858&736164&631628712&29,2916&9,5023\\
809&654481&529475129&28,4429&9,3179&&859&737881&633839779&29,3087&9,5060\\
810&656100&531441000&28,4605&9,3217&&860&739600&636056000&29,3258&9,5097\\
811&657721&533411731&28,4781&9,3255&&861&741321&638277381&29,3428&9,5134\\
812&659344&535387328&28,4956&9,3294&&862&743044&640503928&29,3598&9,5171\\
813&660969&537367797&28,5132&9,3332&&863&744769&642735647&29,3769&9,5207\\
814&662596&539353144&28,5307&9,3370&&864&746496&644972544&29,3939&9,5244\\
815&664225&541343375&28,5482&9,3408&&865&748225&647214625&29,4109&9,5281\\
816&665856&543338496&28,5657&9,3447&&866&749956&649461896&29,4279&9,5317\\
817&667489&545338513&28,5832&9,3485&&867&751689&651714363&29,4449&9,5354\\
818&669124&547343432&28,6007&9,3523&&868&753424&653972032&29,4618&9,5391\\
819&670761&549353259&28,6182&9,3561&&869&755161&656234909&29,4788&9,5427\\
820&672400&551368000&28,6356&9,3599&&870&756900&658503000&29,4958&9,5464\\
821&674041&553387661&28,6531&9,3637&&871&758641&660776311&29,5127&9,5501\\
822&675684&555412248&28,6705&9,3675&&872&760384&663054848&29,5296&9,5537\\
823&677329&557441767&28,6880&9,3713&&873&762129&665338617&29,5466&9,5574\\
824&678976&559476224&28,7054&9,3751&&874&763876&667627624&29,5635&9,5610\\
825&680625&561515625&28,7228&9,3789&&875&765625&669921875&29,5804&9,5647\\
826&682276&563559976&28,7402&9,3827&&876&767376&672221376&29,5973&9,5683\\
827&683929&565609283&28,7576&9,3865&&877&769129&674526133&29,6142&9,5719\\
828&685584&567663552&28,7750&9,3902&&878&770884&676836152&29,6311&9,5756\\
829&687241&569722789&28,7924&9,3940&&879&772641&679151439&29,6479&9,5792\\
830&688900&571787000&28,8097&9,3978&&880&774400&681472000&29,6648&9,5828\\
831&690561&573856191&28,8271&9,4016&&881&776161&683797841&29,6816&9,5865\\
832&692224&575930368&28,8444&9,4053&&882&777924&686128968&29,6985&9,5901\\
833&693889&578009537&28,8617&9,4091&&883&779689&688465387&29,7153&9,5937\\
834&695556&580093704&28,8791&9,4129&&884&781456&690807104&29,7321&9,5973\\
835&697225&582182875&28,8964&9,4166&&885&783225&693154125&29,7489&9,6010\\
836&698896&584277056&28,9137&9,4204&&886&784996&695506456&29,7658&9,6046\\
837&700569&586376253&28,9310&9,4241&&887&786769&697864103&29,7825&9,6082\\
838&702244&588480472&28,9482&9,4279&&888&788544&700227072&29,7993&9,6118\\
839&703921&590589719&28,9655&9,4316&&889&790321&702595369&29,8161&9,6154\\
840&705600&592704000&28,9828&9,4354&&890&792100&704969000&29,8329&9,6190\\
841&707281&594823321&29,0000&9,4391&&891&793881&707347971&29,8496&9,6226\\
842&708964&596947688&29,0172&9,4429&&892&795664&709732288&29,8664&9,6262\\
843&710649&599077107&29,0345&9,4466&&893&797449&712121957&29,8831&9,6298\\
844&712336&601211584&29,0517&9,4503&&894&799236&714516984&29,8998&9,6334\\
845&714025&603351125&29,0689&9,4541&&895&801025&716917375&29,9166&9,6370\\
846&715716&605495736&29,0861&9,4578&&896&802816&719323136&29,9333&9,6406\\
847&717409&607645423&29,1033&9,4615&&897&804609&721734273&29,9500&9,6442\\
848&719104&609800192&29,1204&9,4652&&898&806404&724150792&29,9666&9,6477\\
849&720801&611960049&29,1376&9,4690&&899&808201&726572699&29,9833&9,6513\\
850&722500&614125000&29,1548&9,4727&&900&810000&729000000&30,0000&9,6549\\
\newpage
901&811801&731432701&30,0167&9,6585&&951&904401&860085351&30,8383&9,8339\\
902&813604&733870808&30,0333&9,6620&&952&906304&862801408&30,8545&9,8374\\
903&815409&736314327&30,0500&9,6656&&953&908209&865523177&30,8707&9,8408\\
904&817216&738763264&30,0666&9,6692&&954&910116&868250664&30,8869&9,8443\\
905&819025&741217625&30,0832&9,6727&&955&912025&870983875&30,9031&9,8477\\
906&820836&743677416&30,0998&9,6763&&956&913936&873722816&30,9192&9,8511\\
907&822649&746142643&30,1164&9,6799&&957&915849&876467493&30,9354&9,8546\\
908&824464&748613312&30,1330&9,6834&&958&917764&879217912&30,9516&9,8580\\
909&826281&751089429&30,1496&9,6870&&959&919681&881974079&30,9677&9,8614\\
910&828100&753571000&30,1662&9,6905&&960&921600&884736000&30,9839&9,8648\\
911&829921&756058031&30,1828&9,6941&&961&923521&887503681&31,0000&9,8683\\
912&831744&758550528&30,1993&9,6976&&962&925444&890277128&31,0161&9,8717\\
913&833569&761048497&30,2159&9,7012&&963&927369&893056347&31,0322&9,8751\\
914&835396&763551944&30,2324&9,7047&&964&929296&895841344&31,0483&9,8785\\
915&837225&766060875&30,2490&9,7082&&965&931225&898632125&31,0644&9,8819\\
916&839056&768575296&30,2655&9,7118&&966&933156&901428696&31,0805&9,8854\\
917&840889&771095213&30,2820&9,7153&&967&935089&904231063&31,0966&9,8888\\
918&842724&773620632&30,2985&9,7188&&968&937024&907039232&31,1127&9,8922\\
919&844561&776151559&30,3150&9,7224&&969&938961&909853209&31,1288&9,8956\\
920&846400&778688000&30,3315&9,7259&&970&940900&912673000&31,1448&9,8990\\
921&848241&781229961&30,3480&9,7294&&971&942841&915498611&31,1609&9,9024\\
922&850084&783777448&30,3645&9,7329&&972&944784&918330048&31,1769&9,9058\\
923&851929&786330467&30,3809&9,7364&&973&946729&921167317&31,1929&9,9092\\
924&853776&788889024&30,3974&9,7400&&974&948676&924010424&31,2090&9,9126\\
925&855625&791453125&30,4138&9,7435&&975&950625&926859375&31,2250&9,9160\\
926&857476&794022776&30,4302&9,7470&&976&952576&929714176&31,2410&9,9194\\
927&859329&796597983&30,4467&9,7505&&977&954529&932574833&31,2570&9,9227\\
928&861184&799178752&30,4631&9,7540&&978&956484&935441352&31,2730&9,9261\\
929&863041&801765089&30,4795&9,7575&&979&958441&938313739&31,2890&9,9295\\
930&864900&804357000&30,4959&9,7610&&980&960400&941192000&31,3050&9,9329\\
931&866761&806954491&30,5123&9,7645&&981&962361&944076141&31,3209&9,9363\\
932&868624&809557568&30,5287&9,7680&&982&964324&946966168&31,3369&9,9396\\
933&870489&812166237&30,5450&9,7715&&983&966289&949862087&31,3528&9,9430\\
934&872356&814780504&30,5614&9,7750&&984&968256&952763904&31,3688&9,9464\\
935&874225&817400375&30,5778&9,7785&&985&970225&955671625&31,3847&9,9497\\
936&876096&820025856&30,5941&9,7819&&986&972196&958585256&31,4006&9,9531\\
937&877969&822656953&30,6105&9,7854&&987&974169&961504803&31,4166&9,9565\\
938&879844&825293672&30,6268&9,7889&&988&976144&964430272&31,4325&9,9598\\
939&881721&827936019&30,6431&9,7924&&989&978121&967361669&31,4484&9,9632\\
940&883600&830584000&30,6594&9,7959&&990&980100&970299000&31,4643&9,9666\\
941&885481&833237621&30,6757&9,7993&&991&982081&973242271&31,4802&9,9699\\
942&887364&835896888&30,6920&9,8028&&992&984064&976191488&31,4960&9,9733\\
943&889249&838561807&30,7083&9,8063&&993&986049&979146657&31,5119&9,9766\\
944&891136&841232384&30,7246&9,8097&&994&988036&982107784&31,5278&9,9800\\
945&893025&843908625&30,7409&9,8132&&995&990025&985074875&31,5436&9,9833\\
946&894916&846590536&30,7571&9,8167&&996&992016&988047936&31,5595&9,9866\\
947&896809&849278123&30,7734&9,8201&&997&994009&991026973&31,5753&9,9900\\
948&898704&851971392&30,7896&9,8236&&998&996004&994011992&31,5911&9,9933\\
949&900601&854670349&30,8058&9,8270&&999&998001&997002999&31,6070&9,9967\\
950&902500&857375000&30,8221&9,8305&&1000&1000000&1000000000&31,6228&10,0000\\
\end{longtable}

\section{Scomposizione in fattori}
\label{chap:ScomposizioneInFattori}
{\small \begin{longtable}{lllll}
	\toprule\endhead
	\bottomrule \endfoot
$4=2^{2}$&$68=2^{2}17^{1}$&$126=2^{1}3^{2}7^{1}$&$185=5^{1}37^{1}$&$243=3^{5}$\\
$6=2^{1}3^{1}$&$69=3^{1}23^{1}$&$128=2^{7}$&$186=2^{1}3^{1}31^{1}$&$244=2^{2}61^{1}$\\
$8=2^{3}$&$70=2^{1}5^{1}7^{1}$&$129=3^{1}43^{1}$&$187=11^{1}17^{1}$&$245=5^{1}7^{2}$\\
$9=3^{2}$&$72=2^{3}3^{2}$&$130=2^{1}5^{1}13^{1}$&$188=2^{2}47^{1}$&$246=2^{1}3^{1}41^{1}$\\
$10=2^{1}5^{1}$&$74=2^{1}37^{1}$&$132=2^{2}3^{1}11^{1}$&$189=3^{3}7^{1}$&$247=13^{1}19^{1}$\\
$12=2^{2}3^{1}$&$75=3^{1}5^{2}$&$133=7^{1}19^{1}$&$190=2^{1}5^{1}19^{1}$&$248=2^{3}31^{1}$\\
$14=2^{1}7^{1}$&$76=2^{2}19^{1}$&$134=2^{1}67^{1}$&$192=2^{6}3^{1}$&$249=3^{1}83^{1}$\\
$15=3^{1}5^{1}$&$77=7^{1}11^{1}$&$135=3^{3}5^{1}$&$194=2^{1}97^{1}$&$250=2^{1}5^{3}$\\
$16=2^{4}$&$78=2^{1}3^{1}13^{1}$&$136=2^{3}17^{1}$&$195=3^{1}5^{1}13^{1}$&$252=2^{2}3^{2}7^{1}$\\
$18=2^{1}3^{2}$&$80=2^{4}5^{1}$&$138=2^{1}3^{1}23^{1}$&$196=2^{2}7^{2}$&$253=11^{1}23^{1}$\\
$20=2^{2}5^{1}$&$81=3^{4}$&$140=2^{2}5^{1}7^{1}$&$198=2^{1}3^{2}11^{1}$&$254=2^{1}127^{1}$\\
$21=3^{1}7^{1}$&$82=2^{1}41^{1}$&$141=3^{1}47^{1}$&$200=2^{3}5^{2}$&$255=3^{1}5^{1}17^{1}$\\
$22=2^{1}11^{1}$&$84=2^{2}3^{1}7^{1}$&$142=2^{1}71^{1}$&$201=3^{1}67^{1}$&$256=2^{8}$\\
$24=2^{3}3^{1}$&$85=5^{1}17^{1}$&$143=11^{1}13^{1}$&$202=2^{1}101^{1}$&$258=2^{1}3^{1}43^{1}$\\
$25=5^{2}$&$86=2^{1}43^{1}$&$144=2^{4}3^{2}$&$203=7^{1}29^{1}$&$259=7^{1}37^{1}$\\
$26=2^{1}13^{1}$&$87=3^{1}29^{1}$&$145=5^{1}29^{1}$&$204=2^{2}3^{1}17^{1}$&$260=2^{2}5^{1}13^{1}$\\
$27=3^{3}$&$88=2^{3}11^{1}$&$146=2^{1}73^{1}$&$205=5^{1}41^{1}$&$261=3^{2}29^{1}$\\
$28=2^{2}7^{1}$&$90=2^{1}3^{2}5^{1}$&$147=3^{1}7^{2}$&$206=2^{1}103^{1}$&$262=2^{1}131^{1}$\\
$30=2^{1}3^{1}5^{1}$&$91=7^{1}13^{1}$&$148=2^{2}37^{1}$&$207=3^{2}23^{1}$&$264=2^{3}3^{1}11^{1}$\\
$32=2^{5}$&$92=2^{2}23^{1}$&$150=2^{1}3^{1}5^{2}$&$208=2^{4}13^{1}$&$265=5^{1}53^{1}$\\
$33=3^{1}11^{1}$&$93=3^{1}31^{1}$&$152=2^{3}19^{1}$&$209=11^{1}19^{1}$&$266=2^{1}7^{1}19^{1}$\\
$34=2^{1}17^{1}$&$94=2^{1}47^{1}$&$153=3^{2}17^{1}$&$210=2^{1}3^{1}5^{1}7^{1}$&$267=3^{1}89^{1}$\\
$35=5^{1}7^{1}$&$95=5^{1}19^{1}$&$154=2^{1}7^{1}11^{1}$&$212=2^{2}53^{1}$&$268=2^{2}67^{1}$\\
$36=2^{2}3^{2}$&$96=2^{5}3^{1}$&$155=5^{1}31^{1}$&$213=3^{1}71^{1}$&$270=2^{1}3^{3}5^{1}$\\
$38=2^{1}19^{1}$&$98=2^{1}7^{2}$&$156=2^{2}3^{1}13^{1}$&$214=2^{1}107^{1}$&$272=2^{4}17^{1}$\\
$39=3^{1}13^{1}$&$99=3^{2}11^{1}$&$158=2^{1}79^{1}$&$215=5^{1}43^{1}$&$273=3^{1}7^{1}13^{1}$\\
$40=2^{3}5^{1}$&$100=2^{2}5^{2}$&$159=3^{1}53^{1}$&$216=2^{3}3^{3}$&$274=2^{1}137^{1}$\\
$42=2^{1}3^{1}7^{1}$&$102=2^{1}3^{1}17^{1}$&$160=2^{5}5^{1}$&$217=7^{1}31^{1}$&$275=5^{2}11^{1}$\\
$44=2^{2}11^{1}$&$104=2^{3}13^{1}$&$161=7^{1}23^{1}$&$218=2^{1}109^{1}$&$276=2^{2}3^{1}23^{1}$\\
$45=3^{2}5^{1}$&$105=3^{1}5^{1}7^{1}$&$162=2^{1}3^{4}$&$219=3^{1}73^{1}$&$278=2^{1}139^{1}$\\
$46=2^{1}23^{1}$&$106=2^{1}53^{1}$&$164=2^{2}41^{1}$&$220=2^{2}5^{1}11^{1}$&$279=3^{2}31^{1}$\\
$48=2^{4}3^{1}$&$108=2^{2}3^{3}$&$165=3^{1}5^{1}11^{1}$&$221=13^{1}17^{1}$&$280=2^{3}5^{1}7^{1}$\\
$49=7^{2}$&$110=2^{1}5^{1}11^{1}$&$166=2^{1}83^{1}$&$222=2^{1}3^{1}37^{1}$&$282=2^{1}3^{1}47^{1}$\\
$50=2^{1}5^{2}$&$111=3^{1}37^{1}$&$168=2^{3}3^{1}7^{1}$&$224=2^{5}7^{1}$&$284=2^{2}71^{1}$\\
$51=3^{1}17^{1}$&$112=2^{4}7^{1}$&$169=13^{2}$&$225=3^{2}5^{2}$&$285=3^{1}5^{1}19^{1}$\\
$52=2^{2}13^{1}$&$114=2^{1}3^{1}19^{1}$&$170=2^{1}5^{1}17^{1}$&$226=2^{1}113^{1}$&$286=2^{1}11^{1}13^{1}$\\
$54=2^{1}3^{3}$&$115=5^{1}23^{1}$&$171=3^{2}19^{1}$&$228=2^{2}3^{1}19^{1}$&$287=7^{1}41^{1}$\\
$55=5^{1}11^{1}$&$116=2^{2}29^{1}$&$172=2^{2}43^{1}$&$230=2^{1}5^{1}23^{1}$&$288=2^{5}3^{2}$\\
$56=2^{3}7^{1}$&$117=3^{2}13^{1}$&$174=2^{1}3^{1}29^{1}$&$231=3^{1}7^{1}11^{1}$&$289=17^{2}$\\
$57=3^{1}19^{1}$&$118=2^{1}59^{1}$&$175=5^{2}7^{1}$&$232=2^{3}29^{1}$&$290=2^{1}5^{1}29^{1}$\\
$58=2^{1}29^{1}$&$119=7^{1}17^{1}$&$176=2^{4}11^{1}$&$234=2^{1}3^{2}13^{1}$&$291=3^{1}97^{1}$\\
$60=2^{2}3^{1}5^{1}$&$120=2^{3}3^{1}5^{1}$&$177=3^{1}59^{1}$&$235=5^{1}47^{1}$&$292=2^{2}73^{1}$\\
$62=2^{1}31^{1}$&$121=11^{2}$&$178=2^{1}89^{1}$&$236=2^{2}59^{1}$&$294=2^{1}3^{1}7^{2}$\\
$63=3^{2}7^{1}$&$122=2^{1}61^{1}$&$180=2^{2}3^{2}5^{1}$&$237=3^{1}79^{1}$&$295=5^{1}59^{1}$\\
$64=2^{6}$&$123=3^{1}41^{1}$&$182=2^{1}7^{1}13^{1}$&$238=2^{1}7^{1}17^{1}$&$296=2^{3}37^{1}$\\
$65=5^{1}13^{1}$&$124=2^{2}31^{1}$&$183=3^{1}61^{1}$&$240=2^{4}3^{1}5^{1}$&$297=3^{3}11^{1}$\\
$66=2^{1}3^{1}11^{1}$&$125=5^{3}$&$184=2^{3}23^{1}$&$242=2^{1}11^{2}$&$298=2^{1}149^{1}$\\

$299=13^{1}23^{1}$&$355=5^{1}71^{1}$&$411=3^{1}137^{1}$&$469=7^{1}67^{1}$&$524=2^{2}131^{1}$\\
$300=2^{2}3^{1}5^{2}$&$356=2^{2}89^{1}$&$412=2^{2}103^{1}$&$470=2^{1}5^{1}47^{1}$&$525=3^{1}5^{2}7^{1}$\\
$301=7^{1}43^{1}$&$357=3^{1}7^{1}17^{1}$&$413=7^{1}59^{1}$&$471=3^{1}157^{1}$&$526=2^{1}263^{1}$\\
$302=2^{1}151^{1}$&$358=2^{1}179^{1}$&$414=2^{1}3^{2}23^{1}$&$472=2^{3}59^{1}$&$527=17^{1}31^{1}$\\
$303=3^{1}101^{1}$&$360=2^{3}3^{2}5^{1}$&$415=5^{1}83^{1}$&$473=11^{1}43^{1}$&$528=2^{4}3^{1}11^{1}$\\
$304=2^{4}19^{1}$&$361=19^{2}$&$416=2^{5}13^{1}$&$474=2^{1}3^{1}79^{1}$&$529=23^{2}$\\
$305=5^{1}61^{1}$&$362=2^{1}181^{1}$&$417=3^{1}139^{1}$&$475=5^{2}19^{1}$&$530=2^{1}5^{1}53^{1}$\\
$306=2^{1}3^{2}17^{1}$&$363=3^{1}11^{2}$&$418=2^{1}11^{1}19^{1}$&$476=2^{2}7^{1}17^{1}$&$531=3^{2}59^{1}$\\
$308=2^{2}7^{1}11^{1}$&$364=2^{2}7^{1}13^{1}$&$420=2^{2}3^{1}5^{1}7^{1}$&$477=3^{2}53^{1}$&$532=2^{2}7^{1}19^{1}$\\
$309=3^{1}103^{1}$&$365=5^{1}73^{1}$&$422=2^{1}211^{1}$&$478=2^{1}239^{1}$&$533=13^{1}41^{1}$\\
$310=2^{1}5^{1}31^{1}$&$366=2^{1}3^{1}61^{1}$&$423=3^{2}47^{1}$&$480=2^{5}3^{1}5^{1}$&$534=2^{1}3^{1}89^{1}$\\
$312=2^{3}3^{1}13^{1}$&$368=2^{4}23^{1}$&$424=2^{3}53^{1}$&$481=13^{1}37^{1}$&$535=5^{1}107^{1}$\\
$314=2^{1}157^{1}$&$369=3^{2}41^{1}$&$425=5^{2}17^{1}$&$482=2^{1}241^{1}$&$536=2^{3}67^{1}$\\
$315=3^{2}5^{1}7^{1}$&$370=2^{1}5^{1}37^{1}$&$426=2^{1}3^{1}71^{1}$&$483=3^{1}7^{1}23^{1}$&$537=3^{1}179^{1}$\\
$316=2^{2}79^{1}$&$371=7^{1}53^{1}$&$427=7^{1}61^{1}$&$484=2^{2}11^{2}$&$538=2^{1}269^{1}$\\
$318=2^{1}3^{1}53^{1}$&$372=2^{2}3^{1}31^{1}$&$428=2^{2}107^{1}$&$485=5^{1}97^{1}$&$539=7^{2}11^{1}$\\
$319=11^{1}29^{1}$&$374=2^{1}11^{1}17^{1}$&$429=3^{1}11^{1}13^{1}$&$486=2^{1}3^{5}$&$540=2^{2}3^{3}5^{1}$\\
$320=2^{6}5^{1}$&$375=3^{1}5^{3}$&$430=2^{1}5^{1}43^{1}$&$488=2^{3}61^{1}$&$542=2^{1}271^{1}$\\
$321=3^{1}107^{1}$&$376=2^{3}47^{1}$&$432=2^{4}3^{3}$&$489=3^{1}163^{1}$&$543=3^{1}181^{1}$\\
$322=2^{1}7^{1}23^{1}$&$377=13^{1}29^{1}$&$434=2^{1}7^{1}31^{1}$&$490=2^{1}5^{1}7^{2}$&$544=2^{5}17^{1}$\\
$323=17^{1}19^{1}$&$378=2^{1}3^{3}7^{1}$&$435=3^{1}5^{1}29^{1}$&$492=2^{2}3^{1}41^{1}$&$545=5^{1}109^{1}$\\
$324=2^{2}3^{4}$&$380=2^{2}5^{1}19^{1}$&$436=2^{2}109^{1}$&$493=17^{1}29^{1}$&$546=2^{1}3^{1}7^{1}13^{1}$\\
$325=5^{2}13^{1}$&$381=3^{1}127^{1}$&$437=19^{1}23^{1}$&$494=2^{1}13^{1}19^{1}$&$548=2^{2}137^{1}$\\
$326=2^{1}163^{1}$&$382=2^{1}191^{1}$&$438=2^{1}3^{1}73^{1}$&$495=3^{2}5^{1}11^{1}$&$549=3^{2}61^{1}$\\
$327=3^{1}109^{1}$&$384=2^{7}3^{1}$&$440=2^{3}5^{1}11^{1}$&$496=2^{4}31^{1}$&$550=2^{1}5^{2}11^{1}$\\
$328=2^{3}41^{1}$&$385=5^{1}7^{1}11^{1}$&$441=3^{2}7^{2}$&$497=7^{1}71^{1}$&$551=19^{1}29^{1}$\\
$329=7^{1}47^{1}$&$386=2^{1}193^{1}$&$442=2^{1}13^{1}17^{1}$&$498=2^{1}3^{1}83^{1}$&$552=2^{3}3^{1}23^{1}$\\
$330=2^{1}3^{1}5^{1}11^{1}$&$387=3^{2}43^{1}$&$444=2^{2}3^{1}37^{1}$&$500=2^{2}5^{3}$&$553=7^{1}79^{1}$\\
$332=2^{2}83^{1}$&$388=2^{2}97^{1}$&$445=5^{1}89^{1}$&$501=3^{1}167^{1}$&$554=2^{1}277^{1}$\\
$333=3^{2}37^{1}$&$390=2^{1}3^{1}5^{1}13^{1}$&$446=2^{1}223^{1}$&$502=2^{1}251^{1}$&$555=3^{1}5^{1}37^{1}$\\
$334=2^{1}167^{1}$&$391=17^{1}23^{1}$&$447=3^{1}149^{1}$&$504=2^{3}3^{2}7^{1}$&$556=2^{2}139^{1}$\\
$335=5^{1}67^{1}$&$392=2^{3}7^{2}$&$448=2^{6}7^{1}$&$505=5^{1}101^{1}$&$558=2^{1}3^{2}31^{1}$\\
$336=2^{4}3^{1}7^{1}$&$393=3^{1}131^{1}$&$450=2^{1}3^{2}5^{2}$&$506=2^{1}11^{1}23^{1}$&$559=13^{1}43^{1}$\\
$338=2^{1}13^{2}$&$394=2^{1}197^{1}$&$451=11^{1}41^{1}$&$507=3^{1}13^{2}$&$560=2^{4}5^{1}7^{1}$\\
$339=3^{1}113^{1}$&$395=5^{1}79^{1}$&$452=2^{2}113^{1}$&$508=2^{2}127^{1}$&$561=3^{1}11^{1}17^{1}$\\
$340=2^{2}5^{1}17^{1}$&$396=2^{2}3^{2}11^{1}$&$453=3^{1}151^{1}$&$510=2^{1}3^{1}5^{1}17^{1}$&$562=2^{1}281^{1}$\\
$341=11^{1}31^{1}$&$398=2^{1}199^{1}$&$454=2^{1}227^{1}$&$511=7^{1}73^{1}$&$564=2^{2}3^{1}47^{1}$\\
$342=2^{1}3^{2}19^{1}$&$399=3^{1}7^{1}19^{1}$&$455=5^{1}7^{1}13^{1}$&$512=2^{9}$&$565=5^{1}113^{1}$\\
$343=7^{3}$&$400=2^{4}5^{2}$&$456=2^{3}3^{1}19^{1}$&$513=3^{3}19^{1}$&$566=2^{1}283^{1}$\\
$344=2^{3}43^{1}$&$402=2^{1}3^{1}67^{1}$&$458=2^{1}229^{1}$&$514=2^{1}257^{1}$&$567=3^{4}7^{1}$\\
$345=3^{1}5^{1}23^{1}$&$403=13^{1}31^{1}$&$459=3^{3}17^{1}$&$515=5^{1}103^{1}$&$568=2^{3}71^{1}$\\
$346=2^{1}173^{1}$&$404=2^{2}101^{1}$&$460=2^{2}5^{1}23^{1}$&$516=2^{2}3^{1}43^{1}$&$570=2^{1}3^{1}5^{1}19^{1}$\\
$348=2^{2}3^{1}29^{1}$&$405=3^{4}5^{1}$&$462=2^{1}3^{1}7^{1}11^{1}$&$517=11^{1}47^{1}$&$572=2^{2}11^{1}13^{1}$\\
$350=2^{1}5^{2}7^{1}$&$406=2^{1}7^{1}29^{1}$&$464=2^{4}29^{1}$&$518=2^{1}7^{1}37^{1}$&$573=3^{1}191^{1}$\\
$351=3^{3}13^{1}$&$407=11^{1}37^{1}$&$465=3^{1}5^{1}31^{1}$&$519=3^{1}173^{1}$&$574=2^{1}7^{1}41^{1}$\\
$352=2^{5}11^{1}$&$408=2^{3}3^{1}17^{1}$&$466=2^{1}233^{1}$&$520=2^{3}5^{1}13^{1}$&$575=5^{2}23^{1}$\\
$354=2^{1}3^{1}59^{1}$&$410=2^{1}5^{1}41^{1}$&$468=2^{2}3^{2}13^{1}$&$522=2^{1}3^{2}29^{1}$&$576=2^{6}3^{2}$\\
$578=2^{1}17^{2}$&$634=2^{1}317^{1}$&$690=2^{1}3^{1}5^{1}23^{1}$&$745=5^{1}149^{1}$&$799=17^{1}47^{1}$\\
$579=3^{1}193^{1}$&$635=5^{1}127^{1}$&$692=2^{2}173^{1}$&$746=2^{1}373^{1}$&$800=2^{5}5^{2}$\\
$580=2^{2}5^{1}29^{1}$&$636=2^{2}3^{1}53^{1}$&$693=3^{2}7^{1}11^{1}$&$747=3^{2}83^{1}$&$801=3^{2}89^{1}$\\
$581=7^{1}83^{1}$&$637=7^{2}13^{1}$&$694=2^{1}347^{1}$&$748=2^{2}11^{1}17^{1}$&$802=2^{1}401^{1}$\\
$582=2^{1}3^{1}97^{1}$&$638=2^{1}11^{1}29^{1}$&$695=5^{1}139^{1}$&$749=7^{1}107^{1}$&$803=11^{1}73^{1}$\\
$583=11^{1}53^{1}$&$639=3^{2}71^{1}$&$696=2^{3}3^{1}29^{1}$&$750=2^{1}3^{1}5^{3}$&$804=2^{2}3^{1}67^{1}$\\
$584=2^{3}73^{1}$&$640=2^{7}5^{1}$&$697=17^{1}41^{1}$&$752=2^{4}47^{1}$&$805=5^{1}7^{1}23^{1}$\\
$585=3^{2}5^{1}13^{1}$&$642=2^{1}3^{1}107^{1}$&$698=2^{1}349^{1}$&$753=3^{1}251^{1}$&$806=2^{1}13^{1}31^{1}$\\
$586=2^{1}293^{1}$&$644=2^{2}7^{1}23^{1}$&$699=3^{1}233^{1}$&$754=2^{1}13^{1}29^{1}$&$807=3^{1}269^{1}$\\
$588=2^{2}3^{1}7^{2}$&$645=3^{1}5^{1}43^{1}$&$700=2^{2}5^{2}7^{1}$&$755=5^{1}151^{1}$&$808=2^{3}101^{1}$\\
$589=19^{1}31^{1}$&$646=2^{1}17^{1}19^{1}$&$702=2^{1}3^{3}13^{1}$&$756=2^{2}3^{3}7^{1}$&$810=2^{1}3^{4}5^{1}$\\
$590=2^{1}5^{1}59^{1}$&$648=2^{3}3^{4}$&$703=19^{1}37^{1}$&$758=2^{1}379^{1}$&$812=2^{2}7^{1}29^{1}$\\
$591=3^{1}197^{1}$&$649=11^{1}59^{1}$&$704=2^{6}11^{1}$&$759=3^{1}11^{1}23^{1}$&$813=3^{1}271^{1}$\\
$592=2^{4}37^{1}$&$650=2^{1}5^{2}13^{1}$&$705=3^{1}5^{1}47^{1}$&$760=2^{3}5^{1}19^{1}$&$814=2^{1}11^{1}37^{1}$\\
$594=2^{1}3^{3}11^{1}$&$651=3^{1}7^{1}31^{1}$&$706=2^{1}353^{1}$&$762=2^{1}3^{1}127^{1}$&$815=5^{1}163^{1}$\\
$595=5^{1}7^{1}17^{1}$&$652=2^{2}163^{1}$&$707=7^{1}101^{1}$&$763=7^{1}109^{1}$&$816=2^{4}3^{1}17^{1}$\\
$596=2^{2}149^{1}$&$654=2^{1}3^{1}109^{1}$&$708=2^{2}3^{1}59^{1}$&$764=2^{2}191^{1}$&$817=19^{1}43^{1}$\\
$597=3^{1}199^{1}$&$655=5^{1}131^{1}$&$710=2^{1}5^{1}71^{1}$&$765=3^{2}5^{1}17^{1}$&$818=2^{1}409^{1}$\\
$598=2^{1}13^{1}23^{1}$&$656=2^{4}41^{1}$&$711=3^{2}79^{1}$&$766=2^{1}383^{1}$&$819=3^{2}7^{1}13^{1}$\\
$600=2^{3}3^{1}5^{2}$&$657=3^{2}73^{1}$&$712=2^{3}89^{1}$&$767=13^{1}59^{1}$&$820=2^{2}5^{1}41^{1}$\\
$602=2^{1}7^{1}43^{1}$&$658=2^{1}7^{1}47^{1}$&$713=23^{1}31^{1}$&$768=2^{8}3^{1}$&$822=2^{1}3^{1}137^{1}$\\
$603=3^{2}67^{1}$&$660=2^{2}3^{1}5^{1}11^{1}$&$714=2^{1}3^{1}7^{1}17^{1}$&$770=2^{1}5^{1}7^{1}11^{1}$&$824=2^{3}103^{1}$\\
$604=2^{2}151^{1}$&$662=2^{1}331^{1}$&$715=5^{1}11^{1}13^{1}$&$771=3^{1}257^{1}$&$825=3^{1}5^{2}11^{1}$\\
$605=5^{1}11^{2}$&$663=3^{1}13^{1}17^{1}$&$716=2^{2}179^{1}$&$772=2^{2}193^{1}$&$826=2^{1}7^{1}59^{1}$\\
$606=2^{1}3^{1}101^{1}$&$664=2^{3}83^{1}$&$717=3^{1}239^{1}$&$774=2^{1}3^{2}43^{1}$&$828=2^{2}3^{2}23^{1}$\\
$608=2^{5}19^{1}$&$665=5^{1}7^{1}19^{1}$&$718=2^{1}359^{1}$&$775=5^{2}31^{1}$&$830=2^{1}5^{1}83^{1}$\\
$609=3^{1}7^{1}29^{1}$&$666=2^{1}3^{2}37^{1}$&$720=2^{4}3^{2}5^{1}$&$776=2^{3}97^{1}$&$831=3^{1}277^{1}$\\
$610=2^{1}5^{1}61^{1}$&$667=23^{1}29^{1}$&$721=7^{1}103^{1}$&$777=3^{1}7^{1}37^{1}$&$832=2^{6}13^{1}$\\
$611=13^{1}47^{1}$&$668=2^{2}167^{1}$&$722=2^{1}19^{2}$&$778=2^{1}389^{1}$&$833=7^{2}17^{1}$\\
$612=2^{2}3^{2}17^{1}$&$669=3^{1}223^{1}$&$723=3^{1}241^{1}$&$779=19^{1}41^{1}$&$834=2^{1}3^{1}139^{1}$\\
$614=2^{1}307^{1}$&$670=2^{1}5^{1}67^{1}$&$724=2^{2}181^{1}$&$780=2^{2}3^{1}5^{1}13^{1}$&$835=5^{1}167^{1}$\\
$615=3^{1}5^{1}41^{1}$&$671=11^{1}61^{1}$&$725=5^{2}29^{1}$&$781=11^{1}71^{1}$&$836=2^{2}11^{1}19^{1}$\\
$616=2^{3}7^{1}11^{1}$&$672=2^{5}3^{1}7^{1}$&$726=2^{1}3^{1}11^{2}$&$782=2^{1}17^{1}23^{1}$&$837=3^{3}31^{1}$\\
$618=2^{1}3^{1}103^{1}$&$674=2^{1}337^{1}$&$728=2^{3}7^{1}13^{1}$&$783=3^{3}29^{1}$&$838=2^{1}419^{1}$\\
$620=2^{2}5^{1}31^{1}$&$675=3^{3}5^{2}$&$729=3^{6}$&$784=2^{4}7^{2}$&$840=2^{3}3^{1}5^{1}7^{1}$\\
$621=3^{3}23^{1}$&$676=2^{2}13^{2}$&$730=2^{1}5^{1}73^{1}$&$785=5^{1}157^{1}$&$841=29^{2}$\\
$622=2^{1}311^{1}$&$678=2^{1}3^{1}113^{1}$&$731=17^{1}43^{1}$&$786=2^{1}3^{1}131^{1}$&$842=2^{1}421^{1}$\\
$623=7^{1}89^{1}$&$679=7^{1}97^{1}$&$732=2^{2}3^{1}61^{1}$&$788=2^{2}197^{1}$&$843=3^{1}281^{1}$\\
$624=2^{4}3^{1}13^{1}$&$680=2^{3}5^{1}17^{1}$&$734=2^{1}367^{1}$&$789=3^{1}263^{1}$&$844=2^{2}211^{1}$\\
$625=5^{4}$&$681=3^{1}227^{1}$&$735=3^{1}5^{1}7^{2}$&$790=2^{1}5^{1}79^{1}$&$845=5^{1}13^{2}$\\
$626=2^{1}313^{1}$&$682=2^{1}11^{1}31^{1}$&$736=2^{5}23^{1}$&$791=7^{1}113^{1}$&$846=2^{1}3^{2}47^{1}$\\
$627=3^{1}11^{1}19^{1}$&$684=2^{2}3^{2}19^{1}$&$737=11^{1}67^{1}$&$792=2^{3}3^{2}11^{1}$&$847=7^{1}11^{2}$\\
$628=2^{2}157^{1}$&$685=5^{1}137^{1}$&$738=2^{1}3^{2}41^{1}$&$793=13^{1}61^{1}$&$848=2^{4}53^{1}$\\
$629=17^{1}37^{1}$&$686=2^{1}7^{3}$&$740=2^{2}5^{1}37^{1}$&$794=2^{1}397^{1}$&$849=3^{1}283^{1}$\\
$630=2^{1}3^{2}5^{1}7^{1}$&$687=3^{1}229^{1}$&$741=3^{1}13^{1}19^{1}$&$795=3^{1}5^{1}53^{1}$&$850=2^{1}5^{2}17^{1}$\\
$632=2^{3}79^{1}$&$688=2^{4}43^{1}$&$742=2^{1}7^{1}53^{1}$&$796=2^{2}199^{1}$&$851=23^{1}37^{1}$\\
$633=3^{1}211^{1}$&$689=13^{1}53^{1}$&$744=2^{3}3^{1}31^{1}$&$798=2^{1}3^{1}7^{1}19^{1}$&$852=2^{2}3^{1}71^{1}$\\
$854=2^{1}7^{1}61^{1}$&$909=3^{2}101^{1}$&$963=3^{2}107^{1}$&$1018=2^{1}509^{1}$&$1075=5^{2}43^{1}$\\
$855=3^{2}5^{1}19^{1}$&$910=2^{1}5^{1}7^{1}13^{1}$&$964=2^{2}241^{1}$&$1020=2^{2}3^{1}5^{1}17^{1}$&$1076=2^{2}269^{1}$\\
$856=2^{3}107^{1}$&$912=2^{4}3^{1}19^{1}$&$965=5^{1}193^{1}$&$1022=2^{1}7^{1}73^{1}$&$1077=3^{1}359^{1}$\\
$858=2^{1}3^{1}11^{1}13^{1}$&$913=11^{1}83^{1}$&$966=2^{1}3^{1}7^{1}23^{1}$&$1023=3^{1}11^{1}31^{1}$&$1078=2^{1}7^{2}11^{1}$\\
$860=2^{2}5^{1}43^{1}$&$914=2^{1}457^{1}$&$968=2^{3}11^{2}$&$1024=2^{10}$&$1079=13^{1}83^{1}$\\
$861=3^{1}7^{1}41^{1}$&$915=3^{1}5^{1}61^{1}$&$969=3^{1}17^{1}19^{1}$&$1025=5^{2}41^{1}$&$1080=2^{3}3^{3}5^{1}$\\
$862=2^{1}431^{1}$&$916=2^{2}229^{1}$&$970=2^{1}5^{1}97^{1}$&$1026=2^{1}3^{3}19^{1}$&$1081=23^{1}47^{1}$\\
$864=2^{5}3^{3}$&$917=7^{1}131^{1}$&$972=2^{2}3^{5}$&$1027=13^{1}79^{1}$&$1082=2^{1}541^{1}$\\
$865=5^{1}173^{1}$&$918=2^{1}3^{3}17^{1}$&$973=7^{1}139^{1}$&$1028=2^{2}257^{1}$&$1083=3^{1}19^{2}$\\
$866=2^{1}433^{1}$&$920=2^{3}5^{1}23^{1}$&$974=2^{1}487^{1}$&$1029=3^{1}7^{3}$&$1084=2^{2}271^{1}$\\
$867=3^{1}17^{2}$&$921=3^{1}307^{1}$&$975=3^{1}5^{2}13^{1}$&$1030=2^{1}5^{1}103^{1}$&$1085=5^{1}7^{1}31^{1}$\\
$868=2^{2}7^{1}31^{1}$&$922=2^{1}461^{1}$&$976=2^{4}61^{1}$&$1032=2^{3}3^{1}43^{1}$&$1086=2^{1}3^{1}181^{1}$\\
$869=11^{1}79^{1}$&$923=13^{1}71^{1}$&$978=2^{1}3^{1}163^{1}$&$1034=2^{1}11^{1}47^{1}$&$1088=2^{6}17^{1}$\\
$870=2^{1}3^{1}5^{1}29^{1}$&$924=2^{2}3^{1}7^{1}11^{1}$&$979=11^{1}89^{1}$&$1035=3^{2}5^{1}23^{1}$&$1089=3^{2}11^{2}$\\
$871=13^{1}67^{1}$&$925=5^{2}37^{1}$&$980=2^{2}5^{1}7^{2}$&$1036=2^{2}7^{1}37^{1}$&$1090=2^{1}5^{1}109^{1}$\\
$872=2^{3}109^{1}$&$926=2^{1}463^{1}$&$981=3^{2}109^{1}$&$1037=17^{1}61^{1}$&$1092=2^{2}3^{1}7^{1}13^{1}$\\
$873=3^{2}97^{1}$&$927=3^{2}103^{1}$&$982=2^{1}491^{1}$&$1038=2^{1}3^{1}173^{1}$&$1094=2^{1}547^{1}$\\
$874=2^{1}19^{1}23^{1}$&$928=2^{5}29^{1}$&$984=2^{3}3^{1}41^{1}$&$1040=2^{4}5^{1}13^{1}$&$1095=3^{1}5^{1}73^{1}$\\
$875=5^{3}7^{1}$&$930=2^{1}3^{1}5^{1}31^{1}$&$985=5^{1}197^{1}$&$1041=3^{1}347^{1}$&$1096=2^{3}137^{1}$\\
$876=2^{2}3^{1}73^{1}$&$931=7^{2}19^{1}$&$986=2^{1}17^{1}29^{1}$&$1042=2^{1}521^{1}$&$1098=2^{1}3^{2}61^{1}$\\
$878=2^{1}439^{1}$&$932=2^{2}233^{1}$&$987=3^{1}7^{1}47^{1}$&$1043=7^{1}149^{1}$&$1099=7^{1}157^{1}$\\
$879=3^{1}293^{1}$&$933=3^{1}311^{1}$&$988=2^{2}13^{1}19^{1}$&$1044=2^{2}3^{2}29^{1}$&$1100=2^{2}5^{2}11^{1}$\\
$880=2^{4}5^{1}11^{1}$&$934=2^{1}467^{1}$&$989=23^{1}43^{1}$&$1045=5^{1}11^{1}19^{1}$&$1101=3^{1}367^{1}$\\
$882=2^{1}3^{2}7^{2}$&$935=5^{1}11^{1}17^{1}$&$990=2^{1}3^{2}5^{1}11^{1}$&$1046=2^{1}523^{1}$&$1102=2^{1}19^{1}29^{1}$\\
$884=2^{2}13^{1}17^{1}$&$936=2^{3}3^{2}13^{1}$&$992=2^{5}31^{1}$&$1047=3^{1}349^{1}$&$1104=2^{4}3^{1}23^{1}$\\
$885=3^{1}5^{1}59^{1}$&$938=2^{1}7^{1}67^{1}$&$993=3^{1}331^{1}$&$1048=2^{3}131^{1}$&$1105=5^{1}13^{1}17^{1}$\\
$886=2^{1}443^{1}$&$939=3^{1}313^{1}$&$994=2^{1}7^{1}71^{1}$&$1050=2^{1}3^{1}5^{2}7^{1}$&$1106=2^{1}7^{1}79^{1}$\\
$888=2^{3}3^{1}37^{1}$&$940=2^{2}5^{1}47^{1}$&$995=5^{1}199^{1}$&$1052=2^{2}263^{1}$&$1107=3^{3}41^{1}$\\
$889=7^{1}127^{1}$&$942=2^{1}3^{1}157^{1}$&$996=2^{2}3^{1}83^{1}$&$1053=3^{4}13^{1}$&$1108=2^{2}277^{1}$\\
$890=2^{1}5^{1}89^{1}$&$943=23^{1}41^{1}$&$998=2^{1}499^{1}$&$1054=2^{1}17^{1}31^{1}$&$1110=2^{1}3^{1}5^{1}37^{1}$\\
$891=3^{4}11^{1}$&$944=2^{4}59^{1}$&$999=3^{3}37^{1}$&$1055=5^{1}211^{1}$&$1111=11^{1}101^{1}$\\
$892=2^{2}223^{1}$&$945=3^{3}5^{1}7^{1}$&$1000=2^{3}5^{3}$&$1056=2^{5}3^{1}11^{1}$&$1112=2^{3}139^{1}$\\
$893=19^{1}47^{1}$&$946=2^{1}11^{1}43^{1}$&$1001=7^{1}11^{1}13^{1}$&$1057=7^{1}151^{1}$&$1113=3^{1}7^{1}53^{1}$\\
$894=2^{1}3^{1}149^{1}$&$948=2^{2}3^{1}79^{1}$&$1002=2^{1}3^{1}167^{1}$&$1058=2^{1}23^{2}$&$1114=2^{1}557^{1}$\\
$895=5^{1}179^{1}$&$949=13^{1}73^{1}$&$1003=17^{1}59^{1}$&$1059=3^{1}353^{1}$&$1115=5^{1}223^{1}$\\
$896=2^{7}7^{1}$&$950=2^{1}5^{2}19^{1}$&$1004=2^{2}251^{1}$&$1060=2^{2}5^{1}53^{1}$&$1116=2^{2}3^{2}31^{1}$\\
$897=3^{1}13^{1}23^{1}$&$951=3^{1}317^{1}$&$1005=3^{1}5^{1}67^{1}$&$1062=2^{1}3^{2}59^{1}$&$1118=2^{1}13^{1}43^{1}$\\
$898=2^{1}449^{1}$&$952=2^{3}7^{1}17^{1}$&$1006=2^{1}503^{1}$&$1064=2^{3}7^{1}19^{1}$&$1119=3^{1}373^{1}$\\
$899=29^{1}31^{1}$&$954=2^{1}3^{2}53^{1}$&$1007=19^{1}53^{1}$&$1065=3^{1}5^{1}71^{1}$&$1120=2^{5}5^{1}7^{1}$\\
$900=2^{2}3^{2}5^{2}$&$955=5^{1}191^{1}$&$1008=2^{4}3^{2}7^{1}$&$1066=2^{1}13^{1}41^{1}$&$1121=19^{1}59^{1}$\\
$901=17^{1}53^{1}$&$956=2^{2}239^{1}$&$1010=2^{1}5^{1}101^{1}$&$1067=11^{1}97^{1}$&$1122=2^{1}3^{1}11^{1}17^{1}$\\
$902=2^{1}11^{1}41^{1}$&$957=3^{1}11^{1}29^{1}$&$1011=3^{1}337^{1}$&$1068=2^{2}3^{1}89^{1}$&$1124=2^{2}281^{1}$\\
$903=3^{1}7^{1}43^{1}$&$958=2^{1}479^{1}$&$1012=2^{2}11^{1}23^{1}$&$1070=2^{1}5^{1}107^{1}$&$1125=3^{2}5^{3}$\\
$904=2^{3}113^{1}$&$959=7^{1}137^{1}$&$1014=2^{1}3^{1}13^{2}$&$1071=3^{2}7^{1}17^{1}$&$1126=2^{1}563^{1}$\\
$905=5^{1}181^{1}$&$960=2^{6}3^{1}5^{1}$&$1015=5^{1}7^{1}29^{1}$&$1072=2^{4}67^{1}$&$1127=7^{2}23^{1}$\\
$906=2^{1}3^{1}151^{1}$&$961=31^{2}$&$1016=2^{3}127^{1}$&$1073=29^{1}37^{1}$&$1128=2^{3}3^{1}47^{1}$\\
$908=2^{2}227^{1}$&$962=2^{1}13^{1}37^{1}$&$1017=3^{2}113^{1}$&$1074=2^{1}3^{1}179^{1}$&$1130=2^{1}5^{1}113^{1}$\\

$1131=3^{1}13^{1}29^{1}$&$1183=7^{1}13^{2}$&$1239=3^{1}7^{1}59^{1}$&$1293=3^{1}431^{1}$&$1347=3^{1}449^{1}$\\
$1132=2^{2}283^{1}$&$1184=2^{5}37^{1}$&$1240=2^{3}5^{1}31^{1}$&$1294=2^{1}647^{1}$&$1348=2^{2}337^{1}$\\
$1133=11^{1}103^{1}$&$1185=3^{1}5^{1}79^{1}$&$1241=17^{1}73^{1}$&$1295=5^{1}7^{1}37^{1}$&$1349=19^{1}71^{1}$\\
$1134=2^{1}3^{4}7^{1}$&$1186=2^{1}593^{1}$&$1242=2^{1}3^{3}23^{1}$&$1296=2^{4}3^{4}$&$1350=2^{1}3^{3}5^{2}$\\
$1135=5^{1}227^{1}$&$1188=2^{2}3^{3}11^{1}$&$1243=11^{1}113^{1}$&$1298=2^{1}11^{1}59^{1}$&$1351=7^{1}193^{1}$\\
$1136=2^{4}71^{1}$&$1189=29^{1}41^{1}$&$1244=2^{2}311^{1}$&$1299=3^{1}433^{1}$&$1352=2^{3}13^{2}$\\
$1137=3^{1}379^{1}$&$1190=2^{1}5^{1}7^{1}17^{1}$&$1245=3^{1}5^{1}83^{1}$&$1300=2^{2}5^{2}13^{1}$&$1353=3^{1}11^{1}41^{1}$\\
$1138=2^{1}569^{1}$&$1191=3^{1}397^{1}$&$1246=2^{1}7^{1}89^{1}$&$1302=2^{1}3^{1}7^{1}31^{1}$&$1354=2^{1}677^{1}$\\
$1139=17^{1}67^{1}$&$1192=2^{3}149^{1}$&$1247=29^{1}43^{1}$&$1304=2^{3}163^{1}$&$1355=5^{1}271^{1}$\\
$1140=2^{2}3^{1}5^{1}19^{1}$&$1194=2^{1}3^{1}199^{1}$&$1248=2^{5}3^{1}13^{1}$&$1305=3^{2}5^{1}29^{1}$&$1356=2^{2}3^{1}113^{1}$\\
$1141=7^{1}163^{1}$&$1195=5^{1}239^{1}$&$1250=2^{1}5^{4}$&$1306=2^{1}653^{1}$&$1357=23^{1}59^{1}$\\
$1142=2^{1}571^{1}$&$1196=2^{2}13^{1}23^{1}$&$1251=3^{2}139^{1}$&$1308=2^{2}3^{1}109^{1}$&$1358=2^{1}7^{1}97^{1}$\\
$1143=3^{2}127^{1}$&$1197=3^{2}7^{1}19^{1}$&$1252=2^{2}313^{1}$&$1309=7^{1}11^{1}17^{1}$&$1359=3^{2}151^{1}$\\
$1144=2^{3}11^{1}13^{1}$&$1198=2^{1}599^{1}$&$1253=7^{1}179^{1}$&$1310=2^{1}5^{1}131^{1}$&$1360=2^{4}5^{1}17^{1}$\\
$1145=5^{1}229^{1}$&$1199=11^{1}109^{1}$&$1254=2^{1}3^{1}11^{1}19^{1}$&$1311=3^{1}19^{1}23^{1}$&$1362=2^{1}3^{1}227^{1}$\\
$1146=2^{1}3^{1}191^{1}$&$1200=2^{4}3^{1}5^{2}$&$1255=5^{1}251^{1}$&$1312=2^{5}41^{1}$&$1363=29^{1}47^{1}$\\
$1147=31^{1}37^{1}$&$1202=2^{1}601^{1}$&$1256=2^{3}157^{1}$&$1313=13^{1}101^{1}$&$1364=2^{2}11^{1}31^{1}$\\
$1148=2^{2}7^{1}41^{1}$&$1203=3^{1}401^{1}$&$1257=3^{1}419^{1}$&$1314=2^{1}3^{2}73^{1}$&$1365=3^{1}5^{1}7^{1}13^{1}$\\
$1149=3^{1}383^{1}$&$1204=2^{2}7^{1}43^{1}$&$1258=2^{1}17^{1}37^{1}$&$1315=5^{1}263^{1}$&$1366=2^{1}683^{1}$\\
$1150=2^{1}5^{2}23^{1}$&$1205=5^{1}241^{1}$&$1260=2^{2}3^{2}5^{1}7^{1}$&$1316=2^{2}7^{1}47^{1}$&$1368=2^{3}3^{2}19^{1}$\\
$1152=2^{7}3^{2}$&$1206=2^{1}3^{2}67^{1}$&$1261=13^{1}97^{1}$&$1317=3^{1}439^{1}$&$1369=37^{2}$\\
$1154=2^{1}577^{1}$&$1207=17^{1}71^{1}$&$1262=2^{1}631^{1}$&$1318=2^{1}659^{1}$&$1370=2^{1}5^{1}137^{1}$\\
$1155=3^{1}5^{1}7^{1}11^{1}$&$1208=2^{3}151^{1}$&$1263=3^{1}421^{1}$&$1320=2^{3}3^{1}5^{1}11^{1}$&$1371=3^{1}457^{1}$\\
$1156=2^{2}17^{2}$&$1209=3^{1}13^{1}31^{1}$&$1264=2^{4}79^{1}$&$1322=2^{1}661^{1}$&$1372=2^{2}7^{3}$\\
$1157=13^{1}89^{1}$&$1210=2^{1}5^{1}11^{2}$&$1265=5^{1}11^{1}23^{1}$&$1323=3^{3}7^{2}$&$1374=2^{1}3^{1}229^{1}$\\
$1158=2^{1}3^{1}193^{1}$&$1211=7^{1}173^{1}$&$1266=2^{1}3^{1}211^{1}$&$1324=2^{2}331^{1}$&$1375=5^{3}11^{1}$\\
$1159=19^{1}61^{1}$&$1212=2^{2}3^{1}101^{1}$&$1267=7^{1}181^{1}$&$1325=5^{2}53^{1}$&$1376=2^{5}43^{1}$\\
$1160=2^{3}5^{1}29^{1}$&$1214=2^{1}607^{1}$&$1268=2^{2}317^{1}$&$1326=2^{1}3^{1}13^{1}17^{1}$&$1377=3^{4}17^{1}$\\
$1161=3^{3}43^{1}$&$1215=3^{5}5^{1}$&$1269=3^{3}47^{1}$&$1328=2^{4}83^{1}$&$1378=2^{1}13^{1}53^{1}$\\
$1162=2^{1}7^{1}83^{1}$&$1216=2^{6}19^{1}$&$1270=2^{1}5^{1}127^{1}$&$1329=3^{1}443^{1}$&$1379=7^{1}197^{1}$\\
$1164=2^{2}3^{1}97^{1}$&$1218=2^{1}3^{1}7^{1}29^{1}$&$1271=31^{1}41^{1}$&$1330=2^{1}5^{1}7^{1}19^{1}$&$1380=2^{2}3^{1}5^{1}23^{1}$\\
$1165=5^{1}233^{1}$&$1219=23^{1}53^{1}$&$1272=2^{3}3^{1}53^{1}$&$1331=11^{3}$&$1382=2^{1}691^{1}$\\
$1166=2^{1}11^{1}53^{1}$&$1220=2^{2}5^{1}61^{1}$&$1273=19^{1}67^{1}$&$1332=2^{2}3^{2}37^{1}$&$1383=3^{1}461^{1}$\\
$1167=3^{1}389^{1}$&$1221=3^{1}11^{1}37^{1}$&$1274=2^{1}7^{2}13^{1}$&$1333=31^{1}43^{1}$&$1384=2^{3}173^{1}$\\
$1168=2^{4}73^{1}$&$1222=2^{1}13^{1}47^{1}$&$1275=3^{1}5^{2}17^{1}$&$1334=2^{1}23^{1}29^{1}$&$1385=5^{1}277^{1}$\\
$1169=7^{1}167^{1}$&$1224=2^{3}3^{2}17^{1}$&$1276=2^{2}11^{1}29^{1}$&$1335=3^{1}5^{1}89^{1}$&$1386=2^{1}3^{2}7^{1}11^{1}$\\
$1170=2^{1}3^{2}5^{1}13^{1}$&$1225=5^{2}7^{2}$&$1278=2^{1}3^{2}71^{1}$&$1336=2^{3}167^{1}$&$1387=19^{1}73^{1}$\\
$1172=2^{2}293^{1}$&$1226=2^{1}613^{1}$&$1280=2^{8}5^{1}$&$1337=7^{1}191^{1}$&$1388=2^{2}347^{1}$\\
$1173=3^{1}17^{1}23^{1}$&$1227=3^{1}409^{1}$&$1281=3^{1}7^{1}61^{1}$&$1338=2^{1}3^{1}223^{1}$&$1389=3^{1}463^{1}$\\
$1174=2^{1}587^{1}$&$1228=2^{2}307^{1}$&$1282=2^{1}641^{1}$&$1339=13^{1}103^{1}$&$1390=2^{1}5^{1}139^{1}$\\
$1175=5^{2}47^{1}$&$1230=2^{1}3^{1}5^{1}41^{1}$&$1284=2^{2}3^{1}107^{1}$&$1340=2^{2}5^{1}67^{1}$&$1391=13^{1}107^{1}$\\
$1176=2^{3}3^{1}7^{2}$&$1232=2^{4}7^{1}11^{1}$&$1285=5^{1}257^{1}$&$1341=3^{2}149^{1}$&$1392=2^{4}3^{1}29^{1}$\\
$1177=11^{1}107^{1}$&$1233=3^{2}137^{1}$&$1286=2^{1}643^{1}$&$1342=2^{1}11^{1}61^{1}$&$1393=7^{1}199^{1}$\\
$1178=2^{1}19^{1}31^{1}$&$1234=2^{1}617^{1}$&$1287=3^{2}11^{1}13^{1}$&$1343=17^{1}79^{1}$&$1394=2^{1}17^{1}41^{1}$\\
$1179=3^{2}131^{1}$&$1235=5^{1}13^{1}19^{1}$&$1288=2^{3}7^{1}23^{1}$&$1344=2^{6}3^{1}7^{1}$&$1395=3^{2}5^{1}31^{1}$\\
$1180=2^{2}5^{1}59^{1}$&$1236=2^{2}3^{1}103^{1}$&$1290=2^{1}3^{1}5^{1}43^{1}$&$1345=5^{1}269^{1}$&$1396=2^{2}349^{1}$\\
$1182=2^{1}3^{1}197^{1}$&$1238=2^{1}619^{1}$&$1292=2^{2}17^{1}19^{1}$&$1346=2^{1}673^{1}$&$1397=11^{1}127^{1}$\\

$1398=2^{1}3^{1}233^{1}$&$1455=3^{1}5^{1}97^{1}$&$1510=2^{1}5^{1}151^{1}$&$1564=2^{2}17^{1}23^{1}$&$1622=2^{1}811^{1}$\\
$1400=2^{3}5^{2}7^{1}$&$1456=2^{4}7^{1}13^{1}$&$1512=2^{3}3^{3}7^{1}$&$1565=5^{1}313^{1}$&$1623=3^{1}541^{1}$\\
$1401=3^{1}467^{1}$&$1457=31^{1}47^{1}$&$1513=17^{1}89^{1}$&$1566=2^{1}3^{3}29^{1}$&$1624=2^{3}7^{1}29^{1}$\\
$1402=2^{1}701^{1}$&$1458=2^{1}3^{6}$&$1514=2^{1}757^{1}$&$1568=2^{5}7^{2}$&$1625=5^{3}13^{1}$\\
$1403=23^{1}61^{1}$&$1460=2^{2}5^{1}73^{1}$&$1515=3^{1}5^{1}101^{1}$&$1569=3^{1}523^{1}$&$1626=2^{1}3^{1}271^{1}$\\
$1404=2^{2}3^{3}13^{1}$&$1461=3^{1}487^{1}$&$1516=2^{2}379^{1}$&$1570=2^{1}5^{1}157^{1}$&$1628=2^{2}11^{1}37^{1}$\\
$1405=5^{1}281^{1}$&$1462=2^{1}17^{1}43^{1}$&$1517=37^{1}41^{1}$&$1572=2^{2}3^{1}131^{1}$&$1629=3^{2}181^{1}$\\
$1406=2^{1}19^{1}37^{1}$&$1463=7^{1}11^{1}19^{1}$&$1518=2^{1}3^{1}11^{1}23^{1}$&$1573=11^{2}13^{1}$&$1630=2^{1}5^{1}163^{1}$\\
$1407=3^{1}7^{1}67^{1}$&$1464=2^{3}3^{1}61^{1}$&$1519=7^{2}31^{1}$&$1574=2^{1}787^{1}$&$1631=7^{1}233^{1}$\\
$1408=2^{7}11^{1}$&$1465=5^{1}293^{1}$&$1520=2^{4}5^{1}19^{1}$&$1575=3^{2}5^{2}7^{1}$&$1632=2^{5}3^{1}17^{1}$\\
$1410=2^{1}3^{1}5^{1}47^{1}$&$1466=2^{1}733^{1}$&$1521=3^{2}13^{2}$&$1576=2^{3}197^{1}$&$1633=23^{1}71^{1}$\\
$1411=17^{1}83^{1}$&$1467=3^{2}163^{1}$&$1522=2^{1}761^{1}$&$1577=19^{1}83^{1}$&$1634=2^{1}19^{1}43^{1}$\\
$1412=2^{2}353^{1}$&$1468=2^{2}367^{1}$&$1524=2^{2}3^{1}127^{1}$&$1578=2^{1}3^{1}263^{1}$&$1635=3^{1}5^{1}109^{1}$\\
$1413=3^{2}157^{1}$&$1469=13^{1}113^{1}$&$1525=5^{2}61^{1}$&$1580=2^{2}5^{1}79^{1}$&$1636=2^{2}409^{1}$\\
$1414=2^{1}7^{1}101^{1}$&$1470=2^{1}3^{1}5^{1}7^{2}$&$1526=2^{1}7^{1}109^{1}$&$1581=3^{1}17^{1}31^{1}$&$1638=2^{1}3^{2}7^{1}13^{1}$\\
$1415=5^{1}283^{1}$&$1472=2^{6}23^{1}$&$1527=3^{1}509^{1}$&$1582=2^{1}7^{1}113^{1}$&$1639=11^{1}149^{1}$\\
$1416=2^{3}3^{1}59^{1}$&$1473=3^{1}491^{1}$&$1528=2^{3}191^{1}$&$1584=2^{4}3^{2}11^{1}$&$1640=2^{3}5^{1}41^{1}$\\
$1417=13^{1}109^{1}$&$1474=2^{1}11^{1}67^{1}$&$1529=11^{1}139^{1}$&$1585=5^{1}317^{1}$&$1641=3^{1}547^{1}$\\
$1418=2^{1}709^{1}$&$1475=5^{2}59^{1}$&$1530=2^{1}3^{2}5^{1}17^{1}$&$1586=2^{1}13^{1}61^{1}$&$1642=2^{1}821^{1}$\\
$1419=3^{1}11^{1}43^{1}$&$1476=2^{2}3^{2}41^{1}$&$1532=2^{2}383^{1}$&$1587=3^{1}23^{2}$&$1643=31^{1}53^{1}$\\
$1420=2^{2}5^{1}71^{1}$&$1477=7^{1}211^{1}$&$1533=3^{1}7^{1}73^{1}$&$1588=2^{2}397^{1}$&$1644=2^{2}3^{1}137^{1}$\\
$1421=7^{2}29^{1}$&$1478=2^{1}739^{1}$&$1534=2^{1}13^{1}59^{1}$&$1589=7^{1}227^{1}$&$1645=5^{1}7^{1}47^{1}$\\
$1422=2^{1}3^{2}79^{1}$&$1479=3^{1}17^{1}29^{1}$&$1535=5^{1}307^{1}$&$1590=2^{1}3^{1}5^{1}53^{1}$&$1646=2^{1}823^{1}$\\
$1424=2^{4}89^{1}$&$1480=2^{3}5^{1}37^{1}$&$1536=2^{9}3^{1}$&$1591=37^{1}43^{1}$&$1647=3^{3}61^{1}$\\
$1425=3^{1}5^{2}19^{1}$&$1482=2^{1}3^{1}13^{1}19^{1}$&$1537=29^{1}53^{1}$&$1592=2^{3}199^{1}$&$1648=2^{4}103^{1}$\\
$1426=2^{1}23^{1}31^{1}$&$1484=2^{2}7^{1}53^{1}$&$1538=2^{1}769^{1}$&$1593=3^{3}59^{1}$&$1649=17^{1}97^{1}$\\
$1428=2^{2}3^{1}7^{1}17^{1}$&$1485=3^{3}5^{1}11^{1}$&$1539=3^{4}19^{1}$&$1594=2^{1}797^{1}$&$1650=2^{1}3^{1}5^{2}11^{1}$\\
$1430=2^{1}5^{1}11^{1}13^{1}$&$1486=2^{1}743^{1}$&$1540=2^{2}5^{1}7^{1}11^{1}$&$1595=5^{1}11^{1}29^{1}$&$1651=13^{1}127^{1}$\\
$1431=3^{3}53^{1}$&$1488=2^{4}3^{1}31^{1}$&$1541=23^{1}67^{1}$&$1596=2^{2}3^{1}7^{1}19^{1}$&$1652=2^{2}7^{1}59^{1}$\\
$1432=2^{3}179^{1}$&$1490=2^{1}5^{1}149^{1}$&$1542=2^{1}3^{1}257^{1}$&$1598=2^{1}17^{1}47^{1}$&$1653=3^{1}19^{1}29^{1}$\\
$1434=2^{1}3^{1}239^{1}$&$1491=3^{1}7^{1}71^{1}$&$1544=2^{3}193^{1}$&$1599=3^{1}13^{1}41^{1}$&$1654=2^{1}827^{1}$\\
$1435=5^{1}7^{1}41^{1}$&$1492=2^{2}373^{1}$&$1545=3^{1}5^{1}103^{1}$&$1600=2^{6}5^{2}$&$1655=5^{1}331^{1}$\\
$1436=2^{2}359^{1}$&$1494=2^{1}3^{2}83^{1}$&$1546=2^{1}773^{1}$&$1602=2^{1}3^{2}89^{1}$&$1656=2^{3}3^{2}23^{1}$\\
$1437=3^{1}479^{1}$&$1495=5^{1}13^{1}23^{1}$&$1547=7^{1}13^{1}17^{1}$&$1603=7^{1}229^{1}$&$1658=2^{1}829^{1}$\\
$1438=2^{1}719^{1}$&$1496=2^{3}11^{1}17^{1}$&$1548=2^{2}3^{2}43^{1}$&$1604=2^{2}401^{1}$&$1659=3^{1}7^{1}79^{1}$\\
$1440=2^{5}3^{2}5^{1}$&$1497=3^{1}499^{1}$&$1550=2^{1}5^{2}31^{1}$&$1605=3^{1}5^{1}107^{1}$&$1660=2^{2}5^{1}83^{1}$\\
$1441=11^{1}131^{1}$&$1498=2^{1}7^{1}107^{1}$&$1551=3^{1}11^{1}47^{1}$&$1606=2^{1}11^{1}73^{1}$&$1661=11^{1}151^{1}$\\
$1442=2^{1}7^{1}103^{1}$&$1500=2^{2}3^{1}5^{3}$&$1552=2^{4}97^{1}$&$1608=2^{3}3^{1}67^{1}$&$1662=2^{1}3^{1}277^{1}$\\
$1443=3^{1}13^{1}37^{1}$&$1501=19^{1}79^{1}$&$1554=2^{1}3^{1}7^{1}37^{1}$&$1610=2^{1}5^{1}7^{1}23^{1}$&$1664=2^{7}13^{1}$\\
$1444=2^{2}19^{2}$&$1502=2^{1}751^{1}$&$1555=5^{1}311^{1}$&$1611=3^{2}179^{1}$&$1665=3^{2}5^{1}37^{1}$\\
$1445=5^{1}17^{2}$&$1503=3^{2}167^{1}$&$1556=2^{2}389^{1}$&$1612=2^{2}13^{1}31^{1}$&$1666=2^{1}7^{2}17^{1}$\\
$1446=2^{1}3^{1}241^{1}$&$1504=2^{5}47^{1}$&$1557=3^{2}173^{1}$&$1614=2^{1}3^{1}269^{1}$&$1668=2^{2}3^{1}139^{1}$\\
$1448=2^{3}181^{1}$&$1505=5^{1}7^{1}43^{1}$&$1558=2^{1}19^{1}41^{1}$&$1615=5^{1}17^{1}19^{1}$&$1670=2^{1}5^{1}167^{1}$\\
$1449=3^{2}7^{1}23^{1}$&$1506=2^{1}3^{1}251^{1}$&$1560=2^{3}3^{1}5^{1}13^{1}$&$1616=2^{4}101^{1}$&$1671=3^{1}557^{1}$\\
$1450=2^{1}5^{2}29^{1}$&$1507=11^{1}137^{1}$&$1561=7^{1}223^{1}$&$1617=3^{1}7^{2}11^{1}$&$1672=2^{3}11^{1}19^{1}$\\
$1452=2^{2}3^{1}11^{2}$&$1508=2^{2}13^{1}29^{1}$&$1562=2^{1}11^{1}71^{1}$&$1618=2^{1}809^{1}$&$1673=7^{1}239^{1}$\\
$1454=2^{1}727^{1}$&$1509=3^{1}503^{1}$&$1563=3^{1}521^{1}$&$1620=2^{2}3^{4}5^{1}$&$1674=2^{1}3^{3}31^{1}$\\
$1675=5^{2}67^{1}$&$1728=2^{6}3^{3}$&$1781=13^{1}137^{1}$&$1835=5^{1}367^{1}$&$1890=2^{1}3^{3}5^{1}7^{1}$\\
$1676=2^{2}419^{1}$&$1729=7^{1}13^{1}19^{1}$&$1782=2^{1}3^{4}11^{1}$&$1836=2^{2}3^{3}17^{1}$&$1891=31^{1}61^{1}$\\
$1677=3^{1}13^{1}43^{1}$&$1730=2^{1}5^{1}173^{1}$&$1784=2^{3}223^{1}$&$1837=11^{1}167^{1}$&$1892=2^{2}11^{1}43^{1}$\\
$1678=2^{1}839^{1}$&$1731=3^{1}577^{1}$&$1785=3^{1}5^{1}7^{1}17^{1}$&$1838=2^{1}919^{1}$&$1893=3^{1}631^{1}$\\
$1679=23^{1}73^{1}$&$1732=2^{2}433^{1}$&$1786=2^{1}19^{1}47^{1}$&$1839=3^{1}613^{1}$&$1894=2^{1}947^{1}$\\
$1680=2^{4}3^{1}5^{1}7^{1}$&$1734=2^{1}3^{1}17^{2}$&$1788=2^{2}3^{1}149^{1}$&$1840=2^{4}5^{1}23^{1}$&$1895=5^{1}379^{1}$\\
$1681=41^{2}$&$1735=5^{1}347^{1}$&$1790=2^{1}5^{1}179^{1}$&$1841=7^{1}263^{1}$&$1896=2^{3}3^{1}79^{1}$\\
$1682=2^{1}29^{2}$&$1736=2^{3}7^{1}31^{1}$&$1791=3^{2}199^{1}$&$1842=2^{1}3^{1}307^{1}$&$1897=7^{1}271^{1}$\\
$1683=3^{2}11^{1}17^{1}$&$1737=3^{2}193^{1}$&$1792=2^{8}7^{1}$&$1843=19^{1}97^{1}$&$1898=2^{1}13^{1}73^{1}$\\
$1684=2^{2}421^{1}$&$1738=2^{1}11^{1}79^{1}$&$1793=11^{1}163^{1}$&$1844=2^{2}461^{1}$&$1899=3^{2}211^{1}$\\
$1685=5^{1}337^{1}$&$1739=37^{1}47^{1}$&$1794=2^{1}3^{1}13^{1}23^{1}$&$1845=3^{2}5^{1}41^{1}$&$1900=2^{2}5^{2}19^{1}$\\
$1686=2^{1}3^{1}281^{1}$&$1740=2^{2}3^{1}5^{1}29^{1}$&$1795=5^{1}359^{1}$&$1846=2^{1}13^{1}71^{1}$&$1902=2^{1}3^{1}317^{1}$\\
$1687=7^{1}241^{1}$&$1742=2^{1}13^{1}67^{1}$&$1796=2^{2}449^{1}$&$1848=2^{3}3^{1}7^{1}11^{1}$&$1903=11^{1}173^{1}$\\
$1688=2^{3}211^{1}$&$1743=3^{1}7^{1}83^{1}$&$1797=3^{1}599^{1}$&$1849=43^{2}$&$1904=2^{4}7^{1}17^{1}$\\
$1689=3^{1}563^{1}$&$1744=2^{4}109^{1}$&$1798=2^{1}29^{1}31^{1}$&$1850=2^{1}5^{2}37^{1}$&$1905=3^{1}5^{1}127^{1}$\\
$1690=2^{1}5^{1}13^{2}$&$1745=5^{1}349^{1}$&$1799=7^{1}257^{1}$&$1851=3^{1}617^{1}$&$1906=2^{1}953^{1}$\\
$1691=19^{1}89^{1}$&$1746=2^{1}3^{2}97^{1}$&$1800=2^{3}3^{2}5^{2}$&$1852=2^{2}463^{1}$&$1908=2^{2}3^{2}53^{1}$\\
$1692=2^{2}3^{2}47^{1}$&$1748=2^{2}19^{1}23^{1}$&$1802=2^{1}17^{1}53^{1}$&$1853=17^{1}109^{1}$&$1909=23^{1}83^{1}$\\
$1694=2^{1}7^{1}11^{2}$&$1749=3^{1}11^{1}53^{1}$&$1803=3^{1}601^{1}$&$1854=2^{1}3^{2}103^{1}$&$1910=2^{1}5^{1}191^{1}$\\
$1695=3^{1}5^{1}113^{1}$&$1750=2^{1}5^{3}7^{1}$&$1804=2^{2}11^{1}41^{1}$&$1855=5^{1}7^{1}53^{1}$&$1911=3^{1}7^{2}13^{1}$\\
$1696=2^{5}53^{1}$&$1751=17^{1}103^{1}$&$1805=5^{1}19^{2}$&$1856=2^{6}29^{1}$&$1912=2^{3}239^{1}$\\
$1698=2^{1}3^{1}283^{1}$&$1752=2^{3}3^{1}73^{1}$&$1806=2^{1}3^{1}7^{1}43^{1}$&$1857=3^{1}619^{1}$&$1914=2^{1}3^{1}11^{1}29^{1}$\\
$1700=2^{2}5^{2}17^{1}$&$1754=2^{1}877^{1}$&$1807=13^{1}139^{1}$&$1858=2^{1}929^{1}$&$1915=5^{1}383^{1}$\\
$1701=3^{5}7^{1}$&$1755=3^{3}5^{1}13^{1}$&$1808=2^{4}113^{1}$&$1859=11^{1}13^{2}$&$1916=2^{2}479^{1}$\\
$1702=2^{1}23^{1}37^{1}$&$1756=2^{2}439^{1}$&$1809=3^{3}67^{1}$&$1860=2^{2}3^{1}5^{1}31^{1}$&$1917=3^{3}71^{1}$\\
$1703=13^{1}131^{1}$&$1757=7^{1}251^{1}$&$1810=2^{1}5^{1}181^{1}$&$1862=2^{1}7^{2}19^{1}$&$1918=2^{1}7^{1}137^{1}$\\
$1704=2^{3}3^{1}71^{1}$&$1758=2^{1}3^{1}293^{1}$&$1812=2^{2}3^{1}151^{1}$&$1863=3^{4}23^{1}$&$1919=19^{1}101^{1}$\\
$1705=5^{1}11^{1}31^{1}$&$1760=2^{5}5^{1}11^{1}$&$1813=7^{2}37^{1}$&$1864=2^{3}233^{1}$&$1920=2^{7}3^{1}5^{1}$\\
$1706=2^{1}853^{1}$&$1761=3^{1}587^{1}$&$1814=2^{1}907^{1}$&$1865=5^{1}373^{1}$&$1921=17^{1}113^{1}$\\
$1707=3^{1}569^{1}$&$1762=2^{1}881^{1}$&$1815=3^{1}5^{1}11^{2}$&$1866=2^{1}3^{1}311^{1}$&$1922=2^{1}31^{2}$\\
$1708=2^{2}7^{1}61^{1}$&$1763=41^{1}43^{1}$&$1816=2^{3}227^{1}$&$1868=2^{2}467^{1}$&$1923=3^{1}641^{1}$\\
$1710=2^{1}3^{2}5^{1}19^{1}$&$1764=2^{2}3^{2}7^{2}$&$1817=23^{1}79^{1}$&$1869=3^{1}7^{1}89^{1}$&$1924=2^{2}13^{1}37^{1}$\\
$1711=29^{1}59^{1}$&$1765=5^{1}353^{1}$&$1818=2^{1}3^{2}101^{1}$&$1870=2^{1}5^{1}11^{1}17^{1}$&$1925=5^{2}7^{1}11^{1}$\\
$1712=2^{4}107^{1}$&$1766=2^{1}883^{1}$&$1819=17^{1}107^{1}$&$1872=2^{4}3^{2}13^{1}$&$1926=2^{1}3^{2}107^{1}$\\
$1713=3^{1}571^{1}$&$1767=3^{1}19^{1}31^{1}$&$1820=2^{2}5^{1}7^{1}13^{1}$&$1874=2^{1}937^{1}$&$1927=41^{1}47^{1}$\\
$1714=2^{1}857^{1}$&$1768=2^{3}13^{1}17^{1}$&$1821=3^{1}607^{1}$&$1875=3^{1}5^{4}$&$1928=2^{3}241^{1}$\\
$1715=5^{1}7^{3}$&$1769=29^{1}61^{1}$&$1822=2^{1}911^{1}$&$1876=2^{2}7^{1}67^{1}$&$1929=3^{1}643^{1}$\\
$1716=2^{2}3^{1}11^{1}13^{1}$&$1770=2^{1}3^{1}5^{1}59^{1}$&$1824=2^{5}3^{1}19^{1}$&$1878=2^{1}3^{1}313^{1}$&$1930=2^{1}5^{1}193^{1}$\\
$1717=17^{1}101^{1}$&$1771=7^{1}11^{1}23^{1}$&$1825=5^{2}73^{1}$&$1880=2^{3}5^{1}47^{1}$&$1932=2^{2}3^{1}7^{1}23^{1}$\\
$1718=2^{1}859^{1}$&$1772=2^{2}443^{1}$&$1826=2^{1}11^{1}83^{1}$&$1881=3^{2}11^{1}19^{1}$&$1934=2^{1}967^{1}$\\
$1719=3^{2}191^{1}$&$1773=3^{2}197^{1}$&$1827=3^{2}7^{1}29^{1}$&$1882=2^{1}941^{1}$&$1935=3^{2}5^{1}43^{1}$\\
$1720=2^{3}5^{1}43^{1}$&$1774=2^{1}887^{1}$&$1828=2^{2}457^{1}$&$1883=7^{1}269^{1}$&$1936=2^{4}11^{2}$\\
$1722=2^{1}3^{1}7^{1}41^{1}$&$1775=5^{2}71^{1}$&$1829=31^{1}59^{1}$&$1884=2^{2}3^{1}157^{1}$&$1937=13^{1}149^{1}$\\
$1724=2^{2}431^{1}$&$1776=2^{4}3^{1}37^{1}$&$1830=2^{1}3^{1}5^{1}61^{1}$&$1885=5^{1}13^{1}29^{1}$&$1938=2^{1}3^{1}17^{1}19^{1}$\\
$1725=3^{1}5^{2}23^{1}$&$1778=2^{1}7^{1}127^{1}$&$1832=2^{3}229^{1}$&$1886=2^{1}23^{1}41^{1}$&$1939=7^{1}277^{1}$\\
$1726=2^{1}863^{1}$&$1779=3^{1}593^{1}$&$1833=3^{1}13^{1}47^{1}$&$1887=3^{1}17^{1}37^{1}$&$1940=2^{2}5^{1}97^{1}$\\
$1727=11^{1}157^{1}$&$1780=2^{2}5^{1}89^{1}$&$1834=2^{1}7^{1}131^{1}$&$1888=2^{5}59^{1}$&$1941=3^{1}647^{1}$\\
$1942=2^{1}971^{1}$&$1995=3^{1}5^{1}7^{1}19^{1}$&$2050=2^{1}5^{2}41^{1}$&$2105=5^{1}421^{1}$&$2160=2^{4}3^{3}5^{1}$\\
$1943=29^{1}67^{1}$&$1996=2^{2}499^{1}$&$2051=7^{1}293^{1}$&$2106=2^{1}3^{4}13^{1}$&$2162=2^{1}23^{1}47^{1}$\\
$1944=2^{3}3^{5}$&$1998=2^{1}3^{3}37^{1}$&$2052=2^{2}3^{3}19^{1}$&$2107=7^{2}43^{1}$&$2163=3^{1}7^{1}103^{1}$\\
$1945=5^{1}389^{1}$&$2000=2^{4}5^{3}$&$2054=2^{1}13^{1}79^{1}$&$2108=2^{2}17^{1}31^{1}$&$2164=2^{2}541^{1}$\\
$1946=2^{1}7^{1}139^{1}$&$2001=3^{1}23^{1}29^{1}$&$2055=3^{1}5^{1}137^{1}$&$2109=3^{1}19^{1}37^{1}$&$2165=5^{1}433^{1}$\\
$1947=3^{1}11^{1}59^{1}$&$2002=2^{1}7^{1}11^{1}13^{1}$&$2056=2^{3}257^{1}$&$2110=2^{1}5^{1}211^{1}$&$2166=2^{1}3^{1}19^{2}$\\
$1948=2^{2}487^{1}$&$2004=2^{2}3^{1}167^{1}$&$2057=11^{2}17^{1}$&$2112=2^{6}3^{1}11^{1}$&$2167=11^{1}197^{1}$\\
$1950=2^{1}3^{1}5^{2}13^{1}$&$2005=5^{1}401^{1}$&$2058=2^{1}3^{1}7^{3}$&$2114=2^{1}7^{1}151^{1}$&$2168=2^{3}271^{1}$\\
$1952=2^{5}61^{1}$&$2006=2^{1}17^{1}59^{1}$&$2059=29^{1}71^{1}$&$2115=3^{2}5^{1}47^{1}$&$2169=3^{2}241^{1}$\\
$1953=3^{2}7^{1}31^{1}$&$2007=3^{2}223^{1}$&$2060=2^{2}5^{1}103^{1}$&$2116=2^{2}23^{2}$&$2170=2^{1}5^{1}7^{1}31^{1}$\\
$1954=2^{1}977^{1}$&$2008=2^{3}251^{1}$&$2061=3^{2}229^{1}$&$2117=29^{1}73^{1}$&$2171=13^{1}167^{1}$\\
$1955=5^{1}17^{1}23^{1}$&$2009=7^{2}41^{1}$&$2062=2^{1}1031^{1}$&$2118=2^{1}3^{1}353^{1}$&$2172=2^{2}3^{1}181^{1}$\\
$1956=2^{2}3^{1}163^{1}$&$2010=2^{1}3^{1}5^{1}67^{1}$&$2064=2^{4}3^{1}43^{1}$&$2119=13^{1}163^{1}$&$2173=41^{1}53^{1}$\\
$1957=19^{1}103^{1}$&$2012=2^{2}503^{1}$&$2065=5^{1}7^{1}59^{1}$&$2120=2^{3}5^{1}53^{1}$&$2174=2^{1}1087^{1}$\\
$1958=2^{1}11^{1}89^{1}$&$2013=3^{1}11^{1}61^{1}$&$2066=2^{1}1033^{1}$&$2121=3^{1}7^{1}101^{1}$&$2175=3^{1}5^{2}29^{1}$\\
$1959=3^{1}653^{1}$&$2014=2^{1}19^{1}53^{1}$&$2067=3^{1}13^{1}53^{1}$&$2122=2^{1}1061^{1}$&$2176=2^{7}17^{1}$\\
$1960=2^{3}5^{1}7^{2}$&$2015=5^{1}13^{1}31^{1}$&$2068=2^{2}11^{1}47^{1}$&$2123=11^{1}193^{1}$&$2177=7^{1}311^{1}$\\
$1961=37^{1}53^{1}$&$2016=2^{5}3^{2}7^{1}$&$2070=2^{1}3^{2}5^{1}23^{1}$&$2124=2^{2}3^{2}59^{1}$&$2178=2^{1}3^{2}11^{2}$\\
$1962=2^{1}3^{2}109^{1}$&$2018=2^{1}1009^{1}$&$2071=19^{1}109^{1}$&$2125=5^{3}17^{1}$&$2180=2^{2}5^{1}109^{1}$\\
$1963=13^{1}151^{1}$&$2019=3^{1}673^{1}$&$2072=2^{3}7^{1}37^{1}$&$2126=2^{1}1063^{1}$&$2181=3^{1}727^{1}$\\
$1964=2^{2}491^{1}$&$2020=2^{2}5^{1}101^{1}$&$2073=3^{1}691^{1}$&$2127=3^{1}709^{1}$&$2182=2^{1}1091^{1}$\\
$1965=3^{1}5^{1}131^{1}$&$2021=43^{1}47^{1}$&$2074=2^{1}17^{1}61^{1}$&$2128=2^{4}7^{1}19^{1}$&$2183=37^{1}59^{1}$\\
$1966=2^{1}983^{1}$&$2022=2^{1}3^{1}337^{1}$&$2075=5^{2}83^{1}$&$2130=2^{1}3^{1}5^{1}71^{1}$&$2184=2^{3}3^{1}7^{1}13^{1}$\\
$1967=7^{1}281^{1}$&$2023=7^{1}17^{2}$&$2076=2^{2}3^{1}173^{1}$&$2132=2^{2}13^{1}41^{1}$&$2185=5^{1}19^{1}23^{1}$\\
$1968=2^{4}3^{1}41^{1}$&$2024=2^{3}11^{1}23^{1}$&$2077=31^{1}67^{1}$&$2133=3^{3}79^{1}$&$2186=2^{1}1093^{1}$\\
$1969=11^{1}179^{1}$&$2025=3^{4}5^{2}$&$2078=2^{1}1039^{1}$&$2134=2^{1}11^{1}97^{1}$&$2187=3^{7}$\\
$1970=2^{1}5^{1}197^{1}$&$2026=2^{1}1013^{1}$&$2079=3^{3}7^{1}11^{1}$&$2135=5^{1}7^{1}61^{1}$&$2188=2^{2}547^{1}$\\
$1971=3^{3}73^{1}$&$2028=2^{2}3^{1}13^{2}$&$2080=2^{5}5^{1}13^{1}$&$2136=2^{3}3^{1}89^{1}$&$2189=11^{1}199^{1}$\\
$1972=2^{2}17^{1}29^{1}$&$2030=2^{1}5^{1}7^{1}29^{1}$&$2082=2^{1}3^{1}347^{1}$&$2138=2^{1}1069^{1}$&$2190=2^{1}3^{1}5^{1}73^{1}$\\
$1974=2^{1}3^{1}7^{1}47^{1}$&$2031=3^{1}677^{1}$&$2084=2^{2}521^{1}$&$2139=3^{1}23^{1}31^{1}$&$2191=7^{1}313^{1}$\\
$1975=5^{2}79^{1}$&$2032=2^{4}127^{1}$&$2085=3^{1}5^{1}139^{1}$&$2140=2^{2}5^{1}107^{1}$&$2192=2^{4}137^{1}$\\
$1976=2^{3}13^{1}19^{1}$&$2033=19^{1}107^{1}$&$2086=2^{1}7^{1}149^{1}$&$2142=2^{1}3^{2}7^{1}17^{1}$&$2193=3^{1}17^{1}43^{1}$\\
$1977=3^{1}659^{1}$&$2034=2^{1}3^{2}113^{1}$&$2088=2^{3}3^{2}29^{1}$&$2144=2^{5}67^{1}$&$2194=2^{1}1097^{1}$\\
$1978=2^{1}23^{1}43^{1}$&$2035=5^{1}11^{1}37^{1}$&$2090=2^{1}5^{1}11^{1}19^{1}$&$2145=3^{1}5^{1}11^{1}13^{1}$&$2195=5^{1}439^{1}$\\
$1980=2^{2}3^{2}5^{1}11^{1}$&$2036=2^{2}509^{1}$&$2091=3^{1}17^{1}41^{1}$&$2146=2^{1}29^{1}37^{1}$&$2196=2^{2}3^{2}61^{1}$\\
$1981=7^{1}283^{1}$&$2037=3^{1}7^{1}97^{1}$&$2092=2^{2}523^{1}$&$2147=19^{1}113^{1}$&$2197=13^{3}$\\
$1982=2^{1}991^{1}$&$2038=2^{1}1019^{1}$&$2093=7^{1}13^{1}23^{1}$&$2148=2^{2}3^{1}179^{1}$&$2198=2^{1}7^{1}157^{1}$\\
$1983=3^{1}661^{1}$&$2040=2^{3}3^{1}5^{1}17^{1}$&$2094=2^{1}3^{1}349^{1}$&$2149=7^{1}307^{1}$&$2199=3^{1}733^{1}$\\
$1984=2^{6}31^{1}$&$2041=13^{1}157^{1}$&$2095=5^{1}419^{1}$&$2150=2^{1}5^{2}43^{1}$&$2200=2^{3}5^{2}11^{1}$\\
$1985=5^{1}397^{1}$&$2042=2^{1}1021^{1}$&$2096=2^{4}131^{1}$&$2151=3^{2}239^{1}$&$2201=31^{1}71^{1}$\\
$1986=2^{1}3^{1}331^{1}$&$2043=3^{2}227^{1}$&$2097=3^{2}233^{1}$&$2152=2^{3}269^{1}$&$2202=2^{1}3^{1}367^{1}$\\
$1988=2^{2}7^{1}71^{1}$&$2044=2^{2}7^{1}73^{1}$&$2098=2^{1}1049^{1}$&$2154=2^{1}3^{1}359^{1}$&$2204=2^{2}19^{1}29^{1}$\\
$1989=3^{2}13^{1}17^{1}$&$2045=5^{1}409^{1}$&$2100=2^{2}3^{1}5^{2}7^{1}$&$2155=5^{1}431^{1}$&$2205=3^{2}5^{1}7^{2}$\\
$1990=2^{1}5^{1}199^{1}$&$2046=2^{1}3^{1}11^{1}31^{1}$&$2101=11^{1}191^{1}$&$2156=2^{2}7^{2}11^{1}$&$2206=2^{1}1103^{1}$\\
$1991=11^{1}181^{1}$&$2047=23^{1}89^{1}$&$2102=2^{1}1051^{1}$&$2157=3^{1}719^{1}$&$2208=2^{5}3^{1}23^{1}$\\
$1992=2^{3}3^{1}83^{1}$&$2048=2^{11}$&$2103=3^{1}701^{1}$&$2158=2^{1}13^{1}83^{1}$&$2209=47^{2}$\\
$1994=2^{1}997^{1}$&$2049=3^{1}683^{1}$&$2104=2^{3}263^{1}$&$2159=17^{1}127^{1}$&$2210=2^{1}5^{1}13^{1}17^{1}$\\
$2211=3^{1}11^{1}67^{1}$&$2264=2^{3}283^{1}$&$2320=2^{4}5^{1}29^{1}$&$2374=2^{1}1187^{1}$&$2430=2^{1}3^{5}5^{1}$\\
$2212=2^{2}7^{1}79^{1}$&$2265=3^{1}5^{1}151^{1}$&$2321=11^{1}211^{1}$&$2375=5^{3}19^{1}$&$2431=11^{1}13^{1}17^{1}$\\
$2214=2^{1}3^{3}41^{1}$&$2266=2^{1}11^{1}103^{1}$&$2322=2^{1}3^{3}43^{1}$&$2376=2^{3}3^{3}11^{1}$&$2432=2^{7}19^{1}$\\
$2215=5^{1}443^{1}$&$2268=2^{2}3^{4}7^{1}$&$2323=23^{1}101^{1}$&$2378=2^{1}29^{1}41^{1}$&$2433=3^{1}811^{1}$\\
$2216=2^{3}277^{1}$&$2270=2^{1}5^{1}227^{1}$&$2324=2^{2}7^{1}83^{1}$&$2379=3^{1}13^{1}61^{1}$&$2434=2^{1}1217^{1}$\\
$2217=3^{1}739^{1}$&$2271=3^{1}757^{1}$&$2325=3^{1}5^{2}31^{1}$&$2380=2^{2}5^{1}7^{1}17^{1}$&$2435=5^{1}487^{1}$\\
$2218=2^{1}1109^{1}$&$2272=2^{5}71^{1}$&$2326=2^{1}1163^{1}$&$2382=2^{1}3^{1}397^{1}$&$2436=2^{2}3^{1}7^{1}29^{1}$\\
$2219=7^{1}317^{1}$&$2274=2^{1}3^{1}379^{1}$&$2327=13^{1}179^{1}$&$2384=2^{4}149^{1}$&$2438=2^{1}23^{1}53^{1}$\\
$2220=2^{2}3^{1}5^{1}37^{1}$&$2275=5^{2}7^{1}13^{1}$&$2328=2^{3}3^{1}97^{1}$&$2385=3^{2}5^{1}53^{1}$&$2439=3^{2}271^{1}$\\
$2222=2^{1}11^{1}101^{1}$&$2276=2^{2}569^{1}$&$2329=17^{1}137^{1}$&$2386=2^{1}1193^{1}$&$2440=2^{3}5^{1}61^{1}$\\
$2223=3^{2}13^{1}19^{1}$&$2277=3^{2}11^{1}23^{1}$&$2330=2^{1}5^{1}233^{1}$&$2387=7^{1}11^{1}31^{1}$&$2442=2^{1}3^{1}11^{1}37^{1}$\\
$2224=2^{4}139^{1}$&$2278=2^{1}17^{1}67^{1}$&$2331=3^{2}7^{1}37^{1}$&$2388=2^{2}3^{1}199^{1}$&$2443=7^{1}349^{1}$\\
$2225=5^{2}89^{1}$&$2279=43^{1}53^{1}$&$2332=2^{2}11^{1}53^{1}$&$2390=2^{1}5^{1}239^{1}$&$2444=2^{2}13^{1}47^{1}$\\
$2226=2^{1}3^{1}7^{1}53^{1}$&$2280=2^{3}3^{1}5^{1}19^{1}$&$2334=2^{1}3^{1}389^{1}$&$2391=3^{1}797^{1}$&$2445=3^{1}5^{1}163^{1}$\\
$2227=17^{1}131^{1}$&$2282=2^{1}7^{1}163^{1}$&$2335=5^{1}467^{1}$&$2392=2^{3}13^{1}23^{1}$&$2446=2^{1}1223^{1}$\\
$2228=2^{2}557^{1}$&$2283=3^{1}761^{1}$&$2336=2^{5}73^{1}$&$2394=2^{1}3^{2}7^{1}19^{1}$&$2448=2^{4}3^{2}17^{1}$\\
$2229=3^{1}743^{1}$&$2284=2^{2}571^{1}$&$2337=3^{1}19^{1}41^{1}$&$2395=5^{1}479^{1}$&$2449=31^{1}79^{1}$\\
$2230=2^{1}5^{1}223^{1}$&$2285=5^{1}457^{1}$&$2338=2^{1}7^{1}167^{1}$&$2396=2^{2}599^{1}$&$2450=2^{1}5^{2}7^{2}$\\
$2231=23^{1}97^{1}$&$2286=2^{1}3^{2}127^{1}$&$2340=2^{2}3^{2}5^{1}13^{1}$&$2397=3^{1}17^{1}47^{1}$&$2451=3^{1}19^{1}43^{1}$\\
$2232=2^{3}3^{2}31^{1}$&$2288=2^{4}11^{1}13^{1}$&$2342=2^{1}1171^{1}$&$2398=2^{1}11^{1}109^{1}$&$2452=2^{2}613^{1}$\\
$2233=7^{1}11^{1}29^{1}$&$2289=3^{1}7^{1}109^{1}$&$2343=3^{1}11^{1}71^{1}$&$2400=2^{5}3^{1}5^{2}$&$2453=11^{1}223^{1}$\\
$2234=2^{1}1117^{1}$&$2290=2^{1}5^{1}229^{1}$&$2344=2^{3}293^{1}$&$2401=7^{4}$&$2454=2^{1}3^{1}409^{1}$\\
$2235=3^{1}5^{1}149^{1}$&$2291=29^{1}79^{1}$&$2345=5^{1}7^{1}67^{1}$&$2402=2^{1}1201^{1}$&$2455=5^{1}491^{1}$\\
$2236=2^{2}13^{1}43^{1}$&$2292=2^{2}3^{1}191^{1}$&$2346=2^{1}3^{1}17^{1}23^{1}$&$2403=3^{3}89^{1}$&$2456=2^{3}307^{1}$\\
$2238=2^{1}3^{1}373^{1}$&$2294=2^{1}31^{1}37^{1}$&$2348=2^{2}587^{1}$&$2404=2^{2}601^{1}$&$2457=3^{3}7^{1}13^{1}$\\
$2240=2^{6}5^{1}7^{1}$&$2295=3^{3}5^{1}17^{1}$&$2349=3^{4}29^{1}$&$2405=5^{1}13^{1}37^{1}$&$2458=2^{1}1229^{1}$\\
$2241=3^{3}83^{1}$&$2296=2^{3}7^{1}41^{1}$&$2350=2^{1}5^{2}47^{1}$&$2406=2^{1}3^{1}401^{1}$&$2460=2^{2}3^{1}5^{1}41^{1}$\\
$2242=2^{1}19^{1}59^{1}$&$2298=2^{1}3^{1}383^{1}$&$2352=2^{4}3^{1}7^{2}$&$2407=29^{1}83^{1}$&$2461=23^{1}107^{1}$\\
$2244=2^{2}3^{1}11^{1}17^{1}$&$2299=11^{2}19^{1}$&$2353=13^{1}181^{1}$&$2408=2^{3}7^{1}43^{1}$&$2462=2^{1}1231^{1}$\\
$2245=5^{1}449^{1}$&$2300=2^{2}5^{2}23^{1}$&$2354=2^{1}11^{1}107^{1}$&$2409=3^{1}11^{1}73^{1}$&$2463=3^{1}821^{1}$\\
$2246=2^{1}1123^{1}$&$2301=3^{1}13^{1}59^{1}$&$2355=3^{1}5^{1}157^{1}$&$2410=2^{1}5^{1}241^{1}$&$2464=2^{5}7^{1}11^{1}$\\
$2247=3^{1}7^{1}107^{1}$&$2302=2^{1}1151^{1}$&$2356=2^{2}19^{1}31^{1}$&$2412=2^{2}3^{2}67^{1}$&$2465=5^{1}17^{1}29^{1}$\\
$2248=2^{3}281^{1}$&$2303=7^{2}47^{1}$&$2358=2^{1}3^{2}131^{1}$&$2413=19^{1}127^{1}$&$2466=2^{1}3^{2}137^{1}$\\
$2249=13^{1}173^{1}$&$2304=2^{8}3^{2}$&$2359=7^{1}337^{1}$&$2414=2^{1}17^{1}71^{1}$&$2468=2^{2}617^{1}$\\
$2250=2^{1}3^{2}5^{3}$&$2305=5^{1}461^{1}$&$2360=2^{3}5^{1}59^{1}$&$2415=3^{1}5^{1}7^{1}23^{1}$&$2469=3^{1}823^{1}$\\
$2252=2^{2}563^{1}$&$2306=2^{1}1153^{1}$&$2361=3^{1}787^{1}$&$2416=2^{4}151^{1}$&$2470=2^{1}5^{1}13^{1}19^{1}$\\
$2253=3^{1}751^{1}$&$2307=3^{1}769^{1}$&$2362=2^{1}1181^{1}$&$2418=2^{1}3^{1}13^{1}31^{1}$&$2471=7^{1}353^{1}$\\
$2254=2^{1}7^{2}23^{1}$&$2308=2^{2}577^{1}$&$2363=17^{1}139^{1}$&$2419=41^{1}59^{1}$&$2472=2^{3}3^{1}103^{1}$\\
$2255=5^{1}11^{1}41^{1}$&$2310=2^{1}3^{1}5^{1}7^{1}11^{1}$&$2364=2^{2}3^{1}197^{1}$&$2420=2^{2}5^{1}11^{2}$&$2474=2^{1}1237^{1}$\\
$2256=2^{4}3^{1}47^{1}$&$2312=2^{3}17^{2}$&$2365=5^{1}11^{1}43^{1}$&$2421=3^{2}269^{1}$&$2475=3^{2}5^{2}11^{1}$\\
$2257=37^{1}61^{1}$&$2313=3^{2}257^{1}$&$2366=2^{1}7^{1}13^{2}$&$2422=2^{1}7^{1}173^{1}$&$2476=2^{2}619^{1}$\\
$2258=2^{1}1129^{1}$&$2314=2^{1}13^{1}89^{1}$&$2367=3^{2}263^{1}$&$2424=2^{3}3^{1}101^{1}$&$2478=2^{1}3^{1}7^{1}59^{1}$\\
$2259=3^{2}251^{1}$&$2315=5^{1}463^{1}$&$2368=2^{6}37^{1}$&$2425=5^{2}97^{1}$&$2479=37^{1}67^{1}$\\
$2260=2^{2}5^{1}113^{1}$&$2316=2^{2}3^{1}193^{1}$&$2369=23^{1}103^{1}$&$2426=2^{1}1213^{1}$&$2480=2^{4}5^{1}31^{1}$\\
$2261=7^{1}17^{1}19^{1}$&$2317=7^{1}331^{1}$&$2370=2^{1}3^{1}5^{1}79^{1}$&$2427=3^{1}809^{1}$&$2481=3^{1}827^{1}$\\
$2262=2^{1}3^{1}13^{1}29^{1}$&$2318=2^{1}19^{1}61^{1}$&$2372=2^{2}593^{1}$&$2428=2^{2}607^{1}$&$2482=2^{1}17^{1}73^{1}$\\
$2263=31^{1}73^{1}$&$2319=3^{1}773^{1}$&$2373=3^{1}7^{1}113^{1}$&$2429=7^{1}347^{1}$&$2483=13^{1}191^{1}$\\
$2484=2^{2}3^{3}23^{1}$&$2534=2^{1}7^{1}181^{1}$&$2587=13^{1}199^{1}$&$2640=2^{4}3^{1}5^{1}11^{1}$&$2697=3^{1}29^{1}31^{1}$\\
$2485=5^{1}7^{1}71^{1}$&$2535=3^{1}5^{1}13^{2}$&$2588=2^{2}647^{1}$&$2641=19^{1}139^{1}$&$2698=2^{1}19^{1}71^{1}$\\
$2486=2^{1}11^{1}113^{1}$&$2536=2^{3}317^{1}$&$2589=3^{1}863^{1}$&$2642=2^{1}1321^{1}$&$2700=2^{2}3^{3}5^{2}$\\
$2487=3^{1}829^{1}$&$2537=43^{1}59^{1}$&$2590=2^{1}5^{1}7^{1}37^{1}$&$2643=3^{1}881^{1}$&$2701=37^{1}73^{1}$\\
$2488=2^{3}311^{1}$&$2538=2^{1}3^{3}47^{1}$&$2592=2^{5}3^{4}$&$2644=2^{2}661^{1}$&$2702=2^{1}7^{1}193^{1}$\\
$2489=19^{1}131^{1}$&$2540=2^{2}5^{1}127^{1}$&$2594=2^{1}1297^{1}$&$2645=5^{1}23^{2}$&$2703=3^{1}17^{1}53^{1}$\\
$2490=2^{1}3^{1}5^{1}83^{1}$&$2541=3^{1}7^{1}11^{2}$&$2595=3^{1}5^{1}173^{1}$&$2646=2^{1}3^{3}7^{2}$&$2704=2^{4}13^{2}$\\
$2491=47^{1}53^{1}$&$2542=2^{1}31^{1}41^{1}$&$2596=2^{2}11^{1}59^{1}$&$2648=2^{3}331^{1}$&$2705=5^{1}541^{1}$\\
$2492=2^{2}7^{1}89^{1}$&$2544=2^{4}3^{1}53^{1}$&$2597=7^{2}53^{1}$&$2649=3^{1}883^{1}$&$2706=2^{1}3^{1}11^{1}41^{1}$\\
$2493=3^{2}277^{1}$&$2545=5^{1}509^{1}$&$2598=2^{1}3^{1}433^{1}$&$2650=2^{1}5^{2}53^{1}$&$2708=2^{2}677^{1}$\\
$2494=2^{1}29^{1}43^{1}$&$2546=2^{1}19^{1}67^{1}$&$2599=23^{1}113^{1}$&$2651=11^{1}241^{1}$&$2709=3^{2}7^{1}43^{1}$\\
$2495=5^{1}499^{1}$&$2547=3^{2}283^{1}$&$2600=2^{3}5^{2}13^{1}$&$2652=2^{2}3^{1}13^{1}17^{1}$&$2710=2^{1}5^{1}271^{1}$\\
$2496=2^{6}3^{1}13^{1}$&$2548=2^{2}7^{2}13^{1}$&$2601=3^{2}17^{2}$&$2653=7^{1}379^{1}$&$2712=2^{3}3^{1}113^{1}$\\
$2497=11^{1}227^{1}$&$2550=2^{1}3^{1}5^{2}17^{1}$&$2602=2^{1}1301^{1}$&$2654=2^{1}1327^{1}$&$2714=2^{1}23^{1}59^{1}$\\
$2498=2^{1}1249^{1}$&$2552=2^{3}11^{1}29^{1}$&$2603=19^{1}137^{1}$&$2655=3^{2}5^{1}59^{1}$&$2715=3^{1}5^{1}181^{1}$\\
$2499=3^{1}7^{2}17^{1}$&$2553=3^{1}23^{1}37^{1}$&$2604=2^{2}3^{1}7^{1}31^{1}$&$2656=2^{5}83^{1}$&$2716=2^{2}7^{1}97^{1}$\\
$2500=2^{2}5^{4}$&$2554=2^{1}1277^{1}$&$2605=5^{1}521^{1}$&$2658=2^{1}3^{1}443^{1}$&$2717=11^{1}13^{1}19^{1}$\\
$2501=41^{1}61^{1}$&$2555=5^{1}7^{1}73^{1}$&$2606=2^{1}1303^{1}$&$2660=2^{2}5^{1}7^{1}19^{1}$&$2718=2^{1}3^{2}151^{1}$\\
$2502=2^{1}3^{2}139^{1}$&$2556=2^{2}3^{2}71^{1}$&$2607=3^{1}11^{1}79^{1}$&$2661=3^{1}887^{1}$&$2720=2^{5}5^{1}17^{1}$\\
$2504=2^{3}313^{1}$&$2558=2^{1}1279^{1}$&$2608=2^{4}163^{1}$&$2662=2^{1}11^{3}$&$2721=3^{1}907^{1}$\\
$2505=3^{1}5^{1}167^{1}$&$2559=3^{1}853^{1}$&$2610=2^{1}3^{2}5^{1}29^{1}$&$2664=2^{3}3^{2}37^{1}$&$2722=2^{1}1361^{1}$\\
$2506=2^{1}7^{1}179^{1}$&$2560=2^{9}5^{1}$&$2611=7^{1}373^{1}$&$2665=5^{1}13^{1}41^{1}$&$2723=7^{1}389^{1}$\\
$2507=23^{1}109^{1}$&$2561=13^{1}197^{1}$&$2612=2^{2}653^{1}$&$2666=2^{1}31^{1}43^{1}$&$2724=2^{2}3^{1}227^{1}$\\
$2508=2^{2}3^{1}11^{1}19^{1}$&$2562=2^{1}3^{1}7^{1}61^{1}$&$2613=3^{1}13^{1}67^{1}$&$2667=3^{1}7^{1}127^{1}$&$2725=5^{2}109^{1}$\\
$2509=13^{1}193^{1}$&$2563=11^{1}233^{1}$&$2614=2^{1}1307^{1}$&$2668=2^{2}23^{1}29^{1}$&$2726=2^{1}29^{1}47^{1}$\\
$2510=2^{1}5^{1}251^{1}$&$2564=2^{2}641^{1}$&$2615=5^{1}523^{1}$&$2669=17^{1}157^{1}$&$2727=3^{3}101^{1}$\\
$2511=3^{4}31^{1}$&$2565=3^{3}5^{1}19^{1}$&$2616=2^{3}3^{1}109^{1}$&$2670=2^{1}3^{1}5^{1}89^{1}$&$2728=2^{3}11^{1}31^{1}$\\
$2512=2^{4}157^{1}$&$2566=2^{1}1283^{1}$&$2618=2^{1}7^{1}11^{1}17^{1}$&$2672=2^{4}167^{1}$&$2730=2^{1}3^{1}5^{1}7^{1}13^{1}$\\
$2513=7^{1}359^{1}$&$2567=17^{1}151^{1}$&$2619=3^{3}97^{1}$&$2673=3^{5}11^{1}$&$2732=2^{2}683^{1}$\\
$2514=2^{1}3^{1}419^{1}$&$2568=2^{3}3^{1}107^{1}$&$2620=2^{2}5^{1}131^{1}$&$2674=2^{1}7^{1}191^{1}$&$2733=3^{1}911^{1}$\\
$2515=5^{1}503^{1}$&$2569=7^{1}367^{1}$&$2622=2^{1}3^{1}19^{1}23^{1}$&$2675=5^{2}107^{1}$&$2734=2^{1}1367^{1}$\\
$2516=2^{2}17^{1}37^{1}$&$2570=2^{1}5^{1}257^{1}$&$2623=43^{1}61^{1}$&$2676=2^{2}3^{1}223^{1}$&$2735=5^{1}547^{1}$\\
$2517=3^{1}839^{1}$&$2571=3^{1}857^{1}$&$2624=2^{6}41^{1}$&$2678=2^{1}13^{1}103^{1}$&$2736=2^{4}3^{2}19^{1}$\\
$2518=2^{1}1259^{1}$&$2572=2^{2}643^{1}$&$2625=3^{1}5^{3}7^{1}$&$2679=3^{1}19^{1}47^{1}$&$2737=7^{1}17^{1}23^{1}$\\
$2519=11^{1}229^{1}$&$2573=31^{1}83^{1}$&$2626=2^{1}13^{1}101^{1}$&$2680=2^{3}5^{1}67^{1}$&$2738=2^{1}37^{2}$\\
$2520=2^{3}3^{2}5^{1}7^{1}$&$2574=2^{1}3^{2}11^{1}13^{1}$&$2627=37^{1}71^{1}$&$2681=7^{1}383^{1}$&$2739=3^{1}11^{1}83^{1}$\\
$2522=2^{1}13^{1}97^{1}$&$2575=5^{2}103^{1}$&$2628=2^{2}3^{2}73^{1}$&$2682=2^{1}3^{2}149^{1}$&$2740=2^{2}5^{1}137^{1}$\\
$2523=3^{1}29^{2}$&$2576=2^{4}7^{1}23^{1}$&$2629=11^{1}239^{1}$&$2684=2^{2}11^{1}61^{1}$&$2742=2^{1}3^{1}457^{1}$\\
$2524=2^{2}631^{1}$&$2577=3^{1}859^{1}$&$2630=2^{1}5^{1}263^{1}$&$2685=3^{1}5^{1}179^{1}$&$2743=13^{1}211^{1}$\\
$2525=5^{2}101^{1}$&$2578=2^{1}1289^{1}$&$2631=3^{1}877^{1}$&$2686=2^{1}17^{1}79^{1}$&$2744=2^{3}7^{3}$\\
$2526=2^{1}3^{1}421^{1}$&$2580=2^{2}3^{1}5^{1}43^{1}$&$2632=2^{3}7^{1}47^{1}$&$2688=2^{7}3^{1}7^{1}$&$2745=3^{2}5^{1}61^{1}$\\
$2527=7^{1}19^{2}$&$2581=29^{1}89^{1}$&$2634=2^{1}3^{1}439^{1}$&$2690=2^{1}5^{1}269^{1}$&$2746=2^{1}1373^{1}$\\
$2528=2^{5}79^{1}$&$2582=2^{1}1291^{1}$&$2635=5^{1}17^{1}31^{1}$&$2691=3^{2}13^{1}23^{1}$&$2747=41^{1}67^{1}$\\
$2529=3^{2}281^{1}$&$2583=3^{2}7^{1}41^{1}$&$2636=2^{2}659^{1}$&$2692=2^{2}673^{1}$&$2748=2^{2}3^{1}229^{1}$\\
$2530=2^{1}5^{1}11^{1}23^{1}$&$2584=2^{3}17^{1}19^{1}$&$2637=3^{2}293^{1}$&$2694=2^{1}3^{1}449^{1}$&$2750=2^{1}5^{3}11^{1}$\\
$2532=2^{2}3^{1}211^{1}$&$2585=5^{1}11^{1}47^{1}$&$2638=2^{1}1319^{1}$&$2695=5^{1}7^{2}11^{1}$&$2751=3^{1}7^{1}131^{1}$\\
$2533=17^{1}149^{1}$&$2586=2^{1}3^{1}431^{1}$&$2639=7^{1}13^{1}29^{1}$&$2696=2^{3}337^{1}$&$2752=2^{6}43^{1}$\\
\end{longtable}}
\section{Tavola pitagorica}
\label{sec:TavolaPitagorica}

%\begin{table}[!ht]
%\centering
%\renewcommand\arraystretch{1.9}
%\begin{tabular}{*{11}{|m{.5cm}}|}
%\cline{2-11}
%\multicolumn{1}{c|} {}& 1 & 2 & 3 & 4 & 5 & 6 & 7 & 8 & 9 & 10 \\\hline
%1 & 1 & 2 & 3 & 4 & 5 & 6 & 7 & 8 & 9 & 10 \\\hline
%2 & 2 & 4 & 6 & 8 & 10 & 12 & 14 & 16 & 18 & 20 \\\hline
%3 & 3 & 6 & 9 & 12 & 15 & 18 & 21 & 24 & 27 & 30 \\\hline
%4 & 4 & 8 & 12 & 16 & 20 & 24 & 28 & 32 & 36 & 40 \\\hline
%5 & 5 & 10 & 15 & 20 & 25 & 30 & 35 & 40 & 45 & 50 \\\hline
%6 & 6 & 12 & 18 & 24 & 30 & 36 & 42 & 48 & 54 & 60 \\\hline
%7 & 7 & 14 & 21 & 28 & 35 & 42 & 49 & 56 & 63 & 70 \\\hline
%8 & 8 & 16 & 24 & 32 & 40 & 48 & 56 & 64 & 72 & 80 \\\hline
%9 & 9 & 18 & 27 & 36 & 45 & 54 & 63 & 72 & 81 & 90 \\\hline
%10 & 10 & 20 & 30 & 40 & 50 & 60 & 70 & 80 & 90 & 100 \\\hline
%\end{tabular} 
%\caption{Tavola pitagorica}
%\label{tab:Tavolapitagorica}
%\end{table}
\begin{table}[H]
\centering
%codice Enrico Gregorio Guit
\def\mybox#1{\fbox{\kern.5cm
  \vrule height .5cm depth .5cm width 0pt \makebox(0,0){#1}%
  \kern.5cm}}
\def\riga#1{%
  #1&\number\numexpr#1*1\relax
    &\number\numexpr#1*2\relax
    &\number\numexpr#1*3\relax
    &\number\numexpr#1*4\relax
    &\number\numexpr#1*5\relax
    &\number\numexpr#1*6\relax
    &\number\numexpr#1*7\relax
    &\number\numexpr#1*8\relax
    &\number\numexpr#1*9\relax
    &\number\numexpr#1*10\relax}
\leavevmode\vbox{\fboxsep=0pt \offinterlineskip
\halign{&\mybox{#}\kern-\fboxrule\cr
\omit&1&2&3&4&5&6&7&8&9&10\cr\noalign{\kern-\fboxrule}
\riga{1}\cr\noalign{\kern-\fboxrule}
\riga{2}\cr\noalign{\kern-\fboxrule}
\riga{3}\cr\noalign{\kern-\fboxrule}
\riga{4}\cr\noalign{\kern-\fboxrule}
\riga{5}\cr\noalign{\kern-\fboxrule}
\riga{6}\cr\noalign{\kern-\fboxrule}
\riga{7}\cr\noalign{\kern-\fboxrule}
\riga{8}\cr\noalign{\kern-\fboxrule}
\riga{9}\cr\noalign{\kern-\fboxrule}
\riga{10}\cr\noalign{\kern-\fboxrule}
}} 
\caption{Tavola pitagorica}
\label{tab:tavolepitagorica}
\end{table}\index{Tavola!pitagorica}

\chapter{Altre basi}
\section{Somma binaria}
\begin{center}
	\additiontable{2}
\end{center}\index{Base!binaria!somma}
\section{Prodotto binario}
\begin{center}
	\multiplicationtable{2}
\end{center}\index{Base!binaria!prodotto}
\section{Somma ottale}
\begin{center}
\additiontable{8}
\end{center}\index{Base!ottale!somma}
\section{Prodotto ottale}
\begin{center}
\multiplicationtable{8}
\end{center}\index{Base!ottale!prodotto}
\section{Somma esadecimale}
\begin{center}
	\additiontable{16}
\end{center}\index{Base!esadecimale!somma}
\section{Prodotto esadecimale}
\begin{center}
	\multiplicationtable{16}
\end{center}\index{Base!esadecimale!prodotto}
% !TeX encoding = UTF-8
% !TeX spellcheck = it_IT
% !TeX root = formulario.tex
% 18/11/2017 :: 9:18:03 :: 
\chapter{Proporzioni}
\section{Vocabolario}
\[a:b=c:d\]
\begin{description}
	\item[a,c] antecedenti
	\item[b,d] conseguenti
	\item[a,d] estremi
	\item[b,c] medi
\end{description}\index{Proporzione!antecedenti}\index{Proporzione!conseguenti}\index{Proporzione!estremo}\index{Proporzione!medio}
\section{Proprietà fondamentale proporzioni}
\begin{equation}                       
	\text{Se\;}a:b=c:d\quad\Rightarrow\quad a\cdot d=b\cdot c
\end{equation}\index{Proporzione!proprietà fondamentale}
\section{Proprietà dell'invertire}
\begin{equation}                       
\text{Se\;}a:b=c:d\quad\Rightarrow\quad a\cdot b:a=d:c
\end{equation}\index{Proporzione!invertire}
\section{Proprietà del permutare}
\begin{equation}                       
\text{Se\;}a:b=c:d\quad\Rightarrow\quad \begin{cases}
d:b=c:a\\
a:c=b:d\\
d:c=b:d
\end{cases}
\end{equation}\index{Proporzione!permutare}
\section{Proprietà del comporre}
\begin{equation}                       
\text{Se\;}a:b=c:d\quad\Rightarrow\quad \begin{cases}
(a+b):b=(c+d):d\\
(a+b):a=(c+d):c
\end{cases}
\end{equation}\index{Proporzione!comporre}
\section{Proprietà dello scomporre}
\begin{equation}                       
\text{Se\;}a:b=c:d\quad\Rightarrow\quad \begin{cases}
(a-b):b=(c-d):d\qquad a>b\quad c>d\\
(a-b):a=(c-d):c
\end{cases}
\end{equation}\index{Proporzione!scomporre}
\section{Estremo incognito}
\begin{equation}                       
\text{Se\;}a:b=c:x\quad\Rightarrow\quad x=\dfrac{b\cdot c}{a}
\end{equation}\index{Proporzione!estremo incognito}
\section{Medio incognito}
\begin{equation}                       
\text{Se\;}a:x=c:d\quad\Rightarrow\quad x=\dfrac{a\cdot d}{c}
\end{equation}\index{Proporzione!medio incognito}
\section{Medio proporzionale}
\begin{equation}                       
\text{Se\;}a:x=x:d\quad\Rightarrow\quad x=\sqrt{a\cdot d}
\end{equation}\index{Proporzione!medio proporzionale}
% !TeX encoding = UTF-8
% !TeX spellcheck = it_IT
% !TeX root = formulario.tex
\chapter{Catena di rapporti}
\section{Definizioni e proprietà}
\begin{defn}[Catena di rapporti]
	\begin{equation*}
	a:b=c:d=e:f
	\end{equation*}\index{Catena di rapporti}
\end{defn}
\begin{prop}[Proprietà del comporre]
\begin{equation*}                       
\text{Se\;}x:a=y:b=z:c\quad\Rightarrow\quad \begin{cases}
(x+y+z):(a+b+c)=x:a\\
(x+y+z):(a+b+c)=y:b\\
(x+y+z):(a+b+c)=z:c
\end{cases}
\end{equation*}\index{Catena di rapporti!comporre}
\end{prop}



% !TeX encoding = UTF-8
% !TeX spellcheck = it_IT
% !TeX root = formulario.tex
% 16/11/2017 :: 18:16:18 :: 
\chapter{La percentuale}
\section{Vocabolario}
\begin{description}
	\item[r] ragione o tasso
	\item[p] percentuale
	\item[T] totale
\end{description}
\begin{align}
r:100=&p:T\\
p=&\dfrac{T\cdot r}{100}\\
r=&\dfrac{p\cdot100}{T}\\
T=&\dfrac{p\cdot100}{r}\\
\end{align}\index{Percentuale}
% !TeX encoding = UTF-8
% !TeX spellcheck = it_IT
% !TeX root = formulario.tex
% 18/11/2017 :: 9:18:03 :: 

\chapter{Sconto percentuale}
\section{Vocabolario}
\begin{description}
	\item[Pi] Prezzo iniziale
	\item[r] Tasso di sconto
	\item[Sc] Sconto effettivo
	\item[Ps] Prezzo scontato
\end{description}\section{Prezzo scontato}
\begin{equation}
Ps=Pi-Sc
\end{equation}\index{Sconto!prezzo scontato}
\section{Sconto effettivo}
\begin{equation}
Sc=Pi-Ps
\end{equation}\index{Sconto!effettivo}

\section{Sconto effettivo percentuale}
\begin{equation}
Sc=\dfrac{Pi\cdot r}{100}
\end{equation}\index{Sconto!effettivo}
\section{Prezzo iniziale}
\begin{equation}
Pi=\dfrac{Sc\cdot 100}{Pi}
\end{equation}\index{Sconto!prezzo iniziale}

\section{Tasso di sconto}
\begin{align}
r=&\dfrac{Sc\cdot 100}{Pi}\\
r=&\dfrac{(Pi-Ps)\cdot 100}{Pi}
\end{align}\index{Sconto!tasso}
\chapter{Geometria}
\section{Triangolo}
\begin{tcolorbox}[sidebyside,righthand width=7cm,colback=white,colframe=white,fonttitle=\bfseries	]
\includestandalone[width=5.5cm]{geometria/triangoloSca}
\tcblower
	\begin{align*}
		A=&\dfrac{b\cdot h}{2}&h=&\dfrac{2A}{b}\\
		b=&\dfrac{2A}{h}&P=&a+b+c\\
	p=&\frac{P}{2}&	A=&\sqrt{p(p-a)(p-b)(p-c)}
	\end{align*}
\end{tcolorbox}\index{Triangolo!area}\index{Triangolo!perimetro}\index{Triangolo!altezza}\index{Erone!formula}
\section{Triangolo rettangolo}
\begin{tcolorbox}[sidebyside,righthand width=9cm,colback=white,colframe=white,fonttitle=\bfseries	]
	\includestandalone[width=5.5cm]{geometria/triangoloRet}
	\tcblower
	\begin{align*}
	A=&\dfrac{b\cdot c}{2}&h=&\dfrac{b\cdot c}{a}\\
	a^2=&b^2+c^2&a=&\sqrt{b^2+c^2}\\
	b=&\sqrt{a^2-c^2}&c=&\sqrt{a^2-b^2}\\
	b^2=&p_1a&c^2=&p_2a\\
	h^2=&p_1p_2&h=&\sqrt{p_1p_2}\\
	a=&p_1+p_2
	\end{align*}
\end{tcolorbox}\index{Triangolo!rettangolo!area}\index{Triangolo!rettangolo!altezza}\index{Triangolo!rettangolo!cateto}\index{Triangolo!rettangolo!ipotenusa}\index{Pitagora!teorema}\index{Euclide!primo teorema}\index{Euclide!secondo teorema}
\section{Triangolo equilatero}
\begin{tcolorbox}[sidebyside,righthand width=9cm,colback=white,colframe=white,fonttitle=\bfseries	]
	\includestandalone[width=4.5cm]{geometria/triangoloiso}
	\tcblower
	\begin{align*}
	A=&\dfrac{h\cdot l}{2}&h=&\dfrac{\sqrt{3}}{2}l\\
	A=&\dfrac{\sqrt{3}}{4}l^2&l=&\dfrac{2\sqrt{3}}{3}h\\
	A=&\dfrac{\sqrt{3}}{3}h^2&P=&3l
	\end{align*}\index{Triangolo!equilatero!area}\index{Triangolo!equilatero!altezza}\index{Triangolo!equilatero!lato}\index{Triangolo!equilatero!perimetro}
\end{tcolorbox}
\section{Triangolo isoscele}
\begin{tcolorbox}[sidebyside,righthand width=9cm,colback=white,colframe=white,fonttitle=\bfseries	]
	\includestandalone[width=4.5cm]{geometria/triangoloequi}
	\tcblower
	\begin{align*}
	A=&\dfrac{h\cdot b}{2}&h=&\dfrac{2A}{b}\\
	b=&\dfrac{2A}{h}&P=&b+2l\\
	l=&\sqrt{h^2+\frac{b^2}{4}}	
	\end{align*}\index{Triangolo!isoscele!area}\index{Triangolo!isoscele!altezza}\index{Triangolo!isoscele!lato}\index{Triangolo!isoscele!perimetro}
\end{tcolorbox}
\section{Trapezio}
\begin{tcolorbox}[sidebyside,righthand width=9cm,colback=white,colframe=white,fonttitle=\bfseries	]
	\includestandalone[width=5.5cm]{geometria/trapezio}
	\tcblower
	\begin{align*}
	A=&\dfrac{B+b}{2}h\\
	P=&B+b+l_1+l_2
	\end{align*}
\end{tcolorbox}\index{Trapezio}\index{Trapezio!area}\index{Trapezio!perimetro}

\section{Trapezio isoscele}
\begin{tcolorbox}[sidebyside,righthand width=9cm,colback=white,colframe=white,fonttitle=\bfseries	]
	\includestandalone[width=5.5cm]{geometria/trapezioiso}
	\tcblower
	\begin{align*}
	P=&B+b+2l\\
	l=&\sqrt{h^2+\left(\frac{B-b}{2}\right)^2}\\
	h=&\sqrt{l^2-\left(\frac{B-b}{2}\right)^2}\\
	\frac{B-b}{2}=&\sqrt{l^2-h^2}
	\end{align*}
\end{tcolorbox}\index{Trapezio!isoscele}\index{Trapezio!isoscele!perimetro}\index{Trapezio!isoscele!lato obliquo}\index{Trapezio!isoscele!altezza}
\section{Trapezio rettangolo}
\begin{tcolorbox}[sidebyside,righthand width=9cm,colback=white,colframe=white,fonttitle=\bfseries	]
	\includestandalone[width=5.5cm]{geometria/trapezioret}
	\tcblower
	\begin{align*}
	l=&\sqrt{(B-b)^2+h^2}&\\
	h=&\sqrt{l^2-(B-b)^2}&\\
	B-b=&\sqrt{l^2-h^2}&\\
	D=&\sqrt{h^2+B^2}&d=&\sqrt{h^2+b^2}\\
	B=&\sqrt{D^2-h^2}&b=&\sqrt{d^2-h^2}\\
	h=&\sqrt{D^2-B^2}&h=&\sqrt{d^2-b^2}\\	
	\end{align*}
\end{tcolorbox}\index{Trapezio!rettangolo}\index{Trapezio!rettangolo!lato obliquo}\index{Trapezio!rettangolo!altezza}\index{Trapezio!rettangolo!base}\index{Trapezio!rettangolo!diagonali}
\section{Parallelogramma}
\begin{tcolorbox}[sidebyside,righthand width=9cm,colback=white,colframe=white,fonttitle=\bfseries	]
	\includestandalone[width=4.4cm]{geometria/parallelogramma}
	\tcblower
	\begin{align*}
	A=&b\cdot h&h=&\dfrac{A}{c}&b=&\dfrac{A}{h}\\
	P=&2a+2b
	\end{align*}
\end{tcolorbox}\index{Parallelogramma}
\section{Quadrato}\index{Quadrato!area}\index{Quadrato!lato}\index{Quadrato!diagonale}\index{Quadrato!periodo}
\begin{tcolorbox}[sidebyside,righthand width=9cm,colback=white,colframe=white,fonttitle=\bfseries	]
	\includestandalone[width=4.4cm]{geometria/quadrato}
	\tcblower
	\begin{align*}
	A=&l^2&d=&l\sqrt{2}\\
	l=&\sqrt{A}&l=&\dfrac{\sqrt{2}}{2}d\\
	P=&4l&l=\dfrac{P}{4}
	\end{align*}
\end{tcolorbox}
\section{Rettangolo}
\begin{tcolorbox}[sidebyside,righthand width=9cm,colback=white,colframe=white,fonttitle=\bfseries	]
	\includestandalone[width=4.4cm]{geometria/rettangolo}
	\tcblower
	\begin{align*}
	A=&b\cdot h&d=&\sqrt{b^2+h^2}\\
	b=&\sqrt{d^2-h^2}&h=&\sqrt{d^2-b^2}\\
	P=&2(b+h)
	\end{align*}
\end{tcolorbox}\index{Rettangolo!area}\index{Rettangolo!perimetro}\index{Rettangolo!diagonale}
\section{Rombo}
\begin{tcolorbox}[sidebyside,righthand width=9cm,colback=white,colframe=white,fonttitle=\bfseries	]
	\includestandalone[width=4.4cm]{geometria/rombo}
	\tcblower
	\begin{align*}
	A=&\dfrac{d_1\cdot d_2}{2} & P=&4a\\
	d_1=&\dfrac{2A}{d_2}
		\end{align*}
\end{tcolorbox}\index{Rombo!area}\index{Rombo!perimetro}
\section{Circonferenza}\index{Circonferenza}\index{Cerchio}
\begin{tcolorbox}[sidebyside,righthand width=9cm,colback=white,colframe=white,fonttitle=\bfseries	]
	\includestandalone[width=4.4cm]{geometria/circonferenza}
	\tcblower
	\begin{align*}
	A=&\pi r^2 & P=&2\pi r	\\
	r=&\sqrt{\dfrac{A}{\pi}}
	\end{align*}
\end{tcolorbox}\index{Circonferenza!perimetro}\index{Cerchio!area}\index{Circonferenza!raggio}
\section{Poligoni regolari}
\begin{tcolorbox}[sidebyside,righthand width=9cm,colback=white,colframe=white,fonttitle=\bfseries	]
	\includestandalone[width=4.5cm]{geometria/poligoniregolari}
	\tcblower
	\begin{align*}
	A=&\dfrac{P\cdot a}{2} & P=&nl	\\
	a=&l\cdot f & l=&\dfrac{a}{f}	\\
	A=&l^2\cdot\phi&l=&\sqrt{\dfrac{A}{\phi}}
	\end{align*}
\end{tcolorbox}
\newpage
\begin{center}
	\begin{tabular}{lcll}
		\toprule
Poligono	&  Lati&  Fisso f&Fisso $\phi$ \\ 
Triangolo equilatero	& 3 & 0.289 &0.433\\ 
Quadrato	& 4 & 0.5&1 \\ 
Pentagono	& 5 &0.688 &1.720 \\ 
Esagono	& 6 &0.866 &2.598 \\ 
Ettagono	& 7 &1.038&3.634 \\ 
Ottagono	& 8 &1.207&4.828 \\ 
Ennagono	& 9&  1.374&6.182\\ 
Decagono	& 10 & 1.539&7.694 \\ 
Dodecagono	&  12&  1.866&11.196\\
\bottomrule
\end{tabular}\captionof{table}{Poligoni regolari }\index{Poligono!regolare}
\end{center}

\chapter{Geometria solida}
\section{Cono}
\begin{tcolorbox}[sidebyside,righthand width=9cm,colback=white,colframe=white,fonttitle=\bfseries	]
	\includestandalone[width=2.2cm]{geometria/cono}
	\tcblower
	\begin{align}
V=&\dfrac{\pi r^2h}{3}&r=&\sqrt{\dfrac{3V}{\pi h}}&h=&\dfrac{3V}{\pi r^2}\\
a=&\sqrt{r^2+h^2}&S_{lat}=&\pi r a&S_{tot}=&S_{lat}+S_{ba}
	\end{align}
\end{tcolorbox}\index{Cono!volume}\index{Cono!raggio}
\section{Cilindro}
\begin{tcolorbox}[sidebyside,righthand width=9cm,colback=white,colframe=white,fonttitle=\bfseries	]
	\includestandalone[width=2.2cm]{geometria/cilindro}
	\tcblower
	\begin{align}
	V=&\pi r^2h&r=&\sqrt{\dfrac{V}{\pi h}}&h=&\dfrac{V}{\pi r^2}\\
	S_{lat}=&\pi r h&S_{tot}=&S_{lat}+S_{ba}
	\end{align}
\end{tcolorbox}
\section{Parallelepipedo rettangolo}
\begin{tcolorbox}[sidebyside,righthand width=9cm,colback=white,colframe=white,fonttitle=\bfseries	]
	\includestandalone[width=4cm]{geometria/parallepipedoret}
	\tcblower
	\begin{align}
	V=&abh&V=&S_{ba}h
	\end{align}
\end{tcolorbox}
% !TeX encoding = UTF-8
% !TeX spellcheck = it_IT
% !TeX root = formulario.tex
\chapter{Funzioni logiche}
\section{Tavole di verità}
\begin{center}\index{Funzione!logica}\index{AND}\index{NAND}\index{OR}\index{NOR}\index{XOR}\index{XNOR}\index{NOT}
		\begin{tabular}{@{}cc@{\hspace{2cm}}cc@{}}
	\begin{truthtable}{AND}
	\toprule
	$A$&$B$&$AB$\\
	\midrule           
	0&0&0\\
	1&0&0\\
	0&1&0\\
	1&1&1\\
	\bottomrule
	\end{truthtable}
	& \cport{and} &
	\begin{truthtable}{NAND}
	\toprule
	$A$&$B$&$A\mathbin{\overline{\wedge}}B$\\
	\midrule
	0&0&1\\
	1&0&1\\
	0&1&1\\
	1&1&0\\
	\bottomrule
	\end{truthtable}
	& \cport{nand} \\
	\addlinespace[3ex]
	\begin{truthtable}{OR}
	\toprule
	$A$&$B$&$A+B$\\
	\midrule         
	0&0&0\\
	1&0&1\\
	0&1&1\\
	1&1&1\\
	\bottomrule
	\end{truthtable}
	& \cport{or} &
	\begin{truthtable}{NOR}
	\toprule
	$A$&$B$&$A\mathbin{\overline{\vee}}B$\\
	\midrule
	0&0&1\\
	1&0&0\\
	0&1&0\\
	1&1&0\\
	\bottomrule
	\end{truthtable}
	& \cport{nor} \\
	\addlinespace[3ex]
	\begin{truthtable}{XOR}
	\toprule
	$A$&$B$&$A\XOR B$\\
	\midrule         
	0&0&0\\
	1&0&1\\
	0&1&1\\
	1&1&0\\
	\bottomrule
	\end{truthtable}
	& \cport{xor} &
	\begin{truthtable}{XNOR}
	\toprule
	$A$&$B$&$A\mathbin{\overline{\XOR}}B$\\
	\midrule
	0&0&1\\
	1&0&0\\
	0&1&0\\
	1&1&1\\
	\bottomrule
	\end{truthtable}
	& \cport{xnor} \\
	\addlinespace[3ex]
	\multicolumn{4}{c}{%
		\begin{tabular}{@{}cc@{}}
		\begin{truthtable}[2]{NOT}
		\toprule
		$A$&$\overline{A}$\\
		\midrule         
		0&1\\
		1&0\\
		\bottomrule
		\end{truthtable}
		& \cport{not}
		\end{tabular}}
	\end{tabular}
	\captionof{table}{Porte logiche}
\end{center}
\section{Composizione porte}
\begin{center}
		\begin{tabular}{ccc}
	\toprule
	\begin{circuitikz} \draw
	(0,0) node[and port] (myand) {}
	(1,0) node[not port] (mynot) {}
	(myand.out) -- (mynot.in)
	;\end{circuitikz}&&\begin{circuitikz} \draw
	(0,0) node[nor port]  {}
	;\end{circuitikz} \\
	AND +  NOT &=&NAND \\
	\midrule
	\begin{circuitikz} \draw
	(0,0) node[or port] (myor) {}
	(1,0) node[not port] (mynot) {}
	(myor.out) -- (mynot.in)
	;\end{circuitikz}&& \begin{circuitikz} \draw
	(0,0) node[nor port]  {}
	;\end{circuitikz} \\
	OR +  NOT &=&NOR  \\ 
	\midrule
	\begin{circuitikz} \draw
	(0,0) node[xor port] (myxor) {}
	(1,0) node[not port] (mynot) {}
	(myxor.out) -- (mynot.in)
	;\end{circuitikz}&& \begin{circuitikz} \draw
	(0,0) node[xnor port]  {}
	;\end{circuitikz} \\
	XOR +  NOT &=&XNOR  \\ 
	\bottomrule
	\end{tabular} 
	\captionof{table}{Composizione porte}%
\end{center}
\section{Proprietà AND e OR}
\begin{align*}
\overline{\overline{A}}=&A\\
A+0=&A&A\cdot&1=A\\
A+1=&1&A\cdot&0=0\\
A+A=&A&A\cdot&A=A\\
A+\overline{A}=&1&\overline{A}\cdot&A=0\\
\end{align*}\index{AND}\index{OR}
\begin{align*}
A+AB=&A&A\left(A+B\right)=&A\\
A+\overline{A}B=&A+B&A\left(\overline{A}+B\right)=&AB\\
\left(A+B\right)\left(A+C\right)=&A+BC&AB+AC=&A\left(B+C\right)\\
\left(A+B\right)\left(\overline{A}+C\right)=&\overline{A}B+AC&AB+\overline{A}C=&\left(\overline{A}+B\right)\left(A+C\right)\\
\overline{A\cdot B}=&\overline{A}+\overline{B}&\overline{A+B}=&\overline{A}\cdot\overline{B}\\
\end{align*}
%\twocolumn
% !TeX encoding = UTF-8
% !TeX spellcheck = it_IT
% !TeX root = formulario.tex
\chapter{Numeri classificazione}
{\centering
	\includestandalone{grafici/Numeri}
	\captionof{figure}{Classificazione numeri}
\par}\index{Numero!classificazione}
% !TeX encoding = UTF-8
% !TeX spellcheck = it_IT
% !TeX root = formulario.tex

\chapter{Numeri naturali}

\section{Simbolo}
\begin{equation}
\Ni
\end{equation}\index{Numero!naturale}
\section{Addizione}
L'addizione è una funzione definita:  $\funzione{+}{\Ni\times\Ni}{\Ni}$
\begin{equation}\index{Numero!naturale!addizione}
\overbrace{a}^{addendo}+\overbrace{b}^{addendo}=\overbrace{c}^{somma}
\end{equation}\index{Numero!naturale!addendo}\index{Numero!naturale!somma}
\subsection{Associativa}
\begin{equation}
(a+b)+c=a+(b+c)
\end{equation}\index{Numero!naturale!somma associativa}
\subsection{Commutativa}
\begin{equation}
a+b=b+a
\end{equation}\index{Numero!naturale!somma commutativa}
\subsection{Elemento neutro}\
\begin{equation}
a+0=0+a=a
\end{equation}\index{Numero!naturale!somma elemento neutro}\index{Elemento!neutro}
\section{Prodotto}
Il prodotto è una funzione che\index{Numero!naturale!prodotto} $\funzione{\times}{\Ni\times\Ni}{\Ni}$
\begin{equation}
\overbrace{a}^{fattore}\times\overbrace{b}^{fattore}=\overbrace{c}^{prodotto}
\end{equation}\index{Fattore}\index{Numero!naturale!prodotto}
\subsection{Associativa}
\begin{equation}
(a\times b)\times c=a\times(b\times c) 
\end{equation}\index{Numero!naturale!prodotto associativa}
\subsection{Commutativa}
\begin{equation}
a\times b=b\times a
\end{equation}\index{Numero!naturale!prodotto commutativa}
\subsection{Elemento neutro}
\begin{equation}
a\times 1=1\times a=a
\end{equation}\index{Numero!naturale!prodotto elemento neutro}
\subsection{Assorbente}
\begin{equation}
a\times 0=0\times a=0
\end{equation}\index{Numero!naturale!prodotto assorbente}
\section{Sottrazione}
\begin{equation}
\overbrace{a}^{minuendo}-\overbrace{b}^{sottraendo}=\overbrace{c}^{Sottrazione}\quad a\geq b
\end{equation}\index{Minuendo}\index{Sottraendo}\index{Numero!naturale!sottrazione}
\subsection{Invariantiva sottrazione}
\begin{equation}
a-b=(a+c)-(b+c)\quad a>b
\end{equation}\index{Numero!naturale!sottrazione invariantiva}
\begin{equation}
a-b=(a-c)-(b-c)\quad a>b\quad a-c,b-c\in\Ni
\end{equation}\index{Numero!naturale!sottrazione invariantiva}
 \section{Divisione}
 Non sempre possibile.
\begin{equation}
\overbrace{a}^{dividendo}\div\overbrace{b}^{divisore}=\overbrace{c}^{quoziente}\quad b\neq 0
\end{equation}\index{Dividendo}\index{Divisore}\index{Numero!naturale!quoziente}
\subsection{Invariantiva quoziente}
\begin{equation}
a\div b=(a\times c)\div (b\times c)\quad b\neq 0\quad c \neq 0
\end{equation}\index{Numero!naturale!quoziente invariantiva}
\begin{equation}
a\div b=(a\div c)\div (b\div c)\quad b\neq 0\quad c \neq 0
\end{equation}\index{Numero!naturale!quoziente invariantiva}
\section{Proprietà distributive}
\subsection{Moltiplicazione addizione}
\begin{equation}
(a+b)\times c=a\times c+b\times c
\end{equation}\index{Numero!naturale!distributiva moltiplicazione addizione}
\subsection{Moltiplicazione sottrazione}
\begin{equation}
(a-b)\times c=a\times c-b\times c\quad a>b\quad c\neq 0
\end{equation}\index{Numero!naturale!distributiva moltiplicazione sottrazione}
\subsection{Divisione addizione}
\begin{equation}
(a+b)\div c=a\div c+b\div c\quad c\neq 0
\end{equation}\index{Numero!naturale!distributiva divisione addizione}
\subsection{Divisione sottrazione}
\begin{equation}
(a-b)\div c=a\div c+b\div c\quad a>b\quad c\neq 0
\end{equation}\index{Numero!naturale!distributiva divisione sottrazione}
\section{Riepilogo}
\begin{center}
	\begin{tabular}{lCC}
\toprule
Proprietà	& Somma & Prodotto  \\ 
associativa	& (a+b)+c=a+(b+c) & (a\times b)\times c=a\times(b\times c) \\ 
commutativa	&a+b=b+a  &a\times b=b\times a  \\ 
elemento neutro	&a+0=0+a=a  & a\times 1=1\times a=a \\ 
distributiva	&(a+b)\times c=a\times c+b\times c &  \\ 
assorbimento	&  & a\times 0=0\times a=0 \\ 
\bottomrule
\end{tabular}
\captionof{table}{Proprietà numeri naturali}
\end{center}
% !TeX encoding = UTF-8
% !TeX spellcheck = it_IT
% !TeX root = formulario.tex
\chapter{Numeri interi}
\section{Simbolo}
\begin{equation}
\Z
\end{equation}\index{Numero!intero}
\section{Valore assoluto}
\begin{equation}
\abs{a}=\begin{cases}
	a&a>0\\
	-a&a<0\\
	0&a=0
\end{cases}
\end{equation}\index{Valore assoluto}\index{Numero!intero!valore assoluto}
\section{Vocabolario}
\begin{description}
	\item[Concordi] Due numeri interi che hanno segno uguale\index{Numero!intero!concordi}
	\item[Discordi]  Due numeri interi che hanno segno diverso\index{Numero!intero!disconcordi}
	\item[Opposti] Uguale valore assoluto, segno opposto\index{Numero!intero!opposti}
\end{description}
\section{Somma}
\begin{enumerate}
	\item I numeri a e b sono concordi. In questo caso la somma è un numero intero con lo stesso segno di a e di b e che ha per valore assoluto la somma dei valori assoluti.
	\item I numeri a e b sono discordi. In questo caso la somma è un numero intero con il segno del numero di valore assoluto maggiore e che ha per valore assoluto la differenza tra i valori assoluti.
	\item I due numeri sono opposti. La somma è zero.
\end{enumerate}\index{Numero!intero!somma}\index{Valore!assoluto}
\section{Somma}
\begin{equation}
\overbrace{a}^{addendo}+\overbrace{b}^{addendo}=\overbrace{c}^{somma}
\end{equation}\index{Addendo}\index{Somma}
\subsection{Associativa}
\begin{equation}
(a+b)+c=a+(b+c)
\end{equation}\index{Somma!associativa}
\subsection{Commutativa}
\begin{equation}
a+b=b+a
\end{equation}\index{Somma!commutativa}
\subsection{Elemento  neutro}
\begin{equation}
a+0=0+a=a
\end{equation}\index{Somma!elemento!neutro}
\subsection{Opposto}
\begin{equation}
a+(-a)=(-a)+a=0
\end{equation}\index{Somma!opposto}
\section{Prodotto}
\begin{enumerate}
	\item I numeri a e b sono concordi. Il prodotto fra i due numeri è un numero di segno positivo e per valore assoluto il prodotto dei valori assoluti.
	\item I numeri a e b sono discordi. Il prodotto fra i due numeri è un numero di segno negativo e per valore assoluto il prodotto dei valori assoluti.
\end{enumerate}\index{Prodotto!segno}

{\centering
	\begin{tabular}{ccc}
		\toprule
		$+$ & $+$ & $+$ \\ 
		$-$ & $-$ & $+$ \\ 
		$+$ & $-$ & $-$ \\ 
		$-$ & $+$ & $-$ \\ 
		\bottomrule
	\end{tabular}\index{Numero!intero!segno prodotto}
	\captionof{table}{Segno prodotto} 
\par}
\begin{equation}
\overbrace{a}^{fattore}\times\overbrace{b}^{fattore}=\overbrace{c}^{prodotto}
\end{equation}\index{Fattore}\index{Prodotto}
\subsection{Associativa}
\begin{equation}
(a\times b)\times c=a\times(b\times c) 
\end{equation}\index{Numero!intero!prodotto associativa}
\subsection{Commutativa}
\begin{equation}
a\times b=b\times \index{Numero!intero!prodotto commutativa}
\end{equation}\index{Numero!intero!prodotto commutativa}
\subsection{Elemento neutro}
\begin{equation}
a\times 1=1\times a=a
\end{equation}\index{Numero!intero!prodotto elemento neutro}
\section{Sottrazione}
La sottrazione e la somma coincidono
\begin{equation}
a-b=a+(-b)
\end{equation}
 \section{Divisione}
  Non è sempre possibile.
\begin{equation}
\overbrace{a}^{dividendo}\div\overbrace{b}^{divisore}=\overbrace{c}^{quoziente}\quad b\neq 0
\end{equation}\index{Dividendo}\index{Divisore}\index{Quoziente}
\subsection{Segno}
\begin{enumerate}
	\item I numeri a e b sono concordi. Il quoziente fra i due numeri è un numero di segno positivo e per valore assoluto il quoziente dei valori assoluti.
	\item I numeri a e b sono discordi. Il quoziente fra i due numeri, è un numero che ha  segno negativo e per valore assoluto il quoziente dei valori assoluti.
\end{enumerate}\index{Quoziente!segno}
\subsection{Invariantiva quoziente}
\begin{equation}
a\div b=(a\times c)\div (b\times c)\quad b\neq 0\quad c \neq 0
\end{equation}\index{Quoziente!invariantiva}
\begin{equation}
a\div b=(a\div c)\div (b\div c)\quad b\neq 0\quad c \neq 0
\end{equation}\index{Numero!intero!quoziente invariantiva}
\section{Proprietà distributive}
\subsection{Moltiplicazione addizione}
\begin{equation}
(a+b)\times c=a\times c+b\times c
\end{equation}\index{Distributiva!moltiplicazione!addizione}
\subsection{Moltiplicazione sottrazione}
\begin{equation}
(a-b)\times c=a\times c-b\times c\quad a>b\quad c\neq 0
\end{equation}\index{Distributiva!moltiplicazione!sottrazione}
\subsection{Divisione addizione}
\begin{equation}
(a+b)\div c=a\div c+b\div c\quad c\neq 0
\end{equation}\index{Distributiva!divisione!addizione}
\subsection{Divisione sottrazione}
\begin{equation}
(a-b)\div c=a\div c-b\div c\quad a>b\quad c\neq 0
\end{equation}\index{Distributiva!divisione!sottrazione}
\section{Riepilogo}

{\centering
	\begin{tabular}{lLL}
		\toprule
		Proprietà	& Somma & Prodotto  \\ 
		\midrule
		associativa	& (a+b)+c=a+(b+c) & (a\times b)\times c=a\times(b\times c) \\ 
		commutativa	&a+b=b+a  &a\times b=b\times a  \\ 
		elemento neutro	&a+0=0+a=a  & a\times 1=1\times a =a\\ 
		inverso&(-a)+a=a+(-a)=0\\
		distributiva	&(a+b)\times c =a\times c+b\times c &  \\ 
		assorbimento	&  & a\times 0=0\times a=0 \\ 
		\bottomrule
	\end{tabular}
	\captionof{table}{Proprietà numeri interi}
\par}\index{Proprietà!associativa}\index{Proprietà!commutativa}\index{Elemento!neutro}\index{Proprietà!distributiva}\index{Proprietà!assorbimento}
\include{NumRazionali}
\chapter{Potenze}
\section{Definizione}
\begin{align}
a^n=&\overbrace{a\times a\times\cdots\times a}^{n{}\mbox{volte}}\\
a^0=&1&a\neq0\\
a^1=&a\\
0^n=&0&n>0\\
1^n=&1%
% 1/10/2017 :: 13:56:37 :: a^{-n}=&\left(\dfrac{1}{a}\right)^n
\end{align}\index{Potenza!definzione}\index{Potenza!proprietà}
\section{Prodotto di potenze che hanno la stessa base}
\begin{equation}
a^n\cdot a^m=a^{n+m}
\end{equation}\index{Potenza!prodotto!stessa base}
\section{Quoziente di potenze che hanno la stessa base}
\begin{equation}
a^n\div a^m=a^{n-m}
\end{equation}\index{Potenza!quoziente!stessa base}
\section{Potenza di potenze}
\begin{equation}
(a^n)^m=a^{n\cdot m}
\end{equation}\index{Potenza!di potenze}
\section{Prodotto di potenze che hanno lo stesso esponente}
\begin{equation}
a^n\times b^n=(a\times b)^n
\end{equation}\index{Potenza!prodotto!stesso esponente}
\section{Divisione di potenze che hanno lo stesso esponente}
\begin{equation}
a^n\div b^n=(a\div b )^n
\end{equation}\index{Potenza!quoziente!stesso esponente}
\section{Esponente negativo}
\begin{equation}
a^{-n}=\left(\dfrac{1}{a}\right)^n
\end{equation}\index{Potenza!esponente!negativo}
\section{Esponente frazionario}
\begin{align}
\sqrt[n]{a}=&a^{\frac{1}{n}}\\
\sqrt[n]{a^m}=&a^{\frac{m}{n}}
\end{align}\index{Potenza!esponente!frazionario}
	
\chapter{Monomi}
\section{Definizione}
Un monomio è il prodotto fra una parte numerica\index{Monomio!parte numerica} e una parte letterale\index{Monomio!parte letterale} che non contiene divisioni.
\section{Grado} 
Somma degli esponenti della parte letterale.\index{Monomio!grado}
\section{Monomio zero}
Il monomio con parte numerica zero è chiamato monomio zero.\index{Monomio!zero}
\section{Somma}
\subsection{Somma monomi simili}
La somma di due monomi simili è un monomio che ha la stessa parte letteraria e per parte numerica la somma algebrica delle parti numeriche.\index{Monomio!somma simili}
\begin{equation}
ab^2+3ab^2=4ab^2
\end{equation}
\subsection{Somma monomi non simili}
La somma di due monomi non simili  sono i due monomi non simili.\index{Monomio!somma non simili}
\begin{equation}
2a^3b^2+3ab^2=2a^3b^2+3ab^2
\end{equation}
\section{Prodotto}
Il prodotto di due monomi è un monomio che ha per parte numerica il prodotto algebrico delle parti numeriche e per parte letterale il prodotto delle parti numeriche.\index{Monomio!prodotto}
\chapter{Polinomi}
\section{Definizione}
Un polinomio è la somma di monomi non simili.\index{Polinomio!definizione}
\section{Grado polinomio}
Il grado di un polinomio è il grado maggiore fra i 
monomi che lo compongono.\index{Polinomio!grado}
\section{Monomio per binomio}
\begin{equation*}
a(b+c)=ab+ac
\end{equation*}\index{Prodotto!monomio!binomio}
\begin{equation*}
1(1+2)=11+12
\end{equation*}
\section{Binomio per binomio}
\begin{equation*}
(a+b)(c+d)=ac+ad+bc+bd
\end{equation*}\index{Prodotto!binomio!binomio}
\begin{equation*}
(1+2)(1+2)=11+12+21+22
\end{equation*}
\section{Quadrato del binomio}
\begin{align*}
(a+b)^2=&a^2+b^2+2ab\\
(a-b)^2=&a^2+b^2-2ab
\end{align*}\index{Prodotto!quadrato!binomio}
\begin{equation*}
(1+2)^2=1^2+2^2+2(1)(2)
\end{equation*}
\section{Quadrato del trinomio}
\begin{equation*}
(a+b+c)^2=a^2+b^2+c^2+2ab+2ac+2bc
\end{equation*}\index{Prodotto!quadrato!trinomio}
\begin{equation*}
(1+2+3)^2=1^2+2^2+3^2+2(1)(2)+2(1)(3)+2(2)(3)
\end{equation*}
\section{Cubo binomio}
\begin{equation*}
(a+b)^3=a^3+b^3+3a^2b+3ab^2
\end{equation*}\index{Prodotto!cubo!binomio}
\begin{equation*}
(1+2)^3=1^3+2^3+3(1)^2 2+3(1)2^2
\end{equation*}
\section{Differenza di quadrati}
\begin{equation*}
(a-b)(a+b)=a^2-b^2
\end{equation*}\index{Prodotto!differenza!quadrati}
\begin{equation*}
(1-2)(1+2)=1^2-2^2
\end{equation*}
	
\section{Triangolo di Tartaglia}
\label{sec:TriangolodiTartaglia}
\begin{center}
	\setlength{\tabcolsep}{0pt}
\begin{tabular}{l<{\qquad}*{21}{T}} 
$n=0$&&&&&&&&&1\\
$n=1$&&&&&&&&1&&1\\
$n=2$&&&&&&&1&&2&&1\\
$n=3$&&&&&&1&&3&&3&&1\\
$n=4$&&&&&1&&4&&6&&4&&1\\
$n=5$&&&&1&&5&&10&&10&&5&&1\\
$n=6$&&&1&&6&&15&&20&&15&&6&&1\\
$n=7$&&1&&7&&21&&35&&35&&21&&7&&1\\
$n=8$&1&&8&&28&&56&&70&&56&&28&&8&&1\\
\end{tabular}
\captionof{table}{Triangolo di Tartaglia}
\label{tab:trinagoloditartaglia}
\end{center}


% !TeX encoding = UTF-8
% !TeX spellcheck = it_IT
% !TeX root = formulario.tex
\chapter{Scomposizioni}
\section{Raccoglimento totale}
\begin{equation}
ab+ac=a(b+c)
\end{equation}\index{Scomposizione!raccoglimento!totale}
\begin{equation}
(a+b)c+(a+b)d=(a+b)(c+d)
\end{equation}
\section{Raccoglimento parziale}
\begin{equation}
ac+ad+bc+bd=a(c+d)+b(c+d)=(c+b)(a+b)
\end{equation}\index{Scomposizione!raccoglimento!parziale}
\section{Quadrato del binomio}
\begin{equation}
	a^2+b^2+2ab=(a+b)^2
\end{equation}\index{Scomposizione!quadrato!del binomio}
\section{Quadrato del trinomio}
\begin{equation}
a^2+b^2+c^2+2ab+2ac+2bc=(a+b+c)^2
\end{equation}\index{Scomposizione!quadrato!del trinomio}
\section{Cubo binomio}
\begin{equation}
a^3+b^3+3a^2b+3ab^2=(a+b)^3
\end{equation}\index{Scomposizione!cubo!del binomio}
\section{Differenza di quadrati}
\begin{equation}
a^2-b^2=(a-b)(a+b)
\end{equation}\index{Scomposizione!differenza!quadrati}
\section{Differenza di cubi}
\begin{equation}
a^3-b^3=(a-b)(a^2+ab+b^2)
\end{equation}\index{Scomposizione!differenza!di cubi}
\section{Somma di cubi}
\begin{equation}
a^3+b^3=(a+b)(a^2-ab+b^2)
\end{equation}\index{Scomposizione!somma!di cubi}
\section{Trinomio particolare}
\begin{equation}
x^2+sx+p=(x+a)(x+b)\quad\begin{cases}
a+b=s\\
a\cdot b=p
\end{cases}\index{Scomposizione!trimomio!particolare}
\end{equation}


% !TeX encoding = UTF-8
% !TeX spellcheck = it_IT
% !TeX root = formulario.tex
\chapter{Radicali}
\section{Glossario}
\begin{equation*}
\sqrt[n]{a}=b\\
\end{equation*}
$n$ indice, $a$ radicando, $b$ radice e $\sqrt[n]{a}$ radicale
\section{Definizione}
\begin{align*}
\intertext{n pari}
\sqrt[n]{a}=&b\Longleftrightarrow b^n=a&a\geq0\quad b\geq0\quad n\in\Nz\\
\intertext{n dispari}
\sqrt[n]{a}=&b\Longleftrightarrow b^n=a&a,b\in\R\quad n\in\Nz
\end{align*}\index{Radicale!definizione}
\section{Segno}
\begin{align*}
\intertext{n pari}
\sqrt[n]{a}&&\text{sempre positivo}\\
\intertext{n dispari}
\sqrt[n]{a}&&\text{segno radicando}
\end{align*}\index{Radicale!segno}
\section{Proprietà invariantiva}
\begin{align*}
\sqrt[n\cdot s]{a^{m\cdot s}}=&\sqrt[n]{a^m}&\forall a\geq 0\quad n,m,s\in\Nz\\
\sqrt[n]{a^m}=&\sqrt[n\cdot s]{a^{m\cdot s}}&\forall a\geq 0\quad n,m,s\in\Nz
\end{align*}\index{Radicale!proprietà!invariantiva}\index{Proprietà!invariantiva}
\section{Proprietà}
\begin{align*}
\intertext{n pari}
\sqrt[n]{a^n}=&\abs{a}\\
\intertext{n dispari}
\sqrt[n]{a^n}=&a
\end{align*}
\section{Prodotto con indici uguali}
\begin{align*}
\sqrt[n]{a}\sqrt[n]{b}=&\sqrt[n]{ab}\\
\sqrt[n]{ab}=&\sqrt[n]{a}\sqrt[n]{b}
\end{align*}\index{Radicale!prodotto!indici uguali}
\section{Prodotto con indici diversi}
\begin{enumerate}
	\item Per eseguire $\sqrt[n]{a^p}\cdot\sqrt[m]{b^q}$
	\item Calcolare il $\mcm(m,n)$ che diventerà l'indice  delle due radici
	\item Dividere il  $\mcm$ per $n$ e il risultato lo moltiplico per $p$
	\item Dividere il  $\mcm$ per $m$ e il risultato lo moltiplico per $q$
	\item Eseguiamo  il prodotto tra le due radici che ora hanno lo stesso indice.
\end{enumerate}\index{Radicale!prodotto!indici diversi}
\section{Divisione con indici uguali}
\begin{align*}
\dfrac{\sqrt[n]{a}}{\sqrt[n]{b}}=&\sqrt[n]{\dfrac{a}{b}}&b\neq 0\\
\sqrt[n]{\dfrac{a}{b}}=&\dfrac{\sqrt[n]{a}}{\sqrt[n]{b}}&b\neq 0
\end{align*}\index{Radicale!quoziente!indici uguali}
\section{Divisione con indici diversi}
\begin{enumerate}
	\item Per eseguire
	$\dfrac{\sqrt[n]{a^p}}{\sqrt[m]{b^q}}$
	\item Calcolare il $\mcm(m,n)$ che diventerà l'indice  delle due radici
	\item Dividere il  $\mcm$ per $n$ e il risultato lo moltiplico per $p$
	\item Divider il  $\mcm$ per $m$ e il risultato lo moltiplico per $q$
	\item Eseguire la divisione fra le due radici che ora hanno lo stesso indice.
\end{enumerate}\index{Radicale!quoziente!indici diversi}
\section{Potenza radicale}
\begin{align*}
\left(\sqrt[n]{a}\right)^m=&\sqrt[n]{a^m}\\
\sqrt[n]{a^m}=&\left(\sqrt[n]{a}\right)^m\\
\end{align*}\index{Radicale!potenza}
\section{Radice di radice}
\begin{equation*}
\sqrt[n]{\sqrt[m]{a}}=\sqrt[n\cdot m]{a}
\end{equation*}\index{Radicale!radice!di radice}
\section{Trasporto di un termine dentro al segno di radice}
\begin{equation*}
b\sqrt[n]{a}=\sqrt[n]{b^na}\quad a\geq 0,b\geq 0 
\end{equation*}\index{Radicale!trasporto!dentro}
\section{Trasporto di un termine fuori dal segno di radice}
\begin{equation*}
\sqrt[n]{a^nb}=\abs{a}\sqrt[n]{b}
\end{equation*}
\begin{align*}
\sqrt[n]{a^m}&\quad m\geq n\\
m=&n\cdot q+r\\
\sqrt[n]{a^m}=&\sqrt[n]{a^{n\cdot q+r}}\\
=&a^q\sqrt[n]{a^r}
\end{align*}\index{Radicale!trasporto!fuori}
\section{Somma di radicali}
Due radicali sono simili se hanno stesso indice e radicando.\index{Radicale!radici!simili}
\begin{align*}
b\sqrt[n]{a^m}+c\sqrt[n]{a^m}=&(b+c)\sqrt[n]{a^m}\\
b\sqrt[n]{a}+c\sqrt[n]{a^m}=&b\sqrt[n]{a}+c\sqrt[n]{a^m}
\end{align*}
\chapter{Razionalizzazione di frazioni}
\section{Radicale quadrato al denominatore}
\begin{align*}
\dfrac{b}{\sqrt{a}}=&\dfrac{b}{\sqrt{a}}\cdot\dfrac{\sqrt{a}}{\sqrt{a}}\\
=&\dfrac{b\sqrt{a}}{a}
\end{align*}\index{Radicale!razionalizzazione}
\section{Radicale di indice qualunque al denominatore}
\begin{align*}\index{Radicale!razionalizzazione}
\intertext{radici qualunque $n>m$}
\dfrac{b}{\sqrt[n]{a^m}}=&\\
=&\dfrac{b}{\sqrt[n]{a^m}}\cdot\dfrac{\sqrt[n]{a^{n-m}}}{\sqrt[n]{a^{n-m}}}\\
=&\dfrac{b\sqrt[n]{a^{n-m}}}{a}
\intertext{radici qualunque $m>n$}
\intertext{Prima trasporto fuori il termine e poi procediamo come prima.}\nonumber
\end{align*}
\section{Differenza di quadrati}
\begin{align*}
\dfrac{b}{\sqrt{a}+\sqrt{c}}=&\\
=&\dfrac{b}{\sqrt{a}+\sqrt{c}}\cdot\dfrac{\sqrt{a}-\sqrt{c}}{\sqrt{a}-\sqrt{c}}\\
=&\dfrac{b(\sqrt{a}-\sqrt{c})}{a-c}
\end{align*}\index{Radicale!razionalizzazione!differenza quadrati}
\begin{align*}
\dfrac{b}{\sqrt{a}-\sqrt{c}}=&\\
=&\dfrac{b}{\sqrt{a}+\sqrt{c}}\cdot\dfrac{\sqrt{a}+\sqrt{c}}{\sqrt{a}+\sqrt{c}}\\
=&\dfrac{b(\sqrt{a}-\sqrt{c})}{a-c}
\end{align*}
\section{Somma e differenza di cubi}
\begin{align*}
\dfrac{b}{\sqrt[3]{a}+\sqrt[3]{c}}=&\\
=&\dfrac{b}{\sqrt[3]{a}+\sqrt[3]{c}}\cdot\dfrac{\sqrt[3]{a}-\sqrt[3]{ac}+\sqrt[3]{c}}{\sqrt[3]{a}-\sqrt[3]{ac}+\sqrt[3]{c}}\\
=&\dfrac{b(\sqrt[3]{a}-\sqrt[3]{ac}+\sqrt[3]{c})}{a+c}
\end{align*}\index{Radicale!razionalizzazione!differenza cubi}
\begin{align*}
\dfrac{b}{\sqrt[3]{a}-\sqrt[3]{c}}=&\\
=&\dfrac{b}{\sqrt[3]{a}-\sqrt[3]{c}}\cdot\dfrac{\sqrt[3]{a}+\sqrt[3]{ac}+\sqrt[3]{c}}{\sqrt[3]{a}+\sqrt[3]{ac}+\sqrt[3]{c}}\\
=&\dfrac{b(\sqrt[3]{a}+\sqrt[3]{ac}+\sqrt[3]{c})}{a-c}
\end{align*}\index{Radicale!razionalizzazione!somma cubi}
\chapter{Esponente frazionario}
\section{Definizione}
\begin{align*}
\sqrt[n]{a}=&a^{\frac{1}{n}}&a\geq 0\quad n\in\Ni
\quad n\neq 0\\
\sqrt[n]{a^m}=&a^{\frac{m}{n}}&a\geq 0\quad m,n\in\Ni
\quad n\neq 0
\end{align*}\index{Potenza!esponente!frazionario}
\section{Prodotto tra radicali}
\begin{equation*}
a^{\frac{p}{n}}\cdot a^{\frac{q}{m}}=a^{\frac{p}{n}+\frac{q}{m}}
\end{equation*}\index{Radicale!prodotto}
\section{Quoziente tra radicali}
\begin{equation*}
a^{\frac{p}{n}}\div a^{\frac{q}{m}}=a^{\frac{p}{n}-\frac{q}{m}}
\end{equation*}\index{Radicale!quoziente}
\section{Radicale di radicale}
\begin{equation*}
\left(a^{\frac{p}{n}}\right)^{\frac{q}{m}}=a^{\frac{p}{n}\cdot\frac{q}{m}}
\end{equation*}\index{Radicale!radice di radice}
\section{Prodotto fra radicali che hanno stesso indice}
\begin{equation*}
a^{\frac{p}{n}}\cdot b^{\frac{p}{n}}=\left(a\cdot b\right)^{\frac{p}{n}}
\end{equation*}
\section{Quoziente fra radicali che hanno stesso indice}
\begin{equation*}
a^{\frac{p}{n}}\div b^{\frac{p}{n}}=\left(a\div b\right)^{\frac{p}{n}}
\end{equation*}
\section{Potenze con esponente negativo}
\begin{equation*}
a^{-\frac{p}{n}}=\sqrt[b]{\left(\dfrac{1}{a}\right)^p}\quad a>0\quad n\in\Ni
\quad n\neq 0
\end{equation*}\index{Potenza!esponente frazionario!negativo}

% !TeX encoding = UTF-8
% !TeX spellcheck = it_IT
% !TeX root = formulario.tex
\chapter{Equazioni di primo grado}
\section{Definizione}
\begin{align*}
ax+b&={}0\quad a\neq 0\\
x_1&=-\dfrac{b}{a}
\end{align*}\index{Equazione!primo grado}
\section{Classificazione delle equazioni}
{\centering\captionof{table}{Classificazione equazioni primo grado}
	\begin{tabular}{cl}
		\toprule
		Coefficienti&Tipo\\
		\midrule
		$a=0$\quad $b\neq 0$	& Equazione impossibile  \\ 
		$a=0$\quad $b=0$	& Equazione indeterminata\\ 
		$a\neq0$	& Equazione determinata  \\ 
		\bottomrule
	\end{tabular}\par}\index{Equazione!primo grado!classificazione}
\section{Risoluzione}
Metodo separazione variabili\index{Equazione!primo grado!separazione}
\chapter{Equazioni di secondo grado}
\section{Equazioni pure}
\begin{align*}
ax^2+c=0&\quad a\neq 0
\intertext{se $a$ e $c$ discordi}
x_{1,2}=&\pm\sqrt{-\dfrac{c}{a}}
\intertext{se $a$ e $c$ concordi non si hanno soluzioni}\nonumber
\end{align*}\index{Equazione!secondo grado!pure}
\section{Equazioni spurie}
\begin{align*}
ax^2+bx=&0\quad a\neq 0\\
x(ax+b)=&0\\
x_1=&0\\
ax+b=&0\\
x_2=&-\dfrac{b}{a}
\end{align*}\index{Equazione!secondo grado!spurie}
\section{Equazioni monomia}
\begin{equation*}
ax^2=0\quad a\neq 0
\end{equation*}\index{Equazione!secondo grado!monomia}
\begin{equation*}
x_{1,2}=0
\end{equation*}
\section{Equazioni complete}
\begin{align*}
ax^2+bx+c=&0\quad a\neq 0\\
x_{1,2}=&\dfrac{-b\pm\sqrt{b^2-4ac}}{2a}
\end{align*}\index{Equazione!secondo grado!complete}
\section{Il delta e la classificazione delle soluzioni}
\begin{equation*}
\Delta=b^2-4ac
\end{equation*}\index{Equazione!secondo grado!discriminante}\index{Discriminante}\index{Delta}
\index{Equazione!secondo grado!delta}
\begin{equation*}
ax^2+bx+c=0\quad\begin{cases}
\text{Se $\Delta >0$ Soluzioni distinte}\\
\text{Se $\Delta =0$ Soluzioni coincidenti}\\
\text{Se $\Delta <0$ Nessuna soluzione reale}\\
\end{cases}
\end{equation*}\index{Equazione!secondo grado!classificazione soluzioni}
\section{Proprietà soluzioni}
\begin{equation*}
\begin{cases}
x_1+x_2=-\dfrac{b}{a}\\
x_1\cdot x_2=\dfrac{c}{a}
\end{cases}\index{Equazione!secondo grado!somma soluzioni}\index{Equazione!secondo grado!prodotto soluzioni}
\end{equation*}
\section{Scomposizione trinomio}
\begin{align*}
\intertext{$\Delta>0$}
ax^2+bx+c=&a(x-x_1)(x-x_2)\\
\intertext{$\Delta=0$}
ax^2+bx+c=&a(x-x_1)^2\\
\intertext{$\Delta<0$}
ax^2+bx+c=&a\left[\left(x+\dfrac{b}{2a}\right)^2+\dfrac{4ac-b^2}{4a^2}\right]
\end{align*}\index{Equazione!secondo grado!scomposizione trinomio}\index{Scomposizione!trinomio}\index{Delta}\index{Discriminante}
\chapter{Equazioni binomie}
\begin{align*}
ax^n+b=&0&a\neq 0\\
x^n=&-\frac{b}{a}
\intertext{Se $n$ pari}
x=&\pm\sqrt[n]{-\frac{b}{a}}&\begin{cases}
\text{$a$ e $b$ concordi}& \text{non ha soluzione}\\
\text{$a$ e $b$ discordi}& x=\pm\sqrt[n]{-\frac{b}{a}}\\
\end{cases}\\
\intertext{Se $n$ dispari}
x=&\sqrt[n]{-\frac{b}{a}}\\
\end{align*}\index{Equazione!binomie}
\chapter{Equazioni biquadratiche}
\begin{align*}
ax^4+bx^2+c=&0&a\neq 0\\
x^2=&y\\
ay^2+by+c=&0\\
x^2=&y_1\\
x^2=&y_2
\end{align*}\index{Equazione!biquadratica}
\chapter{Equazioni trinomie}
\begin{align*}
ax^{2n}+bx^n+c=&0&a\neq 0\\
x^n=&y\\
ay^2+by+c=&0\\
\intertext{$\Delta<0$}
\intertext{l'equazione trinomia non ha soluzione}
\intertext{$\Delta=0$}
x^n=-\frac{b}{2a}
\intertext{$\Delta>0$}
x^n=&\dfrac{-b+\sqrt{b^2-4ac}}{2a}\\
x^n=&\dfrac{-b-\sqrt{b^2-4ac}}{2a}\\
\end{align*}\index{Equazione!trinomie}
\chapter{Sistemi lineari}
\[\begin{cases}
a_{1}x+b_{1}y=c_{1}\\
a_2x+b_2y=c_2
\end{cases}\]\index{Sistema!lineare}
\section{Metodo di sostituzione}
\begin{enumerate}
	\item Risolvo una delle due equazioni rispetto ad una variabile.
	\item Sostituisco la variabile trovata nell'altra equazione.
	\item Risolvo l'equazione in un'incognita trovata.
	\item Sostituisco il risultato trovato nell'altra equazione e semplifico.
\end{enumerate}\index{Sistema!lineare!sostituzione}
\section{Metodo del confronto}
\begin{enumerate}
	\item Risolvo rispetto la stessa incognita entrambi le equazioni.
	\item Come prima equazione metto a confronto i risultati ottenuti.
	\item Come seconda equazione scelgo una delle due soluzioni.
	\item Risolvo la prima equazione in una sola incognita e sostituisco il risultato nella seconda e semplifico.
\end{enumerate}\index{Sistema!lineare!confronto}
\section{Metodo di riduzione}
\begin{enumerate}
	\item Ordino le due equazioni in modo che in ogni colonna vi sia la stessa incognita
	\item Modifico un'equazione o entrambe in modo un'incognita abbia lo stesso coefficiente (o il suo opposto)
	\item Sommo o sottraggo le due equazioni. Scompare un'incognita risolvo rispetto all'altra.
	\item Procedo in modo analogo con l'altra incognita.
\end{enumerate}\index{Sistema!lineare!riduzione}
\section{Metodo di Cramer}
\begin{equation}
x=\dfrac{\begin{vmatrix}
	c_{1}&b_{1}\\
	c_2&b_2
	\end{vmatrix}}{\begin{vmatrix}
	a_{1}&b_{1}\\
	a_2&b_2
	\end{vmatrix}}
\end{equation}
\begin{equation}
y=\dfrac{\begin{vmatrix}
	a_{1}&c_{1}\\
	a_2&c_2
	\end{vmatrix}}{\begin{vmatrix}
	a_{1}&b_{1}\\
	a_2&b_2
	\end{vmatrix}}
\end{equation}\index{Sistema!lineare!Cramer}
\section{Classificazione delle soluzioni}
\begin{align}
\dfrac{a_1}{a_2}\neq&\dfrac{b_1}{b_2}& &a_2\neq 0\quad b_2\neq 0&\text{Determinato}\\
\dfrac{a_1}{a_2}=&\dfrac{b_1}{b_2}=\dfrac{c_1}{c_2}& &a_2\neq 0\quad b_2\neq 0\quad c_2\neq 0&\text{Indeterminato}\\
\dfrac{a_1}{a_2}=&\dfrac{b_1}{b_2}\neq\dfrac{c_1}{c_2}& &a_2\neq 0\quad b_2\neq 0\quad c_2\neq 0&\text{Impossibile}\\
\end{align}\index{Sistema!lineare!determinato}\index{Sistema!lineare!indeterminato}\index{Sistema!lineare!impossibile}
% !TeX encoding = UTF-8
% !TeX spellcheck = it_IT
% !TeX root = formulario.tex
\chapter{Numeri complessi}
\section{Unità immaginaria}
\begin{equation*}
\uimm^2=-1
\end{equation*}\index{Unità immaginaria}
\begin{equation*}
\uimm=\sqrt{-1}
\end{equation*}
\section{Numero complesso}
\begin{equation*}
z=a+b\uimm\quad z\in\Co\quad a,b\in\R
\end{equation*}\index{Numero!complesso}
\section{Uguaglianza numeri complessi}
\begin{equation*}
a+b\uimm=c+d\uimm\quad\Longleftrightarrow\quad a=c\quad b=d  \quad a,b,c,d\in\R
\end{equation*}\index{Numero!complesso!uguaglianza}
\section{Parte reale}
\begin{equation*}
z=a+b\uimm\quad\Re(z)=a\quad a,b\in\R
\end{equation*}\index{Numero!complesso!parte reale}
\section{Parte immaginaria}
\begin{equation*}
z=a+b\uimm\quad\Im(z)=b\quad a,b\in\R
\end{equation*}\index{Numero!complesso!parte immaginaria}
\section{Modulo}
\begin{equation*}
z=a+b\uimm\quad r=\abs{z}=\sqrt{a^2+b^2}\quad a,b\in\R
\end{equation*}\index{Numero!complesso!modulo}
\section{Complessi coniugati}
\begin{align*}
z=a+b\uimm\quad\conj{z}=a-b\uimm\quad a,b\in\R\\
\conj{\conj{z}}=&z\\
\end{align*}\index{Numero!complesso!coniugato}
\section{Complessi opposti}
\begin{equation*}
z=a+b\uimm\quad\-z=-a-b\uimm\quad a,b\in\R
\end{equation*}\index{Numero!complesso!opposto}
\section{Somma di numeri complessi}
\begin{align*}
z_1=&a+b\uimm\\
z_2=&c+d\uimm\\
z_1+z_2=&a+c+(b+d)\uimm
\end{align*}\index{Numero!complesso!somma}
\section{Somma di numeri complessi coniugati}
\begin{align*}
z=&a+b\uimm\\
\conj{z}=&a-b\uimm\\
z+\conj{z}=&a+a=2a\\
\conj{z\pm w}=&\conj{z}\pm\conj{w}\\
\end{align*}\index{Numero!complesso!coniugato}
\section{Differenza di numeri complessi}
\begin{align*}
z_1=&a+b\uimm\\
z_2=&c+d\uimm\\
z_1+z_2=&a-c+(b-d)\uimm
\end{align*}\index{Numero!complesso!differenza}
\section{Differenza di numeri complessi coniugati}
\begin{align*}
z=&a+b\uimm\\
\conj{z}=&a-b\uimm\\
z-\conj{z}=&+2b\uimm
\end{align*}\index{Numero!complesso!coniugato}
\section{Prodotto di numeri complessi}
\begin{align*}
z_1=&a+b\uimm\\
z_2=&c+d\uimm\\
z_1\cdot z_2=&(ac-bd)+(ad+bc)\uimm
\end{align*}\index{Numero!complesso!prodotto}
\section{Prodotto di numeri complessi coniugati}
\begin{align*}
z=&a+b\uimm\\
\conj{z}=&a-b\uimm\\
z\cdot\conj{z}=&a^2+b^2\\
\conj{zw}=&\conj{z}\conj{w}
\end{align*}
\section{Reciproco numero complesso }
\begin{align*}
z_1=&a+b\uimm\\
z_2=&c+d\uimm\\
z_1\cdot z_2=&1\\
z_2=&\dfrac{\conj{z_1}}{\abs{z_1}^2}=z^{-1}_1\\
\end{align*}\index{Numero!complesso!reciproco}
\section{Quoziente di due numeri complessi}
\begin{align*}
z_1=&a+b\uimm\\
z_2=&c+d\uimm\\
\dfrac{z_1}{z_2}=&\dfrac{z_1}{z_2}\cdot\dfrac{\conj{z}_2}{\conj{z}_2}\\
\dfrac{z_1}{z_2}=&\dfrac{a+b\uimm}{c+d\uimm}\cdot\dfrac{c-d\uimm}{c-d\uimm}\\
=&\dfrac{ac+bd+(bc-ad)\uimm}{c^2+d^2}
\end{align*}\index{Numero!complesso!quoziente}
\section{Prodotto per scalare}
\begin{align*}
z=&a+b\uimm\\
k\in&\R\\
k\cdot z=&k\cdot a+k\cdot b\uimm\\
\end{align*}\index{Numero!complesso!prodotto per scalare}

% !TeX encoding = UTF-8
% !TeX spellcheck = it_IT
% !TeX root = formulario.tex
\chapter{Disequazioni di primo grado}
\section{Definizione verso}
\begin{description}
	\item[$>$] maggiore
	\item[$\geq$] maggiore o uguale
	\item[$<$] minore
	\item[$\leq$] minore o uguale
\end{description}\index{Verso!definizione}
\section{Definizione}
Una disequazione di primo grado è una delle seguenti espressione
\begin{equation}
ax+b\quad\begin{cases}
>0\\
\geq 0\\
<0\\
\leq 0
\end{cases}\quad a\neq 0
\end{equation}\index{Disequazioni!primo grado!definizione}
\section{Disequazioni di primo grado risoluzione}
\begin{enumerate}
	\item Risolvo la disequazione separando le variabili
	\item se cambio di segno a tutto cambio di verso
	\item Ottengo la soluzione
\end{enumerate}\index{Disequazioni!primo grado!risoluzione}
% !TeX encoding = UTF-8
% !TeX spellcheck = it_IT
% !TeX root = formulario.tex

\chapter{Disequazioni di secondo grado}
\section{Definizione}
Una disequazione di secondo grado è una delle seguenti espressione
\begin{equation}
ax^2+bx+c\quad\begin{cases}
>0\\
\geq 0\\
<0\\
\leq 0
\end{cases}\quad a\neq 0
\end{equation}\index{Disequazioni!secondo grado!definizione}
\section{Risoluzione disequazione secondo grado}
\begin{enumerate}
	\item Risolvo l'equazione associata $ax^2+bx+c=0$ (trinomio)
	\item In base alle soluzioni dell'equazione e al delta costruisco il grafico
	\item Leggendo il  grafico risolvo la disequazione
\end{enumerate}\index{Disequazioni!secondo grado!risoluzione}
\section{Trinomio di secondo grado}
\begin{align}
\rosso{a}x^2+\verde{b}x+\blu{c}=&0\quad \rosso{a}\neq 0\\
x_{1,2}=&\dfrac{-\verde{b}\pm\sqrt{\verde{b}^2-4\rosso{a}\blu{c}}}{2\rosso{a}}\\
\Delta=&\verde{b}^2-4\rosso{a}\blu{c}
\end{align}\index{Disequazioni!secondo grado}\index{Delta}\index{Discriminante}
\section{Delta classificazione soluzioni}
\begin{equation}
\rosso{a}x^2+\verde{b}x+\blu{c}=0\quad \rosso{a}\neq 0\quad\begin{cases}
\text{Se $\Delta >0$ Soluzioni distinte}\\
\text{Se $\Delta =0$ Soluzioni coincidenti}\\
\text{Se $\Delta <0$ Nessuna soluzione reale}\\
\end{cases}
\end{equation}\index{Discriminante}\index{Delta}
%\begin{table}
\section{Segno del trinomio delta maggiore di zero}{\textDelta maggiore di zero}
\begin{center}\index{Disequazioni!secondo grado!segno trinomio}
		\begin{tikzpicture}
	\tkzTabInit[color,lgt=5,espcl=3]%
	{$x$ / .8,$\Delta>0$\\ Il segno di\\ $ax^2+bx+c$ /2}%
	{$-\infty$,$x_1$,$x_2$,$+\infty$}%
	\tkzTabLine{ , \genfrac{}{}{0pt}{0}{\text{segno di}}{a}, z
		, \genfrac{}{}{0pt}{0}{\text{segno}}{\text{opposto di}\ a}, z
		, \genfrac{}{}{0pt}{0}{\text{segno di}}{a}, }
	\end{tikzpicture}
	\captionof{figure}{Regola del DICE}
\end{center}\index{Delta}\index{Discriminante}
\section{Segno del trinomio delta uguale a zero}{\textDelta uguale a zero}
\begin{center}
		\begin{tikzpicture}
	\tkzTabInit[color,lgt=5,espcl=3]%
	{$x$ / .8, $\Delta=0$\\ Il segno di\\ $ax^2+bx+c$ / 2}%
	{$-\infty$,$x_1$,$+\infty$}%
	\tkzTabLine{ , \genfrac{}{}{0pt}{0}{\text{segno di}}{ a} , z
		, \genfrac{}{}{0pt}{0}{\text{segno di}}{a}, }
	\end{tikzpicture}
	\captionof{figure}{Regola Segue a}
\end{center}
\section{Segno del trinomio delta minore di zero}{\textDelta minore di zero}
\begin{center}
		\begin{tikzpicture}
	\tkzTabInit[color,lgt=5,espcl=5]%
	{$x$/.8,$\Delta<0$\\ Il segno di\\ $ax^2+bx+c$/2}%
	{$-\infty$,$+\infty$}%
	\tkzTabLine{ , \genfrac{}{}{0pt}{0}{\text{segno di}}{ a}, }
	\end{tikzpicture}
	\captionof{figure}{Regola  Solo a}
\end{center}\index{Delta}\index{Discriminante}
\section{Equazione e disequazione di secondo grado}
\begin{center}
	\begin{tabular}{CCCC}
\toprule
	& \text{Equazione} & \text{Disequazione} & \text{Disequazione} \\[.5cm]
a>0&ax^2+bx+c=0  & ax^2+bx+c>0 & ax^2+bx+c<0 \\[.5cm] 
\Delta>0	& \begin{aligned}
x_{1,2}=-\dfrac{b\pm\sqrt{b^2-4ac}}{2a}\\ x_1<x_2
\end{aligned} & x<x_1 \vee x>x_2& x_1<x<x_2 \\[.5cm]
\Delta=0	&
	x_{1,2}=-\dfrac{b}{2a} & x\neq-\dfrac{b}{2a}& \nexists\text{ soluzione} \\[.5cm] 
\Delta<0	& \nexists\text{ soluzione} & \forall\quad x &\nexists\text{ soluzione} \\[.5cm] 
\midrule
a<0	&ax^2+bx+c=0  & ax^2+bx+c>0 & ax^2+bx+c<0 \\[.5cm] 
\Delta>0	& \begin{aligned}
	x_{1,2}=-\dfrac{b\pm\sqrt{b^2-4ac}}{2a}\\x_1<x_2
\end{aligned} &  x_1<x<x_2& x<x_1 \vee x>x_2\\[.5cm] 
\Delta=0	&
x_{1,2}=-\dfrac{b}{2a} &\nexists\text{ soluzione}& x\neq-\dfrac{b}{2a}  \\[.5cm]
\Delta<0	& \nexists\text{ soluzione} & \nexists\text{ soluzione}& \forall\quad x \\
\bottomrule
 \end{tabular}
	\captionof{figure}{Equazione e disequazione di secondo grado}
\end{center}\index{Delta}\index{Discriminante}
\begin{figure}[ht]
	\centering
	\begin{subfigure}[b]{0.4\textwidth}\captionsetup{skip=10pt}\caption{$\Delta>0\; a>0$ }
	\centering\includestandalone{geometria/parabolaApDMz}
	\end{subfigure}\qquad
	\begin{subfigure}[b]{0.4\textwidth}\captionsetup{skip=10pt}	\caption{$\Delta>0\; a<0$ }
	\centering\includestandalone{geometria/parabolaAmDMz}
		\end{subfigure}\qquad\qquad
	\begin{subfigure}[b]{0.4\textwidth}\captionsetup{skip=10pt}\caption{$\Delta=0\; a>0$ }
	\centering\includestandalone{geometria/parabolaApDUz}
\end{subfigure}\qquad
\begin{subfigure}[b]{0.4\textwidth}\captionsetup{skip=10pt}\caption{$\Delta=0\; a<0$ }
	\centering\includestandalone{geometria/parabolaAmDUz}
\end{subfigure}\qquad\qquad
	\begin{subfigure}[b]{0.4\textwidth}\captionsetup{skip=10pt}	\caption{$\Delta<0\; a>0$ }
	\centering\includestandalone{geometria/parabolaApDmiz}
\end{subfigure}\qquad
\begin{subfigure}[b]{0.4\textwidth}	\captionsetup{skip=10pt}\caption{$\Delta<0\; a<0$ }
	\centering\includestandalone{geometria/parabolaAmDmiz}
\end{subfigure}
	\caption{Disequazioni secondo grado}
\end{figure}

\begin{figure}[ht]
	\centering
	\begin{subfigure}[b]{0.4\textwidth}\captionsetup{skip=10pt}\caption{$\Delta>0\; a>0$ }
		\centering\includestandalone{geometria/parabolaApDMzC}
	\end{subfigure}\qquad
	\begin{subfigure}[b]{0.4\textwidth}\captionsetup{skip=10pt}	\caption{$\Delta>0\; a<0$ }
		\centering\includestandalone{geometria/parabolaAmDMzC}
	\end{subfigure}\qquad\qquad
	\begin{subfigure}[b]{0.4\textwidth}\captionsetup{skip=10pt}\caption{$\Delta=0\; a>0$ }
		\centering\includestandalone{geometria/parabolaApDUzC}
	\end{subfigure}\qquad
	\begin{subfigure}[b]{0.4\textwidth}\captionsetup{skip=10pt}\caption{$\Delta=0\; a<0$ }
		\centering\includestandalone{geometria/parabolaAmDUzC}
	\end{subfigure}\qquad\qquad
	\begin{subfigure}[b]{0.4\textwidth}\captionsetup{skip=10pt}	\caption{$\Delta<0\; a>0$ }
		\centering\includestandalone{geometria/parabolaApDmizC}
	\end{subfigure}\qquad
	\begin{subfigure}[b]{0.4\textwidth}	\captionsetup{skip=10pt}\caption{$\Delta<0\; a<0$ }
		\centering\includestandalone{geometria/parabolaAmDmizC}
	\end{subfigure}
	\caption{Disequazioni secondo grado}
\end{figure}

%\begin{tabular}{ccc}
%	& $a>0$ &$a<0$  \\ 
%$\Delta>0$	& \includestandalone{geometria/parabolaApDMz} & \includestandalone{geometria/parabolaAmDMz} \\[0.5cm]
%$\Delta=0$	& \includestandalone{geometria/parabolaApDUz} & \includestandalone{geometria/parabolaAmDUz} \\[0.5cm]
%$\Delta<0$	& \includestandalone{geometria/parabolaApDmiz} & \includestandalone{geometria/parabolaAmDmiz} \\[0.5cm]
%\end{tabular} 
% !TeX encoding = UTF-8
% !TeX spellcheck = it_IT
% !TeX root = formulario.tex
\chapter{Disequazioni equazioni frazionarie}
\section{Definizione disequazione}
Una disequazione è frazionaria se ha al denominatore un'incognita.\index{Disequazione!frazionaria}
\section{Forma normale}
Una disequazione frazionaria è in forma normale se
\begin{equation*}
\dfrac{f(x)}{g(x)}\begin{cases}
>0\\
\geq 0\\
<0\\
\leq 0
\end{cases}\quad g(x)\neq 0
\end{equation*}
\section{Risoluzione disequazione frazionaria}
\begin{enumerate}
	\item Trovo il segno del numeratore
	\item Trovo il segno del denominatore
	\item Sovrappongo i grafici dei segni
	\item Leggo il grafico e trovo la soluzione
\end{enumerate}\index{Disequazione!frazionaria!risoluzione}
\section{Definizione equazione}
Un'equazione è frazionaria se ha al denominatore un'incognita.\index{Equazione!frazionaria}
\section{Forma normale}
Un'equazione frazionaria è in forma normale se
\begin{equation*}
\dfrac{f(x)}{g(x)}=0\quad g(x)\neq 0
\end{equation*}
\section{Risoluzione equazione frazionaria}
\begin{enumerate}
	\item Trovare i valori di $x$ per cui $g(x)=0$
	\item Risolvere $f(x)=0$ 
	\item Verificare se le soluzioni annullano il denominatore 
\end{enumerate}\index{Equazione!frazionaria!risoluzione}
{\centering\captionof{table}{Equazioni e disequazioni frazionarie}\index{Disequazione!frazionaria}\index{Equazione!frazionaria}
	\begin{tabular}{Cp{0.4\textwidth}}
\toprule
	& Soluzione \\ 
\midrule
\dfrac{f(x)}{g(x)}=0	& Ha soluzione per quei valori di $x$ per cui $f(x)=0$ e $g(x)\neq 0$  \\ 
\dfrac{f(x)}{g(x)}>0	& Ha soluzione per quei valori di $x$ per cui $f(x)$ e $g(x)$ sono concordi e $g(x)\neq 0$  \\ 
\dfrac{f(x)}{g(x)}<0	& Ha soluzione per quei valori di $x$ per cui $f(x)$ e $g(x)$ sono discordi e $g(x)\neq 0$  \\ 
\bottomrule
\end{tabular}\par}
% !TeX encoding = UTF-8
% !TeX spellcheck = it_IT
% !TeX root = formulario.tex
\chapter{Prodotto di disequazioni e equazioni}
\section{Forma normale}
Una disequazione prodotto è in forma normale se
\begin{equation*}
f(x)\cdot g(x)\begin{cases}
>0\\
\geq 0\\
<0\\
\leq 0
\end{cases}
\end{equation*}\index{Disequazione!prodotto}
\section{Risoluzione disequazione prodotto}
\begin{enumerate}
	\item Trovare il segno dei fattori
	\item Sovrapporre i grafici dei segni
	\item Leggere il grafico e trovare la soluzione
\end{enumerate}\index{Disequazione!prodotto!risoluzione}
\section{Forma normale}
Un'equazione prodotto è in forma normale se
\begin{equation*}
f(x)\cdot g(x)=0
\end{equation*}
\section{Risoluzione equazione prodotto}
\begin{enumerate}
	\item Risolvere, ponendoli uguali a zero, i fattori che compongono il prodotto.
\end{enumerate}\index{Equazione!prodotto!risoluzione}
\begin{center}
	\begin{tabular}{Cp{0.4\textwidth}}
		\toprule
		& Soluzione \\ 
		\midrule
		f(x)\cdot g(x)=0	& Ha soluzione per quei valori di $x$ per cui $f(x)=0$ e $g(x)= 0$  \\ 
		f(x)\cdot g(x)>0	& Ha soluzione per quei valori di $x$ per cui $f(x)$ e $g(x)$ sono concordi\\ 
	f(x)\cdot g(x)<0& Ha soluzione per quei valori di $x$ per cui $f(x)$ e $g(x)$ sono discordi\\ 
		\bottomrule
	\end{tabular}\captionof{table}{Equazioni e disequazioni prodotto}\index{Disequazione!prodotto}\index{Equazione!prodotto}
\end{center}
\chapter{Misura di angoli}
\section{Gradi sessagesimali}
\begin{align*}
\ang{1}=&\dfrac{angolo giro}{360}\\
\ang{;1;}=&\dfrac{\ang{1}}{60}\\
\ang{;;1}=&\dfrac{\ang{;1;}}{60}=\dfrac{\ang{1}}{3600}
\end{align*}\index{Angoli!gradi!sessagesimali}
\begin{equation*}
\alpha=x^\circ y'z''
\end{equation*}
\section{Gradi sessa-decimali}
\begin{equation*}
\beta=x.y^\circ\quad\text{x parte intera, y parte decimale}
\end{equation*}\index{Angoli!gradi!sessa-decimali}
\section{Conversione da gradi sessagesimali a gradi sessa-decimali}
\begin{align*}
\alpha=&x^\circ y'z''\\
\beta=&x^\circ+\left(\dfrac{y}{60}\right)^\circ+\left(\dfrac{z}{3600}\right)^\circ
\end{align*}\index{Angoli!gradi!conversione}
\section{Conversione da gradi sessa-decimali  a gradi sessagesimali}
\begin{enumerate}
	\item Inizio
	\item $\beta=x.y^\circ$
	\item La parte intera di $\beta$ $x^\circ$ sono i gradi
	\item Per ottenerei i minuti $m'=(x.y^\circ-x^\circ)*60$
	\item $m'=z.k'$
	\item $z$ la parte intera di $m$ sono i minuti 
	\item Per ottenere i secondi
	$s''=(z.k'-z')*60$ 
	\item $s''=p.q''$
	\item $p$ la parte intera di $s$ sono i secondi
	\item Stop
\end{enumerate}
\section{Radianti}
\begin{center}
	\includestandalone{geometria/radianti}
	\captionof{figure}{Radianti}
\end{center}\index{Angoli!radianti}
\begin{align*}
\rho=\dfrac{arco}{raggio}=\dfrac{l}{r}
\end{align*}\index{Angoli!radianti}
\section{Da radianti a gradi sessa-decimali}
\begin{equation*}\index{Angoli!conversione!radianti gradi}
\alpha=\dfrac{\ang{180}}{\pi}\rho
\end{equation*}
\section{Da gradi sessa-decimali a radianti}
\begin{equation*}
\rho=\dfrac{\pi}{\ang{180}}\alpha
\end{equation*}\index{Angoli!conversione!gradi radianti}	
% !TeX encoding = UTF-8
% !TeX spellcheck = it_IT
% !TeX root = formulario.tex
%\onecolumn
\chapter{Goniometria}\label{ch:goniometria}
%{\centering\captionof{table}{Valori particolari di funzioni goniometriche}
%	%	\renewcommand{\arraystretch}{2}
%	\begin{tabular}{cccccc}
%		\toprule
%		Gradi & Radianti & Seno & Coseno & Tangente & Cotangente \\ [.25cm]
%		%\midrule
%		$\ang{0}$ & 0 & 0 & 1 & 0 & n.e. \\ [.25cm] 
%%	\midrule%
%	$\ang{15}$ &$\dfrac{1}{12}\pi$ &$\dfrac{1}{4}\left(\sqrt{6}-\sqrt{2}\right)$&$\dfrac{1}{4}\left(\sqrt{6}+\sqrt{2}\right)$&$2-\sqrt{3}$& $2+\sqrt{3}$ \\ [.25cm]
%%		\hline%
%	%
%		$\ang{18}$&$\dfrac{1}{10}\pi$& $\dfrac{1}{4}\left(\sqrt{5}-1\right)$ & $\dfrac{1}{4}\sqrt{10+2\sqrt{5}}$ & $\dfrac{1}{5}\sqrt{25-10\sqrt{5}}$ & $\sqrt{5+2\sqrt{5}}$ \\ [.25cm]
%	%	\hline%
%		$\ang{22;30;}$&$\dfrac{1}{8}\pi$&$\dfrac{1}{2}\sqrt{2-\sqrt{2}}$&$\dfrac{1}{2}\sqrt{2+\sqrt{2}}$&$\sqrt{2}-1$&$\sqrt{2}+1$ \\ [.25cm]
%%		\hline%
%		$\ang{30}$&$\dfrac{1}{6}\pi$&$\dfrac{1}{2}$&$\dfrac{\sqrt{3}}{2}$&$\dfrac{\sqrt{3}}{3}$&$\sqrt{3}$\\ [.25cm]
%%		\hline%
%		$\ang{36}$&$\dfrac{1}{5}\pi$&$\dfrac{1}{4}\sqrt{10-2\sqrt{5}}$&$\dfrac{1}{4}\left(\sqrt{5}+1\right)$&$\sqrt{5-2\sqrt{5}}$&$\dfrac{1}{5}\sqrt{25+10\sqrt{5}}$\\ [.4cm]
%%		\hline%
%		$\ang{45}$&$\dfrac{1}{4}\pi$&$\dfrac{\sqrt{2}}{2}$& $\dfrac{\sqrt{2}}{2}$ & 1 & 1 \\ [.4cm]
%%		\hline%
%		$\ang{54}$&$\dfrac{3}{10}\pi$& $\dfrac{1}{4}\left(\sqrt{5}+1\right)$ & $\dfrac{1}{4}\sqrt{10-2\sqrt{5}}$ & $\dfrac{1}{5}\sqrt{25+10\sqrt{5}}$ & $\sqrt{5-2\sqrt{5}}$ \\ [.25cm]
%%		\hline%
%		$\ang{60}$&$\dfrac{1}{3}\pi$&$\dfrac{\sqrt{3}}{2}$&$\dfrac{1}{2}$&$\sqrt{3}$&$\dfrac{\sqrt{3}}{3}$\\ [.25cm]
%%		\hline%
%$\ang{65;30;}$&$\dfrac{3}{8}\pi$&$\dfrac{1}{2}\sqrt{2+\sqrt{2}}$&$\dfrac{1}{2}\sqrt{2-\sqrt{2}}$&$\sqrt{2}+1$&$\sqrt{2}-1$ \\ [.25cm]
%%		\hline%
%		$\ang{72}$&$\dfrac{2}{5}\pi$&$\dfrac{1}{4}\sqrt{10+2\sqrt{5}}$&$\dfrac{1}{4}\left(\sqrt{5}-1\right)$&$\sqrt{5+2\sqrt{5}}$&$\dfrac{1}{5}\sqrt{25-10\sqrt{5}}$\\ [.4cm]
%%		\hline%
%		$\ang{75}$ &$\dfrac{5}{12}\pi$ &$\dfrac{1}{4}\left(\sqrt{6}+\sqrt{2}\right)$&$\dfrac{1}{4}\left(\sqrt{6}-\sqrt{2}\right)$&$2+\sqrt{3}$& $2-\sqrt{3}$ \\ [.25cm]
%%		\hline%
%		$\ang{90}$&$\dfrac{\pi}{2}$&1&0&n.e.&0\\[.25cm]
%	%	\hline%
%		$\ang{180}$&$\pi$&0&-1& 0 &n.e.\\ [.25cm]
%	%	\hline%
%		$\ang{270}$&$\dfrac{3}{2}\pi$&-1&0&n.e.&0\\ [.25cm]
%%		\hline%
%		$\ang{360}$&$2\pi$&0&1&0&n.e.\\ [.25cm]
%		\bottomrule%
%	\end{tabular}
%\par}
%\twocolumn
\begin{table}
		\begin{tabular}{cccccc}
		\toprule
		Gradi & Radianti & Seno & Coseno & Tangente & Cotangente \\ [.25cm]
		%\midrule
		$\ang{0}$ & 0 & 0 & 1 & 0 & n.e. \\ [.25cm] 
		%	\midrule%
		$\ang{15}$ &$\dfrac{1}{12}\pi$ &$\dfrac{1}{4}\left(\sqrt{6}-\sqrt{2}\right)$&$\dfrac{1}{4}\left(\sqrt{6}+\sqrt{2}\right)$&$2-\sqrt{3}$& $2+\sqrt{3}$ \\ [.25cm]
		%		\hline%
		%
		$\ang{18}$&$\dfrac{1}{10}\pi$& $\dfrac{1}{4}\left(\sqrt{5}-1\right)$ & $\dfrac{1}{4}\sqrt{10+2\sqrt{5}}$ & $\dfrac{1}{5}\sqrt{25-10\sqrt{5}}$ & $\sqrt{5+2\sqrt{5}}$ \\ [.25cm]
		%	\hline%
		$\ang{22;30;}$&$\dfrac{1}{8}\pi$&$\dfrac{1}{2}\sqrt{2-\sqrt{2}}$&$\dfrac{1}{2}\sqrt{2+\sqrt{2}}$&$\sqrt{2}-1$&$\sqrt{2}+1$ \\ [.25cm]
		%		\hline%
		$\ang{30}$&$\dfrac{1}{6}\pi$&$\dfrac{1}{2}$&$\dfrac{\sqrt{3}}{2}$&$\dfrac{\sqrt{3}}{3}$&$\sqrt{3}$\\ [.25cm]
		%		\hline%
		$\ang{36}$&$\dfrac{1}{5}\pi$&$\dfrac{1}{4}\sqrt{10-2\sqrt{5}}$&$\dfrac{1}{4}\left(\sqrt{5}+1\right)$&$\sqrt{5-2\sqrt{5}}$&$\dfrac{1}{5}\sqrt{25+10\sqrt{5}}$\\ [.4cm]
		%		\hline%
		$\ang{45}$&$\dfrac{1}{4}\pi$&$\dfrac{\sqrt{2}}{2}$& $\dfrac{\sqrt{2}}{2}$ & 1 & 1 \\ [.4cm]
		%		\hline%
		$\ang{54}$&$\dfrac{3}{10}\pi$& $\dfrac{1}{4}\left(\sqrt{5}+1\right)$ & $\dfrac{1}{4}\sqrt{10-2\sqrt{5}}$ & $\dfrac{1}{5}\sqrt{25+10\sqrt{5}}$ & $\sqrt{5-2\sqrt{5}}$ \\ [.25cm]
		%		\hline%
		$\ang{60}$&$\dfrac{1}{3}\pi$&$\dfrac{\sqrt{3}}{2}$&$\dfrac{1}{2}$&$\sqrt{3}$&$\dfrac{\sqrt{3}}{3}$\\ [.25cm]
		%		\hline%
		$\ang{65;30;}$&$\dfrac{3}{8}\pi$&$\dfrac{1}{2}\sqrt{2+\sqrt{2}}$&$\dfrac{1}{2}\sqrt{2-\sqrt{2}}$&$\sqrt{2}+1$&$\sqrt{2}-1$ \\ [.25cm]
		%		\hline%
		$\ang{72}$&$\dfrac{2}{5}\pi$&$\dfrac{1}{4}\sqrt{10+2\sqrt{5}}$&$\dfrac{1}{4}\left(\sqrt{5}-1\right)$&$\sqrt{5+2\sqrt{5}}$&$\dfrac{1}{5}\sqrt{25-10\sqrt{5}}$\\ [.4cm]
		%		\hline%
		$\ang{75}$ &$\dfrac{5}{12}\pi$ &$\dfrac{1}{4}\left(\sqrt{6}+\sqrt{2}\right)$&$\dfrac{1}{4}\left(\sqrt{6}-\sqrt{2}\right)$&$2+\sqrt{3}$& $2-\sqrt{3}$ \\ [.25cm]
		%		\hline%
		$\ang{90}$&$\dfrac{\pi}{2}$&1&0&n.e.&0\\[.25cm]
		%	\hline%
		$\ang{180}$&$\pi$&0&-1& 0 &n.e.\\ [.25cm]
		%	\hline%
		$\ang{270}$&$\dfrac{3}{2}\pi$&-1&0&n.e.&0\\ [.25cm]
		%		\hline%
		$\ang{360}$&$2\pi$&0&1&0&n.e.\\ [.25cm]
		\bottomrule%
	\end{tabular}
	\caption{Valori particolari di funzioni goniometriche}\label{tab:Valori-funzioni-goniometriche}
\end{table}
\begin{table}
	\centering
	\begin{tabular}{CCCC}
	\toprule
	\multicolumn{1}{c}{Giro}	&  \multicolumn{1}{c}{RAD}&  \multicolumn{1}{c}{DEG}& \multicolumn{1}{c}{GRAD} \\ 
\midrule
	0	& 0 &\ang{0;;}  & 0^g \\ 
	 \dfrac{1}{12}& \dfrac{\pi}{6} & \ang{30;;} &  33.3^g\\[.25cm] 
	\dfrac{1}{8}& \dfrac{\pi}{4} & \ang{45;;} &  50^g\\[.25cm] 	
	\dfrac{1}{6}& \dfrac{\pi}{3} & \ang{60;;} &  66.6^g\\[.25cm] 
	\dfrac{1}{4}& \dfrac{\pi}{2} & \ang{90;;} &  100^g\\[.25cm] 
	\dfrac{1}{2}& \pi & \ang{180;;} &  200^g\\[.25cm] 
	\dfrac{3}{4}& \dfrac{3}{2}\pi & \ang{270;;} &  300^g\\[.25cm] 
	1& 2\pi & \ang{360;;} &  400^g\\[.25cm] 
	\bottomrule
	\end{tabular} 
	\caption{Radianti, gradi, gradi centesimali }\label{tab:Radianti_gradi_gradi_cent}
\end{table}
\section{Circonferenza goniometrica}\label{sec:circonferenza-goniometrica}
\begin{defn}[Circonferenza goniometrica]\label{defn:Circoferenza-Gonimetrica}
Dato un sistema di riferimento cartesiana, una circonferenza goniometria è una circonferenza con centro nell'origine e raggio unitario
\begin{equation*}
x^2+y^2=1
\end{equation*}\index{Circonferenza!goniometrica}
\end{defn}
\section{Definizione coseno}\label{sec:definizione-coseno}
\begin{defn}[Coseno]\label{defn:coseno1}
Data una circonferenza goniometrica ed un angolo $\alpha$ diremo coseno dell'angolo $\alpha$, l'ascissa del punto $P$.
\end{defn}\index{Coseno!definizione}
\begin{figure}
	\centering
	\includestandalone{geometria/cosenodefinizione}
	\caption{Definizione coseno}
	\label{fig:cosenodefinizione}
\end{figure}\index{Funzione!coseno!definizione}
%{\centering
%	\includestandalone{geometria/cosenodefinizione}
%	\captionof{figure}{Definizione coseno}\par}\index{Funzione!coseno!definizione}
\begin{prop}[Il coseno è limitato]\label{prop:Cosenolimitato}
Il coseno è limitato\index{Funzione!limitata!coseno}\index{Coseno!limitato}
e assume i valori compresi
\begin{equation*}
-1\leq \cos\alpha \leq 1
\end{equation*}
\end{prop}
\begin{prop}[Periodo coseno]\label{Prop:periodocoseno}
Il coseno è periodico. 
Se l'angolo è misurato in gradi il coseno è periodico di periodo $k\ang{360}$\index{Funzione!periodica!coseno}
\begin{equation*}
\cos(\alpha+k\ang{360;;})=\cos\alpha
\end{equation*}
Se l'angolo è misurato in radianti il coseno è periodico di periodo $2k\pi$\index{Funzione!periodica!coseno}
\begin{equation*}
\cos(\alpha+2k\pi)=\cos\alpha
\end{equation*}
\end{prop}
\section{Definizione seno}\label{sec:definizione-seno}
\begin{defn}[Definizione seno]
Data una circonferenza goniometrica ed un angolo $\alpha$ chiamo seno l'ordinata del punto $P$.\index{Seno!definizione}
\end{defn}
%{\centering
%	\includestandalone{geometria/senodefinizione}
%	\captionof{figure}{Definizione seno}\par}\index{Funzione!seno!definizione}
\begin{figure}
	\centering
	\includestandalone{geometria/senodefinizione}
	\caption{Definzione seno}
	\label{fig:senodefinizione}
\end{figure}\index{Funzione!seno!definizione}
\begin{prop}[Il seno è limitato]\label{prop:senolimitato}
	Il seno è limitato\index{Funzione!limitata!seno}\index{Seno!limitato}
e	assume i valori compresi
	\begin{equation*}
	-1\leq \sin\alpha \leq 1
	\end{equation*}
\end{prop}
\begin{prop}[Periodo seno]\label{Prop:periodoseno}
	Il seno è periodico. 
	Se l'angolo è misurato in gradi il seno è periodico di periodo $k\ang{360}$\index{Funzione!periodica!seno}
	\begin{equation*}
	\sin(\alpha+k\ang{360;;})=\sin\alpha
	\end{equation*}
	Se l'angolo è misurato in radianti il seno è periodico di periodo $2k\pi$\index{Funzione!periodica!seno}
	\begin{equation*}
	\sin(\alpha+2k\pi)=\sin\alpha
	\end{equation*}
\end{prop}
\section{Definizione tangente}
\begin{defn}[Definizione tangente]
	Data una circonferenza goniometrica ed un angolo $\alpha$ diremo tangente dell'angolo $\alpha$ l'ordinata del punto $T$ intersezione tra il prolungamento del raggio e la tangente alla circonferenza passante per il punto $(1,0)$  .\index{Funzione!tangente!definizione}
\end{defn}
\begin{prop}[La tangente non è limitata]\label{prop:tangentenonlimitata}
La tangente è non limitata e assume i seguenti valori:
\begin{equation*}
-\infty<\tan\alpha<\infty
\end{equation*}\index{Funzione!illimitata!tangente}
\end{prop}
\begin{figure}
	\centering
	\includestandalone{geometria/tangentedefinizione}
	\caption{Definizione tangente}
	\label{fig:tangentedefinizione}
\end{figure}\index{Funzione!tangente!definizione}
%{\centering
%	\includestandalone{geometria/tangentedefinizione}
%	\captionof{figure}{Definizione tangente}\par}\index{Funzione!tangente!definizione}
\begin{prop}[Periodo tangente]\label{prop:PeriodoTangente}
	La tangente è periodica.
	Se l'angolo è espresso in gradi la tangente è periodica di periodo $k\ang{180}$\index{Funzione!periodica!tangente}
	\begin{equation*}
	\tan(\alpha+k\ang{180;;})=\tan\alpha
	\end{equation*}
	Se l'angolo è espresso in radianti la tangente è periodica di periodo $k\pi$\index{Funzione!periodica!tangente}
	\begin{equation*}
	\tan(\alpha+k\pi)=\tan\alpha
	\end{equation*}
\end{prop}
\section{Definizione cotangente}
\begin{defn}[Definizione cotangente]
	Data una circonferenza goniometrica ed un angolo $\alpha$ diremo cotangente dell'angolo $\alpha$ l'ascissa del punto $C$ intersezione tra il prolungamento del raggio e la tangente alla circonferenza passante per il punto $(0,1)$  .\index{Funzione!cotangente!definizione}
\end{defn}
\begin{prop}[La cotangente non è limitata]\label{prop:cotangentenonlimitata}
	La tangente è non limitata e assume i seguenti valori:
	\begin{equation*}
	-\infty<\cot\alpha< \infty
	\end{equation*}\index{Funzione!illimitata!cotangente}
\end{prop}
\begin{figure}
	\centering
	\documentclass[10pt]{standalone}
\usepackage{amsmath}
\usepackage{pgf,tikz}
\usetikzlibrary{calc}
\usepackage{mathrsfs}
\usetikzlibrary{arrows}
\pagestyle{empty}
\begin{document}
	

		\begin{tikzpicture}[>=triangle  45]
		\pgfmathsetmacro{\raggio}{3};
		\pgfmathsetmacro{\angolob}{50};
		\pgfmathsetmacro{\angolo}{\angolob};
		\pgfmathsetmacro{\y}{3};
		\coordinate [label=below right:$O$] (oo)  at (0,0);
		\coordinate[label=above:$P$] (P) at ({\raggio*cos(\angolo)},{\raggio*sin(\angolo)});
		\coordinate(Q) at (\raggio,0);
		\coordinate[label=above:$P$] (P) at ({\raggio*cos(\angolo)},{\raggio*sin(\angolo)});
		\coordinate(Q) at (0,\raggio);
		\coordinate[label=above right:$C$] (T) at ({\raggio*(cot(\angolo ))},\raggio);
		\coordinate[label=above:$\cot\alpha$] (M)at ($(T)!0.5!(Q)$);
		\draw[->] (-\raggio-1,0) -- (\raggio+1,0) node[above] {$x$} ;
		\draw[->] (0,-\raggio-1) -- (0,\raggio+1) node[left] {$y$} ;
		\draw (-\raggio-1,\raggio) -- (\raggio+1,\raggio) ;
		\draw (oo) circle (\raggio) ;
		\draw[->] (\raggio/\y,0 ) arc (0:\angolo:\raggio/\y) ;
		\draw (\angolo/2:\raggio/\y) node[ above right]  {$\alpha$};
		\draw (oo)-- (P) ;
		\draw (T)-- (oo) ;
		\filldraw[black] (oo) circle(1pt);
		\filldraw[black] (P) circle(1pt);
		\filldraw[black] (T) circle(1pt);
		\filldraw[black] (Q) circle(1pt);
		\end{tikzpicture}
\end{document}
	\caption{Definizione cotangente}
	\label{fig:cotangentedefinizione}
\end{figure}\index{Funzione!cotangente!definizione}
 %{\centering
%	\includestandalone{geometria/cotangentedefinizione}
%	\captionof{figure}{Definizione cotangente}\par}\index{Funzione!cotangente!definizione}
\begin{prop}[Periodo cotangente]\label{prop:PeriodoCotangente}
	La cotangente è periodica.
	Se l'angolo è espresso in gradi la cotangente è periodica di periodo $k\ang{180}$\index{Funzione!periodica!cotangente}
	\begin{equation*}
	\cot(\alpha+k\ang{180;;})=\cot\alpha
	\end{equation*}
	Se l'angolo è espresso in radianti la cotangente è periodica di periodo $k\pi$\index{Funzione!periodica!cotangente}
	\begin{equation*}
	\cot(\alpha+k\pi)=\cot\alpha
	\end{equation*}
\end{prop}

\section{Relazioni fondamentali goniometria}\label{sec:relazioni-fondamentali-goniometria}
\begin{align*}	
	&\cos^{2}\alpha+\sin^{2}\alpha=1\\[.25cm]\index{Goniometria!relazione!fondamentale}
	&\tan\alpha={}\dfrac{\sin\alpha}{\cos\alpha}&\alpha\neq\dfrac{\pi}{2}+k\pi\\[.25cm]\index{Tangente!definizione}
	&\cot\alpha={}\dfrac{\cos\alpha}{\sin\alpha}&\alpha\neq k\pi\\ \index{Cotangente!definizione}
	&\tan\alpha\cot\alpha={}1&\alpha\neq\dfrac{\pi}{2}+k\pi\quad\alpha\neq k\pi
\end{align*}\index{Funzione!seno}\index{Funzione!coseno}\index{Funzione!tangente}\index{Funzione!cotangente}
\section{Relazioni derivate}\label{sec:relazioni-derivate}
\begin{align*}
\cos^{2}\alpha=&{}1-\sin^{2}\alpha\\\index{Coseno!dato!seno}
\cos\alpha=&{}\pm\sqrt{1-\sin^{2}\alpha}\\
\sin^{2}\alpha=&{}1-\cos^{2}\alpha \\\index{Seno!dato!coseno}
\sin^{2}\alpha=&{}\pm\sqrt{1-\cos^{2}\alpha}\\
\cos^{2}\alpha=&{}\dfrac{1}{1+{\tan}^{2}\alpha} &&\alpha\neq\dfrac{\pi}{2}+k\pi\\
\cos\alpha=&{}\pm\dfrac{1}{\sqrt{1+{\tan}^{2}\alpha}} &&\alpha\neq\dfrac{\pi}{2}+k\pi\\\index{Coseno!dato!tangente}
\sin^{2}\alpha=&{}\dfrac{\tan^{2}\alpha}{1+\tan^{2}\alpha}&&\alpha\neq\dfrac{\pi}{2}+k\pi\\
\sin\alpha=&{}\pm\dfrac{\tan\alpha}{\sqrt{1+\tan^{2}\alpha}}&&\alpha\neq\dfrac{\pi}{2}k\pi\\\index{Seno!dato!tangente}
 \sin\alpha=&{}\tan\alpha\cos\alpha&&\alpha\neq\dfrac{\pi}{2}+k\pi\\\index{Seno!dato!tangente}
\sin^{2}\alpha=&{}\dfrac{1}{1+{\cot}^{2}\alpha} &&\alpha\neq k\pi\\
\cos\alpha=&{}\pm\dfrac{1}{\sqrt{1+{\cot}^{2}\alpha}} &&\alpha\neq k\pi\\\index{Coseno!dato!cotangente}
\cos^{2}\alpha=&{}\dfrac{\cot^{2}\alpha}{1+\cot^{2}\alpha}&&\alpha\neq k\pi\\
\cos\alpha=&{}\pm\dfrac{\cot\alpha}{\sqrt{1+\cot^{2}\alpha}}&&\alpha\neq+k\pi\\\index{Coseno!dato!cotangente}
\cos\alpha=&{}\cot\alpha\sin\alpha&&\alpha\neq k\pi\\\index{Coseno!dato!cotangente} 
\tan\alpha=&{}\pm\dfrac{\sin(x)}{\sqrt{1-\sin^2(x)}}&&\alpha\neq\dfrac{\pi}{2}+k\pi\\\index{Tangente!dato!seno}
\tan\alpha=&{}\pm\dfrac{\sqrt{1-\cos^2(x)}}{\cos(x)}
&&\alpha\neq\dfrac{\pi}{2}+k\pi\\ \index{Tangente!dato!coseno}
\cot\alpha=&{}\pm\dfrac{\cos(x)}{\sqrt{1-\cos^2(x)}}&&\alpha\neq k\pi\\\index{Cotangente!dato!coseno}
\cot\alpha=&{}\pm\dfrac{\sqrt{1-\sin^2(x)}}{\sin(x)}
&&\alpha\neq k\pi\index{Cotangente!dato!seno}
\end{align*}
%{\centering\index{Seno!quadrato}\index{Coseno!quadrato}\index{Tangente!quadrato}%\index{Cotangente!quadrato}
%	\begin{tabular}{LCCC}
%		\toprule 
%		& \sin(x) & \cos(x) &\tan(x)  \\
%		\midrule
%		\sin(x)	&\sin(x)  & \pm\sqrt{1-\cos^2(x)} & \pm\dfrac{\tan(x)}{\sqrt{1+\tan^2(x)}} \\[.5cm]
%		\cos(x)	&\pm\sqrt{1-\sin^2(x)}  & \cos(x) & \pm\dfrac{1}{\sqrt{1+\tan^2(x)}} \\[.5cm]
%		\tan(x)	& \pm\dfrac{\sin(x)}{\sqrt{1-\sin^2(x)}} &\pm\dfrac{\sqrt{1-\cos^2(x)}}{\cos(x)}   & \tan(x) \\[.5cm]
%		\bottomrule
%	\end{tabular}\captionof{table}{Relazioni derivate}
%\par}
\section{Formule di addizione e sottrazione}\label{sec:formule-di-addizione}
\begin{align*}
\cos\left(\alpha-\beta\right)=&{}\cos\alpha\cos\beta+\sin\alpha\sin\beta\\\index{Coseno!differenza!angoli}
\cos\left(\alpha+\beta\right)=&{}\cos\alpha\cos\beta-\sin\alpha\sin\beta\\\index{Coseno!somma!angoli}
\sin\left(\alpha-\beta\right)=&{}\sin\alpha\cos\beta-\cos\alpha\sin\beta\\\index{Seno!differenza angoli}
\sin\left(\alpha+\beta\right)=&{}\sin\alpha\cos\beta+\cos\alpha\sin\beta\\\index{Seno!somma!angoli}
\tan\left(\alpha-\beta\right)=&{}\dfrac{\tan\alpha-\tan\beta}{1+\tan\alpha\tan\beta} &\alpha,\beta,\left(\alpha-\beta\right)\neq\left(2k+1\right)\dfrac{\pi}{2}\\\index{Tangente!differenza!angoli}
\cot\left(\alpha+\beta\right)=&{}\dfrac{\cot\alpha\cot\beta-1}{\cot\beta+\cot\alpha}
&\alpha,\beta,\left(\alpha+\beta\right)\neq k\pi\index{Cotangente!somma!angoli}
\end{align*}
\section{Angoli supplementari}\label{sec:angoli-supplementari}
\begin{align*}
\cos\alpha=&-\cos(\pi-\alpha)\\
\sin\alpha=&\sin(\pi-\alpha)\\
\tan\alpha=&-\tan(\pi-\alpha)\\
\cot\alpha=&-\cot(\pi-\alpha)
\end{align*}
\section{Angoli la cui differenza è \texorpdfstring{$\pi$}{\textpi}}
\begin{align*}
\cos\alpha=&-\cos(\pi+\alpha)\\
\sin\alpha=&-\sin(\pi+\alpha)\\
\tan\alpha=&\tan(\pi+\alpha)\\
\cot\alpha=&\cot(\pi+\alpha)
\end{align*}
\section{Angoli esplementari}\label{sec:angoli-esplementari}
\begin{align*}
\cos\alpha=&\cos(2\pi-\alpha)\\
\sin\alpha=&-\sin(2\pi-\alpha)\\
\tan\alpha=&-\tan(2\pi-\alpha)\\
\cot\alpha=&-\cot(2\pi-\alpha)
\end{align*}
\section{Angoli opposti}\label{sec:angoli-opposti}
\begin{align*}
\cos\alpha=&\cos(-\alpha)\\
\sin\alpha=&-\sin(-\alpha)\\
\tan\alpha=&-\tan(-\alpha)\\
\cot\alpha=&-\cot(-\alpha)
\end{align*}
\section{Angoli complementari}\label{sec:angoli-complementari}
\begin{align*}
\cos\alpha=&\sin(\dfrac{\pi}{2}-\alpha)\\
\sin\alpha=&\cos(\dfrac{\pi}{2}-\alpha)\\
\tan\alpha=&\cot(\dfrac{\pi}{2}-\alpha)\\
\cot\alpha=&\tan(\dfrac{\pi}{2}-\alpha)\\
\end{align*}
\section{Angoli la cui differenza è \texorpdfstring{$\dfrac{\pi}{2}$}{\textpi/2} }
\begin{align*}
\cos\alpha=&\sin(\dfrac{\pi}{2}+\alpha)\\
\sin\alpha=&-\cos(\dfrac{\pi}{2}+\alpha)\\
\tan\alpha=&-\cot(\dfrac{\pi}{2}+\alpha)\\
\cot\alpha=&-\tan(\dfrac{\pi}{2}+\alpha)\\
\end{align*}
\section{Angoli la cui somma è \texorpdfstring{$\pi$}{\textpi}}
\begin{align*}
\cos\alpha=&-\sin(\dfrac{3}{2}\pi-\alpha)\\
\sin\alpha=&-\cos(\dfrac{3}{2}\pi-\alpha)\\
\tan\alpha=&\cot(\dfrac{3}{2}\pi-\alpha)\\
\cot\alpha=&\tan(\dfrac{3}{2}\pi-\alpha)\\
\end{align*}
\section{Angoli la cui differenza è \texorpdfstring{$\dfrac{3}{2}\pi$}{3/2 \textpi}}
\begin{align*} 
\cos\alpha=&-\sin(\dfrac{3}{2}\pi-\alpha)\\
\sin\alpha=&\cos(\dfrac{3}{2}\pi-\alpha)\\
\tan\alpha=&-\cot(\dfrac{3}{2}\pi-\alpha)\\
\cot\alpha=&-\tan(\dfrac{3}{2}\pi-\alpha)\\
\end{align*}
\section{Funzioni inverse}\label{sec:funzioni-inverse}
\begin{align*}
\alpha=&\arcsin x\quad\Longleftrightarrow\quad\sin\alpha=x&& x\in[-1,1]&\alpha\in[-\dfrac{\pi}{2},\dfrac{\pi}{2}]\\
\alpha=&\arccos x\quad\Longleftrightarrow\quad\cos\alpha=x&&x\in[-1,1]&\alpha\in[0,\pi]\\
\alpha=&\arctan x\quad\Longleftrightarrow\quad\tan\alpha=x&& x\in\R
&\alpha\in(-\dfrac{\pi}{2},\dfrac{\pi}{2})
\end{align*}\index{Funzione!inversa!tangente}\index{Funzione!inversa!seno}\index{Funzione!inversa!coseno}		
% !TeX encoding = UTF-8
% !TeX spellcheck = it_IT
% !TeX root = formulario.tex
\chapter{Equazioni goniometriche}
\section{Equazioni goniometriche elementari}
\begin{align*}
\intertext{$x\in[0,2\pi]$}
	\sin(x)=&m &0<m<1\\
	x_1=&\arcsin(m)+2k\pi &k\in\Z\\
	x_2=&\pi-\arcsin(m)+2k\pi\\
\sin(x)=&m &-1<m<0\\
x_1=&\pi+\arcsin(\abs{m})+2k\pi &k\in\Z\\
x_2=&2\pi-\arcsin(\abs{m})+2k\pi\\
\sin(x)=&1\\\index{Equazione!goniometrica!seno}
x_1=&\dfrac{\pi}{2}+2k\pi &k\in\Z\\
\sin(x)=&-1\\
x_1=&\dfrac{3}{2}\pi+2k\pi &k\in\Z\\
\sin(x)=&0\\
x_1=&k\pi &k\in\Z
\intertext{$x\in[0,2\pi]$}
\cos(x)=&m &0<m<1\\\index{Equazione!goniometrica!coseno}
x_1=&\arccos(m)+2k\pi &k\in\Z\\
x_2=&2\pi-\arccos(m)+2k\pi\\
\cos(x)=&m &-1<m<0\\
x_1=&\pi-\arccos(\abs{m})+2k\pi &k\in\Z\\
x_2=&\pi+\arccos(\abs{m}))+2k\pi\\
\cos(x)=&1\\
x_1=&2k\pi &k\in\Z\\
\cos(x)=&-1\\
x_1=&\pi+2k\pi &k\in\Z\\
\cos(x)=&0\\
x_1=&k\dfrac{\pi}{2} &k\in\Z
\intertext{$x\in[0,2\pi]$}
\tan(x)=&m&m\geq0\\\index{Equazione!goniometrica!tangente}
x=&\arctan(m)+k\pi&k\in\Z\\
\tan(x)=&m&m<0\\
x=&\pi-\arctan(\abs{m})+k\pi&k\in\Z
\intertext{$x\in[-\pi,\pi]$}
\sin(x)=&m &0<m<1\\
x_1=&\arcsin(m)+2k\pi &k\in\Z\\
x_2=&\pi-\arcsin(m)+2k\pi\\
\sin(x)=&m &-1<m<0\\
x_1=&-\arcsin(\abs{m})+2k\pi &k\in\Z\\
x_2=&-\pi+\arcsin(\abs{m})+2k\pi\\
\sin(x)=&1\\
x_1=&\dfrac{\pi}{2}+2k\pi &k\in\Z\\
\sin(x)=&-1\\
x_1=&-\dfrac{\pi}{2}+2k\pi &k\in\Z\\
\sin(x)=&0\\
x_1=&k\pi &k\in\Z
\intertext{$x\in[-\pi,\pi]$}
\cos(x)=&m &0<m<1\\
x_1=&\arccos(m)+2k\pi &k\in\Z\\
x_2=&-\arccos(m)+2k\pi\\
\cos(x)=&m &-1<m<0\\
x_1=&\pi-\arccos(\abs{m})+2k\pi &k\in\Z\\
x_2=&-\pi+\arccos(\abs{m})+2k\pi\\
\cos(x)=&1\\
x_1=&2k\pi &k\in\Z\\
\cos(x)=&-1\\
x_1=&-\pi+2k\pi &k\in\Z\\
\cos(x)=&0\\
x_1=&k\dfrac{\pi}{2} &k\in\Z
\intertext{$x\in[-\pi,\pi]$}
\tan(x)=&m&m\geq0\\
x=&\arctan(m)+k\pi&k\in\Z\\
\end{align*}
\section{Equazioni riconducibili ad equazioni elementari}
\begin{align*}
\intertext{Tipo uno}
\sin(\alpha)=&\sin(\beta)\\
\alpha=&\beta+2k\pi\\
\alpha=&\pi-\beta+2k\pi\\\index{Equazione!goniometrica!seno}
\intertext{Tipo due}
\cos(\alpha)=&\cos(\beta)\\
\alpha=&\pm\beta+2k\pi\\\index{Equazione!goniometrica!coseno}
\intertext{Tipo tre}
\tan(\alpha)=\tan(\beta)\\
\alpha=&\beta+k\pi\index{Equazione!goniometrica!coseno}
\end{align*}
\begin{align*}
\intertext{Riconducibili al tipo uno}
\sin(\alpha)=&-\sin(\beta)&&\sin(\alpha)=\sin(-\beta)\\
\sin(\alpha)=&\cos(\beta)&&\sin(\alpha)=\sin(\frac{\pi}{2}-\beta)\\
\sin(\alpha)=&-\cos(\beta)&&\sin(-\alpha)=\sin(\frac{\pi}{2}-\beta)\\
\sin^2(\alpha)=&\sin^2(\beta)
&&\begin{cases}
\sin(\alpha)=\sin(\beta)\\
\sin(\alpha)=-\sin(\beta)
\end{cases}\\
\intertext{Riconducibili al tipo due}
\cos(\alpha)=&-\cos(\beta)&&\cos(\alpha)=\cos(\pi-\beta)\\
\cos^2(\alpha)=&\cos^2(\beta)
&&\begin{cases}
\cos(\alpha)=\cos(\beta)\\
\cos(\alpha)=-\cos(\beta)
\end{cases}\\
\intertext{Riconducibili al tipo tre}
\tan(\alpha)=&-\tan(\beta)&&\tan(\alpha)=\tan(-\beta)\\
\tan^2(\alpha)=&\tan^2(\beta)
&&\begin{cases}
\tan(\alpha)=\tan(\beta)\\
\tan(\alpha)=-\tan(\beta)
\end{cases}
\end{align*}
\section{Equazioni risolvibili tramite  equazioni di secondo grado}
\begin{align*}
a\cos^2\alpha+b\cos\alpha+c=&0\quad a\neq 0\\
\cos\alpha=&t\\
at^2+bt+c=&0
\intertext{$\Delta>0$}
t_1=&\frac{-b+\sqrt{b^2-4ac}}{2a}\\
t_2=&\frac{-b-\sqrt{b^2-4ac}}{2a}
\intertext{se $t_1\in[-1,+1]$}
\cos\alpha=&t_1\\
\intertext{se $t_2\in[-1,+1]$}
\cos\alpha=&t_2
\intertext{$\Delta=0$}
t_1=&-\frac{b}{2a}\\
\intertext{se $t_1\in[-1,+1]$}
\cos\alpha=&t_1
\intertext{$\Delta<0$}
\intertext{Nessuna soluzione}
\end{align*}\index{Equazione!goniometrica!secondo grado}\index{Discriminante}\index{Delta}
\begin{align*}
a\sin^2\alpha+b\sin\alpha+c=&0\quad a\neq 0\\
\sin\alpha=&t\\
at^2+bt+c=&0
\intertext{$\Delta>0$}
t_1=&\frac{-b+\sqrt{b^2-4ac}}{2a}\\
t_2=&\frac{-b-\sqrt{b^2-4ac}}{2a}
\intertext{se $t_1\in[-1,+1]$}
\sin\alpha=&t_1\\
\intertext{se $t_2\in[-1,+1]$}
\sin\alpha=&t_2
\intertext{$\Delta=0$}
t_1=&-\frac{b}{2a}\\
\intertext{se $t_1\in[-1,+1]$}
\sin\alpha=&t_1\\
\intertext{$\Delta<0$}
\intertext{Nessuna soluzione}
\end{align*}\index{Equazione!goniometrica!secondo grado}\index{Discriminante}\index{Delta}
\begin{align*}
a\tan^2\alpha+b\tan\alpha+c=&0\quad a\neq 0\\
\tan\alpha=&t\\
at^2+bt+c=&0
\intertext{$\Delta>0$}
t_1=&\frac{-b+\sqrt{b^2-4ac}}{2a}\\
t_2=&\frac{-b-\sqrt{b^2-4ac}}{2a}
\tan\alpha=&t_1\\
\tan\alpha=&t_2
\intertext{$\Delta=0$}
t_1=&-\frac{b}{2a}\\
\tan\alpha=&t_1\\
\intertext{$\Delta<0$}
\intertext{Nessuna soluzione}
\end{align*}\index{Equazione!goniometrica!secondo grado}\index{Discriminante}\index{Delta}
% !TeX encoding = UTF-8
% !TeX spellcheck = it_IT
% !TeX root = formulario.tex
\chapter{Forma goniometrica numeri complessi}
\section{Argomento}
\begin{align*}
z=&a+b\uimm& z\in\Co\quad a,b\in\R\\[8pt]
\theta=\arg(z)=&\\
&\arctan(\dfrac{b}{a})&\text{se}\ a>0 \\[8pt]
&\arctan(\dfrac{b}{a})+\pi&\text{se}\ a<0 \ \text{e} \ b\geq0\\[8pt]
&\arctan(\dfrac{b}{a})-\pi&\text{se}\ a<0\ \text{e} \ b<0\\[8pt]
&+\dfrac{\pi}{2}&\text{se}\ a=0 \ \text{e} \ b>0\\[8pt]
&-\dfrac{\pi}{2}&\text{se}\ a=0 \ \text{e} \ b<0\\[8pt]
&\text{non definito}&\text{se}\ a=0 \ \text{e} \ b=0
\end{align*}\index{Numero!complesso!argomento}
Se $\theta\in]-\pi,\pi]$
 \section{Modulo}
\begin{equation}
z=a+b\uimm\quad r=\abs{z}=\sqrt{a^2+b^2}\quad a,b\in\R
\end{equation}\index{Numero!complesso!modulo}
\section{Da forma algebrica a goniometrica di numero complesso}
\begin{align*}
z=&a+b\uimm\quad z\in\Co\quad a,b\in\R\\
\theta=&\arg(z)\\
r=&\abs{z}\\
z=&r[\cos\theta+\uimm\sin\theta]
\end{align*}\index{Numero!complesso!forma goniometrica}\index{Numero!complesso!modulo}\index{Numero!complesso!argomento}
\section{Da forma goniometrica ad algebrica di numero complesso}
\begin{align*}
z=&r[\cos\theta+\uimm\sin\theta]\\
a=&r\cos\theta\\
b=&r\sin\theta
\end{align*}\index{Numero!complesso!forma algebrica}\index{Numero!complesso!modulo}\index{Numero!complesso!argomento}

% !TeX encoding = UTF-8
% !TeX spellcheck = it_IT
% !TeX root = formulario.tex
\chapter{Trigonometria}
\section{Triangolo rettangolo}
\begin{center}
		\includestandalone{geometria/triangolopitagorico1}
	\captionof{figure}{Triangolo rettangolo}
\end{center}
\begin{equation*}
\dfrac{\pi}{2}+\beta+\gamma=\pi
\end{equation*}\index{Triangolo!somma!angoli interni}
\begin{align*}
b=&a\sin\beta=a\cos\gamma\\
c=&a\sin\gamma=a\cos\beta
\end{align*}\index{Triangolo!rettangolo!cateto}
\begin{align*}
\sin\beta=&\dfrac{b}{a}=\dfrac{\text{opposto}}{\text{ipotenusa}}\\
\cos\beta=&\dfrac{c}{a}=\dfrac{\text{adiacente}}{\text{ipotenusa}}\\
\sin\gamma=&\dfrac{c}{a}=\dfrac{\text{opposto}}{\text{ipotenusa}}\\
\cos\gamma=&\dfrac{b}{a}=\dfrac{\text{adiacente}}{\text{ipotenusa}}
\end{align*}\index{Triangolo!rettangolo!adiacente}
\index{Triangolo!rettangolo!opposto}
\begin{equation*}
a=\dfrac{b}{\sin\beta}=\dfrac{b}{\cos\gamma}=\dfrac{c}{\sin\gamma}=\dfrac{c}{\cos\beta}
\end{equation*}
\begin{align*}
b=&c\tan\beta\\
c=&b\tan\gamma
\end{align*}
\begin{align*}
\tan\beta=&\dfrac{b}{c}=\dfrac{\text{opposto}}{\text{adiacente}}\\
\tan\gamma=&\dfrac{c}{b}=\dfrac{\text{opposto}}{\text{adiacente}}
\end{align*}\index{Triangolo!rettangolo!adiacente}
\index{Triangolo!rettangolo!opposto}
\section{Triangolo qualunque}
\begin{center}
	\includestandalone{geometria/triangoloisc}
	\captionof{figure}{Triangolo qualunque}
\end{center}
\begin{equation*}
\alpha+\beta+\gamma=\pi
\end{equation*}\index{Triangolo!somma!angoli interni}
\section{Teorema della corda}
\begin{align*}
a=&2R\sin\alpha\\
b=&2R\sin\beta\\
c=&2R\sin\gamma
\end{align*}\index{Corda!teorema}\index{Teorema!corda}
\section{Teorema dei seni}
\begin{equation*}
\dfrac{a}{\sin\alpha}=\dfrac{b}{\sin\beta}=\dfrac{c}{\sin\gamma}=2R
\end{equation*}\index{Triangolo!teorema seni}\index{Seno!teorema}\index{Teorema!seno}
\section{Raggio cerchio circoscritto}
\begin{equation*}
R=\dfrac{a}{2\sin\alpha}=\dfrac{b}{2\sin\beta}=\dfrac{c}{2\sin\gamma}
\end{equation*}\index{Triangolo!raggio!circoscritto}\index{Circonferenza!triangolo!circoscritto}
\section{Teorema di Carnot}
\begin{align*}
a^2=&b^2+c^2-2bc\cos\alpha\\
b^2=&a^2+c^2-2ac\cos\beta\\
c^2=&a^2+b^2-2ab\cos\gamma
\end{align*}\index{Triangolo!teorema Carnot}\index{Carnot!teorema}\index{Teorema!Carnot}\index{Triangolo!lato}
\section{Area triangolo}
\begin{align*}
S=&\dfrac{1}{2}bc\sin\alpha\\
S=&\dfrac{1}{2}ba\sin\gamma\\
S=&\dfrac{1}{2}ac\sin\beta
\end{align*}\index{Triangolo!area}
\begin{align*}
S=&\dfrac{1}{2}a^2\dfrac{\sin\beta\sin\gamma}{\sin\alpha}\\
S=&\dfrac{1}{2}b^2\dfrac{\sin\alpha\sin\gamma}{\sin\beta}\\
S=&\dfrac{1}{2}c^2\dfrac{\sin\alpha\sin\beta}{\sin\gamma}
\end{align*}\index{Triangolo!area}
\section{Risoluzione triangolo rettangolo}
\begin{center}
	\includestandalone{geometria/triangolopitagorico1}
	\captionof{figure}{Triangolo rettangolo}
\end{center}\index{Triangolo!rettagolo!cateto}\index{Triangolo!rettagolo!ipotenusa}\index{Triangolo!rettagolo!angolo acuto}\index{Triangolo!rettagolo!adiacente}\index{Triangolo!rettagolo!opposto}
\subsection{Ipotenusa e angolo acuto noto}
\begin{enumerate}
	\item $a$ noto
	\item $\beta$ noto
\end{enumerate}
\begin{align*}
c=&a\cos\beta\\
b=&a\sin\beta\\
\gamma=&\frac{\pi}{2}-\beta\\
\end{align*}
\subsection{Cateto ed un angolo acuto noto}
\begin{enumerate}
	\item $b$ noto
	\item $\beta$ noto
\end{enumerate}
\begin{align*}
\gamma=&\frac{\pi}{2}-\beta&\gamma=&\frac{\pi}{2}-\beta\\
a=&\frac{b}{\sin\beta}&c=&b\tan\gamma\\
c=&a\cos\beta&a=&\sqrt{b^2+c^2}
\end{align*}
\subsection{Ipotenusa e un cateto noto}
\begin{enumerate}
	\item $a$ noto
	\item $b$ noto
\end{enumerate}
\begin{align*}
c=&\sqrt{a^2-b^2}\\
\beta=&\arcsin\dfrac{b}{a}\\
\gamma=&\frac{\pi}{2}-\beta
\end{align*}
\subsection{Due cateti noti}
\begin{enumerate}
	\item $b$ noto
	\item $c$ noto
\end{enumerate}
\begin{align*}
\beta=&\arctan\frac{b}{c}&a=&\sqrt{b^2+c^2}\\
\gamma=&\frac{\pi}{2}-\beta&\beta=&\arcsin\frac{b}{a}\\
a=&\sqrt{b^2+c^2}&\gamma=&\frac{\pi}{2}-\beta\\
\end{align*}
% !TeX encoding = UTF-8
% !TeX spellcheck = it_IT
% !TeX root = formulario.tex
\chapter{Geometria Analitica}
\section{Distanza tra due punti}
\begin{align*}
A\coord{x_1}{y_2}&& B\coord{x_2}{y_2}&&&  d(AB)=\abs{x_2-x_1}\\
A\coord{x_1}{y_1} && B\coord{x_1}{y_2}&&&  d(AB)=\abs{y_2-y_1}\\
A\coord{x_1}{y_1} && B\coord{x_2}{y_2}&&&  d(AB)=\sqrt{(x_1-x_2)^2+(y_1-y_2)^2}
\end{align*}\index{Distanza!due punti}
\section{Punto medio}
\begin{align*}
A\coord{x_1}{y_1}&& B\coord{x_2}{y_2}&&& M\coord{\dfrac{x_1+x_2}{2}}{\dfrac{y_1+y_2}{2}}
\end{align*}\index{Punto!medio}
\section{Baricentro}
\begin{align*}
A\coord{x_1}{y_1}& &B\coord{x_2}{y_2}&&C\coord{x_3}{y_3}&& G\coord{\dfrac{x_1+x_2+x_3}{3}}{\dfrac{y_1+y_2+y_3}{3}}
\end{align*}\index{Barricentro}

\chapter{Parabola con asse parallelo asse ordinate}
\section{Equazione}
\begin{equation*}
y=ax^2+bx+c\quad a\neq0
\end{equation*}\index{Parabola!asse!parallelo a asse y}\index{Funzione!parabola}\index{Funzione!quadratica}
\section{Asse di simmetria parabola}
\begin{equation*}
x=-\dfrac{b}{2a}\quad a\neq0
\end{equation*}\index{Parabola!asse!simmetria}
\section{Fuoco}
\begin{equation*}
F\coord{-\dfrac{b}{2a}}{\dfrac{1-\Delta}{4a}}\quad  a\neq0\; \Delta=b^2-4ac
\end{equation*}\index{Parabola!fuoco}\index{Discriminante}\index{Delta}
\section{Vertice}
\begin{equation*}
V\coord{-\dfrac{b}{2a}}{-\dfrac{\Delta}{4a}}\quad  a\neq0\; \Delta=b^2-4ac
\end{equation*}\index{Parabola!vertice}\index{Discriminante}\index{Delta}
\section{Direttrice parabola}
\begin{equation*}
y=-\dfrac{1+\Delta}{4a}\quad  a\neq0,\; \Delta=b^2-4ac
\end{equation*}\index{Parabola!direttrice}\index{Discriminante}
\chapter{Parabola con asse parallelo asse ascisse}
\section{Equazione}
\begin{equation}
x=ay^2+by+c\quad a\neq0
\end{equation}\index{Parabola!asse parallelo a asse x}
\section{Asse di simmetria parabola}
\begin{equation}
y=-\dfrac{b}{2a}\quad a\neq0
\end{equation}\index{Parabola!asse simmetria}
\section{Fuoco}
\begin{equation}
F\coord{\dfrac{1-\Delta}{4a}}{-\dfrac{b}{2a}}\quad  a\neq0\; \Delta=b^2-4ac
\end{equation}\index{Parabola!fuoco}\index{Discriminante}\index{Delta}
\section{Vertice}
\begin{equation}
V\coord{-\dfrac{\Delta}{4a}}{-\dfrac{b}{2a}}\quad  a\neq0\; \Delta=b^2-4ac
\end{equation}\index{Parabola!vertice}\index{Discriminante}\index{Delta}
\section{Direttrice parabola}
\begin{equation}
x=-\dfrac{1+\Delta}{4a}\quad  a\neq0\;\Delta=b^2-4ac
\end{equation}\index{Parabola!direttrice}\index{Discriminante}
\chapter{Parabola e retta}
\section{Posizioni retta parabola}
\begin{equation*}
\begin{cases}
y=ax^2+bx+c\\
y=mx+q
\end{cases}
\end{equation*}\index{Parabola!retta}\index{Retta!parabola}
\begin{equation*}
ax^2+(b-m)x+c-q=0\quad\begin{cases}
\text{Se $\Delta >0$ Secante}\\
\text{Se $\Delta =0$ Tangente}\\
\text{Se $\Delta <0$ Esterna}\\
\end{cases}
\end{equation*}\index{Parabola!intersezione!retta}\index{Discriminante}\index{Retta!secante}\index{Retta!tangente}\index{Retta!esterna}\index{Delta}

\include{Circonferenza}
\chapter{Forma cartesiana numeri complessi}
\section{Definizione}
\begin{equation*}
z=a+b\uimm\quad z\in\Co\quad a,b\in\R
\end{equation*} 
{\centering
	\includestandalone{geometria/ncomplessi}
	\captionof{figure}{Numero complesso nel piano}\par}\index{Numero!complesso!forma cartesiana}
\section{Complessi opposti}
\begin{equation*}
z=a+b\uimm\quad\-z=-a-b\uimm\quad a,b\in\R
\end{equation*}\index{Numero!complesso!opposto}
{\centering
	\includestandalone{geometria/ncomplessiopposti}
	\captionof{figure}{Numeri complessi opposti}\par}\index{Numero!complesso!opposto}
\section{Complessi coniugati}
\begin{equation*}
z=a+b\uimm\quad\conj{z}=a-b\uimm\quad a,b\in\R
\end{equation*}\index{Numero!complesso!coniugato}
{\centering
	\includestandalone{geometria/ncomplessiconiugati}
	\captionof{figure}{Numeri complessi coniugati}\par}\index{Numero!complesso!coniugato}
\section{Somma di numeri complessi}
\begin{equation*}
w=u+v
\end{equation*}\index{Numero!complesso!somma}
{\centering
	\includestandalone{geometria/ncomplessisomma}
	\captionof{figure}{Numeri complessi somma}\par}\index{Numero!complesso!somma}
%\onecolumn
%\twocolumn[text]
\chapter{Logaritmi}
\section{Definizione}
Il logaritmo $x$ di $b$ in base $a$ è
\begin{equation*}
x=\log_{a}b \quad\Longleftrightarrow\quad a^x=b\quad a>0,\; a\neq 1,\; b>0
\end{equation*}\index{Logaritmo!definizione}
\section{Logaritmo decimale}
se la base è $e$ il logaritmo si dice naturale e si scrive $\ln$
\begin{equation*}
\log_{e}b=\ln b
\end{equation*}\index{Logaritmo!naturale}
\section{Proprietà fondamentali}
\begin{align*}
a^{\log_{a}b}={}&b&a>0,\; a\neq 1,\; b>0\\
\log_{a}a^c=&c&a>0,\; a\neq 1\\
\log_{a}1=&0&a>0,\; a\neq 1\\
\log_{a}a=&1&a>0,\; a\neq 1
\end{align*}\index{Logaritmo!proprietà!fondamentali}
\section{Logaritmo prodotto}
\begin{equation*}
\log_{a}b\cdot c=\log_{a}b+\log_{a}c\quad a>0,\;a\neq 1,\;b>0,\;c>0 
\end{equation*}\index{Logaritmo!prodotto}
\section{Logaritmo somma}
\begin{equation*}
\log_{a}b+\log_{a}c=\log_{a}b\cdot c\quad a>0,\;a\neq 1,\;b>0,\;c>0 
\end{equation*}\index{Logaritmo!somma}
\section{Logaritmo quoziente}
\begin{equation*}
\log_{a}\dfrac{b}{c}=\log_{a}b-\log_{a}c\quad a>0,\;a\neq 1,\;b>0,\;c>0 
\end{equation*}\index{Logaritmo!quoziente}
\section{Logaritmo differenza}
\begin{equation*}
\log_{a}b-\log_{a}c=\log_{a}\dfrac{b}{c}\quad a>0,\;a\neq 1,\;b>0,\;c>0 
\end{equation*}\index{Logaritmo!differenza}
\section{Logaritmo potenza}
\begin{equation*}
\log_{a}b^m=m\log_{a}b\quad a>0,\;a\neq 1,\;b>0,\;c>0,\;m\in\R
\end{equation*}\index{Logaritmo!potenza}
\section{Logaritmo radicale}
\begin{equation*}
\log_{a}\sqrt[n]{b}=\dfrac{1}{n}\log_{a}b\quad a>0,\;a\neq 1,\;b>0,\;n\in\Nz
\end{equation*}\index{Logaritmo!radicale}
\section{Cambiamento di base}
\begin{equation*}
\log_{a}b=\dfrac{\log_{c}b}{\log_{c}a}\quad a>0,\;a\neq 1,\;c>0,\;c\neq 1,\;b>0
\end{equation*}\index{Logaritmo!cambio di base}
\section{Utili}
\begin{align*}
\log_{a}b=&\dfrac{1}{\log_{b}a}&a>0,\;a\neq 1,\;b>0,\;b\neq 1\\
\log_{\frac{1}{a}}b=&-\log_{a}b&a>0,\;a\neq 1,\;b>0
\end{align*}\index{Logaritmo!reciproco!base}\index{Logaritmo!scambio!base argomento}
\chapter{Intervalli e intorni}
\section{Intervallo chiuso}
\begin{equation*}
\left[a\; ;\; b\right]=\lbrace x\in\R\quad a\leq x\leq b\rbrace
\end{equation*}\index{Intervallo!chiuso}
\section{Intervallo aperto}
\begin{equation*}
\left(a\; ;\; b\right)=\lbrace x\in\R\quad a<x<b\rbrace
\end{equation*}\index{Intervallo!aperto}
\section{Intervallo semi aperto destro}
\begin{equation*}
\left[a\; ;\; b\right)=\lbrace x\in\R\quad a<x<b\rbrace
\end{equation*}\index{Intervallo!semi aperto!destro}
\section{Intervallo semi aperto sinistro}
\begin{equation*}
\left(a\; ;\; b\right]=\lbrace x\in\R\quad a<x\leq b\rbrace
\end{equation*}\index{Intervallo!semi aperto!sinistro}
\section{Intervallo non limitato chiuso a sinistra}
\begin{equation*}
\left[a\; ;\; +\infty\right)=\lbrace x\in\R\quad a\leq x \rbrace
\end{equation*}\index{Intervallo!non limitato!chiuso a sinistra }
\section{Intervallo non limitato aperto a sinistra}
\begin{equation*}
\left(a\; ;\; +\infty\right)=\lbrace x\in\R\quad x> a\rbrace
\end{equation*}\index{Intervallo!non limitato!aperto a sinistra}
\section{Intervallo non limitato chiuso a destra}
\begin{equation*}
\left(-\infty\; ;\; b\right]=\lbrace x\in\R\quad x\leq b\rbrace
\end{equation*}\index{Intervallo!non limitato!chiuso a destra}
\section{Intervallo non limitato aperto a destra}
\begin{equation*}
\left(-\infty\; ;\; b\right)=\lbrace x\in\R\quad x< b\rbrace
\end{equation*}\index{Intervallo!non limitato!aperto a destra}
\section{Intorno circolare di un punto}
Intorno circolare\index{Intorno!circolare} di $x_0$ e raggio $\delta>0$
\begin{equation*}
I(x_0)=\left(x_0-\delta,x_0+\delta
\right)\quad\delta>0
\end{equation*}
\chapter{Funzioni}
\section{Definizione}
Una funzione è una relazione che ad ogni elemento di $x\in A$ dominio\index{Funzione!dominio}, fa corrispondere uno e uno solo elemento appartenente  a $B$ codominio\index{Funzione!codominio}
\begin{equation*}
\function{f}{A}{B}{x}{f(x)}
\end{equation*}\index{Funzione!definizione}
\section{Funzione pari}
Una funzione $\funzione{f}{A}{B}$ è una funzione pari se 
\begin{equation*}
f(-x)=f(x)\quad\forall x\in A
\end{equation*}\index{Funzione!pari}
\section{Funzione dispari}
Una funzione $\funzione{f}{A}{B}$ è una funzione dispari se 
\begin{equation*}
f(-x)=-f(x)\quad\forall x\in A
\end{equation*}\index{Funzione!dispari}
\section{Funzione limitata}
Una funzione $\funzione{f}{A}{B}$ è una funzione limitata se 
\begin{equation*}
\exists\; M\quad \abs{f(x)}<M \quad\forall x\in A
\end{equation*}\index{Funzione!limitata}
\section{Funzione iniettiva}
Una funzione $\funzione{f}{A}{B}$ è iniettiva se
\begin{equation*}
 x_1\neq x_2\quad\Longrightarrow\quad f(x_1)\neq f(x_2)\quad \forall x_1,x_2\in A
\end{equation*}
oppure
\begin{equation*}
f(x_1)= f(x_2)\quad\Longrightarrow\quad  x_1= x_2\quad \forall x_1,x_2\in A
\end{equation*}\index{Funzione!iniettiva}
\section{Funzione suriettiva}
Una funzione $\funzione{f}{A}{B}$ è suriettiva se
\begin{equation*}
	f(A)=B
\end{equation*}\index{Funzione!suriettiva} 
\section{Funzione biettiva}
Una funzione $\funzione{f}{A}{B}$ è biettiva se è contemporaneamente suriettiva e iniettiva\index{Funzione!biettiva}
\section{Funzione periodica}
Una funzione $\funzione{f}{A}{B}$ è periodica di periodo $T>0$ se
\begin{equation*}
f(x+kT)=f(x)\quad k\in Z
\end{equation*}\index{Funzione!periodica}
\section{Funzione crescente in senso stretto}
Una funzione $\funzione{f}{A}{B}$ si dice crescente in senso stretto nell'intervallo  $I\subset A$ se
\begin{equation*}
\forall\; x_1,x_2\in I\quad x_1< x_2\Longrightarrow f(x_1)<f(x_2)
\end{equation*}\index{Funzione!crescente in senso stretto}
\section{Funzione non decrescente}
Una funzione $\funzione{f}{A}{B}$ si dice non decrescente nell'intervallo  $I\subset A$ se
\begin{equation*}
\forall\; x_1,x_2\in I\quad x_1< x_2\Longrightarrow f(x_1)\leq f(x_2)
\end{equation*}\index{Funzione!non decrescente}
\section{Funzione decrescente in senso stretto}
Una funzione $\funzione{f}{A}{B}$ si dice decrescente in senso stretto nell'intervallo  $I\subset A$ se
\begin{equation*}
\forall\; x_1,x_2\in I\quad x_1< x_2\Longrightarrow f(x_1)>f(x_2)
\end{equation*}\index{Funzione!decrescente in senso stretto}
\section{Funzione non crescente}
Una funzione $\funzione{f}{A}{B}$ si dice non crescente nell'intervallo  $I\subset A$ se
\begin{equation*}
\forall\; x_1,x_2\in I\quad x_1< x_2\Longrightarrow f(x_1)\geq f(x_2)
\end{equation*}\index{Funzione!non crescente}
..\section{Zeri funzione}
Data una funzione $\funzione{f}{A}{B}$ $a\in A$ è uno zero per la funzione se $f(a)=0$\index{Funzione!zero}
\section{Funzione algebrica}
Una funzione $\funzione{f}{A}{B}$ si dice algebrica se costruita utilizzando un numero finito di applicazione delle quattro operazioni dell'aritmetica, dell'elevazione a potenza e delle radici.\index{Funzione!algebrica} 
\section{Funzione algebrica razionale}
Una funzione $\funzione{f}{A}{B}$ algebrica è razionale quando la variabile indipendente non si trova sotto il segno di radice\index{Funzione!algebrica!razionale} \section{Funzione algebrica irrazionale}
Una funzione $\funzione{f}{A}{B}$ algebrica è irrazionale quando la variabile indipendente  si trova sotto il segno di radice\index{Funzione!algebrica!irrazionale} 
\section{Funzione algebrica intera}
Una funzione $\funzione{f}{A}{B}$ algebrica è intera quando la variabile indipendente non si trova al denominatore di una frazione\index{Funzione!algebrica!intera}
\section{Funzione algebrica fratta}
Una funzione $\funzione{f}{A}{B}$ algebrica è fratta quando la variabile indipendente si trova al denominatore di una frazione\index{Funzione!algebrica!fratta}	
\section{Funzione trascendente}
	Una funzione $\funzione{f}{A}{B}$  è trascendente quando compaiono operazioni non algebriche\index{Funzione!trascendente} come logaritmo, esponenziale, goniometriche.
\section{Funzione insieme di definizione}
Data una funzione $\funzione{f}{A}{B}$, diremo insieme di definizione della funzione un sotto insieme del dominio in cui la funzione è effettivamente definita.\index{Funzione!insieme!definizione} L'insieme di definizione è noto come campo di esistenza.\index{Funzione!campo!esistenza}

\begin{center}
	\includestandalone{geometria/DominioFunzRazzioIrrazio}
	\captionof{figure}{Insieme di definizione funzioni razionali e irrazionali}
\end{center}\index{Funzione!razionale!insieme definizione}\index{Funzione!irrazionale!insieme definizione}
% !TeX encoding = UTF-8
% !TeX spellcheck = it_IT
% !TeX root = formulario.tex
\chapter{Funzione esponenziale}
\section{Definizione}
Una funzione esponenziale è una funzione del tipo
\begin{equation*}
\function{f}{\R}{\Rpos}{x}{a^x}\quad a>0\quad a\neq1
\end{equation*}\index{Funzione!esponenziale!definizione}
\section{Caso base maggiore di uno}
Se la base a è maggiore di uno
\begin{itemize}
	\item La funzione è trascendente\index{Funzione!trascendente}
	\item Il dominio è $\R$
	\item Il codominio è $\Rpos$
	\item La funzione passa per $A(0,1)$
	\item La funzione è crescente in senso stretto $
	\forall\; x_1,x_2\in I\quad x_1< x_2\Longrightarrow f(x_1)<f(x_2)$\index{Funzione!crescente!in senso stretto}
	\item L'asse delle $x$ è un asintoto orizzontale per la funzione $\lim_{x\to-\infty} f(x)=0$\index{Asintoto!orizzontale} 
\end{itemize}
{\centering
	\includestandalone{geometria/expMadiuno}
	\captionof{figure}{Funzione esponenziale a >1}\par}\index{Funzione!esponenziale}
\section{Caso base compresa tra zero e uno}
Se la base a è compresa tra zero e uno
\begin{itemize}
	\item La funzione è trascendente\index{Funzione!trascendente}
	\item Il dominio è $\R$
	\item Il codominio è $\Rpos$
	\item La funzione passa per $A(0,1)$
	\item La funzione è decrescente in senso stretto $
	\forall\; x_1,x_2\in I\quad x_1< x_2\Longrightarrow f(x_1)>f(x_2)$\index{Funzione!decrescente!in senso stretto}
	\item L'asse delle $x$ è un asintoto orizzontale per la funzione $\lim_{x\to +\infty} f(x)=0$\index{Asintoto!orizzontale} 
\end{itemize}
{\centering
	\includestandalone{geometria/expMidiuno}
	\captionof{figure}{Funzione esponenziale 0<a<1}\par}\index{Funzione!esponenziale}

% !TeX encoding = UTF-8
% !TeX spellcheck = it_IT
% !TeX root = formulario.tex
\chapter{Funzione logaritmo}
\section{Definizione}
Una funzione logaritmo è una funzione del tipo
\begin{equation*}
\function{f}{\Rpos}{\R}{x}{\log_{a}x}\quad a>0\quad a\neq1\quad x>0
\end{equation*}
\section{Caso base maggiore di uno}
Se la base a è maggiore di uno
\begin{itemize}
	\item La funzione è trascendente\index{Funzione!trascendente}
	\item Il dominio è $\Rpos$
	\item Il codominio è $\R$
	\item La funzione passa per $A(1,0)$
	\item La funzione è crescente in senso stretto $
	\forall\; x_1,x_2\in I\quad x_1< x_2\Longrightarrow f(x_1)<f(x_2)$\index{Funzione!crescente!in senso stretto}
	\item L'asse delle $y$ è un asintoto verticale per la funzione $\lim_{x\to 0^+} {f(x)}=-\infty$\index{Asintoto!orizzontale} 
\end{itemize}
{\centering
	\includestandalone{geometria/logMadiuno}
	\captionof{figure}{Funzione logaritmo a >1}\par}\index{Funzione!logaritmica}
\section{Caso base compresa tra zero e uno}
Se la base a è compresa tra zero e uno
\begin{itemize}
	\item La funzione è trascendente\index{Funzione!trascendente}
	\item Il dominio è $\Rpos$
	\item Il codominio è $\R$
	\item La funzione passa per $A(1,0)$
	\item La funzione è decrescente in senso stretto $
	\forall\; x_1,x_2\in I\quad x_1< x_2\Longrightarrow f(x_1)>f(x_2)$\index{Funzione!decrescente!in senso stretto}
	\item L'asse delle $y$ è un asintoto verticale per la funzione $\lim_{x\to 0^+} {f(x)}=+\infty$\index{Asintoto!orizzontale} 
\end{itemize}
{\centering
	\includestandalone{geometria/logMidiuno}
	\captionof{figure}{Funzione logaritmica 0<a<1}\par}\index{Funzione!logaritmica}
% !TeX encoding = UTF-8
% !TeX spellcheck = it_IT
% !TeX root = formulario.tex
\chapter{Funzioni irrazionali}\index{Funzione!irrazionale}
\section{Irrazionale intera pari}\index{Funzione!irrazionale!intera}
\begin{equation*}
f(x)=\sqrt[n]{A(x)}\quad n=2,4,6,\dots
\end{equation*}
\subsection{Dominio e positività}
\begin{enumerate}
	\item Dominio: $A(x)\geq 0$
	\item Positività: Sempre positiva
\end{enumerate}
\section{Irrazionale intera dispari}
\begin{equation*}\index{Funzione!irrazionale!dispari}
f(x)=\sqrt[n]{A(x)}\quad n=3,5,7,\dots
\end{equation*}\index{Funzione!irrazionale!pari}
\subsection{Dominio e positività}
\begin{enumerate}
	\item Dominio: Sempre definita
	\item Positività: $A(x)\geq 0$
\end{enumerate}
\section{Irrazionali fratte pari}\index{Funzione!irrazionale!fratta}
\begin{align*}
f(x)=&\dfrac{\sqrt{N(x)}}{D(x)}
\intertext{Dominio}\index{Funzione!irrazionale!dominio}
&\begin{cases}
N(x)\geq 0\\
D(x)\neq0
\end{cases}
\intertext{Positività}
D(x)>0&\wedge Dominio\\
f(x)=&\dfrac{N(x)}{\sqrt{D(x)}}
\intertext{Dominio}
D(x)>&0
\intertext{Positività}
N(x)\geq0&\wedge Dominio\\
f(x)=&\sqrt{\dfrac{N(x)}{D(x)}}\\
\intertext{Dominio}
N(x)\geq0 &\wedge D(x)>0
\intertext{Positività}
\intertext{Sempre positiva}
\end{align*}
\section{Irrazionali fratte dispari}\index{Funzione!irrazionale!fratta}
\begin{align*}
f(x)=&\dfrac{\sqrt[n]{N(x)}}{D(x)}
\intertext{Dominio}\index{Funzione!irrazionale!dominio}
D(x)\neq&0
\intertext{Positività}
&\begin{cases}
N(x)\geq 0\\
D(x)>0
\end{cases}\\
f(x)=&\dfrac{N(x)}{\sqrt[n]{D(x)}}
\intertext{Dominio}
D(x)\neq&0
\intertext{Positività}
&\begin{cases}
N(x)\geq 0\\
D(x)>0
\end{cases}\\
f(x)=&\sqrt[n]{\dfrac{N(x)}{D(x)}}\\
\intertext{Dominio}
D(x)\neq&0
\intertext{Positività}
&\begin{cases}
N(x)\geq 0\\
D(x)>0
\end{cases}
\end{align*}

% !TeX encoding = UTF-8
% !TeX spellcheck = it_IT
% !TeX root = formulario.tex
\chapter{Funzione seno}
\section{Definizione}
\begin{equation*}
y=\sin(x)
\end{equation*}
\section{Proprietà}
\begin{itemize}
	\item La funzione è trascendente\index{Funzione!trascendente}
\item Il dominio è $\R$
\item La funzione è limitata\index{Funzione!limitata!seno}, il codominio è $[-1,1]$
\item La funzione è periodica di periodo $k\ang{360;;}$ o $2k\pi$\index{Funzione!periodica!seno}
\end{itemize}
{\centering
	\includestandalone{geometria/senografico}
	\captionof{figure}{Funzione seno grafico}\par}\index{Funzione!seno}
\chapter{Funzione sinusoide}
\begin{equation*}
y=A\sin(\omega t+\phi)
\end{equation*}
\section{Proprietà}
\begin{itemize}
	\item La funzione è trascendente\index{Funzione!trascendente}
	\item Il dominio è $\R$
	\item $t$ si misura  in \SI[parse-numbers=false]{}{\second}
	\item $\tau=\dfrac{2\pi}{\omega}$ periodo si misura  in \SI[parse-numbers=false]{}{\second}
	\item $f$ frequenza si misura in  \SI[parse-numbers=false]{}{1\per\second}= \SI[parse-numbers=false]{}{\hertz}
	\item $\phi$ sfasamento si misura in \SI[parse-numbers=false]{}{\radian}
\item $A$ ampiezza\index{Funzione!sinusoidale!ampiezza}
\item $\omega=\dfrac{2\pi}{\tau}$ si misura in  \SI[parse-numbers=false]{}{\radian\per\second} pulsazione\index{Funzione!sinusoide!pulsazione} o velocità angolare 
	\item La funzione è limitata\index{Funzione!limitata!seno}, il codominio è $[-A,A]$
\end{itemize}
\include{funzione_cos}
% !TeX encoding = UTF-8
% !TeX spellcheck = it_IT
% !TeX root = formulario.tex
\chapter{Funzione tangente}
\section{Definizione}
\begin{equation*}
y=\tan(x)\quad x\neq\dfrac{\pi}{2}+k\pi\quad k\in\Z
\end{equation*}
\section{Proprietà}
\begin{itemize}
	\item La funzione è trascendente\index{Funzione!trascendente}
\item Il dominio è $\dfrac{\pi}{2}+k\pi\quad k\in\Z$
\item La funzione è non limitata\index{Funzione!illimitata!tangente}, il codominio è $\R$
\item La funzione è periodica di periodo $k\ang{180;;}$ o $k\pi$\index{Funzione!periodica!coseno}
\end{itemize}
{\centering
	\includestandalone{geometria/tangentegrafico}
	\captionof{figure}{Funzione tangente grafico}\par}\index{Funzione!seno}
\chapter{Limiti}
\section{Limite finito per x che tende a valore finito}
Data la funzione$\funzione{f}{D}{\R}$,definita nel suo dominio $D$ a valore in $\R$  diremo che la funzione $f$ tende al limite $l$ per $x$ che tende a $x_0$, quando, comunque preso un intorno $I(l)$, esiste un intorno $I(x_0)$ tale che: comunque preso $x$ diverso da $x_0$, appartenente  all'intorno $I(x_0)$ risulti che $f(x)$ appartenga all'intorno $I(l)$. In simboli
\begin{equation}
\lim_{x\to x_0}f(x)=l
\end{equation}
\begin{equation}
\forall\; I(l)\; \exists\; I(x_0) : \forall x\in I(x_0)-\lbrace x_0\rbrace \longrightarrow f(x)\in I(l)
\end{equation}\index{Limite!finito!finito}
\section{Limite infinito per x che tende a valore finito}
Data la funzione$\funzione{f}{D}{\R}$,definita nel suo dominio $D$ a valore in $\R$  diremo che la funzione $f$ tende al limite $\infty$ per $x$ che tende a $x_0$, quando, comunque preso un intorno $I(\infty)$, esiste un intorno $I(x_0)$ tale che: comunque preso $x$ diverso da $x_0$, appartenente  all'intorno $I(x_0)$ risulti che $f(x)$ appartenga all'intorno $I(\infty)$. In simboli
\begin{equation}
\lim_{x\to x_0}f(x)=\infty
\end{equation}
\begin{equation}
\forall\; I(\infty)\; \exists\; I(x_0) : \forall x\in I(x_0)-\lbrace x_0\rbrace \longrightarrow f(x)\in I(\infty)
\end{equation}\index{Limite!infinito!finito}
\section{Limite finito per x che tende a valore infinito}
Data la funzione$\funzione{f}{D}{\R}$,definita nel suo dominio $D$ a valore in $\R$  diremo che la funzione $f$ tende al limite $l$ per $x$ che tende a $\infty$, quando, comunque preso un intorno $I(l)$, esiste un intorno $I(\infty)$ tale che: comunque preso $x$ appartenente  all'intorno $I(\infty)$ risulti che $f(x)$ appartenga all'intorno $I(l)$. In simboli
\begin{equation}
\lim_{x\to \infty}f(x)=l
\end{equation}
\begin{equation}
\forall\; I(l)\; \exists\; I(\infty) : \forall x\in I(\infty) \longrightarrow f(x)\in I(l)
\end{equation}\index{Limite!finito!infinito}
\section{Limite infinito per x che tende a valore finito}
Data la funzione$\funzione{f}{D}{\R}$,definita nel suo dominio $D$ a valore in $\R$  diremo che la funzione $f$ tende al limite $\infty$ per $x$ che tende a $x_0$, quando, comunque preso un intorno $I(\infty)$, esiste un intorno $I(x_0)$ tale che: comunque preso $x$ diverso da $x_0$, appartenente  all'intorno $I(x_0)$ risulti che $f(x)$ appartenga all'intorno $I(\infty)$. In simboli
\begin{equation}
\lim_{x\to x_0}f(x)=\infty
\end{equation}
\begin{equation}
\forall\; I(\infty)\; \exists\; I(x_0) : \forall x\in I(x_0)-\lbrace x_0\rbrace \longrightarrow f(x)\in I(\infty)
\end{equation}\index{Limite!infinito!finito}
\section{Limite infinito per x che tende a valore infinito}
Data la funzione$\funzione{f}{D}{\R}$,definita nel suo dominio $D$ a valore in $\R$  diremo che la funzione $f$ tende al limite $\infty$ per $x$ che tende a $\infty$, quando, comunque preso un intorno $I(\infty)$, esiste un intorno $I(\infty)$ tale che: comunque preso $x$ appartenente  all'intorno $I(\infty)$ risulti che $f(x)$ appartenga all'intorno $I(\infty)$. In simboli
\begin{equation}
\lim_{x\to \infty}f(x)=\infty
\end{equation}
\begin{equation}
\forall\; I(\infty)\; \exists\; I(\infty) : \forall x\in I(\infty) \longrightarrow f(x)\in I(\infty)
\end{equation}\index{Limite!infinito!infinito}
\section{Operazioni}
\begin{center}
 \begin{tabular}{FFFFFF}
\toprule
\lim_{x\to a}f(x) & \lim_{x\to a}g(x) & \lim_{x\to a}f(x)+g(x) &\lim_{x\to a}f(x)\cdot g(x) &\lim_{x\to a}\dfrac{1}{g(x)} &\lim_{x\to a}\dfrac{f(x)}{g(x)} \\[0.8cm] 

m & \pm\infty& \pm\infty & \pm\infty &0 & 0 \\[0.8cm] 

\pm\infty & m & \pm\infty & \pm\infty & \dfrac{1}{m} &\pm\infty \\[0.8cm]
 
0 &\pm\infty &\pm\infty & \text{Ind} & 0 & 0 \\[0.8cm] 

\pm\infty & 0 & \pm\infty & \text{Ind} & \pm\infty & \pm\infty \\[0.8cm]
 
+\infty & +\infty & +\infty&+\infty &0 & \text{Ind} \\[0.8cm]
 
-\infty & -\infty & -\infty & +\infty & 0 & \text{Ind} \\[0.8cm] 

+\infty & -\infty & \text{Ind} & -\infty & 0 & \text{Ind} \\[0.8cm]
\bottomrule
\end{tabular}\index{Limite!operazioni}
\captionof{table}{Simboli matematici}
\end{center}
\chapter{Asintoti}
\section{Asintoto verticale}
La funzione $\funzione{f}{A}{B}$ ha un asintoto verticale per $a\in A$ se
\begin{equation*}
\lim_{x\to a^+} f(x)=\pm\infty
\end{equation*}\index{Funzione!asintoto!verticale}\index{Asintoto!verticale}
oppure
\begin{equation*}
\lim_{x\to a^-} f(x)=\pm\infty
\end{equation*}\index{Funzione!asintoto!verticale}
\section{Asintoto orizzontale}
La retta $y=c$ è un asintoto orizzontale per la funzione  $\funzione{f}{A}{B}$  se \begin{equation*}
\lim_{x\to\pm\infty} f(x)=c
\end{equation*}\index{Asintoto!orizzontale}\index{Funzione!asintoto!orizzontale}
\section{Asintoto obliquo}
La retta $y=mx+q$ è un asintoto obliquo per la funzione  $\funzione{f}{A}{B}$ se
\begin{equation*}
\lim_{x\to +\infty} [f(x)-(mx+q)]=0
\end{equation*}
o analogamente
\begin{equation*}
\lim_{x\to -\infty} [f(x)-(mx+q)]=0
\end{equation*}\index{Asintoto!obliquo}\index{Funzione!asintoto!obliquo}
\begin{equation*}
\lim_{x\to -\infty} \dfrac{f(x)}{x}=m
\end{equation*}
\begin{equation*}
\lim_{x\to -\infty} [f(x)-mx)]=q
\end{equation*}
% !TeX encoding = UTF-8
% !TeX spellcheck = it_IT
% !TeX root = formulario.tex
\chapter{Derivata}
\section{Definizione}
\begin{equation}
\dfrac{dy}{dx}=\lim_{h \to 0}\dfrac{f(x+h)+f(x)}{h}=f'(x)
\end{equation}\index{Derivata!definizione}
\section{Derivate funzioni}
\begin{equation}
\OpD{a}=0\quad a\in\R
\end{equation}\index{Derivata!costante}
\begin{equation}
\OpD{ax}=a\quad a\in\R
\end{equation}
\begin{equation}
\OpD{x^n}=nx^{n-1}
\end{equation}\index{Derivata!funzione potenza}
\begin{equation}
\OpD{\dfrac{1}{x}}=-\dfrac{1}{x^2}\quad x\neq 0
\end{equation}\index{Derivata!reciproco}
\begin{equation}
\OpD{\sqrt{x}}=\dfrac{1}{2\sqrt{x}}\quad x> 0
\end{equation}\index{Derivata!radice}
\begin{equation}
\OpD{k\cdot f( x )}=k\cdot\OpD{f(x)}\quad a\in\R
\end{equation}\index{Derivata!costante per funzione}
\begin{equation}
\OpD{f( x )+g( x )+h( x )}=\OpD{f( x )}+\OpD{g( x )}+\OpD{h( x )}
\end{equation}\index{Derivata!somma funzioni}
\begin{equation}
\OpD{f( x )\cdot g( x )}=\OpD{f( x )}\cdot g( x )+f( x )\cdot\OpD{g( x )}
\end{equation}\index{Derivata!prodotto funzioni}
\begin{equation}
\OpD{\dfrac{f( x )}{g( x )}}=\dfrac{\OpD{f( x )}\cdot g( x )-f( x )\cdot\OpD{g( x )}} {[g( x )]^2}
\end{equation}\index{Derivata!quoziente funzioni}
% !TeX encoding = UTF-8
% !TeX spellcheck = it_IT
% !TeX root = formulario.tex
\chapter{Statistica}
\section{Media aritmetica semplice}
\begin{equation}
M=\dfrac{\sum x_i}{n}
\end{equation}\index{Media!aritmetica!semplice}
\section{Media aritmetica ponderata}
\begin{equation}
M=\dfrac{\sum x_i n_i}{n}
\end{equation}\index{Media!aritmetica!ponderata}
\section{Scarti dalla media aritmetica}
\begin{equation}
d_i=x_i-M
\end{equation}\index{Scarto!media!aritmetica}
\section{Scarti ponderati media aritmetica}
\begin{equation}
d_i=(x_i-M)n_i
\end{equation}\index{Scarto!ponderato!media aritmetica}
\section{Varianza}
\begin{equation}
\sigma^2=\dfrac{\sum(x_i-M)^2}{n}
\end{equation}\index{Varianza}
\section{Scarto quadratico medio}
\begin{equation}
\sigma=\sqrt{\dfrac{\sum(x_i-M)^2}{n}}
\end{equation}\index{Scarto!quadratico!medio}
% !TeX encoding = UTF-8
% !TeX spellcheck = it_IT
% !TeX root = formulario.tex
\chapter{Le unità di misura del SI}
\label{sec:UnitaDiMisura}
\section{Conversioni e costanti}
\label{sec:ConversioniECostanti}
%\begin{table}[H]
%\centering
\begin{center}
	\begin{tabular}{lcl}
		\toprule
		nome&simbolo&\multicolumn{1}{c}{valore}\\
		\midrule
		elettronvolt&\si{\electronvolt}&\SI{1}{\electronvolt}=\SI{1,602 176 46e-19}{\joule}\\
		chilowattora&\si{\kilo\watt\hour}&\SI{1}{\kilo\watt\hour}=\SI{3,6e6}{\joule}\\
		velocità della luce nel vuoto&c&\SI{299792458}{\metre\per\second\tothe{1}}\\
		permeabilità magnetica del vuoto&$\mu_0$&$4\pi 10^{-7}$\si{\newton \per\ampere\tothe{2}}\\
		costante di Coulomb&k&\SI{8,99e9}{\newton\metre\tothe{2}\per\coulomb\tothe{2}}\\
		costante dielettrica del vuoto&$\epsilon_0$&\SI{8,854 187 81762e-12}{\farad\per\metre}\\
		costante di gravitazione universale&$G$&\num[separate-uncertainty]{6,67428(67)e-11}\si{\metre\tothe{3}\per\kilogram\per\second\tothe{2}}\\
		accelerazione di gravità (livello del mare)&g&\SI{9,80665}{\metre\per\second\tothe{2}}\\
		\bottomrule
	\end{tabular}
	\captionof{table}{Costanti}\index{Costanti}
	\label{Tab:Costanti}
\end{center}
%\end{table}
\section{Tabella prefissi SI}
\label{sec:TabellaPrefissiSI}
\begin{center}
	%\begin{table}[H]
%\centering 
\begin{tabular}{llclr}
\toprule
\multicolumn{1}{c}{\textbf{Prefisso}}&\textbf{Simbolo}&\multicolumn{1}{c}{\textbf{Nome}}&\multicolumn{1}{c}{\textbf{$10^n$}}& \multicolumn{1}{c}{\textbf{Equivalente decimale}}\\
\midrule
yotta&\si{\yotta}&Quadrilione&\num{e24}&\num{1000000000000000000000000}\\
zetta&\si{\zetta}&Triliardo&\num{e21}&\num{1000000000000000000000}\\
exa&\si{\exa}&Trilione&\num{e18}&\num{1000000000000000000}\\
peta&\si{\peta}&Biliardo&\num{e15}&\num{1000000000000000}\\
tera&\si{\tera}&Bilione&\num{e12}&\num{1000000000000}\\
giga&\si{\giga}&Miliardo&\num{e9}&\num{1000000000}\\
mega&\si{\mega}&Milione&\num{e6}&\num{1000000}\\
kilo&\si{\kilo}&Mille&\num{e3}&\num{1000}\\
hecto&\si{\hecto}&Cento&\num{e2}&\num{100}\\
deca&\si{\deca}&Dieci&\num{e1}&\num{10}\\
deci&\si{\deci}&Decimo&\num{e-1}&\num{0,1}\\
centi&\si{\centi}&Centesimo&\num{e-2}&\num{0,01}\\
milli&\si{\milli}&Millesimo&\num{e-3}&\num{0,001}\\
micro&\si{\micro}&Milionesimo&\num{e-6}&\num{0,000001}\\
nano&\si{\nano}&Miliardesimo&\num{e-9}&\num{0,000000001}\\
pico&\si{\pico}&Bilionesimo&\num{e-12}&\num{0,000000000001}\\
femto&\si{\femto}&Biliardesimo&\num{e-15}&\num{0,000000000000001}\\
atto&\si{\atto}&Trilionesimo&\num{e-18}&\num{0,000000000000000001}\\
zepto&\si{\zepto}&Triliardesimo&\num{e-21}&\num{0,000000000000000000001}\\
yocto&\si{\yocto}&Quadrilionesimo&\num{e-24}&\num{0,000000000000000000000001}\\
\bottomrule
\end{tabular}
\captionof{table}{Prefissi del Sistema Internazionale}\index{Sistema internazionale!prefissi}
\label{PrefissidelSistemaInternazionale}
%
\end{center}
%\begin{table}
\begin{center}
	%	\centering
	\begin{tabular}{llrrrrrrr}
		\toprule	
			&&\multicolumn{1}{c}{\textbf{Kilo}} &\multicolumn{1}{c}{\textbf{Hecto}} & \multicolumn{1}{c}{\textbf{Deca}} &  & \multicolumn{1}{c}{\textbf{Deci}} & \multicolumn{1}{c}{\textbf{Centi}} & \multicolumn{1}{c}{\textbf{Milli}} \\ 
			&&\multicolumn{1}{c}{\textbf{\si{\kilo}}} & \multicolumn{1}{c}{\textbf{\si{\hecto}}} & \multicolumn{1}{c}{\textbf{\si{\deca}}} &  & \multicolumn{1}{c}{\textbf{\si{\deci}}} & \multicolumn{1}{c}{\textbf{\si{\centi}}} & \multicolumn{1}{c}{\textbf{\si{\milli}}} \\ 
			\midrule
		\textbf{Kilo}&\si{\kilo}	&\num{1}  &  &  &  &  &  &  \\ 
		\textbf{Hecto}&\si{\hecto}	& \num{10} &\num{1}  &  &  &  &  &  \\ 
		\textbf{Deca}&\si{\deca}	& \num{100} & \num{10} &\num{1}  &  &  &  &  \\ 
	&	& \num{1000} & \num{100} &\num{10}  &\num{1}  &  &  &  \\ 
		\textbf{Deci}&\si{\deci}	& \num{10000} &\num{1000}  &\num{100}  &\num{10}  & \num{1} &  &  \\ 
		\textbf{Centi}&\si{\centi}	&\num{100000}  & \num{10000} & \num{1000} & \num{100} &\num{10}  & \num{1} &  \\ 
		\textbf{Milli}&\si{\milli}	&\num{1000000}  &\num{100000}  & \num{10000} &\num{1000}  & \num{100} & \num{10} & \num{1} \\ 
		\bottomrule
	\end{tabular} 
	\captionof{table}{Equivalenze da Kilo a Milli}\index{Conversione!tavole}
\end{center}
%\end{table}
%\begin{table}
%	\centering
\begin{center}
		\begin{tabular}{ll*{7}{r}}
		\toprule	
		&\multicolumn{1}{c}{\textbf{Milli}}  & \multicolumn{1}{c}{\textbf{Centi}} & \multicolumn{1}{c}{\textbf{Deci}} &  & \multicolumn{1}{c}{\textbf{Deca}} & \multicolumn{1}{c}{\textbf{Hecto}} & \multicolumn{1}{c}{\textbf{Kilo}} \\ 
        &&\multicolumn{1}{c}{\textbf{\si{\kilo}}}& \multicolumn{1}{c}{\textbf{\si{\hecto}}}& \multicolumn{1}{c}{\textbf{\si{\deca}}} &  & \multicolumn{1}{c}{\textbf{\si{\deci}}}& \multicolumn{1}{c}{\textbf{\si{\centi}}}& \multicolumn{1}{c}{\textbf{\si{\milli}}} \\
		\midrule
		Milli&\si{\milli}	&\num{1}  &  &  &  &  &  &  \\ 
		Centi&\si{\centi}	& \num{0,1} &      \num{1}  &  &  &  &  &  \\ 
		Deci&\si{\deci}	& \num{0,01} &     \num{0,1} &    \num{1}  &  &  &  &  \\ 
				&	& \num{0,001} &    \num{0,01} &   \num{0,1}    & \num{1}  &  &  &  \\ 
		Deca&\si{\deca}	& \num{0,0001} &   \num{0,001}  & \num{0,01}  &  \num{0,1}  & \num{1} &  &  \\ 
		Hecto&\si{\hecto}	& \num{0,00001}  & \num{0,0001} & \num{0,001} &  \num{0,01}   &\num{0,1}  & \num{1} &  \\ 
		Kilo&\si{\kilo}	& \num{0,000001}  &\num{0,00001}& \num{0,0001} & \num{0,001}  & \num{0,01} & \num{0,1} & \num{1} \\ 
		\bottomrule
	\end{tabular} 
	\captionof{table}{Equivalenze da \si{\milli} a \si{\kilo}}
\end{center}
%\end{table}
\section{Unità di misura fondamentali SI}
\label{sec:UnitàDiMisuraFondamentaliSI}
%\begin{table}[H]
%\centering
\begin{center}
	\begin{tabular}{lccc}
\toprule
Grandezza fisica&Simbolo della Grandezza& Nome dell'unità SI&Simbolo dell'unità SI\\
\midrule
lunghezza&L&metro&\si{\metre}\\
massa&M&chilogrammo&\si{\kilogram}\\
intervallo di tempo&T&secondo&\si{\second}\\
intensità di corrente&I,i&ampere&\si{\ampere}\\
temperatura assoluta&T&kelvin&\si{\kelvin}\\
quantità di sostanza&N&mole&\si{\mole}\\
intensità luminosa&J&candela&\si{\candela}\\
\bottomrule
\end{tabular}
\captionof{table}{Unit\'a fondamentali}\index{Sistema internazionale!unità !di misura}
\end{center}

\begin{center}
	\begin{tabular}{lcl}
		\toprule 
		\multicolumn{1}{c}{nome}&simbolo&\multicolumn{1}{c}{equivalenza SI}\\
		\midrule
		minuto&\si{\minute}&$\SI{1}{\minute}=\SI{60}{\second}$\\
		ora&\si{\hour}&$\SI{1}{\hour}=\SI{60}{\minute}=\SI{3600}{\second}$\\
		giorno&\si{\day}&$\SI{1}{\day}=\SI{24}{\hour}=\SI{1440}{\minute} = \SI{86400}{\second}$\\
		litro&\si{\litre}&$\SI{1}{\litre}=\SI{1}{\deci\metre\tothe{3}}=\SI{e-3}{\metre\tothe{3}}$\\
		\bottomrule
	\end{tabular}
	\captionof{table}{Unit\'a non SI}\index{Sistema internazionale!unità!non SI}
\end{center}
\begin{center}
	\begin{tabular}{clcl}
\toprule
genere&nome&simbolo&\multicolumn{1}{c}{valore}\\
\midrule
\multirow{6}*{massa}&grammo&\si{\gram}&\SI{e-3}{\kilo\gram}\\
&milligrammo&\si{\milli\gram}&\SI{e-6}{\kilo\gram}\\
&microgrammo&\si{\micro\gram}&\SI{e-9}{\kilo\gram}\\
&nanogrammo&\si{\nano\gram}&\SI{e-12}{\kilo\gram}\\
&picogrammo&\si{\pico\gram}&\SI{e-15}{\kilo\gram}\\
&femtogrammo&\si{\femto\gram}&\SI{e-18}{\kilo\gram}\\
%atomic mass \atomicmass u \amu
\midrule
\multirow{7}*{lunghezza}&chilometro&\si{\kilo\metre}&\SI{e3}{\metre}\\
&decimetro&\si{\deci\metre}&\SI{e-1}{\metre}\\
&centimetro&\si{\centi\metre}&\SI{e-2}{\metre}\\
&millimetro&\si{\milli\metre}&\SI{e-3}{\metre}\\
&micrometro&\si{\micro\metre}&\SI{e-6}{\metre}\\
&nanometro&\si{\nano\metre}&\SI{e-9}{\metre}\\
&picometro&\si{\pico\metre}&\SI{e-12}{\metre}\\
\midrule
\multirow{4}*{tempo}&millisecondo&\si{\milli\second}&\SI{e-3}{\second}\\
&microsecondo&\si{\micro\second}&\SI{e-6}{\second}\\
&nanosecondo&\si{\nano\second}&\SI{e-9}{\second}\\
&picosecondo&\si{\pico\second}&\SI{e-12}{\second}\\
\midrule
\multirow{5}*{corrente}&chiloampere&\si{\kilo\ampere}&\SI{e3}{\ampere}\\
&milliampere& \si{\milli\ampere}&\SI{e-3}{\ampere}\\
&microampere&\si{\micro\ampere}&\SI{e-6}{\ampere}\\
&nanoampere&\si{\nano\ampere}&\SI{e-9}{\ampere}\\
&picoampere &\si{\pico\ampere}&\SI{e-12}{\ampere}\\
\midrule
\multirow{4}*{frequenza}&terahertz &\si{\tera\hertz}&\SI{e12}{\hertz}\\
&gigahertz&\si{\giga\hertz}&\SI{e9}{\hertz}\\
&megahertz&\si{\mega\hertz}&\SI{e6}{\hertz}\\
&kilohertz&\si{\kilo\hertz}&\SI{e3}{\hertz}\\
\midrule
\multirow{2}*{potenziale}&kilovolt&\si{\kilo\volt}&\SI{e3}{\volt}\\
&millivolt&\si{\milli\volt}&\SI{e-3}{\volt}\\
\midrule
\multirow{3}*{potenza}&megawatt&\si{\mega\watt}&\SI{e6}{\watt}\\
&kilowatt& \si{\kilo\watt}&\SI{e3}{\watt}\\
&milliwatt& \si{\milli\watt}&\SI{e-3}{\watt}\\
\midrule
\multirow{5}*{capacità}&millifarad&\si{\milli\farad}&\SI{e-3}{\farad}\\
&microfarad&\si{\micro\farad}&\SI{e-6}{\farad}\\
&nanofarad&\si{\nano\farad}&\SI{e-9}{\farad}\\
&picofarad&\si{\pico\farad}&\SI{e-12}{\farad}\\
&femtofarad&\si{\femto\farad}&\SI{e-15}{\farad}\\
\midrule
\multirow{3}*{resistenza}&gigaohm&\si{\giga\ohm}&\SI{e9}{\ohm}\\
&megohm&\si{\mega\ohm}&\SI{e6}{\ohm}\\
&kilohm&\si{\kilo\ohm}&\SI{e3}{\ohm}\\
\midrule
\multirow{6}*{energia}&kilojoule&\si{\kilo\joule}&\SI{e3}{\joule}\\ 
&teraelettronvolt& \si{\tera\electronvolt}&\SI{e12}{\electronvolt}\\
&gigaelettronvolt&\si{\giga\electronvolt}&\SI{e9}{\electronvolt}\\
&megaelettronvolt&\si{\mega\electronvolt}&\SI{e6}{\electronvolt}\\
&kiloelectronvolt&\si{\kilo\electronvolt}&\SI{e3}{\electronvolt}\\
&millielettronvolt&\si{\milli\electronvolt}&\SI{e-3}{\electronvolt}\\
\bottomrule
\end{tabular}
\captionof{table}{Multipli unit\'a di misura}\index{Sistema internazionale!multipli}
\end{center}
%\end{table}
\section{Unità di misura derivate SI}
\begin{center}
	\begin{tabular}{llclFl}
		\toprule
		Grandezza fisica& \multicolumn{1}{c}{Nome SI} & \multicolumn{1}{c}{Simbolo SI} & \multicolumn{1}{c}{Unità SI}& \multicolumn{1}{c}{Dimensioni}& \multicolumn{1}{l}{Equivalenza SI}\\ 
		\midrule
		area  &  &  &\si{\meter\squared}&\dif{\si{\Lunghezza\squared}} &\si{\meter\squared} \\ 
		volume  & &  &\si{\meter\cubed}& \dif{\si{\Lunghezza\cubed}} &\si{\meter\cubed} \\ 
		\midrule
		velocità &  &  &\si{\meter\per\second}  &\dif{\si{\Lunghezza\per\Tempo}} &  \si{\meter\per\second} \\
		accelerazione &  & &\si{\meter\per\square\second} &\dif{\si{\Lunghezza\per\Tempo\squared}} & \si{\meter\per\square\second} \\ 
		\midrule 
		angolo piano & radiante   & \si{\radian}& \si{\meter\per\meter}&\dif{\si{\Lunghezza\per\Lunghezza}}  &1\\
		velocità angolare &  &  &\si{\radian\per\second}& \dif{\si{\per\Tempo}}  & \si{\per\second} \\
		accelerazione angolare &  &  &\si{\radian\per\second\squared}& \dif{\si{\per\Tempo\squared}}  & \si{\per\second\squared} \\ 
		frequenza &  hertz & \si{\hertz} &\si{\per\second}&\dif{\si{\per\Tempo}} &\si{\per\second} \\ 
		\midrule 
		forza &   newton& \si{\newton} &\si{\kilogram\meter\per\square\second} & \dif{\si{\per\Tempo\squared}}&\si{\kilogram\meter\per\square\second} \\ 
		pressione  & pascal& \si{\pascal}&\si{\newton\per\square\meter}&\dif{\si{\Massa\per\Lunghezza\per\Tempo\squared}} & \si{\kilogram\per\meter\per\square\second} \\ 
		lavoro energia  & joule& \si{\joule}&\si{\newton\meter}&\dif{\si{\Massa\Lunghezza\squared\per\Tempo\squared}} & \si{\kilogram\meter\squared\per\square\second} \\ 
		potenza  & watt& \si{\watt}&\si{\joule\per\second}&\dif{\si{\Massa\per\Lunghezza\squared\per\Tempo\cubed}} & \si{\kilogram\meter\squared\per\second\cubed} \\ 
		\midrule
		carica elettrica & coulomb& \si{\coulomb}&\si{\ampere\second}&\dif{\si{\Corrente\Tempo}} & \si{\ampere\second} \\ 
		potenziale elettrico & volt& \si{\volt}&\si{\joule\per\coulomb}&\dif{\si{\Massa\Lunghezza\squared\per\Tempo\cubed\per\Corrente}} &\si{\kilogram\meter\squared\per\second\cubed\per\ampere}  \\ 
		resistenza elettrica & ohm& \si{\ohm}&\si{\volt\per\ampere}&\dif{\si{\per\Massa\per\Lunghezza\Tempo\cubed\per\Corrente\squared}} &\si{\kilogram\meter\squared\per\second\cubed\per\square\ampere}  \\ 
		conduttanza elettrica & siemens& \si{\siemens}&\si{\ampere\per\volt}&\dif{\si{\Corrente\squared\Tempo\cubed\per\Massa\per\Lunghezza\squared}} &\si{\ampere\squared\second\cubed\per\kilogram\squared\per\meter\squared}  \\ 
		capacità elettrica & farad& \si{\farad}&\si{\coulomb\per\volt}&\dif{\si{\Corrente\squared\Tempo\tothe{4}\per\Massa\per\Lunghezza\squared}} &\si{\ampere\squared\second\tothe{4}\per\kilogram\per\meter\squared}  \\ 
		flusso magnetico & weber& \si{\weber}&\si{\volt\second}&\dif{\si{\Massa\Lunghezza\squared\per\Tempo\squared\per\Corrente}
		} &\si{\kg\meter\squared\per\second\squared\per\ampere}  \\ 
		densità flusso magnetico & tesla& \si{\tesla}&\si{\volt\second\per\meter\squared}&\dif{\si{\Massa\per\Tempo\squared\per\Corrente}} &\si{\kg\per\second\squared\per\ampere}  \\ 
		induttanza & henry& \si{\henry}&\si{\volt\second\per\ampere}&\dif{\si{\Massa\Lunghezza\squared\per\Tempo\squared\per\Corrente\squared}} &\si{\kg\meter\squared\per\second\squared\per\ampere\squared}  \\ 
		\bottomrule
	\end{tabular}\captionof{table}{SI unità derivate}\index{Sistema internazionale!unità!derivate} 
\end{center}
\chapter{Cinematica}
\section{Medie}
\begin{equation*}
v_m=\dfrac{s_2-s_1}{\Delta t}=\dfrac{\Delta s}{\Delta t}
\end{equation*}\index{Velocità!media}
\begin{equation*}
a_m=\dfrac{v_2-v_1}{\Delta t}=\dfrac{\Delta v}{\Delta t}
\end{equation*}
\section{Moto rettilineo uniformemente accelerato}
\begin{align*}
v=&v_0+at\\
t=\dfrac{v-v_0}{a}\\
s=&s_0+\frac{1}{2}(v_x+v_0)t\\
s=&s_0+v_0t+\frac{1}{2}at^2\\
v^2=&v_{0}^2+2a(s-s_0)\\
a=&\frac{v^2-v_{0}^2}{2a(s-s_0)}\\
t=&\frac{2(x-x_0)}{v_0+v}\\
t=&\frac{v_x-v_0}{a}
\end{align*}\index{Moto uniforme!velocità}\index{Moto uniforme!spazio}\index{Moto uniforme!tempo}\index{Moto uniforme!accelerazione}
\section{Moto rettilineo costante}
\begin{align*}
a=&0\\
v=&v_0\\
s=&s_0+v_0t\\
\end{align*}\index{Moto uniforme!velocità}\index{Moto uniforme!spazio}\index{Moto uniforme!tempo}\index{Moto uniforme!accelerazione}
\section{Moto in caduta libera}
Sistema di riferimento verticale 
\begin{align*}
v=&v_0-gt\\
s=&\frac{1}{2}(v_0+v)t\\
s=&v_0t-\frac{1}{2}gt^2\\
v^2=&v_{0}^2-2gs\\
t=&\frac{v_0-v}{g}\\
s=&\frac{v_{0}^2-v^2}{2g}
\end{align*}
\chapter{Moto circolare uniforme}
{\centering
	\includestandalone[width=.6\linewidth]{geometria/velocitangolare}
	\captionof{figure}{Velocità e accelerazione centripeta}\par}
Modulo velocità, velocità angolare, modulo accelerazione centripeta, frequenza, periodo costanti. \index{Velocità!angolare}\index{Accelerazione!centripeta}\index{Frequenza}\index{Periodo}
\section{Frequenza Periodo}
\begin{align*}
f=&\dfrac{1}{T}\\
T=&\dfrac{1}{f}\\
\end{align*}
\section{Velocità tangenziale}
\begin{align*}
v=&\dfrac{s}{t}\\
v=&\dfrac{2\pi r}{T}\\
r=&\dfrac{vT}{2\pi}\\
T=&\dfrac{2\pi r}{v}\\
\end{align*}
\section{Velocità angolare}
\begin{align*}
\omega=&\dfrac{2\pi}{T}\\
\omega=&2\pi f\\
T=&\dfrac{2\pi}{\omega}\\
v=&\omega r\\
\omega=&\dfrac{v}{r}\\
r=&\dfrac{v}{\omega}
\end{align*}
\section{Accelerazione centripeta}
\begin{align*}
a_c=&\dfrac{v^2}{r}\\
a_c=&\omega^2 r\\
v=&\sqrt{a_c r}\\
r=&\dfrac{v^2}{a_c}\\
\omega=&\sqrt{\dfrac{a_c}{r}}\\
r=&\dfrac{a_c}{\omega^2}
\end{align*}
\section{Legge oraria}
\begin{equation*}
\phi_t=\phi_0+\omega t
\end{equation*}
\section{Equazioni parametriche}
\begin{align*}
x(t)=&r\cos(\omega t+\phi_0)\\
y(t)=&r\sin(\omega t+\phi_0)\\
v_x(t)=&-r\omega\sin(\omega t+\phi_0)\\
v_y(t)=&r\omega\cos(\omega t+\phi_0)\\
a_x(t)=&-r\omega^2\cos(\omega t+\phi_0)=-\omega^2 x(t)\\
a_y(t)=&-r\omega^2\sin(\omega t+\phi_0)=-\omega^2 y(t)\\
\end{align*}		
\backmatter	
 \addcontentsline{toc}{chapter}{\indexname}
 \printindex
\appendix
\chapter{Mezzi usati}
\CDMezziUsati
	\twocolumn
\glsaddall
\printglossaries
\onecolumn
\nocite{*}
\addcontentsline{toc}{chapter}{\bibname}
\printbibliography
\end{document}
