\chapter{Arrotondamento}
\section{Troncamento}
\begin{defn}[Approssimazione per difetto o troncamento]
Dato un numero decimale, fissata una cifra decimale sono scartate le cifre successive.
\end{defn}\index{Arrotondamento!troncamento}
\section{Eccesso}
\begin{defn}[Approssimazione per eccesso]
	Dato un numero decimale, fissata una cifra decimale questa viene aumentata di uno. Le rimanenti vengono scartate.
\end{defn}\index{Arrotondamento!eccesso}
\section{Caso generale}
\begin{defn}[Regola del cinque]
	Fissato la cifra decimale rispetto a cui si fa l'arrotondamento:
	\begin{itemize}
		\item Se la cifra successiva è minore di cinque, cinque compreso, si tronca alla cifra decimale fissata
		\item Se la cifra successiva è maggiore di cinque, cinque escluso, si aggiunge uno alla cifra di arrotondamento e si scarta il resto.
	\end{itemize}
\end{defn}\index{Arrotondamento!regola del cinque}