% !TeX encoding = UTF-8
% !TeX spellcheck = it_IT
% !TeX root = formulario.tex
\chapter{Numeri interi}
\section{Simbolo}
\begin{equation}
\Z
\end{equation}\index{Numero!intero}
\section{Valore assoluto}
\begin{equation}
\abs{a}=\begin{cases}
	a&a>0\\
	-a&a<0\\
	0&a=0
\end{cases}
\end{equation}\index{Valore assoluto}\index{Numero!intero!valore assoluto}
\section{Vocabolario}
\begin{description}
	\item[Concordi] Due numeri interi che hanno segno uguale\index{Numero!intero!concordi}
	\item[Discordi]  Due numeri interi che hanno segno diverso\index{Numero!intero!disconcordi}
	\item[Opposti] Uguale valore assoluto, segno opposto\index{Numero!intero!opposti}
\end{description}
\section{Somma}
\begin{enumerate}
	\item I numeri a e b sono concordi. In questo caso la somma è un numero intero con lo stesso segno di a e di b e che ha per valore assoluto la somma dei valori assoluti.
	\item I numeri a e b sono discordi. In questo caso la somma è un numero intero con il segno del numero di valore assoluto maggiore e che ha per valore assoluto la differenza tra i valori assoluti.
	\item I due numeri sono opposti. La somma è zero.
\end{enumerate}\index{Numero!intero!somma}\index{Valore!assoluto}
\section{Somma}
\begin{equation}
\overbrace{a}^{addendo}+\overbrace{b}^{addendo}=\overbrace{c}^{somma}
\end{equation}\index{Addendo}\index{Somma}
\subsection{Associativa}
\begin{equation}
(a+b)+c=a+(b+c)
\end{equation}\index{Somma!associativa}
\subsection{Commutativa}
\begin{equation}
a+b=b+a
\end{equation}\index{Somma!commutativa}
\subsection{Elemento  neutro}
\begin{equation}
a+0=0+a=a
\end{equation}\index{Somma!elemento!neutro}
\subsection{Opposto}
\begin{equation}
a+(-a)=(-a)+a=0
\end{equation}\index{Somma!opposto}
\section{Prodotto}
\begin{enumerate}
	\item I numeri a e b sono concordi. Il prodotto fra i due numeri è un numero di segno positivo e per valore assoluto il prodotto dei valori assoluti.
	\item I numeri a e b sono discordi. Il prodotto fra i due numeri è un numero di segno negativo e per valore assoluto il prodotto dei valori assoluti.
\end{enumerate}\index{Prodotto!segno}

{\centering\captionof{table}{Segno prodotto} 
	\begin{tabular}{ccc}
		\toprule
		$+$ & $+$ & $+$ \\ 
		$-$ & $-$ & $+$ \\ 
		$+$ & $-$ & $-$ \\ 
		$-$ & $+$ & $-$ \\ 
		\bottomrule
	\end{tabular}\index{Numero!intero!segno prodotto}
	\par}
\begin{equation}
\overbrace{a}^{fattore}\times\overbrace{b}^{fattore}=\overbrace{c}^{prodotto}
\end{equation}\index{Fattore}\index{Prodotto}
\subsection{Associativa}
\begin{equation}
(a\times b)\times c=a\times(b\times c) 
\end{equation}\index{Numero!intero!prodotto associativa}
\subsection{Commutativa}
\begin{equation}
a\times b=b\times \index{Numero!intero!prodotto commutativa}
\end{equation}\index{Numero!intero!prodotto commutativa}
\subsection{Elemento neutro}
\begin{equation}
a\times 1=1\times a=a
\end{equation}\index{Numero!intero!prodotto elemento neutro}
\section{Sottrazione}
La sottrazione e la somma coincidono
\begin{equation}
a-b=a+(-b)
\end{equation}
 \section{Divisione}
  Non è sempre possibile.
\begin{equation}
\overbrace{a}^{dividendo}\div\overbrace{b}^{divisore}=\overbrace{c}^{quoziente}\quad b\neq 0
\end{equation}\index{Dividendo}\index{Divisore}\index{Quoziente}
\subsection{Segno}
\begin{enumerate}
	\item I numeri a e b sono concordi. Il quoziente fra i due numeri è un numero di segno positivo e per valore assoluto il quoziente dei valori assoluti.
	\item I numeri a e b sono discordi. Il quoziente fra i due numeri, è un numero che ha  segno negativo e per valore assoluto il quoziente dei valori assoluti.
\end{enumerate}\index{Quoziente!segno}
\subsection{Invariantiva quoziente}
\begin{equation}
a\div b=(a\times c)\div (b\times c)\quad b\neq 0\quad c \neq 0
\end{equation}\index{Quoziente!invariantiva}
\begin{equation}
a\div b=(a\div c)\div (b\div c)\quad b\neq 0\quad c \neq 0
\end{equation}\index{Numero!intero!quoziente invariantiva}
\section{Proprietà distributive}
\subsection{Moltiplicazione addizione}
\begin{equation}
(a+b)\times c=a\times c+b\times c
\end{equation}\index{Distributiva!moltiplicazione!addizione}
\subsection{Moltiplicazione sottrazione}
\begin{equation}
(a-b)\times c=a\times c-b\times c\quad a>b\quad c\neq 0
\end{equation}\index{Distributiva!moltiplicazione!sottrazione}
\subsection{Divisione addizione}
\begin{equation}
(a+b)\div c=a\div c+b\div c\quad c\neq 0
\end{equation}\index{Distributiva!divisione!addizione}
\subsection{Divisione sottrazione}
\begin{equation}
(a-b)\div c=a\div c-b\div c\quad a>b\quad c\neq 0
\end{equation}\index{Distributiva!divisione!sottrazione}
\section{Riepilogo}

{\centering	\captionof{table}{Proprietà numeri interi}
	\begin{tabular}{lLL}
		\toprule
		Proprietà	& Somma & Prodotto  \\ 
		\midrule
		associativa	& (a+b)+c=a+(b+c) & (a\times b)\times c=a\times(b\times c) \\ 
		commutativa	&a+b=b+a  &a\times b=b\times a  \\ 
		elemento neutro	&a+0=0+a=a  & a\times 1=1\times a =a\\ 
		inverso&(-a)+a=a+(-a)=0\\
		distributiva	&(a+b)\times c =a\times c+b\times c &  \\ 
		assorbimento	&  & a\times 0=0\times a=0 \\ 
		\bottomrule
	\end{tabular}
\par}\index{Proprietà!associativa}\index{Proprietà!commutativa}\index{Elemento!neutro}\index{Proprietà!distributiva}\index{Proprietà!assorbimento}