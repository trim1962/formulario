% !TeX encoding = UTF-8
% !TeX spellcheck = it_IT
% !TeX root = formulario.tex
\chapter{Disequazioni equazioni frazionarie}
\section{Definizione disequazione}
Una disequazione è frazionaria se ha al denominatore un'incognita.\index{Disequazione!frazionaria}
\section{Forma normale}
Una disequazione frazionaria è in forma normale se
\begin{equation}
\dfrac{f(x)}{g(x)}\begin{cases}
>0\\
\geq 0\\
<0\\
\leq 0
\end{cases}\quad g(x)\neq 0
\end{equation}
\section{Risoluzione disequazione frazionaria}
\begin{enumerate}
	\item Trovo il segno del numeratore
	\item Trovo il segno del denominatore
	\item Sovrappongo i grafici dei segni
	\item Leggo il grafico e trovo la soluzione
\end{enumerate}\index{Disequazione!frazionaria!risoluzione}
\section{Definizione equazione}
Un'equazione è frazionaria se ha al denominatore un'incognita.\index{Equazione!frazionaria}
\section{Forma normale}
Un'equazione frazionaria è in forma normale se
\begin{equation}
\dfrac{f(x)}{g(x)}=0\quad g(x)\neq 0
\end{equation}
\section{Risoluzione equazione frazionaria}
\begin{enumerate}
	\item Trovare i valori di $x$ per cui $g(x)=0$
	\item Risolvere $f(x)=0$ 
	\item Verificare se le soluzioni annullano il denominatore 
\end{enumerate}\index{Equazione!frazionaria!risoluzione}
\begin{center}
	\begin{tabular}{Cp{0.4\textwidth}}
\toprule
	& Soluzione \\ 
\midrule
\dfrac{f(x)}{g(x)}=0	& Ha soluzione per quei valori di $x$ per cui $f(x)=0$ e $g(x)\neq 0$  \\ 
\dfrac{f(x)}{g(x)}>0	& Ha soluzione per quei valori di $x$ per cui $f(x)$ e $g(x)$ sono concordi e $g(x)\neq 0$  \\ 
\dfrac{f(x)}{g(x)}<0	& Ha soluzione per quei valori di $x$ per cui $f(x)$ e $g(x)$ sono discordi e $g(x)\neq 0$  \\ 
\bottomrule
\end{tabular}\captionof{table}{Equazioni e disequazioni frazionarie}\index{Disequazione!frazionaria}\index{Equazione!frazionaria}
\end{center}