\section{Risoluzione triangolo rettangolo}
\begin{center}
	\includestandalone{geometria/triangolopitagorico1}
	\captionof{figure}{Triangolo rettangolo}
\end{center}\index{Triangolo!rettagolo!cateto}\index{Triangolo!rettagolo!ipotenusa}\index{Triangolo!rettagolo!angolo acuto}\index{Triangolo!rettagolo!adiacente}\index{Triangolo!rettagolo!opposto}
\subsection{Ipotenusa e angolo acuto noto}
\begin{enumerate}
	\item $a$ noto
	\item $\beta$ noto
\end{enumerate}
\begin{align*}
c=&a\cos\beta\\
b=&a\sin\beta\\
\gamma=&\frac{\pi}{2}-\beta\\
\end{align*}
\subsection{Cateto ed un angolo acuto noto}
\begin{enumerate}
	\item $b$ noto
	\item $\beta$ noto
\end{enumerate}
\begin{align*}
\gamma=&\frac{\pi}{2}-\beta&\gamma=&\frac{\pi}{2}-\beta\\
a=&\frac{b}{\sin\beta}&c=&b\tan\gamma\\
c=&a\cos\beta&a=&\sqrt{b^2+c^2}
\end{align*}
\subsection{Ipotenusa e un cateto noto}
\begin{enumerate}
	\item $a$ noto
	\item $b$ noto
\end{enumerate}
\begin{align*}
c=&\sqrt{a^2-b^2}\\
\beta=&\arcsin\dfrac{b}{a}\\
\gamma=&\frac{\pi}{2}-\beta
\end{align*}
\subsection{Due cateti noti}
\begin{enumerate}
	\item $b$ noto
	\item $c$ noto
\end{enumerate}
\begin{align*}
\beta=&\arctan\frac{b}{c}&a=&\sqrt{b^2+c^2}\\
\gamma=&\frac{\pi}{2}-\beta&\beta=&\arcsin\frac{b}{a}\\
a=&\sqrt{b^2+c^2}&\gamma=&\frac{\pi}{2}-\beta\\
\end{align*}