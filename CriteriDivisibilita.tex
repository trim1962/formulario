\chapter{Tavole Numeriche}
\section{Criteri di divisibilità}
\label{sec:CriteridiDivisibilita}
{\centering\captionof{table}{Criteri di divisibilità }
	\begin{tabular}{Ccp{0.25\textwidth}p{0.25\textwidth}}
\toprule  N&  &\multicolumn{1}{c}{Regola}&\multicolumn{1}{c}{Esempio}    \\ 
\midrule 2 & Se & l'ultima cifra è pari, cioè è  \numlist{0;2;4;6;8}& \\ 
3 & Se & la somma delle cifre è divisibile per tre & \num{375} $3+7+5=15\div3=5$ infatti $375\div 3=125$ \\ 
 4 & Se & le ultime due cifre sono divisibili per quattro o sono due zeri $\mathbf{00}$& $4\mathbf{60}$ $60\div 4=15$ $469\div 4=115$ \\
 5 & Se & l'ultima cifra è  cinque o zero&\num{10}, \num{735} \\  
 6 & Se & è divisibile contemporaneamente per tre e per due& \num{54} \\  
 8 & Se & ultime tre cifre sono divisibili per 8 o sono tre zeri $\mathbf{000}$& $9\mathbf{872}$ le ultime tre cifre sono divisibili per otto $872\div 8= 109$ $9872\div 8=1234$ \\  
 9 & Se & la somma delle cifre è divisibile per 9& $405$ $4+0+5=9$ $405\div9=45$  \\
 10 & Se & l'ultima sua cifra è zero& \num{100},\num{140}\\
 11 & Se& la differenza della somma delle cifre di posto pari e le cifre di posto dispari è zero o si divide per undici&  $25652$ $(5+5)-(2+6+2)=0$ $25652\div 11=2332$. Esempio \num{4145889} $(4+4+8+9)-(1+5+8=11)$ $4145889\div 11=376899$  \\    
 12 & Se & è divisibile contemporaneamente per tre e per quattro&\num{144}  \\  
 25 & Se & il numero  formato dalle ultime due cifre è divisibile per venticinque&\\
\bottomrule
\end{tabular}
\par}




