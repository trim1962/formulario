\chapter{Intersezioni fra due circonferenze}
\begin{center}
	\includestandalone{geometria/asseradicale}
	\captionof{figure}{Asso radicale}
\end{center}\index{Circonferenza!asse!radicale}\index{Asse!radicale}
\section{Intersezione} Date due circonferenze l'intersezione fra le due circonferenze equivale a: 
\begin{equation*}
\begin{cases}
x^2+y^2+ax+by+c=0\\
x^2+y^2+a_1x+b_1y+c_1=0
\end{cases}
\end{equation*}
Se le due circonferenze non sono concentriche 
\begin{equation*}
\begin{cases}
x^2+y^2+ax+by+c=0\\
(a-a_1)x+(b-b_1)y+c-c_1=0
\end{cases}
\end{equation*}
\section{Asse radicale}
La retta \begin{equation*}
(a-a_1)x+(b-b_1)y+c-c_1=0
\end{equation*} è chiamata asse radicale\index{Circonferenza!asse!radicale}\index{Asse!radicale}