\chapter{Monomi}
\section{Definizione}
Un monomio è il prodotto fra una parte numerica\index{Monomio!parte numerica} e una parte letterale\index{Monomio!parte letterale} che non contiene divisioni.
\section{Grado} 
Somma degli esponenti della parte letterale.\index{Monomio!grado}
\section{Monomio zero}
Il monomio con parte numerica zero è chiamato monomio zero.\index{Monomio!zero}
\section{Somma monomi simili}
La somma di due monomi simili è un monomio che ha la stessa parte letteraria e per parte numerica la somma algebrica delle parti numeriche.\index{Monomio!somma simili}
\begin{equation}
ab^2+3ab^2=4ab^2
\end{equation}
\section{Somma monomi non simili}
La somma di due monomi non simili  sono i due monomi non simili.\index{Monomio!somma non simili}
\begin{equation}
2a^3b^2+3ab^2=2a^3b^2+3ab^2
\end{equation}
\section{Prodotto}
Il prodotto di due monomi è un monomio che ha per parte numerica il prodotto algebrico delle parti numeriche e per parte letterale il prodotto delle parti numeriche.\index{Monomio!prodotto}