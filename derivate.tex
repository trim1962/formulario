% !TeX encoding = UTF-8
% !TeX spellcheck = it_IT
% !TeX root = formulario.tex
\chapter{Derivata}
\section{Definizione}
\begin{equation}
\dfrac{dy}{dx}=\lim_{h \to 0}\dfrac{f(x+h)+f(x)}{h}=f'(x)
\end{equation}\index{Derivata!definizione}
\section{Derivate funzioni}
\begin{equation}
\OpD{a}=0\quad a\in\R
\end{equation}\index{Derivata!costante}
\begin{equation}
\OpD{ax}=a\quad a\in\R
\end{equation}
\begin{equation}
\OpD{x^n}=nx^{n-1}
\end{equation}\index{Derivata!funzione potenza}
\begin{equation}
\OpD{\dfrac{1}{x}}=-\dfrac{1}{x^2}\quad x\neq 0
\end{equation}\index{Derivata!reciproco}
\begin{equation}
\OpD{\sqrt{x}}=\dfrac{1}{2\sqrt{x}}\quad x> 0
\end{equation}\index{Derivata!radice}
\begin{equation}
\OpD{k\cdot f( x )}=k\cdot\OpD{f(x)}\quad a\in\R
\end{equation}\index{Derivata!costante per funzione}
\begin{equation}
\OpD{f( x )+g( x )+h( x )}=\OpD{f( x )}+\OpD{g( x )}+\OpD{h( x )}
\end{equation}\index{Derivata!somma funzioni}
\begin{equation}
\OpD{f( x )\cdot g( x )}=\OpD{f( x )}\cdot g( x )+f( x )\cdot\OpD{g( x )}
\end{equation}\index{Derivata!prodotto funzioni}
\begin{equation}
\OpD{\dfrac{f( x )}{g( x )}}=\dfrac{\OpD{f( x )}\cdot g( x )-f( x )\cdot\OpD{g( x )}} {[g( x )]^2}
\end{equation}\index{Derivata!quoziente funzioni}