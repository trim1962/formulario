\newglossaryentry{addendog}{name={Addendo},description={termine dell'addizione}}
\newglossaryentry{albbing}{name={Albero binario},description={un albero binario è formato da un nodo detto radice da cui si staccano due nodi detti figli. Un nodo senza figli è detto foglia}}
\newglossaryentry{angolnegativoog}{name={Angolo negativo},description={un angolo che per costruirlo si è utilizzato un movimento orario}}
\newglossaryentry{angoloacutog}{name={Angolo acuto},description={un angolo che è minore di un angolo retto}}
\newglossaryentry{angologirog}{name={Angolo giro},description={un angolo in cui i due lati coincidono}}
\newglossaryentry{angolog}{name={Angolo},description={parte di piano compresa fra due semirette che hanno la stessa origine}}
\newglossaryentry{angolopiattg}{name={Angolo piatto},description={un angolo che è la metà di un angolo giro}}
\newglossaryentry{angolopositivog}{name={Angolo positivo},description={un angolo che per costruirlo si è utilizzato un movimento antiorario}}
\newglossaryentry{angolorettog}{name={Angolo retto},description={un angolo che è la metà di un angolo piatto}}
\newglossaryentry{angolottusog}{name={Angolo ottuso},description={un angolo che è maggiore angolo retto}}
\newglossaryentry{circgoniog}{name={Circonferenza goniometrica},description={circonferenza con centro nell'origine e di raggio unitario}}
\newglossaryentry{coefficienteangg}{name={Coefficiente angolare},description={coefficiente $m$, dell'equazione $y=mx+q$ legato all'inclinazione della retta }}
\newglossaryentry{coniunumcompg}{name={Numero coniugato},description=il coniugato di un numero complesso è il numero con la parte immaginaria opposta ${z=a-\uimm b}$}
\newglossaryentry{coorpolog}{name={Coordinate polari},description={un sistema di coordinate polari individua la posizione di un punto $P$ nel piano, tramite una coppia $(\rho;\theta)$ dove il primo numero è la distanza del punto detto polo e da un angolo $\theta$ misurato in senso antiorario da una semiretta di origine il polo}}
\newglossaryentry{costanteg}{name={Costante},description={una costante è un carattere che rappresenta una quantità numerica non nota ma fissata}}
\newglossaryentry{denominatoreg}{name={Denominatore},description={in una frazione è la parte scritta sotto la linea di frazione ed è sempre diverso da zerov}}
\newglossaryentry{differenzag}{name={Differenza},description={risultato sottrazione}}
\newglossaryentry{distanzag}{name={Distanza},description={Esprime la misura della lontananza di due punti. La distanza tra due punti nel piano, si ottiene utlizzando il teorema di Pitagora ed è $d(AB)=\sqrt{(x_1-x_0)^2+(y_1-y_0)^2}$ }}
\newglossaryentry{dividendog}{name={Dividendo},description={primo termine della divisione}}
\newglossaryentry{divisoreg}{name={Divisore},description={secondo termine della divisione}}
\newglossaryentry{dominioequag}{name={Dominio equazione},description={insieme dei valori per cui l'equazione esiste}}
\newglossaryentry{equazioneDetg}{name={Equazione determinata},description={è un'equazione con un numero finito di soluzioni}}
\newglossaryentry{equazioneIdentg}{name={Equazione indeterminata},description={è un'equazione con un numero infinito di soluzioni}}
\newglossaryentry{equazioneImpg}{name={Equazione impossibile},description={è un'equazione che non ha soluzioni}}
\newglossaryentry{equazionealgebricag}{name={Equazione algebrica},description={equazione scritta in forma polinomiale}}
\newglossaryentry{equazioneequig}{name={Equazioni equivalenti},description={sono equazioni che hanno le stesse soluzioni}}
\newglossaryentry{equazionefrag}{name={Equazione frazionaria},description={è un'equazione che ha al denominatore l'incognita}}
\newglossaryentry{equazioneg}{name={Equazione},description={uguaglianza condizionata fra due espressioni algebriche}}
\newglossaryentry{equazioneinterag}{name={Equazione intera},description={nessuna delle incognite appare mai al denominatore delle frazioni che la compongono}}
\newglossaryentry{equazioneirraziog}{name={Equazione irrazionale},description={almeno una delle incognite appare come argomento di radice}}
\newglossaryentry{equazionenormaleg}{name={Forma normale},description={una equazione è in forma normale se tutti i termini sono a primo membro ordinati}}
\newglossaryentry{equazioneraziog}{name={Equazione razionale},description={nessuna delle incognite appare come argomento delle radici}}
\newglossaryentry{fasenumcompg}{name={Fase},description=angolo che nel piano complesso un vettore forma con l'asse reale}
\newglossaryentry{fattoreg}{name={Fattore},description={termine della moltiplicazione}}
\newglossaryentry{figconcag}{name={Figura concava},description={una figura è concava se presi due qualunque suoi punti il segmento che li congiunge non appartiene alla figura}}
\newglossaryentry{figconveg}{name={Figura convessa},description={una figura è convessa se presi due qualunque suoi punti il segmento che li congiunge appartiene alla figura}}
\newglossaryentry{frazineappag}{name={Frazione apparente},description={nella frazione apparente il numeratore è multiplo del denominatore}}
\newglossaryentry{frazinpropg}{name={Frazione impropria},description={nella frazione impropria il numeratore è maggiore del denominatore}}
\newglossaryentry{frazionegeng}{name={Frazione generatrice},description={la frazione corrispondente ad un numero decimale dato}}
\newglossaryentry{frazionesempg}{name={Semplificare una frazione},description={semplificare una frazione significa dividere il numeratore e il denominatore per il loro Massimo Comune Divisore ($\mcd$). Per la proprietà invariantiva la frazione ottenuta è equivalente a quella data}}
\newglossaryentry{frazioniequivg}{name={Frazioni equivalenti},description={due frazioni sono equivalenti quando rappresentano lo stesso quoziente.}}
\newglossaryentry{frazpropg}{name={Frazione propria},description={nella frazione propria il numeratore è minore del denominatore}}
\newglossaryentry{gradog}{name={Grado sessagesimale},description={è la trecentosessantesima parte di un angolo giro}}
\newglossaryentry{gradositemag}{name={Grado sistema},description={il grado di un sistema, di più equazioni in tante incognite,  è il prodotto dei gradi delle equazioni del sistema ridotte in forma normale.}}
\newglossaryentry{insdisg}{name={Insieme discreto},description={un insieme è discreto nel senso che fra un numero e il suo successivo non vi è nessun altro elemento dell'insieme}}
\newglossaryentry{intervallog}{name={Intervallo},description={Insieme ordinato di punti compresi tra un punto $a$ che precede tutti gli elementi e un elemento $b$ che segue tutti gli elementi. Un intervallo è chiuso se comprende gli estremi. Un intorno è aperto se non comprende gli estremi. Un intervallo è chiuso a destra ma non a sinistra se comprende l'estremo destro ma non il sinistro. Un intervallo è chiuso a sinistra ma non a destra se comprende l'estremo sinistro ma non il destro.}}
\newglossaryentry{intervallononlimg}{name={Intervallo non limitato},description={un intorno illimitato a destra è l'insieme dei valori che seguono un elemento fissato $a$. Un intorno illimitato a sinistra è l'insieme dei valori che precedono un valore prefissato $b$}}
\newglossaryentry{intornopuntog}{name={Intorno circolare di un punto},description={Un intorno circolare di un punto è un intervallo aperto del tipo $I(x_0)=\left(x_0-\delta,x_0+\delta\right)$ $\quad\delta>0$},see={intervallog}}
\newglossaryentry{minuendog}{name={Minuendo},description={primo termine sottrazione}}
\newglossaryentry{minutog}{name={Minuto},description={è la sessantesima parte di un grado sessagesimale}}
\newglossaryentry{multiplog}{name={Multiplo},description={un numero $a$ è multiplo di un altro numero $b$ se esiste un numero $c$ tale che $a=b\cdot c$ }}
\newglossaryentry{numcompg}{name={Numero complesso},description=un numero complesso ${z=a+\uimm b}$}
\newglossaryentry{numeratoreg}{name={Numeratore},description={in una frazione è la parte scritta sopra la linea di frazione}}
\newglossaryentry{numerireciprocig}{name={Numeri reciproci},description={due numeri sono reciproci se il loro prodotto è uno}}
\newglossaryentry{numerodecimaleg}{name={Numero decimale},description={numero formato da due parti separate dalla virgola chiamate parte intera e parte decimale}}
\newglossaryentry{numerodecimalfinitog}{name={Numero decimale finito},description={numero decimale con la parte decimale composta da un numero finito di cifre},see={numerodecimaleg}}
\newglossaryentry{numerodecimalinfinitog}{name={Numero decimale infinito},description={numero decimale con la parte decimale composta da un numero infinito di cifre},see={numerodecimaleg}}	
\newglossaryentry{numerodecimalinfinitoperg}{name={Numero decimale periodico},description={numero decimale con la parte decimale composta da un numero finito di cifre, dette periodo, che si ripetono all'infinito},see={numerodecimaleg}}
\newglossaryentry{numerodecimalinfinitopermistg}{name={Numero decimale periodico misto},description={numero decimale con la parte decimale divisa in una parte finita detta antiperiodo e  da un numero finito di cifre, dette periodo, che si ripetono all'infinito},see={numerodecimaleg}}
\newglossaryentry{numng}{name={Numeri naturali},description={insieme numerico $\Ni=\Set{0,1,2,3,\dots,}$}}
\newglossaryentry{numnprimifralorog}{name={Numeri primi fra loro},description={due numeri sono primi fra loro se l'unico numero che li divide entrambi è uno}}	
\newglossaryentry{opaddg}{name={Addizione},description={operazione binaria}}
\newglossaryentry{opdiffg}{name={Sottrazione},description={operazione binaria}}
\newglossaryentry{opdifg}{name={Moltiplicazione},description={operazione binaria}}
\newglossaryentry{opdivfg}{name={Divisione},description={operazione binaria inversa della moltiplicazione}}
\newglossaryentry{parallelogrammag}{name={Parallelogramma},description={quadrilatero con quattro lati paralleli}}
\newglossaryentry{periodofun}{name={Periodo funzione},description=rappresenta il più piccolo valore $T$ per cui ${f(x)=f(x+T)}$ per ogni x }
\newglossaryentry{periodonum}{name={Periodo numero},description={in un numero decimale indica una sequenza di cifre che si ripete ciclicamente nella sua rappresentazione},see={numerodecimaleg}}
\newglossaryentry{pianoArganGaussg}{name={Piano di Argand-Gauss},description=piano complesso in cui l'asse $x$ è l'asse reale mentre l'asse $y$ è la retta immaginaria}
\newglossaryentry{potenzag}{name={Potenza},description={operazione binaria}}
\newglossaryentry{primog}{name={Primo},description={numero divisibile solo per se stesso e per l'unità}}
\newglossaryentry{primomembroequazioneg}{name={Primo membro},description={in una equazione è il termine che si trova a sinistra del segno di uguaglianza}}
\newglossaryentry{primoprincipioequig}{name={Primo principio equivalenza},description={sommando o sottraendo la stessa quantità al primo e al secondo membro di una equazione si ottiene una equazione equivalente a quella data}}
\newglossaryentry{prodottog}{name={Prodotto},description={risultato moltiplicazione}}
\newglossaryentry{quozienteg}{name={Quoziente},description={risultato della divisione}}
\newglossaryentry{radianteg}{name={Radiante},description={un angolo al centro ha l'ampiezza di un radiante quando sottende sulla circonferenza un arco uguale al raggio della circonferenza}}
\newglossaryentry{regolacang}{name={Regola cancellazione},description={se lo stesso termine è al primo e al secondo membro di un'equazione, allora può essere cancellato}}
\newglossaryentry{regolatraspg}{name={Regola trasporto},description={spostando un termine dal primo al secondo membro di un'equazione e viceversa bisogna cambiargli di segno}}
\newglossaryentry{risequag}{name={Risolvere un'equazione},description={trovare le soluzioni dell'equazione}}
\newglossaryentry{scompfatprimig}{name={Scomposizione in fattori primi},description={scrivere un numero come prodotto di numeri primi}}
\newglossaryentry{secondog}{name={Secondo},description={è la sessantesima parte di un minuto}}
\newglossaryentry{secondomembroequazioneg}{name={Secondo membro},description={in una equazione è il termine che si trova a destra del segno di uguaglianza}}
\newglossaryentry{secondoprincipioequig}{name={Secondo principio equivalenza},description={moltiplicando o dividendo per  la stessa quantità diversa da zero il primo e il secondo membro di una equazione si ottiene una equazione equivalente a quella data}}
\newglossaryentry{separazionevarg}{name={Separare le variabili},description={tecnica che consiste nel portare al primo membro di un'equazione le incognite e al secondo membro le parti numeriche}}
\newglossaryentry{sistemaequag}{name={Sistema},description={risolvere contemporaneamente due o più equazioni}}
\newglossaryentry{soluzionequazioneg}{name={Soluzione},description={valore che sostituito all'incognita,  rende vera l'equazione}}
\newglossaryentry{soluzionesistemag}{name={Soluzione sistema},description={una soluzione per un sistema è un insieme ordinato di valori che sono soluzione per ogni equazione del sistema}}
\newglossaryentry{sommag}{name={Somma},description={risultato dell'addizione}}
\newglossaryentry{sottraendog}{name={Sottraendo},description={secondo termine sottrazione}}
\newglossaryentry{ternaPitag}{name={Terna pitagorica},description={tre numeri legati dalla relazione di Pitagora}}
\newglossaryentry{trapeziog}{name={Trapezio},description={quadrilatero con due lati paralleli dette basi e i rimanenti lati obliqui}}
\newglossaryentry{trapezioisog}{name={Trapezio isoscele},description={trapezio con i lati obliqui uguali},see={trapeziog}}
\newglossaryentry{trapezioretg}{name={Trapezio rettangolo},description={trapezio con un lato obliquo perpendicolare alle basi},see={trapeziog}}
\newglossaryentry{triangoloeqig}{name={Triangolo equilatero},description={triangolo con tre lati uguali}}
\newglossaryentry{triangolog}{name={Triangolo},description={poligono ottenuto da tre punti non allineati, ha tre lati e tre vertici}}
\newglossaryentry{triangoloisog}{name={Triangolo isoscele},description={triangolo con due lati uguali}}
\newglossaryentry{triangolorettangolog}{name={Triangolo rettangolo},description={triangolo in cui due lati formano un angolo retto},see={angolorettog}}
\newglossaryentry{triangoloscag}{name={Triangolo scaleno},description={triangolo con tre lati disuguali}}
\newglossaryentry{uniimgg}{name={Unità immaginaria},description=simbolo che ha la proprietà che se è elevato al quadrato vale meno uno}
\newglossaryentry{variabileg}{name={Variabile},description={una variabile è un carattere che rappresenta una quantità numerica non nota}}
 \newglossaryentry{intornosinpuntog}{name={Intorno sinistro di un punto},description={Un intorno sinistro di un punto è un intervallo aperto del tipo $I^{-}(x_0)=\left(x_0-\delta,x_0\right)$    $\delta>0$}, see={intervallog,intornopuntog}}
 \newglossaryentry{intornodespuntog}{name={Intorno destro di un punto},description={Un intorno destro di un punto è un intervallo aperto del tipo $I^{+}(x_0)=\left(x_0,x_0+\delta\right)$ $\delta>0$},see=[vedi anche]{intervallog,intornopuntog}}
 \newglossaryentry{intornopiuinfg}{name={Intorno di più infinto},description={Un intorno di un più infinto è intervallo aperto del tipo $I(+\infty)=\left(M,+\infty\right)$ $M\in\R$},see={intervallog,intornopuntog}}
 \newglossaryentry{intornomenoinfg}{name={Intorno di meno infinto},description={Un intorno di un meno infinto è intervallo aperto del tipo $I(-\infty)=\left(-\infty,M\right)$ $M\in\R$},see={intervallog,intornopuntog}}
  \newglossaryentry{Numeroppostog}{name={Numero opposto},description={Due numeri sono opposti se hanno lo stesso valore assoluto ma segno diverso}}
  \newglossaryentry{latoppostog}{name={Lato opposto},description={In un triangolo, un lato è opposto ad un angolo se il lato non è lato dell'angolo}}
  \newglossaryentry{Latoadiacenteg}{name={Lato adiacente},description={In un triangolo, un lato è adiacentead un angolo se il lato è lato dell'angolo}}
  \newglossaryentry{angolosupplentareg}{name={Angolo supplemetare},description={un angolo che sommato ad un angolo dato forma un angolo piatto}}
   \newglossaryentry{angolocomplementareg}{name={Angolo complementare},description={un angolo che si ottiene sottraendo un angolo dato ad un angolo retto}}
   \newglossaryentry{baricentrog}{name={Baricentro},description={in un triangolo, punto di intersezione fra le mediane},see={medianag}}
   \newglossaryentry{medianag}{name={Mediana},description={in un triangolo, segmento che congiunge un vertice con il punto medio del lato opposto}}
   \newglossaryentry{diametrog}{name={Diametro},description={segmento che passa per il centro di una figura chiusa}}
    \newglossaryentry{disparig}{name={Dispari},description={numero non divisibile per due}}
    \newglossaryentry{disuguaglianzag}{name={Disuguaglianza},description={una delle seguenti relazioni $a<b$ a minore di b, $a>b$ a maggiore di b, $a\leq b$ a minore o uguale a b, $a\geq b$ a maggiore o uguale a b}}
     \newglossaryentry{numerodivisibileg}{name={Numero divisibile},description={numero che può essere diviso esattamente da un altro numero}}
     \newglossaryentry{fascog}{name={Fascio},description={insieme di rette che passano per lo stesso punto o un insieme di rette parallele}}
     \newglossaryentry{fattoreng}{name={Fattore},description={intero che divide esattemente un intero dato}}
     \newglossaryentry{interog}{name={Intero},description={numero che può essere espresso come somma o differnza di due numeri naturali},see={numerorelativog}}
  \newglossaryentry{numerorelativog}{name={Numero relativo},description={numero che può essere positivo, negatico o zero}}